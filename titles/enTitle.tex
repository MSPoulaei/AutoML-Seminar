% !TeX root=main.tex
% در این فایل، عنوان پایان‌نامه، مشخصات خود و چکیده پایان‌نامه را به انگلیسی، وارد کنید.

%%%%%%%%%%%%%%%%%%%%%%%%%%%%%%%%%%%%
\baselineskip=.6cm
\begin{latin}
\latinuniversity{Iran University of Science and Technology}
\latinfaculty{Computer Engineering Department}
\latinsubject{Computer Engineering-Artificial Intelligence}
% \latinfield{Artificial Intelligence}
\latintitle{Review of deep learning methods in brain tumor segmentation in medical images}
\firstlatinsupervisor{Dr. Mohsen Soryani}
%\secondlatinsupervisor{Second Supervisor}
% \firstlatinadvisor{First Advisor}
%\secondlatinadvisor{Second Advisor}
\latinname{Morteza}
\latinsurname{Hajiabadi}
\latinthesisdate{November 2023}
\latinkeywords{Deep learning, segmentation, brain tumor, medical images, artificial intelligence}
\en-abstract{
Diagnosis and segmentation of brain tumors is one of the most challenging and critical issues in medical imaging. Accuracy in identifying tumorous areas is very important because it is essential for treatment decisions and prediction of disease outcomes.
\\
In recent years, thanks to significant advances in the field of deep learning, artificial intelligence models have been able to play an important role in improving the identification of brain tumors. Among these developments, we can mention the use of deep neural networks.
\\
Deep neural networks can analyze brain images using convolutional layers and attention layers and detect tumor areas. These networks allow doctors to determine tumor areas with high accuracy and plan more effective treatment.
\\
Overall, advances in deep learning in brain tumor diagnosis and segmentation have provided clinicians with more powerful tools for treatment decision-making. These advances have helped increase the speed and accuracy of brain tumor diagnosis, which ultimately leads to improved treatment and outcomes for patients. In this seminar report, we examine the recent developments in this field and the challenges and problems of the existing methods.
\\
Our research shows that deep learning methods have much better performance compared to traditional methods. In addition, among deep learning methods, Transformer-based methods have better performance than methods based on Convolutional layers, but they also have problems such as the need for too much data. In recent years, interactive methods for the segmentation of brain tumors have obtained very good results, which has caused more attention of specialists in this field to these methods.
}
\latinfirstPage
\end{latin}
