% !TeX root=main.tex
% در این فایل، عنوان پایان‌نامه، مشخصات خود، متن تقدیمی‌، ستایش، سپاس‌گزاری و چکیده پایان‌نامه را به فارسی، وارد کنید.
% توجه داشته باشید که جدول حاوی مشخصات پروژه/پایان‌نامه/رساله و همچنین، مشخصات داخل آن، به طور خودکار، درج می‌شود.
%%%%%%%%%%%%%%%%%%%%%%%%%%%%%%%%%%%%
% دانشگاه خود را وارد کنید
\university{علم و صنعت ایران}
% دانشکده، آموزشکده و یا پژوهشکده  خود را وارد کنید
\faculty{دانشکده مهندسی کامپیوتر}
% گروه آموزشی خود را وارد کنید
\department{گروه هوش مصنوعی و رباتیک}
% گروه آموزشی خود را وارد کنید
\subject{مهندسی کامپیوتر}
% گرایش خود را وارد کنید
\field{هوش مصنوعی و رباتیکز}
% عنوان پایان‌نامه را وارد کنید
\title{مروری بر روش های یادگیری عمیق در قطعه‌بندی تومورهای مغزی در تصاویرپزشکی}
% نام استاد(ان) راهنما را وارد کنید
\firstsupervisor{دکتر محسن سریانی}
% \secondsupervisor{استاد راهنمای دوم}
% نام استاد(دان) مشاور را وارد کنید. چنانچه استاد مشاور ندارید، دستور پایین را غیرفعال کنید.
% \firstadvisor{استاد مشاور اول}
%\secondadvisor{استاد مشاور دوم}
% نام دانشجو را وارد کنید
\name{مرتضی}
% نام خانوادگی دانشجو را وارد کنید
\surname{حاجی آبادی}
% شماره دانشجویی دانشجو را وارد کنید
\studentID{۴۰۱۷۲۲۰۵۵}
% تاریخ پایان‌نامه را وارد کنید
\thesisdate{آبان ماه ۱۴۰۲}
% به صورت پیش‌فرض برای پایان‌نامه‌های کارشناسی تا دکترا به ترتیب از عبارات «پروژه»، «پایان‌نامه» و »رساله» استفاده می‌شود؛ اگر  نمی‌پسندید هر عنوانی را که مایلید در دستور زیر قرار داده و آنرا از حالت توضیح خارج کنید.
%\projectLabel{پایان‌نامه}

% به صورت پیش‌فرض برای عناوین مقاطع تحصیلی کارشناسی تا دکترا به ترتیب از عبارات «کارشناسی»، «کارشناسی ارشد» و »دکترا» استفاده می‌شود؛ اگر  نمی‌پسندید هر عنوانی را که مایلید در دستور زیر قرار داده و آنرا از حالت توضیح خارج کنید.
%\degree{}

\firstPage
\besmPage
% \davaranPage

% %\vspace{.5cm}
% % در این قسمت اسامی اساتید راهنما، مشاور و داور باید به صورت دستی وارد شوند
% %\renewcommand{\arraystretch}{1.2}
% \begin{center}
% \begin{tabular}{| p{8mm} | p{18mm} | p{.17\textwidth} |p{14mm}|p{.2\textwidth}|c|}
% \hline
% ردیف	& سمت & نام و نام خانوادگی & مرتبه \newline دانشگاهی &	دانشگاه یا مؤسسه &	امضـــــــــــــا\\
% \hline
% ۱  &	استاد راهنما				 & دکتر \newline محمود فتحی & دانشیار & دانشگاه \newline علم و صنعت ایران &  \\
% \hline
% ۲ &     استاد مشاور				 & دکتر \newline ناصر مزینی & استادیار & دانشگاه \newline علم و صنعت ایران & \\
% \hline
% ۳ &      استاد مدعو\newline  خارجی			 & دکتر \newline محمدحسن \newline قاسمیان & استاد & دانشگاه \newline تربیت مدرس & \\
% \hline
% ۴ &	استاد مدعو\newline  خارجی			 & دکتر \newline  نصرالله مقدم & استادیار & دانشگاه \newline  تربیت مدرس& \\
% \hline
% ۵ &	استاد مدعو\newline  داخلی			 & دکتر\newline  رضا برنگی & استادیار & دانشگاه \newline  علم و صنعت ایران & \\
% \hline
% ۶ &	استاد مدعو\newline  داخلی			 & دکتر\newline  محسن سریانی & استادیار & دانشگاه \newline  علم و صنعت ایران & \\
% \hline
% ۷ &	استاد مدعو\newline  داخلی			 &دکتر \newline محمدرضا جاهدمطلق & دانشیار& دانشگاه \newline  علم و صنعت ایران & \\
% \hline
% \end{tabular}
% \end{center}

% \esalatPage
% \mojavezPage


% % چنانچه مایل به چاپ صفحات «تقدیم»، «نیایش» و «سپاس‌گزاری» در خروجی نیستید، خط‌های زیر را با گذاشتن ٪  در ابتدای آنها غیرفعال کنید.
%  % پایان‌نامه خود را تقدیم کنید!

%  \newpage
% \thispagestyle{empty}
% {\Large تقدیم به:}\\
% \begin{flushleft}
% {\huge
% همسر و فرزندانم\\
% \vspace{7mm}
% و\\
% \vspace{7mm}
% پدر و مادرم
% }
% \end{flushleft}


% % سپاس‌گزاری
% \begin{acknowledgementpage}
% سپاس خداوندگار حکیم را که با لطف بی‌کران خود، آدمی را زیور عقل آراست.


% در آغاز وظیفه‌  خود  می‌دانم از زحمات بی‌دریغ استاد  راهنمای خود،  جناب آقای دکتر ...، صمیمانه تشکر و  قدردانی کنم  که قطعاً بدون راهنمایی‌های ارزنده‌  ایشان، این مجموعه  به انجام  نمی‌رسید.

% از جناب  آقای  دکتر ...   که زحمت  مطالعه و مشاوره‌  این رساله را تقبل  فرمودند و در آماده سازی  این رساله، به نحو احسن اینجانب را مورد راهنمایی قرار دادند، کمال امتنان را دارم.

% همچنین لازم می‌دانم از پدید آورندگان بسته زی‌پرشین، مخصوصاً جناب آقای  وفا خلیقی، که این پایان‌نامه با استفاده از این بسته، آماده شده است و همه دوستانمان در گروه پارسی‌لاتک کمال قدردانی را داشته باشم.

%  در پایان، بوسه می‌زنم بر دستان خداوندگاران مهر و مهربانی، پدر و مادر عزیزم و بعد از خدا، ستایش می‌کنم وجود مقدس‌شان را و تشکر می‌کنم از خانواده عزیزم به پاس عاطفه سرشار و گرمای امیدبخش وجودشان، که بهترین پشتیبان من بودند.
% % با استفاده از دستور زیر، امضای شما، به طور خودکار، درج می‌شود.
% \signature 
% \end{acknowledgementpage}
%%%%%%%%%%%%%%%%%%%%%%%%%%%%%%%%%%%%
% کلمات کلیدی پایان‌نامه را وارد کنید
\keywords{یادگیری عمیق، قطعه‌بندی، تومور مغزی، تصاویر پزشکی، هوش مصنوعی}
%چکیده پایان‌نامه را وارد کنید، برای ایجاد پاراگراف جدید از \\ استفاده کنید. اگر خط خالی دشته باشید، خطا خواهید گرفت.
\fa-abstract{تشخیص و قطعه‌بندی تومورهای مغزی یکی از چالش‌برانگیزترین و حیاتی‌ترین مسائل در تصویربرداری پزشکی است. دقت در شناسایی ناحیه‌های توموردار اهمیت بسیار زیادی دارد زیرا برای تصمیم‌گیری درمانی و پیش‌بینی نتایج بیماری اساسی است.
\\
در سال‌های اخیر، به لطف پیشرفت‌های قابل توجه در حوزه یادگیری عمیق، مدل‌های هوش مصنوعی توانسته‌اند تا نقش مهمی در بهبود شناسایی تومورهای مغزی ایفا کنند. از جمله این پیشرفت‌ها می‌توان به استفاده از شبکه‌های عصبی عمیق اشاره کرد.
% ، مدل‌های یادگیری انتقالی و یادگیری نیمه‌نظارتی اشاره کرد.
\\
شبکه‌های عصبی عمیق می‌توانند با استفاده از لایه‌های پیچشی و لایه های توجه تصاویر مغز را تحلیل کرده و نواحی توموردار را تشخیص دهند. این شبکه‌ها به پزشکان این امکان را می‌دهند تا با دقت بالا نواحی توموردار را تعیین کرده و برنامه‌ریزی درمانی موثرتری را انجام دهند.
\\
% یادگیری انتقالی از مدل‌های پیشینی که برای تصویربرداری از دیگر بخش‌های بدن تربیت شده‌اند به منظور افزایش دقت ناحیه‌بندی تومور مغزی استفاده می‌شود. این روش از دانش مدل‌های پیشین به منظور تشخیص تومور مغزی بهره می‌برد.
% \\
% علاوه بر این، یادگیری نیمه‌نظارتی از داده‌های برچسب‌دار و بدون برچسب به منظور بهبود عملکرد مدل‌ها استفاده می‌کند. این امکان را می‌دهد تا با تعداد محدودی داده برچسب‌دار، دقت ناحیه‌بندی را افزایش دهیم.
به طور کلی، پیشرفت‌های یادگیری عمیق در تشخیص و قطعه‌بندی تومور مغزی به پزشکان ابزارهای قوی‌تری برای تصمیم‌گیری درمانی ارائه داده‌اند. این پیشرفت‌ها به افزایش سرعت و دقت تشخیص تومور مغزی کمک کرده‌اند، که در نهایت به بهبود مراحل درمان و نتایج برای بیماران منجر می‌شود. در این گزارش سمینار پیشرفت های اخیر در این زمینه و چالش ها و مشکلات روش های موجود را مورد بررسی قرار می‌دهیم.
\\
بررسی های ما نشان می‌دهد روش های یادگیری عمیق در مقایسه با روش های سنتی از عملکرد بسیار بهتری برخوردارند. علاوه بر آن در بین روش های یادگیری عمیق، روش های مبتنی بر مبدل از عملکرد بهتری نسبت به روش های مبتنی بر لایه های پیچشی برخوردارند اما مشکلاتی مانند نیاز بسیار زیاد به داده را نیز دارند. در سال های اخیر روش های تعاملی برای قطعه‌بندی تومورهای مغزی نتایج بسیار خوبی گرفته اند که باعث توجه بیشتر متخصصان این حوزه به این روش ها شده است.
}
\abstractPage
\newpage\clearpage