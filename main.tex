% !TEX TS-program = XeLaTeX
% Commands for running this example:
% 	 xelatex main
% 	 bibtex8 -W -c cp1256fa main
%      xindy -L persian -C utf8 -M texindy main
% 	 xelatex main
% 	 xelatex main
% End of Commands

%        نمونه پایان‌نامه آماده شده با استفاده از کلاس IUST-Thesis، نگارش 0.6
% 		محمود امین‌طوسی، دانشگاه تربیت معلم سبزوار، http://profsite.sttu.ac.ir/mamintoosi/
% 		گروه پارسی‌لاتک  http://www.parsilatex.com
%        این نسخه، بر اساس نسخه‌ 0.4 از کلاس Tabriz_Thesis آقای وحید دامن‌افشان آماده شده است. http://damanafshan.tk
%        
%        تغییرات:
%        نسخه 0.6:
%        اصلاح مشکل بسته subfig 
%----------------------------------------------------------------------------------------------
%        اگر قصد نوشتن پروژه کارشناسی را دارید، در خط زیر به جای msc، کلمه bsc و اگر قصد نوشتن پروژه دکترا
%        را دارید، کلمه phd را قرار دهید. کلیه تنظیمات لازم، به طور خودکار، اعمال می‌شود.

%        اگر مایلید پایان‌نامه شما دورو باشد به جای oneside  در خط زیر از twoside استفاده کنید
\documentclass[oneside,openany,msc]{IUST-Thesis}

% مشخصات پایان‌نامه را در فایلهای faTitle و enTitle وارد نمایید.

%       فایل commands.tex را مطالعه کنید؛ چون دستورات مربوط به فراخوانی بسته زی‌پرشین 
%       و دیگر بسته‌ها و ... در این فایل قرار دارد و بهتر است که با نحوه استفاده از آنها آشنا شوید.
% در این فایل، دستورها و تنظیمات مورد نیاز، آورده شده است.
%-------------------------------------------------------------------------------------------------------------------

% در ورژن جدید زی‌پرشین برای تایپ متن‌های ریاضی، این سه بسته، حتماً باید فراخوانی شود
\usepackage{amsthm,amssymb,amsmath}
% بسته‌ای برای تنطیم حاشیه‌های بالا، پایین، چپ و راست صفحه
\usepackage[top=40mm, bottom=40mm, left=25mm, right=35mm]{geometry}
% بسته‌‌ای برای ظاهر شدن شکل‌ها و تصاویر متن
\usepackage{graphicx}
% بسته‌ای برای رسم کادر
\usepackage{framed} 
% بسته‌‌ای برای چاپ شدن خودکار تعداد صفحات در صفحه «معرفی پایان‌نامه»
\usepackage{lastpage}
% بسته‌ و دستوراتی برای ایجاد لینک‌های رنگی با امکان جهش
\usepackage[pagebackref=false,colorlinks,linkcolor=blue,citecolor=blue]{hyperref}
% چنانچه قصد پرینت گرفتن نوشته خود را دارید، خط بالا را غیرفعال و  از دستور زیر استفاده کنید چون در صورت استفاده از دستور زیر‌‌، 
% لینک‌ها به رنگ سیاه ظاهر خواهند شد که برای پرینت گرفتن، مناسب‌تر است
%\usepackage[pagebackref=false]{hyperref}
% بسته‌ لازم برای تنظیم سربرگ‌ها
\usepackage{fancyhdr}
%
\usepackage{setspace}
\usepackage{algorithm}
\usepackage{algorithmic}
\usepackage{subfigure}
\usepackage{multirow}
\usepackage{caption}

\usepackage[subfigure]{tocloft}

% بسته‌ای برای ظاهر شدن «مراجع» و «نمایه» در فهرست مطالب
\usepackage[nottoc]{tocbibind}
% دستورات مربوط به ایجاد نمایه
\usepackage{makeidx}
\makeindex
%%%%%%%%%%%%%%%%%%%%%%%%%%
% فراخوانی بسته زی‌پرشین و تعریف قلم فارسی و انگلیسی
\usepackage[extrafootnotefeatures]{xepersian}
\settextfont[Scale=1.2,Path=fonts/]{XBNiloofar.ttf}
% \settextfont[Scale=1.2,Path=fonts/]{BNazanin.ttf}
% \setlatintextfont[Scale=0.9,Path=fonts/]{Times New Roman.ttf}
\setlatintextfont[Scale=0.9]{Times New Roman}

%%%%%%%%%%%%%%%%%%%%%%%%%%
% چنانچه می‌خواهید اعداد در فرمول‌ها، انگلیسی باشد، خط زیر را غیرفعال کنید
\setdigitfont[Scale=1,Path=fonts/]{XBZar.ttf}%{Persian Modern}
%%%%%%%%%%%%%%%%%%%%%%%%%%
% تعریف قلم‌های فارسی و انگلیسی اضافی برای استفاده در بعضی از قسمت‌های متن
% \defpersianfont\titlefont[Scale=1,Path=fonts/]{XBNiloofar.ttf}
\defpersianfont\titlefont[Scale=1,Path=fonts/]{BTitrBd.ttf}
% \defpersianfont\iranic[Scale=1.1,Path=fonts/]{XB Zar Oblique}%Italic}%
% \defpersianfont\nastaliq[Scale=1.2,Path=fonts/]{IranNastaliq}

%%%%%%%%%%%%%%%%%%%%%%%%%%
% دستوری برای حذف کلمه «چکیده»
% \renewcommand{\abstractname}{}
% دستوری برای حذف کلمه «abstract»
% \renewcommand{\latinabstract}{}
% دستوری برای تغییر نام کلمه «اثبات» به «برهان»
\renewcommand\proofname{\textbf{برهان}}
% دستوری برای تغییر نام کلمه «کتاب‌نامه» به «مراجع»
\renewcommand{\bibname}{مراجع}
% دستوری برای تعریف واژه‌نامه انگلیسی به فارسی
\newcommand\persiangloss[2]{#1\dotfill\lr{#2}\\}
% دستوری برای تعریف واژه‌نامه فارسی به انگلیسی 
\newcommand\englishgloss[2]{#2\dotfill\lr{#1}\\}
% تعریف دستور جدید «\پ» برای خلاصه‌نویسی جهت نوشتن عبارت «پروژه/پایان‌نامه/رساله»
\newcommand{\پ}{پروژه/پایان‌نامه/رساله }
% دستور جدید برای واژه نامه
\newcommand{\persianfootnote}[1]{#1}

%\newcommand\BackSlash{\char`\\}

%%%%%%%%%%%%%%%%%%%%%%%%%%
\SepMark{-}

% تعریف و نحوه ظاهر شدن عنوان قضیه‌ها، تعریف‌ها، مثال‌ها و ...
\theoremstyle{definition}
\newtheorem{definition}{تعریف}[section]
\theoremstyle{theorem}
\newtheorem{theorem}[definition]{قضیه}
\newtheorem{lemma}[definition]{لم}
\newtheorem{proposition}[definition]{گزاره}
\newtheorem{corollary}[definition]{نتیجه}
\newtheorem{remark}[definition]{ملاحظه}
\theoremstyle{definition}
\newtheorem{example}[definition]{مثال}

%\renewcommand{\theequation}{\thechapter-\arabic{equation}}
%\def\bibname{مراجع}
\numberwithin{algorithm}{chapter}
\def\listalgorithmname{فهرست الگوریتم‌ها}
\def\listfigurename{فهرست شکل ها}
\def\listtablename{فهرست جداول}

%%%%%%%%%%%%%%%%%%%%%%%%%%%%
% دستورهایی برای سفارشی کردن سربرگ صفحات
% \newcommand{\SetHeader}{
% \csname@twosidetrue\endcsname
% \pagestyle{fancy}
% \fancyhf{} 
% \fancyhead[OL,EL]{\thepage}
% \fancyhead[OR]{\small\rightmark}
% \fancyhead[ER]{\small\leftmark}
% \renewcommand{\chaptermark}[1]{%
% \markboth{\thechapter-\ #1}{}}
% }
%%%%%%%%%%%%5
%\def\MATtextbaseline{1.5}
%\renewcommand{\baselinestretch}{\MATtextbaseline}
\doublespacing
%%%%%%%%%%%%%%%%%%%%%%%%%%%%%
% دستوراتی برای اضافه کردن کلمه «فصل» در فهرست مطالب

\newlength\mylenprt
\newlength\mylenchp
\newlength\mylenapp

\renewcommand\cftpartpresnum{\partname~}
\renewcommand\cftchappresnum{\chaptername~}
\renewcommand\cftchapaftersnum{:}

\settowidth\mylenprt{\cftpartfont\cftpartpresnum\cftpartaftersnum}
\settowidth\mylenchp{\cftchapfont\cftchappresnum\cftchapaftersnum}
\settowidth\mylenapp{\cftchapfont\appendixname~\cftchapaftersnum}
\addtolength\mylenprt{\cftpartnumwidth}
\addtolength\mylenchp{\cftchapnumwidth}
\addtolength\mylenapp{\cftchapnumwidth}

\setlength\cftpartnumwidth{\mylenprt}
\setlength\cftchapnumwidth{\mylenchp}	

\makeatletter
{\def\thebibliography#1{\chapter*{\refname\@mkboth
   {\uppercase{\refname}}{\uppercase{\refname}}}\list
   {[\arabic{enumi}]}{\settowidth\labelwidth{[#1]}
   \rightmargin\labelwidth
   \advance\rightmargin\labelsep
   \advance\rightmargin\bibindent
   \itemindent -\bibindent
   \listparindent \itemindent
   \parsep \z@
   \usecounter{enumi}}
   \def\newblock{}
   \sloppy
   \sfcode`\.=1000\relax}}
\makeatother




\begin{document}

\pagenumbering{harfi}
% !TeX root=main.tex
% در این فایل، عنوان پایان‌نامه، مشخصات خود، متن تقدیمی‌، ستایش، سپاس‌گزاری و چکیده پایان‌نامه را به فارسی، وارد کنید.
% توجه داشته باشید که جدول حاوی مشخصات پروژه/پایان‌نامه/رساله و همچنین، مشخصات داخل آن، به طور خودکار، درج می‌شود.
%%%%%%%%%%%%%%%%%%%%%%%%%%%%%%%%%%%%
% دانشگاه خود را وارد کنید
\university{علم و صنعت ایران}
% دانشکده، آموزشکده و یا پژوهشکده  خود را وارد کنید
\faculty{دانشکده مهندسی کامپیوتر}
% گروه آموزشی خود را وارد کنید
\department{گروه هوش مصنوعی و رباتیک}
% گروه آموزشی خود را وارد کنید
\subject{مهندسی کامپیوتر}
% گرایش خود را وارد کنید
\field{هوش مصنوعی و رباتیکز}
% عنوان پایان‌نامه را وارد کنید
\title{بررسی مدل‌های زبانی بزرگ عامل‌محور در زمینه جستجوی معماری شبکه های عصبی و بهینه سازی ابرپارامتر}
% نام استاد(ان) راهنما را وارد کنید
\firstsupervisor{دکتر محمدرضا محمدی}
\secondsupervisor{دکتر سید صالح اعتمادی}
% نام استاد(دان) مشاور را وارد کنید. چنانچه استاد مشاور ندارید، دستور پایین را غیرفعال کنید.
% \firstadvisor{استاد مشاور اول}
%\secondadvisor{استاد مشاور دوم}
% نام دانشجو را وارد کنید
\name{محمدصادق}
% نام خانوادگی دانشجو را وارد کنید
\surname{پولائی موزیرجی}
% شماره دانشجویی دانشجو را وارد کنید
\studentID{403723196}
% تاریخ پایان‌نامه را وارد کنید
\thesisdate{آبان ماه ۱۴۰۴}
% به صورت پیش‌فرض برای پایان‌نامه‌های کارشناسی تا دکترا به ترتیب از عبارات «پروژه»، «پایان‌نامه» و »رساله» استفاده می‌شود؛ اگر  نمی‌پسندید هر عنوانی را که مایلید در دستور زیر قرار داده و آنرا از حالت توضیح خارج کنید.
%\projectLabel{پایان‌نامه}

% به صورت پیش‌فرض برای عناوین مقاطع تحصیلی کارشناسی تا دکترا به ترتیب از عبارات «کارشناسی»، «کارشناسی ارشد» و »دکترا» استفاده می‌شود؛ اگر  نمی‌پسندید هر عنوانی را که مایلید در دستور زیر قرار داده و آنرا از حالت توضیح خارج کنید.
%\degree{}

\firstPage
\besmPage
% \davaranPage

% %\vspace{.5cm}
% % در این قسمت اسامی اساتید راهنما، مشاور و داور باید به صورت دستی وارد شوند
% %\renewcommand{\arraystretch}{1.2}
% \begin{center}
% \begin{tabular}{| p{8mm} | p{18mm} | p{.17\textwidth} |p{14mm}|p{.2\textwidth}|c|}
% \hline
% ردیف	& سمت & نام و نام خانوادگی & مرتبه \newline دانشگاهی &	دانشگاه یا مؤسسه &	امضـــــــــــــا\\
% \hline
% ۱  &	استاد راهنما				 & دکتر \newline محمود فتحی & دانشیار & دانشگاه \newline علم و صنعت ایران &  \\
% \hline
% ۲ &     استاد مشاور				 & دکتر \newline ناصر مزینی & استادیار & دانشگاه \newline علم و صنعت ایران & \\
% \hline
% ۳ &      استاد مدعو\newline  خارجی			 & دکتر \newline محمدحسن \newline قاسمیان & استاد & دانشگاه \newline تربیت مدرس & \\
% \hline
% ۴ &	استاد مدعو\newline  خارجی			 & دکتر \newline  نصرالله مقدم & استادیار & دانشگاه \newline  تربیت مدرس& \\
% \hline
% ۵ &	استاد مدعو\newline  داخلی			 & دکتر\newline  رضا برنگی & استادیار & دانشگاه \newline  علم و صنعت ایران & \\
% \hline
% ۶ &	استاد مدعو\newline  داخلی			 & دکتر\newline  محسن سریانی & استادیار & دانشگاه \newline  علم و صنعت ایران & \\
% \hline
% ۷ &	استاد مدعو\newline  داخلی			 &دکتر \newline محمدرضا جاهدمطلق & دانشیار& دانشگاه \newline  علم و صنعت ایران & \\
% \hline
% \end{tabular}
% \end{center}

% \esalatPage
% \mojavezPage


% % چنانچه مایل به چاپ صفحات «تقدیم»، «نیایش» و «سپاس‌گزاری» در خروجی نیستید، خط‌های زیر را با گذاشتن ٪  در ابتدای آنها غیرفعال کنید.
%  % پایان‌نامه خود را تقدیم کنید!

%  \newpage
% \thispagestyle{empty}
% {\Large تقدیم به:}\\
% \begin{flushleft}
% {\huge
% همسر و فرزندانم\\
% \vspace{7mm}
% و\\
% \vspace{7mm}
% پدر و مادرم
% }
% \end{flushleft}


% % سپاس‌گزاری
% \begin{acknowledgementpage}
% سپاس خداوندگار حکیم را که با لطف بی‌کران خود، آدمی را زیور عقل آراست.


% در آغاز وظیفه‌  خود  می‌دانم از زحمات بی‌دریغ استاد  راهنمای خود،  جناب آقای دکتر ...، صمیمانه تشکر و  قدردانی کنم  که قطعاً بدون راهنمایی‌های ارزنده‌  ایشان، این مجموعه  به انجام  نمی‌رسید.

% از جناب  آقای  دکتر ...   که زحمت  مطالعه و مشاوره‌  این رساله را تقبل  فرمودند و در آماده سازی  این رساله، به نحو احسن اینجانب را مورد راهنمایی قرار دادند، کمال امتنان را دارم.

% همچنین لازم می‌دانم از پدید آورندگان بسته زی‌پرشین، مخصوصاً جناب آقای  وفا خلیقی، که این پایان‌نامه با استفاده از این بسته، آماده شده است و همه دوستانمان در گروه پارسی‌لاتک کمال قدردانی را داشته باشم.

%  در پایان، بوسه می‌زنم بر دستان خداوندگاران مهر و مهربانی، پدر و مادر عزیزم و بعد از خدا، ستایش می‌کنم وجود مقدس‌شان را و تشکر می‌کنم از خانواده عزیزم به پاس عاطفه سرشار و گرمای امیدبخش وجودشان، که بهترین پشتیبان من بودند.
% % با استفاده از دستور زیر، امضای شما، به طور خودکار، درج می‌شود.
% \signature 
% \end{acknowledgementpage}
%%%%%%%%%%%%%%%%%%%%%%%%%%%%%%%%%%%%
% کلمات کلیدی پایان‌نامه را وارد کنید
\keywords{یادگیری ماشین خودکار، مدل‌های زبانی بزرگ، سیستم‌های عامل-محور، بهینه‌سازی ابرپارامترها، جستجوی معماری عصبی، تولید تقویت‌شده با بازیابی، یادگیری درون‌متنی، انتقال یادگیری}
%چکیده پایان‌نامه را وارد کنید، برای ایجاد پاراگراف جدید از \\ استفاده کنید. اگر خط خالی دشته باشید، خطا خواهید گرفت.
\fa-abstract{
    رویکردهای سنتی یادگیری ماشین خودکار با استفاده از بهینه‌سازی بیزی و الگوریتم‌های تکاملی فاقد تفسیرپذیری بوده و در بهره‌برداری از دانش حوزه‌ای دچار چالش هستند. مدل‌های زبانی بزرگ با توانایی‌های درک زبان طبیعی، استدلال و تولید کد، تحولی بنیادین ارائه می‌دهند. این سمینار بررسی می‌کند چگونه عامل‌های مبتنی بر مدل‌های زبانی بزرگ می‌توانند سیستم‌های یادگیری ماشین خودکار را با عمل به عنوان عامل‌های هوشمند، آگاه از زمینه و تطبیق‌پذیر دگرگون سازند. این پژوهش یک مرور نظام‌مند از رویکردهای اخیر از سه منظر ارائه می‌دهد: (۱) معماری عامل: سامانه‌های تک‌عاملی (بهینه‌سازی مستقیم، عملگرهای تکاملی، کنترل‌گرهای جریان) در مقابل چندعاملی (همکاری مبتنی بر نقش، هماهنگی سلسله‌مراتبی) (۲) منابع دانش: دانش درونی (تاریخچه بهینه‌سازی، یادگیری درون‌متنی) در برابر بازیابی بیرونی (استخراج از ادبیات، مخازن مدل) و (۳) قالب خروجی: پیکربندی‌های ساختاریافته، تولید کد، نمایش‌های درختی و رویکردهای ترکیبی. تحلیل نشان می‌دهد حوزه به طور مساوی بین معماری‌های تک‌عاملی و چندعاملی تقسیم شده، با تکیه اکثر سامانه‌ها بر دانش درونی و بهره‌گیری اندک از دانش بیرونی. قالب‌های خروجی تنوع بالایی دارند و رویکردهای ترکیبی تعادل بین خوانایی ماشینی و بیان‌پذیری را فراهم می‌کنند. سمینار مسائل باز و جهت‌های آینده شامل بهینه‌سازی با منابع محدود، یکپارچه‌سازی دانش پیشرفته و چارچوب‌های چندعاملی را شناسایی می‌کند. پیشنهاد پایان‌نامه برای طراحی سیستم عامل-محور جهت انتخاب و تنظیم مدل در انتقال یادگیری با داده محدود ارائه می‌شود.
}
\abstractPage
\newpage\clearpage
\tableofcontents

\newpage
\listoffigures \newpage
\listoftables  \newpage
% \addcontentsline{toc}{chapter}{\listalgorithmname}
% \listofalgorithms \newpage
% \chapter*{فهرست علائم اختصاری}
\addcontentsline{toc}{chapter}{فهرست علائم اختصاری}

\persiangloss{شتاب گرانش}{$a$ (m/s$^2$)}
\persiangloss{نیرو}{$F$ (N)}


\pagestyle{fancy}
% اگر شما فصل اول  خود را در فایلی به جز chapter1 همراه با این کلاس نوشته‌اید باید چندخط اول chapter1 را در فایل خود کپی کنید.
\pagenumbering{arabic}

\chapter{مقدمه}
\thispagestyle{empty}

\section{شرح مسأله}
فرایند ساخت و بهینه‌سازی مدل‌های یادگیری ماشین به‌طور سنتی، فرایندی پیچیده، زمان‌بر و نیازمند دانش تخصصی عمیق است. متخصصان علم داده ناچارند زمان قابل توجهی را صرف \persianfootnote{مهندسی ویژگی}\LTRfootnote{feature engineering}، \persianfootnote{انتخاب مدل}\LTRfootnote{model selection}، و \persianfootnote{تنظیم ابرپارامترها}\LTRfootnote{hyperparameter tuning} کنند. \persianfootnote{یادگیری ماشین خودکار}\LTRfootnote{Automated Machine Learning (AutoML)} با هدف خودکارسازی این \persianfootnote{خط لوله}\LTRfootnote{pipeline} و \persianfootnote{مردمی‌سازی}\LTRfootnote{democratization} یادگیری ماشین پدیدار شد. با این حال، روش‌های سنتی یادگیری ماشین خودکار، مانند \persianfootnote{بهینه‌سازی بیزی}\LTRfootnote{Bayesian optimization} یا \persianfootnote{الگوریتم‌های تکاملی}\LTRfootnote{evolutionary algorithms}، اغلب به عنوان  بهینه‌سازهای \persianfootnote{جعبه‌سیاه}\LTRfootnote{black-box} عمل می‌کنند که فاقد تفسیرپذیری بوده و در انطباق با مسائل جدید یا بهره‌برداری از دانش برون‌حوزه‌ای دچار چالش هستند.

در سال‌های اخیر، ظهور \persianfootnote{مدل‌های زبانی بزرگ}\LTRfootnote{Large Language Models (LLMs)} با توانایی‌های چشمگیر در \persianfootnote{درک زبان طبیعی}\LTRfootnote{Natural Language Understanding (NLU)}، استدلال و تولید کد، روش جدیدی را معرفی کرده است. این مدل‌ها پتانسیل آن را دارند که از بهینه‌سازهای کور فراتر رفته و به عنوان \persianfootnote{عامل‌}\LTRfootnote{agent}های هوشمند عمل کنند. این عامل‌ها می‌توانند با استفاده از دانش پیشین خود، استدلال گام‌به‌گام، و حتی تعامل با محیط‌های اجرایی و \persianfootnote{پایگاه‌های دانش}\LTRfootnote{knowledge bases}، فرایند یادگیری ماشین خودکار را به شیوه‌ای \persianfootnote{خودمختار}\LTRfootnote{autonomous}، \persianfootnote{تطبیق‌پذیر}\LTRfootnote{adaptive} و \persianfootnote{آگاه از زمینه}\LTRfootnote{context-aware} هدایت کنند. مسئله اصلی که این سمینار به آن می‌پردازد، بررسی این تقاطع نوظهور است: چگونه می‌توان از مدل‌های زبانی بزرگ عامل-محور برای دگرگونی و ارتقای نسل بعدی سیستم‌های یادگیری ماشین خودکار بهره جست؟

\section{معرفی حوزه سمینار}
حوزه اصلی این سمینار، یادگیری ماشین خودکار است. این حوزه به طور سنتی بر توسعه روش‌هایی برای خودکارسازی کامل خط لوله یادگیری ماشین، به‌ویژه چالش‌های محاسباتی پیچیده‌ای چون \persianfootnote{بهینه‌سازی ابرپارامترها}\LTRfootnote{Hyperparameter Optimization (HPO)} و \persianfootnote{جستجوی معماری عصبی}\LTRfootnote{Neural Architecture Search (NAS)}، تمرکز داشته است.

این سمینار یک روش نوظهور و دگرگون‌ساز در این حوزه را بررسی می‌کند که مبتنی بر ادغام مدل‌های زبانی بزرگ است. برخلاف روش‌های کلاسیک یادگیری ماشین خودکار (مانند بهینه‌سازی بیزی یا الگوریتم‌های تکاملی) که اغلب به عنوان بهینه‌سازهای جعبه‌سیاه عمل می‌کنند، مدل‌های زبانی بزرگ با توانایی‌های بی‌نظیر خود در درک دستورالعمل‌های پیچیده، استدلال گام‌به‌گام و تولید کد، پتانسیل ایجاد فرایندهای بهینه‌سازی شفاف‌تر، انعطاف‌پذیرتر و آگاهانه‌تر را فراهم می‌کنند.

نکته کلیدی که این سمینار به آن می‌پردازد، بررسی این مدل‌های زبانی در چارچوب \persianfootnote{سیستم‌های عامل-محور}\LTRfootnote{Agent-based Systems} است. در این دیدگاه، مدل زبانی به عنوان مغز متفکر یک عامل هوشمند عمل می‌کند. این عامل می‌تواند به صورت خودمختار، وظایف پیچیده یادگیری ماشین خودکار را مدیریت کند، از دانش خارجی بهره ببرد، استراتژی‌های جستجو را استدلال نماید، کد اجرایی تولید کند و بر اساس بازخورد، رویکرد خود را تطبیق دهد.

بنابراین، حوزه تخصصی این سمینار، نقطه تلاقی این مفاهیم مطالعه و تحلیل \textbf{کاربرد عامل‌های هوشمند مبتنی بر مدل زبانی بزرگ} به منظور \textbf{ارتقا و تحول در نسل جدید سیستم‌های یادگیری ماشین خودکار }می‌باشد.

\section{ساختار گزارش}
این گزارش در ۴ فصل تهیه شده است. در فصل اول به بیان مقدمه و شرح مسئله پرداخته ایم. در فصل دوم مبانی و مفاهیم حوزه یادگیری ماشین خودکار، بهینه‌سازی ابرپارامترها و مدل های زبانی بزرگ عامل محور شرح داده شده است. در فصل سوم روش های مختلف از دیدگاه های عامل، دانش و نوع خروجی بررسی و مقایسه شده است. در فصل چهارم نیز به جمع‌بندی، نتیجه‌گیری از مطالب آورده شده در گزارش و کارهای آینده پرداخته‌ایم.
			% فصل اول: مقدمه

\chapter{تعاریف و مفاهیم مبنایی }
\thispagestyle{empty}
\section{مقدمه}
برای درک عمیق روش‌های نوین ارائه شده در این گزارش، ضروری است که با مفاهیم و تعاریف پایه‌ای در حوزه‌های یادگیری ماشین خودکار، مدل‌های زبانی بزرگ و سیستم‌های عامل-محور آشنا شویم. این فصل به مرور این مبانی نظری اختصاص دارد و به عنوان سنگ بنایی برای تحلیل‌های ارائه شده در فصل سوم عمل خواهد کرد.

\section{یادگیری خودکار ماشین}
یادگیری ماشین خودکار به فرایند خودکارسازی وظایف تکراری و تخصصی در طراحی و استقرار مدل‌های یادگیری ماشین اطلاق می‌شود. هدف نهایی یادگیری ماشین خودکار، کاهش نیاز به دخالت متخصصان انسانی و تسریع فرایند کشف و اعتبارسنجی مدل‌های کارآمد است. این فرایند معمولاً شامل مراحلی چون \persianfootnote{پیش‌پردازش داده}\LTRfootnote{data preprocessing}، مهندسی ویژگی، انتخاب مدل و بهینه‌سازی ابرپارامتر می‌باشد.

\subsection{بهینه‌سازی ابرپارامتر}
\persianfootnote{ابرپارامترها}\LTRfootnote{Hyperparameters} پارامترهایی در مدل یادگیری ماشین هستند که مقدار آن‌ها پیش از آغاز فرایند آموزش تنظیم می‌شود (برخلاف پارامترها یا وزن‌ها که در طول آموزش یادگرفته می‌شوند). بهینه‌سازی ابرپارامتر فرایند یافتن ترکیبی از ابرپارامترها است که منجر به بهترین \persianfootnote{کارایی}\LTRfootnote{performance} مدل بر روی یک مجموعه داده اعتبارسنجی می‌شود. روش‌های متداول برای این کار شامل \persianfootnote{جستجوی شبکه‌ای}\LTRfootnote{Grid Search}، \persianfootnote{جستجوی تصادفی}\LTRfootnote{Random Search} و بهینه‌سازی بیزی است. این فرایند به دلیل \persianfootnote{هزینه محاسباتی}\LTRfootnote{computational cost} بالای ارزیابی هر \persianfootnote{پیکربندی}\LTRfootnote{configuration}، بسیار چالش‌برانگیز است.

\begin{align}
\lambda^\star &= \arg\min_{\lambda}\; \mathcal{L}_V^\star(\lambda)
= \arg\min_{\lambda}\; \mathcal{L}_V\!\big(\lambda, \mathbf{w}^\star(\lambda)\big), \\
\text{s.t.}\quad \mathbf{w}^\star(\lambda) &= \arg\min_{\mathbf{w}}\; \mathcal{L}_T(\lambda, \mathbf{w}) .
\end{align}

توضیح کوتاه:
در این فرمول:
- $\lambda$: بردار ابرپارامترها
- $\mathbf{w}$: پارامترهای قابل‌آموزش مدل
- $\mathcal{L}_T$: تابع زیان روی داده‌های آموزش (Train)
- $\mathcal{L}_V$: تابع زیان/معیار روی داده‌های اعتبارسنجی (Validation)
- $\mathbf{w}^\star(\lambda)$: وزن‌های بهینه‌ی حاصل از آموزش با ابرپارامترِ $\lambda$
معنا: ابتدا برای هر $\lambda$ مدل را با کمینه‌کردن $\mathcal{L}_T$ آموزش می‌دهیم تا $\mathbf{w}^\star(\lambda)$ به‌دست آید؛
سپس $\lambda$ای را برمی‌گزینیم که $\mathcal{L}_V$ روی مجموعه‌ی اعتبارسنجی را کمینه کند.

\subsection{جستجوی معماری شبکه عصبی}
جستجوی معماری عصبی یکی از زیرشاخه‌های یادگیری ماشین خودکار است که به طور خاص بر خودکارسازی طراحی معماری \persianfootnote{شبکه‌های عصبی عمیق}\LTRfootnote{Deep Neural Networks} تمرکز دارد. به جای تکیه بر \persianfootnote{شهود}\LTRfootnote{intuition} و طراحی دستی توسط متخصصان، جستجوی معماری عصبی از \persianfootnote{الگوریتم‌های جستجو}\LTRfootnote{search algorithms} (مانند \persianfootnote{یادگیری تقویتی}\LTRfootnote{Reinforcement Learning} یا \persianfootnote{روش‌های تکاملی}\LTRfootnote{evolutionary methods}) برای کاوش در \persianfootnote{فضای طراحی}\LTRfootnote{design space} وسیع معماری‌ها استفاده می‌کند. هدف، یافتن معماری‌ای است که بهترین توازن را میان دقت و \persianfootnote{کارایی محاسباتی}\LTRfootnote{computational efficiency} برقرار کند.

\section{مدل های زبانی بزرگ}
مدل‌های زبانی بزرگ مدل‌های زبانی هستند که با استفاده از معماری ترنسفورمر و بر روی حجم عظیمی از داده‌های متنی آموزش دیده‌اند. این مدل‌ها دارای میلیاردها پارامتر هستند و توانایی‌های شگفت‌انگیزی در \persianfootnote{درک زمینه}\LTRfootnote{understanding context}، تولید متن \persianfootnote{منسجم}\LTRfootnote{coherent} و استدلال از خود نشان می‌دهند.

\subsection{تاریخچه}
توسعه مدل‌های زبانی بزرگ با معرفی معماری ترنسفورمر \cite{vaswani2017attention} شتاب گرفت. مدل‌هایی مانند BERT \cite{devlin2019bert} و GPT \cite{brown2020language} \persianfootnote{چارچوب نظری}\LTRfootnote{paradigm} \persianfootnote{پیش‌آموزش}\LTRfootnote{pre-training} و \persianfootnote{ریزتنظیم}\LTRfootnote{fine-tuning} را تثبیت کردند. نسل‌های بعدی این مدل‌ها، با افزایش چشمگیر \persianfootnote{مقیاس}\LTRfootnote{scale} داده و پارامترها، به توانایی‌های \persianfootnote{یادگیری درون-متنی}\LTRfootnote{In-Context Learning (ICL)} دست یافتند که به آن‌ها اجازه می‌دهد وظایف جدید را بدون ریزتنظیم و تنها با دیدن چند مثال در \persianfootnote{دستور}\LTRfootnote{prompt} انجام دهند.

\subsection{معماری}
پایه و اساس اکثر مدل‌های زبانی بزرگ مدرن، معماری ترنسفورمر است که بر مکانیزم \persianfootnote{توجه خودی}\LTRfootnote{self-attention} تکیه دارد. این مکانیزم به مدل اجازه می‌دهد تا \persianfootnote{وابستگی‌های دوربرد}\LTRfootnote{long-range dependencies} را در متن مدل کند و به بخش‌های مختلف ورودی وزن‌های متفاوتی اختصاص دهد. مدل‌ها معمولاً از پشته‌ای از لایه‌های \persianfootnote{رمزگذار}\LTRfootnote{encoder} (مانند \lr{BERT}) یا \persianfootnote{رمزگشا}\LTRfootnote{decoder} (مانند \lr{GPT}) یا هر دو (مانند \lr{T5}) تشکیل شده‌اند \cite{vaswani2017attention}.

\subsection{کاربردها}
مدل‌های زبانی بزرگ کاربردهای متنوعی از جمله \persianfootnote{ترجمه ماشینی}\LTRfootnote{machine translation}~\cite{wang-etal-2023-document-level}، \persianfootnote{خلاصه‌سازی متن}\LTRfootnote{text summarization}، پرسش و پاسخ و تولید محتوا دارند~\cite{minaee2024large}. اخیراً، توانایی آن‌ها در \persianfootnote{تولید کد}\LTRfootnote{code generation}~\cite{gao2023pal} و حل مسائل منطقی~\cite{pan-etal-2023-logic}، درهای جدیدی را برای استفاده از آن‌ها در حوزه‌های فنی‌تر مانند مهندسی نرم‌افزار و یادگیری ماشین خودکار گشوده است.

\subsection{تفکر و عمل}
فراتر از تولید \persianfootnote{پاسخ‌های ایستا}\LTRfootnote{static responses}، مدل‌های زبانی بزرگ می‌توانند برای \persianfootnote{تفکر}\LTRfootnote{Reasoning} و \persianfootnote{عمل}\LTRfootnote{Action} نیز به کار روند. چارچوب‌هایی مانند ReAct \cite{yao2023react} نشان دادند که چگونه یک مدل زبانی بزرگ می‌تواند به صورت \persianfootnote{درهم‌تنیده}\LTRfootnote{interleaved}، \persianfootnote{ردپاهای استدلالی}\LTRfootnote{reasoning traces} (تفکر) و اقدامات (مانند جستجو در وب یا اجرای دستور) تولید کند. این قابلیت، سنگ بنای استفاده از مدل‌های زبانی بزرگ به عنوان هسته تصمیم‌گیرنده در عامل‌های خودمختار است.

\section{عامل}
در زمینه هوش مصنوعی، عامل به سیستمی اطلاق می‌شود که در یک \persianfootnote{محیط}\LTRfootnote{environment} قرار دارد، آن را از طریق \persianfootnote{حسگرها}\LTRfootnote{sensors} ادراک می‌کند و از طریق \persianfootnote{کنشگرها}\LTRfootnote{actuators} بر آن تأثیر می‌گذارد تا به اهداف خود دست یابد. عامل‌های زبانی نوع خاصی از عامل‌ها هستند که از مدل‌های زبانی بزرگ به عنوان موتور استدلال اصلی خود برای پردازش ادراکات (اغلب متنی)، \persianfootnote{برنامه‌ریزی}\LTRfootnote{planning} و انتخاب اقدام استفاده می‌کنند \cite{wang2024survey}.

\subsection[تک‌عاملی]{\persianfootnote{تک‌عاملی}\LTRfootnote{Single-Agent}}
یک سیستم  شامل یک عامل واحد است که تمام وظایف ادراک، استدلال و عمل را به تنهایی انجام می‌دهد. در زمینه یادگیری ماشین خودکار، این می‌تواند یک عامل مبتنی بر مدل زبانی بزرگ باشد که کل خط لوله بهینه‌سازی را از ابتدا تا انتها مدیریت می‌کند \cite{wang2024survey}.

\subsection[چندعاملی]{\persianfootnote{چندعاملی}\LTRfootnote{Multi-Agent}}
سیستم‌های چندعاملی شامل دو یا چند عامل هستند که در یک محیط مشترک با یکدیگر \persianfootnote{تعامل}\LTRfootnote{interact} می‌کنند. این تعامل می‌تواند \persianfootnote{همکارانه}\LTRfootnote{collaborative} (برای دستیابی به یک هدف مشترک) یا \persianfootnote{رقابتی}\LTRfootnote{competitive} باشد. در یادگیری ماشین خودکار، می‌توان از سیستم‌های چندعاملی برای \persianfootnote{تفکیک وظایف}\LTRfootnote{task decomposition} استفاده کرد؛ برای مثال، یک عامل متخصص تحلیل داده، یک عامل متخصص \persianfootnote{تولید معماری}\LTRfootnote{architecture generation} و یک عامل \persianfootnote{منتقد}\LTRfootnote{critic} برای ارزیابی نتایج \cite{wang2024survey}.

\section[تولید تقویت‌شده با بازیابی]{\persianfootnote{تولید تقویت‌شده با بازیابی}\LTRfootnote{Retrieval-Augmented Generation (RAG)}}
تولید تقویت‌شده با بازیابی \cite{Lewis2020RAG} روشی است که مدل‌های زبانی بزرگ را با یک \persianfootnote{سازوکار بازیابی اطلاعات}\LTRfootnote{information retrieval mechanism} خارجی ترکیب می‌کند. به جای تکیه صرف بر دانش پارامتری (ذخیره شده در وزن‌های مدل)، تولید تقویت‌شده با بازیابی ابتدا اطلاعات مرتبط را از یک \persianfootnote{مجموعه متون}\LTRfootnote{corpus} یا \persianfootnote{پایگاه دانش}\LTRfootnote{knowledge base} بازیابی می‌کند و سپس این اطلاعات را به مدل زبانی بزرگ می‌دهد تا پاسخ نهایی را بر اساس آن تولید کند. این روش به کاهش \persianfootnote{توهم}\LTRfootnote{hallucination} و افزایش دقت و \persianfootnote{به‌روز بودن}\LTRfootnote{up-to-dateness} اطلاعات کمک می‌کند \cite{xia2025ragselfreasoning}.

\subsection{پایگاه دانش}
پایگاه دانش در تولید تقویت‌شده با بازیابی معمولاً مجموعه‌ای از \persianfootnote{اسناد}\LTRfootnote{documents} است. این اسناد اغلب به \persianfootnote{قطعات}\LTRfootnote{chunks} کوچکتر تقسیم شده و به صورت \persianfootnote{نمایش‌های برداری}\LTRfootnote{vector representations} (یا \persianfootnote{نهفتگی‌ها}\LTRfootnote{embeddings}) در یک \persianfootnote{پایگاه داده برداری}\LTRfootnote{vector database} ذخیره می‌شوند تا \persianfootnote{بازیابی مبتنی بر شباهت معنایی}\LTRfootnote{semantic similarity retrieval} به سرعت انجام شود.

\subsection{ترکیب با عامل}
عامل‌های هوشمند می‌توانند از تولید تقویت‌شده با بازیابی به عنوان یک \persianfootnote{ابزار}\LTRfootnote{tool} کلیدی استفاده کنند. زمانی که یک عامل یادگیری ماشین خودکار با یک مجموعه داده جدید روبرو می‌شود، می‌تواند از تولید تقویت‌شده با بازیابی برای جستجو در پایگاه دانش استفاده کند. این دانش بازیابی‌شده به عامل کمک می‌کند تا تصمیمات آگاهانه‌تری در مورد مسئله اتخاذ کند \cite{singh2025agenticrag}.

  % فصل دوم: تعاریف و مفاهیم مبنایی 
\chapter{مروری بر کارهای مرتبط}
\thispagestyle{empty}

\section{مقدمه}
\subsection{ظهور اولین روش های یادگیری ماشین خودکار}
\subsection{روش های یادگیری ماشین خودکار در عصر مدل های زبانی بزرگ}


\section{تحلیل بر مبنای معماری عامل}
\subsection{سامانه‌های تک‌عاملی}

بیشینهٔ سامانه‌های مبتنی بر مدل‌های زبانی بزرگ برای خودکارسازی یادگیری ماشین، معماری‌های تک‌عاملی را برمی‌گزینند که در آن یک عامل منفرد کل جریان بهینه‌سازی را مدیریت می‌کند. بر مبنای چگونگی ادغام مدل زبانی در فرایند بهینه‌سازی، این سامانه‌ها در چند پارادایم عملیاتی متمایز قرار می‌گیرند.

\subsubsection{\persianfootnote{بهینه‌سازی مستقیم از طریق دستوردهی تکراری}\protect\LTRfootnote{Direct Optimization through Iterative Prompting}}
در این رویکرد برجسته، مدل‌های زبانی به‌منزلهٔ بهینه‌سازهای \persianfootnote{جعبه‌سیاه}\LTRfootnote{black-box} به‌کار می‌روند که پیکربندی‌ها را پیشنهاد می‌کنند و از راه \persianfootnote{حلقه‌های بازخورد}\LTRfootnote{feedback loops} آن‌ها را پالایش می‌کنند. مدل با تکیه بر تاریخچهٔ \persianfootnote{آزمون‌ها}\LTRfootnote{trial} که به‌صورت \persianfootnote{گفت‌وگوهای چت}\LTRfootnote{chat-style dialogues} یا خلاصه‌های فشرده انباشته می‌شود، \persianfootnote{زمینه}\LTRfootnote{context} را حفظ می‌کند و پالایش تکراری مبتنی بر معیارهای اعتبارسنجی را ممکن می‌سازد \cite{zhang2023usingLLMforHPO, zheng2023GENIUS}. این پارادایم به چارچوب‌های بهینه‌سازی \persianfootnote{بیزی}\LTRfootnote{Bayesian} نیز بسط یافته است؛ جایی که مدل‌های آغاز گرم، نمونه‌گیری از نامزدها و \persianfootnote{مدل‌سازی جانشین}\LTRfootnote{surrogate modeling} را با اتکاء به استدلال زبان طبیعی و مشروط بر تاریخچهٔ بهینه‌سازی انجام می‌دهند \cite{liu2024LLAMBO}. این رویکردها در تنظیمات با بودجه کم کارایی رقابتی نشان می‌دهند و بی‌آن‌که به \persianfootnote{ریزتنظیم}\LTRfootnote{fine-tuning} نیاز داشته باشند، بر \persianfootnote{یادگیری زمینه ای}\LTRfootnote{in-context learning} از توصیف مسئله و بازخورد تجربی تکیه می‌کنند.

\subsubsection{\persianfootnote{عملگرهای تکاملی}\protect\LTRfootnote{Evolutionary Operators}}
راهبردی دیگر، مدل زبانی را به‌عنوان عملگرهای \persianfootnote{جهش}\LTRfootnote{mutation} یا \persianfootnote{ترکیب}\LTRfootnote{crossover} درون چارچوب‌های \persianfootnote{تکاملی}\LTRfootnote{evolutionary} جا می‌دهد. به‌جای جایگزینی الگوریتم‌های جست‌وجوی سنتی، این سامانه‌ها آن‌ها را با تنوع‌های تولیدشده توسط مدل زبانی تقویت می‌کنند. مدل می‌تواند تغییرات معماری مبتنی بر کد را در چارچوب‌های \persianfootnote{کیفیت–تنوع}\LTRfootnote{quality-diversity} بسازد \cite{LLMatic2024} یا به‌صورت عملگرهای \persianfootnote{تطبیقی}\LTRfootnote{adaptive operators} که میان نسل‌ها با \persianfootnote{تنظیم دستور}\LTRfootnote{prompt-tuning} پالایش می‌شوند عمل کند \cite{chen2023Evoprompting}. برخی پیاده‌سازی‌ها \persianfootnote{توانایی‌های بازتابی}\LTRfootnote{reflective capabilities} را نیز می‌گنجانند؛ به این معنا که مدل پیامدهای جهش را تحلیل می‌کند و \persianfootnote{بازخورد زبانی}\LTRfootnote{linguistic feedback} برای هدایت تکرارهای بعدی تولید می‌کند \cite{ji2025RZNAS}. این ادغام، \persianfootnote{پایداری}\LTRfootnote{robustness} جست‌وجوی تکاملی را حفظ می‌کند و در عین حال از \persianfootnote{خلاقیت}\LTRfootnote{creativity} مدل در تولید تنوع‌های معنادار بهره می‌گیرد. نمونه‌ای شاخص، GPT-NAS است که در آن \persianfootnote{مدل زبانی}\LTRfootnote{GPT} به‌منزلهٔ یک \persianfootnote{بازساز}\LTRfootnote{reconstructor} معماری عمل کرده و با ماسک‌گذاری و بازتولید لایه‌ها، نامزدهای نمونه‌برداری‌شده توسط \persianfootnote{الگوریتم ژنتیک}\LTRfootnote{genetic algorithm} را بهبود می‌دهد؛ بنابراین عملاً نقشی هم‌ارز با یک عملگر جهشِ آگاه از زمینه ایفا می‌کند بی‌آن‌که راهبرد جست‌وجوی تکاملی را جایگزین کند \cite{Yu2025GPTNAS}.

\subsubsection{\persianfootnote{کنترل‌گرهای جریان کار}\protect\LTRfootnote{Workflow Controllers}}
در این پارادایم، مدل‌های زبانی به‌مثابه \persianfootnote{هماهنگ‌کننده}\LTRfootnote{orchestrator} برای مدیریت اجزای \persianfootnote{خطّ لوله}\LTRfootnote{pipeline} به کار می‌روند. سامانه با ترکیب دستور‌هایی که \persianfootnote{فرادادهٔ ساختاریافته}\LTRfootnote{structured metadata}-از جمله \persianfootnote{کارت‌های داده}\LTRfootnote{data cards} و \persianfootnote{کارت‌های مدل}\LTRfootnote{model cards}-را در خود دارند، مدل را در مراحل پیاپی از پردازش داده، انتخاب مدل تا تنظیم فراپارامتر هدایت می‌کند \cite{zhang2023AutomlGPTAutomaticMachineLearning, shen2023HuggingGPT}. برخی پیاده‌سازی‌ها برنامه‌های پیچیدهٔ یادگیری ماشین را به \persianfootnote{مولفه‌های ماژولار}\LTRfootnote{modular components} تجزیه می‌کنند که به‌طور جداگانه تولید و با \persianfootnote{آزمون‌های واحد خودکار}\LTRfootnote{automated unit tests} راستی‌آزمایی می‌شوند تا سازگاری تضمین گردد \cite{xu2024largeTextToML}. این رویکرد با شکستن خطوط لولهٔ طولانی و ناهمگون به زیروظایف قابل مدیریت و اتکاء به \persianfootnote{دستوردهی زمینه‌مند}\LTRfootnote{contextual prompting}، انسجام کلّی را حفظ می‌کند.


\subsection{سامانه‌های چندعاملی}

معماری‌های چندعاملی با تفکیک نقش، زیروظایف یادگیری ماشین خودکار را میان عامل‌هایی با قابلیت‌های مکمل توزیع می‌کنند. این سامانه‌ها از \persianfootnote{رهگذر تفکیک کارکردی}\LTRfootnote{functional decomposition} و \persianfootnote{همکاری بین‌عاملی}\LTRfootnote{inter-agent collaboration}، مدیریت پیچیدگی و استدلال پیچیده‌تر را ممکن می‌سازند.

\subsubsection{\persianfootnote{همکاری مبتنی بر نقش}\protect\LTRfootnote{Role-Based Collaboration}}
الگوی پایه، دو عامل تخصصی با مسئولیت‌های متمایز را به‌کار می‌گیرد. در یک ساختار، تولید \persianfootnote{پیکربندی}\LTRfootnote{configuration} از اجرای تجربی جدا می‌شود: عامل سازنده نیازمندی‌ها را تفسیر و پیکربندی‌های پیشنهادی همراه با استدلال ارائه می‌کند و عامل اجراکننده آموزش را انجام داده و نتایج را در گزارش‌های مشترک می‌گنجاند تا چرخه‌های بعدی پیشنهادهای سازنده را تغذیه کند \cite{liu2025agenthpo}. این تقسیم کار بازتاب گردش‌کار متخصصان است و با \persianfootnote{حافظهٔ ترتیبی}\LTRfootnote{episodic memory} انباشته، به عملکرد خودگردان بدون مداخلهٔ انسانی می‌انجامد.

\subsubsection{\persianfootnote{هماهنگی سلسله‌مراتبی}\protect\LTRfootnote{Hierarchical Coordination}}
در ساختارهای پیچیده‌تر، چند عامل تخصصی تحت نظارت یک \persianfootnote{مدیر عامل}\LTRfootnote{Agent Manager} سازمان می‌یابند. مدیر با \persianfootnote{استدلال تقویت‌شده با بازیابی}\LTRfootnote{retrieval-augmented reasoning} طرح‌های متنوعی می‌سازد، آن‌ها را به زیروظایف \persianfootnote{قابل موازی‌سازی}\LTRfootnote{parallelizable subtasks} واگشایی و به عامل مناسب تخصیص می‌دهد و از رهگذر راستی‌آزمایی چندمرحله‌ای و \persianfootnote{حلقه‌های بازنگری}\LTRfootnote{revision loops} نتایج را اعتبارسنجی می‌کند \cite{trirat2025automlagent}. این معماری کلّ خط لولهٔ AutoML را از بازیابی داده تا \persianfootnote{استقرار}\LTRfootnote{deployment} به‌صورت کارآمد پیش می‌برد و \persianfootnote{وابستگی‌های بین‌گامی}\LTRfootnote{inter-step dependencies} را با \persianfootnote{پروتکل‌های هماهنگی ساختاریافته}\LTRfootnote{structured coordination protocols} مدیریت می‌کند.

\subsubsection{\persianfootnote{تیم‌های پژوهش–توسعه}\protect\LTRfootnote{Research-Development Teams}.}
پیشرفته‌ترین ساختار سازمانی، عامل‌ها را در تیم‌های کارکردی تقسیم می‌کند \cite{Yang2025NADER}. \persianfootnote{تیم پژوهش}\LTRfootnote{Research Team} دانش را از \persianfootnote{ادبیات پژوهشی}\LTRfootnote{literature} استخراج می‌کند و با اتکاء به بینش‌های بازیابی‌شده، پیشنهادهای تغییر را می‌سازد؛ \persianfootnote{تیم توسعه}\LTRfootnote{Development Team} این تغییرها را بر \persianfootnote{نمایش‌های گراف}\LTRfootnote{graph representations} اعمال کرده و هم بازخورد فوری و هم استخراج تجربهٔ بلندمدت را فراهم می‌کند. گفت‌وگوهای چندمرحله‌ای میان تیم‌ها یادگیری مستمر از تاریخچهٔ طراحی را ممکن می‌کند و \persianfootnote{پایگاه‌های دادهٔ برداری}\LTRfootnote{vector databases} با \persianfootnote{بازیابی مبتنی بر شباهت}\LTRfootnote{similarity-based retrieval}، دانش و تجربه‌های گذشتهٔ مرتبط را برای هدایت اکتشاف فراخوانی می‌کنند.

پارادایم چندعاملی با افزودن پیچیدگی هماهنگی، در برابر \persianfootnote{توضیح‌پذیری}\LTRfootnote{interpretability} بهبود‌یافته از طریق تفکیک صریح نقش‌ها و کارایی بهتر بهینه‌سازی از رهگذر استدلال تخصصی معامله می‌کند؛ هرچند به مدیریت حافظه پیشرفته برای حفظ سازگاری در تعاملات عامل‌ها و طراحی دقیق پروتکل‌های ارتباطی برای جلوگیری از شکست‌های هماهنگی نیاز دارد.

\section{طبقه بندی بر اساس دانش خارجی}
\subsection{با استفاده از تولید تقویت شده با بازیابی}
\subsection{با استفاده از ابزار های جستجو}
\section{طبقه بندی بر اساس نوع خروجی مدل}
\subsection{خروجی بصورت واژه‌نامه}
\subsection{خروجی بصورت کد برنامه}
\subsection{خروجی بصورت درخت}
\subsection{خروجی بصورت ترکیبی از واژه‌نامه و کد برنامه}
\section{طبقه بندی کار های مرتبط}
همانطور که در جدول \ref{tab:recent-works} مشاهده می‌شود، اکثر روش‌های بررسی شده از مدل‌های زبانی بزرگ چندعاملی استفاده می‌کنند. این رویکرد به دلیل توانایی در تقسیم وظایف و همکاری بین عوامل مختلف، معمولاً عملکرد بهتری را در مسائل پیچیده ارائه می‌دهد. همچنین، بیشتر روش‌های جدید از دانش خارجی بهره می‌برند که می‌تواند به بهبود دقت و کارایی مدل کمک کند. در زمینه نوع خروجی، روش‌های متنوعی وجود دارد که بسته به نیاز مسئله، می‌توانند انتخاب شوند. برای مثال، خروجی بصورت کد برنامه برای مسائل نیازمند پیاده‌سازی عملی مناسب‌تر است، در حالی که خروجی بصورت واژه‌نامه ممکن است برای مسائل تحلیلی کاربردی‌تر باشد. این تنوع در رویکردها نشان‌دهنده انعطاف‌پذیری و قابلیت تطبیق مدل‌های زبانی بزرگ با نیازهای مختلف در حوزه جستجوی معماری شبکه عصبی و بهینه‌سازی ابرپارامتر است.
\begin{table}[t]
    \centering
    \footnotesize
    \setlength{\tabcolsep}{3pt}
    \renewcommand{\arraystretch}{1.2}
    \begin{tabularx}{\textwidth}{@{} Y c c c c c c c l @{}}
        \toprule
        عنوان                                                            & عامل & روش قدیمی & ایجاد کد                      & ابزار  & دانش خارجی   & فضای جستجو & بدون آموزش & وظیفه        \\
        \midrule
        \lr{LLM for HPO}\cite{zhang2023usingLLMforHPO}             & تک   & —         & \xmark                        & \xmark & \xmark & اختیاری    & —          & \lr{HPO}\    \\
        \lr{GENIUS}\cite{zheng2023GENIUS}                                & تک   & —         & \xmark                        & \xmark & \xmark & بله        & —          & \lr{NAS}     \\
        \lr{LLMATIC}\cite{LLMatic2024}                                   & تک   & EA        & \cmark                        & \xmark & \xmark & خیر        & —          & \lr{NAS}     \\
        \lr{Text to ML}\cite{xu2024largeTextToML}                        & چند  & —         & \cmark                        & \xmark & \xmark & خیر        & \cmark     & \lr{AutoML}  \\
        \lr{AgentHPO}\cite{liu2025agenthpo}                              & چند  & —         & \xmark                        & \cmark & \xmark & بله        & —          & \lr{HPO}     \\
        \lr{AutoML-Agent}\cite{trirat2025automlagent}                    & چند  & —         & \cmark                        & \xmark & \cmark & خیر        & \cmark     & \lr{AutoML}  \\
        \lr{LLAMBO}\cite{liu2024LLAMBO}                                  & چند  & BO        & \xmark                        & \xmark & \xmark & بله        & \cmark     & \lr{HPO}     \\
        \lr{Nader}\cite{Yang_2025_NADER}                                 & چند  & —         & \xmark\textsuperscript{\dag}  & \cmark & \cmark & بله        & —          & \lr{NAS} \\
        \lr{RZ-NAS}\cite{ji2025RZNAS}                                    & تک   & EA        & \cmark\textsuperscript{\ddag} & \xmark & \xmark & بله        & \cmark     & \lr{NAS}     \\
        \lr{EvoPrompting}\cite{chen2023Evoprompting}                     & تک   & EA        & \cmark                        & \xmark & \xmark & خیر        & —          & \lr{NAS}     \\
        \lr{AutoML-GPT}\cite{zhang2023AutomlGPTAutomaticMachineLearning} & تک   & —         & \cmark                        & \xmark & \xmark & بله        & —          & \lr{AutoML}  \\
        \lr{HuggingGPT}\cite{shen2023HuggingGPT}                         & چند  & —         & \cmark                        & \xmark & \cmark & خیر        & —          & \lr{AutoML}  \\
        \lr{GPT-NAS}\cite{Yu2025GPTNAS}                                  & تک   & EA        & —                             & \xmark & \xmark & خیر        & —          & \lr{NAS}     \\
        \lr{ML Copilot}\cite{zhang-etal-2024-MLCopilot}                  & چند  & —         & \xmark                        & \xmark & \cmark & خیر        & —          & \lr{AutoML}  \\
        \bottomrule
    \end{tabularx}
    \caption[مقایسهٔ فشردهٔ مقالات مبتنی بر \lr{LLM}]{مقایسهٔ فشردهٔ مقالات مبتنی بر \lr{LLM}. \lr{EA}=الگوریتم‌های تکاملی، \lr{BO}=بهینه‌سازی بیزی. نشانه‌ها: \textsuperscript{\dag}=ساختار درختی به‌جای تولید کد؛ \textsuperscript{\ddag}=تولید کد + تنظیمات.}
    \label{tab:recent-works}
\end{table}
  % فصل سوم: مروری برا کارهای مرتبط 

\chapter{نتیجه گیری و کارهای آینده}
\thispagestyle{empty}
\section{نتیجه‌گیری}
این گزارش به بررسی و تحلیل رویکردهای نوظهور در \persianfootnote{یادگیری ماشین خودکار}\LTRfootnote{Automated Machine Learning (AutoML)} پرداخت که از قابلیت‌های \persianfootnote{مدل‌های زبانی بزرگ}\LTRfootnote{Large Language Models (LLMs)} در قالب \persianfootnote{سیستم‌های عامل-محور}\LTRfootnote{agent-based systems} بهره می‌برند. مرور ادبیات نشان داد که این حوزه به سرعت در حال فاصله گرفتن از \persianfootnote{بهینه‌سازهای جعبه‌سیاه}\LTRfootnote{black-box optimizers} سنتی و حرکت به سوی \persianfootnote{پارادایم‌های}\LTRfootnote{paradigms} \persianfootnote{آگاه از دانش}\LTRfootnote{knowledge-driven}، \persianfootnote{تفسیرپذیر}\LTRfootnote{interpretable} و \persianfootnote{خودمختار}\LTRfootnote{autonomous} است.

ما روش‌های موجود را از سه منظر کلیدی طبقه‌بندی کردیم: \persianfootnote{معماری عامل}\LTRfootnote{agent architecture} (تک‌عاملی در برابر چندعاملی)، \persianfootnote{منابع دانش}\LTRfootnote{knowledge sources} (درونی در برابر بیرونی/RAG) و \persianfootnote{قالب خروجی}\LTRfootnote{output format} (ساختاریافته، کد، یا ترکیبی).

یافته‌ها حاکی از آن است که \persianfootnote{عامل‌های تک‌عاملی}\LTRfootnote{single-agent}، به‌ویژه آن‌هایی که از \persianfootnote{دستوردهی تکراری}\LTRfootnote{iterative prompting} یا \persianfootnote{عملگرهای تکاملی}\LTRfootnote{evolutionary operators} استفاده می‌کنند، برای \persianfootnote{بهینه‌سازی ابرپارامترها}\LTRfootnote{HPO} و \persianfootnote{جستجوی معماری عصبی}\LTRfootnote{NAS} \persianfootnote{محدود}\LTRfootnote{constrained} مؤثر هستند. در مقابل، \persianfootnote{معماری‌های چندعاملی}\LTRfootnote{multi-agent architectures} با \persianfootnote{تفکیک نقش}\LTRfootnote{role specialization} (مانند \persianfootnote{پژوهشگر}\LTRfootnote{researcher} و \persianfootnote{توسعه‌دهنده}\LTRfootnote{developer})، پتانسیل بیشتری برای مدیریت \persianfootnote{خطوط لوله}\LTRfootnote{pipelines} پیچیده و \persianfootnote{بلند-افق}\LTRfootnote{long-horizon} یادگیری ماشین خودکار از خود نشان می‌دهند.

ادغام تولید تقویت‌شده با بازیابی یک پیشرفت کلیدی است که به \persianfootnote{عامل‌ها}\LTRfootnote{agents} اجازه می‌دهد تا از \persianfootnote{دانش ایستا}\LTRfootnote{static knowledge}ی خود فراتر رفته و از \persianfootnote{ادبیات پژوهشی}\LTRfootnote{scientific literature}، \persianfootnote{مخازن کد}\LTRfootnote{code repositories} و \persianfootnote{نتایج آزمایش‌های گذشته}\LTRfootnote{past experimental results} برای اتخاذ \persianfootnote{تصمیمات آگاهانه‌تر}\LTRfootnote{more informed decisions} استفاده کنند. در نهایت، \persianfootnote{قالب خروجی}\LTRfootnote{output format} نشان‌دهنده یک \persianfootnote{توازن}\LTRfootnote{trade-off} میان \persianfootnote{خوانایی ماشینی}\LTRfootnote{machine-readability} (مانند JSON) و \persianfootnote{بیانگری}\LTRfootnote{expressiveness} (مانند تولید کد کامل) است، که رویکردهای \persianfootnote{ترکیبی}\LTRfootnote{hybrid} به عنوان راه‌حلی میانه در حال ظهور هستند. در مجموع، \persianfootnote{یادگیری ماشین خودکار عامل-محور}\LTRfootnote{Agentic AutoML} یک حوزه تحقیقاتی بسیار پویا است که نویدبخش خودکارسازی هوشمندانه‌تر و کارآمدتر فرایندهای علم داده است.

\section{مسائل باز و کارهای قابل انجام}
علی‌رغم پیشرفت‌های هیجان‌انگیز، چالش‌ها و \persianfootnote{مسائل باز}\LTRfootnote{open problems} متعددی در این حوزه وجود دارد که نیازمند پژوهش‌های آتی است:

\begin{itemize}
    \item \textbf{\persianfootnote{ارزیابی و محک‌زنی}\LTRfootnote{Evaluation and Benchmarking}:} در حال حاضر، \persianfootnote{محک‌های}\LTRfootnote{benchmarks} استاندارد و جامعی برای ارزیابی \persianfootnote{عامل‌های}\LTRfootnote{agents} یادگیری ماشین خودکار وجود ندارد. معیارهای ارزیابی باید فراتر از \persianfootnote{کارایی نهایی}\LTRfootnote{final performance} مدل باشند و مواردی چون \persianfootnote{کارایی نمونه}\LTRfootnote{sample efficiency} (تعداد \persianfootnote{ارزیابی‌های}\LTRfootnote{evaluations} گران‌قیمت)، \persianfootnote{هزینه استنتاج}\LTRfootnote{inference cost} مدل زبانی بزرگ، \persianfootnote{تفسیرپذیری}\LTRfootnote{interpretability} فرایند جستجو و \persianfootnote{استحکام}\LTRfootnote{robustness} عامل در برابر \persianfootnote{ورودی‌های نویزی}\LTRfootnote{noisy inputs} یا \persianfootnote{وظایف خارج از توزیع}\LTRfootnote{out-of-distribution tasks} را نیز در بر گیرند.

    \item \textbf{\persianfootnote{کارایی محاسباتی و هزینه}\LTRfootnote{Computational Efficiency and Cost}:} استفاده از مدل‌های زبانی بزرگ قدرتمند (مانند GPT-4) در یک \persianfootnote{حلقه بهینه‌سازی}\LTRfootnote{optimization loop} تکراری می‌تواند بسیار پرهزینه باشد. پژوهش در مورد چگونگی کاهش \persianfootnote{هزینه}\LTRfootnote{cost} (مثلاً با استفاده از \persianfootnote{مدل‌های کوچک‌تر}\LTRfootnote{smaller models} برای \persianfootnote{تصمیمات ساده‌تر}\LTRfootnote{simpler decisions}، \persianfootnote{تقطیر دانش}\LTRfootnote{knowledge distillation}، یا \persianfootnote{ذخیره‌سازی هوشمند}\LTRfootnote{intelligent caching} \persianfootnote{استدلال‌ها}\LTRfootnote{reasoning steps}) ضروری است.

    \item \textbf{\persianfootnote{قابلیت اطمینان و توهم}\LTRfootnote{Reliability and Hallucination}:} \persianfootnote{عامل‌ها}\LTRfootnote{Agents} ممکن است در \persianfootnote{استدلال}\LTRfootnote{reasoning} خود دچار \persianfootnote{خطا}\LTRfootnote{errors} شوند، \persianfootnote{کد}\LTRfootnote{code} \persianfootnote{ناقص}\LTRfootnote{buggy} تولید کنند، یا \persianfootnote{مفاهیم}\LTRfootnote{concepts} را به اشتباه تفسیر کنند (\persianfootnote{توهم}\LTRfootnote{hallucination}). توسعه \persianfootnote{مکانیزم‌های خود-اصلاحی}\LTRfootnote{self-correction mechanisms}، \persianfootnote{اعتبارسنجی دقیق}\LTRfootnote{rigorous validation} \persianfootnote{خروجی‌های}\LTRfootnote{outputs} عامل، و \persianfootnote{حلقه‌های بازخورد انسانی}\LTRfootnote{Human-in-the-Loop (HITL)} برای افزایش \persianfootnote{قابلیت اطمینان}\LTRfootnote{reliability} سیستم‌ها حیاتی است.

    \item \textbf{\persianfootnote{مدیریت حافظه و وظایف بلند-افق}\LTRfootnote{Memory Management and Long-Horizon Tasks}:} \persianfootnote{خطوط لوله}\LTRfootnote{Pipelines} یادگیری ماشین خودکار می‌توانند بسیار طولانی باشند. \persianfootnote{عامل‌ها}\LTRfootnote{Agents} نیاز به \persianfootnote{حافظه}\LTRfootnote{memory} \persianfootnote{بلندمدت}\LTRfootnote{long-term} کارآمد دارند تا \persianfootnote{تجربیات گذشته}\LTRfootnote{past experiences} را به خاطر بسپارند، \persianfootnote{زمینه}\LTRfootnote{context} را حفظ کنند و از \persianfootnote{تکرار خطاها}\LTRfootnote{repeating mistakes} بپرهیزند. این امر با \persianfootnote{محدودیت پنجره زمینه}\LTRfootnote{context window limitations} مدل‌های زبانی بزرگ در تضاد است و نیازمند \persianfootnote{معماری‌های حافظه}\LTRfootnote{memory architectures} پیشرفته (مانند \persianfootnote{حافظه‌های سلسله‌مراتبی}\LTRfootnote{hierarchical memory} یا \persianfootnote{ترکیبی}\LTRfootnote{hybrid}) می‌باشد.

    \item \textbf{\persianfootnote{یادگیری از بازخورد محیطی}\LTRfootnote{Learning from Environmental Feedback}:} چگونگی \persianfootnote{یادگیری}\LTRfootnote{learning} عامل از \persianfootnote{بازخورد}\LTRfootnote{feedback} (مثلاً نتایج \persianfootnote{اعتبارسنجی}\LTRfootnote{validation} یک \persianfootnote{پیکربندی}\LTRfootnote{configuration} یا \persianfootnote{خطای اجرای کد}\LTRfootnote{code execution error}) برای \persianfootnote{پالایش}\LTRfootnote{refine} \persianfootnote{سیاست}\LTRfootnote{policy} \persianfootnote{جستجوی}\LTRfootnote{search} خود، یک \persianfootnote{مسئله باز}\LTRfootnote{open question} است. این موضوع به \persianfootnote{یادگیری تقویتی}\LTRfootnote{Reinforcement Learning} مرتبط است، اما \persianfootnote{فضای حالت و اقدام}\LTRfootnote{state-action space} در یادگیری ماشین خودکار بسیار پیچیده‌تر و \persianfootnote{پربعدتر}\LTRfootnote{higher-dimensional} است.
\end{itemize}

\section{معرفی موضوع مورد نظر برای پایان نامه}
با توجه به \persianfootnote{مسائل باز}\LTRfootnote{open problems} شناسایی شده، یک \persianfootnote{موضوع پژوهشی}\LTRfootnote{research topic} جذاب برای پایان‌نامه می‌تواند «توسعه یک چارچوب \persianfootnote{چندعاملی}\LTRfootnote{Multi-Agent} \persianfootnote{خود-اصلاحگر}\LTRfootnote{Self-Correcting} برای \persianfootnote{جستجوی معماری عصبی}\LTRfootnote{Neural Architecture Search} با \persianfootnote{تقویت دانش}\LTRfootnote{Knowledge Augmentation}» باشد.

\persianfootnote{هدف اصلی}\LTRfootnote{Main objective} این پژوهش، طراحی سیستمی متشکل از چندین عامل تخصصی (مثلاً \persianfootnote{عامل تحلیلگر نیازمندی}\LTRfootnote{Requirement Analyst Agent}، \persianfootnote{عامل معمار}\LTRfootnote{Architect Agent}، \persianfootnote{عامل ارزیاب}\LTRfootnote{Evaluator Agent} و \persianfootnote{عامل منتقد/اصلاحگر}\LTRfootnote{Critic/Corrector Agent}) است. این سیستم باید قادر باشد:

\begin{enumerate}
    \item \textbf{استفاده از تولید تقویت‌شده با بازیابی:} \persianfootnote{عامل معمار}\LTRfootnote{Architect Agent} از تولید تقویت‌شده با بازیابی برای \persianfootnote{بازیابی}\LTRfootnote{retrieve} \persianfootnote{بلوک‌های ساختمانی}\LTRfootnote{building blocks} و \persianfootnote{الگوهای طراحی}\LTRfootnote{design patterns} موفق از \persianfootnote{ادبیات پژوهشی}\LTRfootnote{literature} و \persianfootnote{محک‌های}\LTRfootnote{benchmarks} \persianfootnote{جستجوی معماری عصبی}\LTRfootnote{NAS} (مانند \lr{NAS-Bench}) استفاده کند تا \persianfootnote{فضای جستجو}\LTRfootnote{search space} را به صورت \persianfootnote{آگاهانه}\LTRfootnote{informed} \persianfootnote{هرس}\LTRfootnote{prune} کند.
    \item \textbf{\persianfootnote{تولید کد قابل اعتبارسنجی}\LTRfootnote{Generating Verifiable Code}:} \persianfootnote{عامل معمار}\LTRfootnote{Architect Agent} \persianfootnote{معماری‌های}\LTRfootnote{architectures} پیشنهادی را به صورت \persianfootnote{کد}\LTRfootnote{code} اجرایی (مثلاً \lr{PyTorch} یا \lr{TensorFlow}) تولید کند.
    \item \textbf{\persianfootnote{مکانیزم خود-اصلاحی}\LTRfootnote{Self-Correction Mechanism}:} \persianfootnote{عامل ارزیاب}\LTRfootnote{Evaluator Agent} \persianfootnote{کد}\LTRfootnote{code} را اجرا کرده و \persianfootnote{نتایج}\LTRfootnote{results} \persianfootnote{کارایی}\LTRfootnote{performance} (مانند \persianfootnote{دقت}\LTRfootnote{accuracy} و \persianfootnote{تعداد پارامترها}\LTRfootnote{parameter count}) را گزارش دهد. \persianfootnote{عامل منتقد}\LTRfootnote{Critic Agent} این \persianfootnote{نتایج}\LTRfootnote{results} و \persianfootnote{خطاهای}\LTRfootnote{errors} احتمالی \persianfootnote{اجرا}\LTRfootnote{execution} را \persianfootnote{تحلیل}\LTRfootnote{analyzes} کرده و \persianfootnote{بازخورد}\LTRfootnote{feedback} \persianfootnote{سازنده}\LTRfootnote{constructive} و \persianfootnote{قابل اقدام}\LTRfootnote{actionable} (مثلاً «\persianfootnote{لایه تنگنا}\LTRfootnote{bottleneck layer} بیش از حد \persianfootnote{باریک}\LTRfootnote{narrow} است» یا «\persianfootnote{اتصال کوتاه}\LTRfootnote{skip connection} \persianfootnote{فراموش شده}\LTRfootnote{missing} است») به \persianfootnote{عامل معمار}\LTRfootnote{Architect Agent} ارائه دهد تا \persianfootnote{طراحی}\LTRfootnote{design} خود را در \persianfootnote{تکرار}\LTRfootnote{iteration} بعدی \persianfootnote{پالایش}\LTRfootnote{refine} کند.
\end{enumerate}  % فصل سوم: نتیجه گیری و کارهای آینده
% \include{latexIntro}

% مراجع
\pagestyle{empty}
{
    \onehalfspacing
    % \bibliographystyle{acm-fa}%{chicago-fa}%{plainnat-fa}%
    \bibliographystyle{ieeetr-fa}
    \bibliography{MyReferences}
}


\pagestyle{fancy}

\appendix                           %فصلهای پس از این قسمت به عنوان ضمیمه خواهند آمد.
% اگر شما پیوست اول  خود را در فایلی به جز appendix1 همراه با این کلاس نوشته‌اید باید چندخط اول appendix1 را در فایل خود کپی کنید.
% % !TeX root=main.tex
% دستورات زیر باید در اولین فایل پیوست باشند. آنها را حذف نکنید!
\addtocontents{toc}{
    \protect\renewcommand\protect\cftchappresnum{\appendixname~}%
    \protect\setlength{\cftchapnumwidth}{\mylenapp}}%

\chapter{مدیریت مراجع در لاتک}\label{App:RefMan}
\thispagestyle{empty}

در بخش \ref{Sec:Ref} اشاره شد که با دستور
\lr{\textbackslash bibitem}
می‌توان یک مرجع را تعریف نمود و با فرمان
\lr{\textbackslash cite}
به آن ارجاع داد. این روش برای تعداد مراجع زیاد و تغییرات آنها مناسب نیست. در ادامه به صورت مختصر توضیحی در خصوص برنامه \lr{BibTeX} که همراه با توزیع‌های معروف تک عرضه می‌شود و نحوه استفاده از آن در زی‌پرشین خواهیم داشت.

\section{ مدیریت مراجع با  \texorpdfstring{\lr{Bib\TeX}}{Bib\TeX} }
یکی از روش‌های قدرتمند و انعطاف‌پذیر برای نوشتن مراجع مقالات و مدیریت مراجع در لاتک، استفاده از  \lr{BibTeX} است.
روش کار با  \lr{BibTeX} به این صورت است که مجموعه‌ی همه‌ی مراجعی را که در \پ استفاده کرده یا خواهیم کرد،
در پرونده‌ی جداگانه‌ای نوشته و به آن فایل در سند خودمان به صورت مناسب لینک می‌دهیم.
کنفرانس‌ها یا مجله‌های گوناگون برای نوشتن مراجع، قالب‌ها یا قراردادهای متفاوتی دارند که به آنها استیلهای مراجع گفته می‌شود.
در این حالت به کمک ‌استیل‌های \lr{BibTeX} خواهید توانست تنها با تغییر یک پارامتر در پرونده‌ی ورودی خود، مراجع را مطابق قالب موردنظر تنظیم کنید.
بیشتر مجلات و کنفرانس‌های معتبر یک پرونده‌ی سبک (\lr{BibTeX Style}) با پسوند \lr{bst} در وب‌گاه خود می‌گذارند که برای همین منظور طراحی شده است.

به جز نوشتن مقالات این سبک‌ها کمک بسیار خوبی برای تهیه‌ی مستندات علمی همچون پایان‌نامه‌هاست که فرد می‌تواند هر قسمت از کارش را که نوشت مراجع مربوطه را به بانک مراجع خود اضافه نماید. با داشتن چنین بانکی از مراجع، وی خواهد توانست به راحتی یک یا چند ارجاع به مراجع و یا یک یا چند بخش را حذف یا اضافه ‌نماید؛
مراجع به صورت خودکار مرتب شده و فقط مراجع ارجاع داده شده در قسمت کتاب‌نامه خواهندآمد. قالب مراجع به صورت یکدست مطابق سبک داده شده بوده و نیازی نیست که کاربر درگیر قالب‌دهی به مراجع باشد.
در این جا مجموعه‌ سبک‌های بسته \lr{Persian-bib} که برای  زی‌پرشین آماده شده‌اند به صورت مختصر معرفی شده و روش کار با آن‌ها گفته می‌شود. برای اطلاع بیشتر به راهنمای بسته‌ی \lr{Persian-bib} مراجعه فرمایید.
\subsection{سبک‌های فعلی قابل استفاده در زی‌پرشین}
در حال حاضر فایلهای سبک زیر برای استفاده در زی‌پرشین آماده شده‌اند:

\singlespacing
\begin{description}
    \item [unsrt-fa.bst] این سبک متناظر با \lr{unsrt.bst} می‌باشد. مراجع به ترتیب ارجاع در متن ظاهر می‌شوند.
    \item [plain-fa.bst] این سبک متناظر با \lr{plain.bst} می‌باشد. مراجع بر اساس نام‌خانوادگی نویسندگان، به ترتیب صعودی مرتب می‌شوند.
          همچنین ابتدا مراجع فارسی و سپس مراجع انگلیسی خواهند آمد.
    \item [acm-fa.bst] این سبک متناظر با \lr{acm.bst} می‌باشد. شبیه \lr{plain-fa.bst} است.  قالب مراجع کمی متفاوت است. اسامی نویسندگان انگلیسی با حروف بزرگ انگلیسی نمایش داده می‌شوند. (مراجع مرتب می‌شوند)
    \item [ieeetr-fa.bst] این سبک متناظر با \lr{ieeetr.bst} می‌باشد. (مراجع مرتب نمی‌شوند)
    \item [plainnat-fa.bst] این سبک متناظر با \lr{plainnat.bst} می‌باشد. نیاز به بسته \lr{natbib} دارد. (مراجع مرتب می‌شوند)
    \item [chicago-fa.bst] این سبک متناظر با \lr{chicago.bst} می‌باشد. نیاز به بسته \lr{natbib} دارد. (مراجع مرتب می‌شوند)
    \item [asa-fa.bst] این سبک متناظر با \lr{asa.bst} می‌باشد. نیاز به بسته \lr{natbib} دارد. (مراجع مرتب می‌شوند)
\end{description}
\doublespacing

با استفاده از استیلهای فوق می‌توانید به انواع مختلفی از مراجع فارسی و لاتین ارجاع دهید. به عنوان نمونه مرجع
\cite{Omidali82phdThesis}
یک نمونه پروژه دکترا (به فارسی) و مرجع
\cite{Vahedi87} یک نمونه مقاله مجله فارسی است.
مرجع
\cite{Amintoosi87afzayesh}  یک نمونه  مقاله کنفرانس فارسی و
مرجع
\cite{Pedram80osool} یک نمونه کتاب فارسی با ذکر مترجمان و ویراستاران فارسی است. مرجع
\cite{Khalighi07MscThesis} یک نمونه پروژه کارشناسی ارشد انگلیسی و
\cite{Khalighi87xepersian} هم یک نمونه متفرقه  می‌باشند.

مراجع
\cite{Gonzalez02book,Baker02limits}
نمونه کتاب و مقاله انگلیسی هستند.
استیل مورد استفاده در این \پ \lr{acm-fa} است که خروجی آنرا در بخش مراجع می‌توانید مشاهده کنید.
نمونه  خروجی سبک \lr{asa-fa} در شکل \ref{fig:asafa} آمده است.

\begin{figure}[t]
    \centering
    \includegraphics[width=.8\textwidth]{asa-fa-crop.pdf}
    \caption{نمونه خروجی با سبک \lr{asa-fa}}
    \label{fig:asafa}
\end{figure}

\subsection{ نحوه استفاده از سبک‌های فارسی}


برای استفاده از بیب‌تک باید مراجع خود را در یک فایل با پسوند \lr{bib} ذخیره نمایید. یک فایل \lr{bib} در واقع یک پایگاه داده از مراجع\LTRfootnote{Bibliography Database}  شماست که هر مرجع در آن به عنوان یک رکورد از این پایگاه داده
با قالبی خاص ذخیره می‌شود. به هر رکورد یک مدخل\LTRfootnote{Entry} گفته می‌شود. یک نمونه مدخل برای معرفی کتاب \lr{Digital Image Processing} در ادامه آمده است:

\singlespacing
\begin{LTR}
    \begin{verbatim}
@BOOK{Gonzalez02image,
  AUTHOR =      {Rafael Gonzalez and Richard Woods},
  TITLE =       {Digital Image Processing},
  PUBLISHER =   {Prentice-Hall, Inc.},
  YEAR =        {2006},
  EDITION =     {3rd},
  ADDRESS =     {Upper Saddle River, NJ, USA}
}
\end{verbatim}
\end{LTR}
\doublespacing

در مثال فوق، \lr{@BOOK} مشخصه‌ی شروع یک مدخل مربوط به یک کتاب و \lr{Gonzalez02book} برچسبی است که به این مرجع منتسب شده است.
این برچسب بایستی یکتا باشد. برای آنکه فرد به راحتی بتواند برچسب مراجع خود را به خاطر بسپارد و حتی‌الامکان برچسب‌ها متفاوت با هم باشند معمولاً از قوانین خاصی به این منظور استفاده می‌شود. یک قانون می‌تواند فامیل نویسنده‌ی اول+دورقم سال نشر+اولین کلمه‌ی عنوان اثر باشد. به \lr{AUTHOR} و $\dots$ و \lr{ADDRESS} فیلدهای این مدخل گفته می‌شود؛ که هر یک با مقادیر مربوط به مرجع مقدار گرفته‌اند. ترتیب فیلدها مهم نیست.

انواع متنوعی از مدخل‌ها برای اقسام مختلف مراجع همچون کتاب، مقاله‌ی کنفرانس و مقاله‌ی ژورنال وجود دارد که برخی فیلدهای آنها با هم متفاوت است.
نام فیلدها بیانگر نوع اطلاعات آن می‌باشد. مثالهای ذکر شده در فایل \lr{MyReferences.bib} کمک خوبی به شما خواهد بود.
%این فایل یک فایل متنی بوده و با ویرایشگرهای معمول همچون \lr{Notepad++} قابل ویرایش می‌باشد. برنامه‌هایی همچون 
%\lr{TeXMaker}
% امکاناتی برای نوشتن این مدخل‌ها دارند و به صورت خودکار فیلدهای مربوطه را در فایل \lr{bib}  شما قرار می‌دهند.  
با استفاده از سبک‌های فارسی آماده شده، محتویات هر فیلد می‌تواند به فارسی نوشته شود، ترتیب مراجع و نحوه‌ی چینش فیلدهای هر مرجع را سبک مورد استفاده  مشخص خواهد کرد.

نکته: بدون اعمال تنظیمات موردنیاز \lr{Bib\TeX} در \lr{TeXWorks}، مراجع فارسی در استیل‌هایی که مراجع را به صورت مرتب شده چاپ می‌کنند، ترتیب کاملاً درستی نخواهند داشت. برای توضیحات بیشتر \cite{persianbib87userguide} را ببینید یا به سایت پارسی‌لاتک مراجعه فرمایید. تنظیمات موردنیاز در \lr{TeXMaker} اصلاح شده اعمال شده‌اند.

\textbf{برای درج مراجع خود لازم نیست نگران موارد فوق باشید. در فایل
    \lr{MyReferences.bib}
    که همراه با این \پ هست، موارد مختلفی درج شده است و کافیست مراجع خود را جایگزین موارد مندرج در آن نمایید.
}

پس از قرار دادن مراجع خود، یک بار \lr{XeLaTeX} را روی سند خود اجرا نمایید، سپس \lr{bibtex} و پس از آن دوبار \lr{XeLaTeX} را. در \lr{TeXMaker} کلید \lr{F11} و در \lr{TeXWorks} هم گزینه‌ی \lr{BibTeX} از منوی \lr{Typeset}، \lr{BibTeX} را روی سند شما اجرا می‌کنند.

برای بسیاری از مقالات لاتین حتی لازم نیست که مدخل مربوط به آنرا خودتان بنویسید. با جستجوی نام مقاله + کلمه \lr{bibtex}  در اینترنت سایتهای بسیاری همچون \lr{ACM} و \lr{ScienceDirect} را خواهید یافت که مدخل \lr{bibtex} مربوط به مقاله شما را دارند و کافیست آنرا به انتهای فایل \lr{MyReferences} اضافه کنید.

از هر یک از سبکهای \lr{Persian-bib} می‌توانید استفاده کنید، البته اگر از سه استیل آخر استفاده می‌کنید و مایلید که مراجع شما شماره بخورند باید بسته \lr{natbib} را با گزینه \lr{numbers} فراخوانی نمایید.
		% پیوست اول: مدیریت مراجع در لاتک
% % !TeX root=main.tex

\chapter{‌جدول، نمودار و الگوریتم در لاتک}\label{App:Latex:More}
\thispagestyle{empty}

در این بخش نمونه مثالهایی از جدول، نمودار و الگوریتم در لاتک را خواهیم دید.
\section{مدلهای حرکت دوبعدی}
بسیاری از اوقات حرکت بین دو تصویر از یک صحنه با یکی از مدلهای پارامتری ذکر شده در جدول \eqref{tab:MotionModels} قابل مدل نمودن می‌باشد.  
\begin{table}[ht]
\caption{مدلهای تبدیل.}
\label{tab:MotionModels}
\centering
\onehalfspacing
\begin{tabular}{|r|c|l|r|}
\hline نام مدل & درجه آزادی & تبدیل مختصات & توضیح \\ 
\hline انتقالی & ۲ & $\begin{aligned} x'=x+t_x \\ y'=y+t_y \end{aligned}$  &  انتقال دوبعدی\\ 
\hline اقلیدسی & ۳ & $\begin{aligned} x'=xcos\theta - ysin\theta+t_x \\ y'=xsin\theta+ycos\theta+t_y \end{aligned}$  &  انتقالی+دوران \\ 
\hline مشابهت & ۴ & $\begin{aligned} x'=sxcos\theta - sysin\theta+t_x \\ y'=sxsin\theta+sycos\theta+t_y  \end{aligned}$  & اقلیدسی+تغییرمقیاس \\ 
\hline آفین & ۶ & $\begin{aligned} x'=a_{11}x+a_{12}y+t_x \\ y'=a_{21}x+a_{22}y+t_y \end{aligned}$  & مشابهت+اریب‌شدگی \\ 
\hline  پروجکتیو & ۸ & $\begin{aligned} x'&=(m_1x+m_2y+m_3)/D \\ y'&=(m_4x+m_5y+m_6)/D \\ D&=m_7x+m_8y+1 \end{aligned}$  & آفین+\lr{keystone+chirping} \\ 
\hline  شارنوری & $\infty $ & $\begin{aligned} x'=x+v_x(x,y) \\ y'=y+v_y(x,y) \end{aligned}$  &  حرکت آزاد\\ 
\hline 
\end{tabular} 
\end{table}

\section{ماتریس}

شناخته‌شده‌ترین روش تخمین ماتریس هوموگرافی الگوریتم تبدیل خطی مستقیم (\lr{DLT\LTRfootnote{Direct Linear Transform}}) است.  فرض کنید چهار زوج نقطهٔ متناظر در دو تصویر در دست هستند،  $\mathbf{x}_i\leftrightarrow\mathbf{x}'_i$   و تبدیل با رابطهٔ
  $\mathbf{x}'_i = H\mathbf{x}_i$
  نشان داده می‌شود که در آن:
\[\mathbf{x}'_i=(x'_i,y'_i,w'_i)^\top  \]
و
\[ H=\left[
\begin{array}{ccc}
h_1 & h_2 & h_3 \\ 
h_4 & h_5 & h_6 \\ 
h_7 & h_8 & h_9
\end{array} 
\right]\]
رابطه زیر را برای الگوریتم  \eqref{alg:DLT} لازم دارم.
\begin{equation}\label{eq:DLT_Ah}
\left[
\begin{array}{ccc}
0^\top & -w'_i\mathbf{x}_i^\top & y'_i\mathbf{x}_i^\top \\ 
w'_i\mathbf{x}_i & 0^\top & -x'_i\mathbf{x}_i^\top \\ 
- y'_i\mathbf{x}_i^\top & x'_i\mathbf{x}_i^\top & 0^\top
\end{array} 
\right]
\left(
\begin{array}{c}
\mathbf{h}^1 \\ 
\mathbf{h}^2 \\ 
\mathbf{h}^3
\end{array} 
\right)=0
\end{equation}

\section{الگوریتم با دستورات فارسی}
با مفروضات فوق، الگوریتم \lr{DLT} به صورت نشان داده شده در الگوریتم \eqref{alg:DLT}  خواهد بود.
\begin{algorithm}[t]
\onehalfspacing
\caption{الگوریتم \lr{DLT} برای تخمین ماتریس هوموگرافی.} \label{alg:DLT}
\begin{algorithmic}[1]
\REQUIRE $n\geq4$ زوج نقطهٔ متناظر در دو تصویر 
${\mathbf{x}_i\leftrightarrow\mathbf{x}'_i}$،\\
\ENSURE ماتریس هوموگرافی $H$ به نحوی‌که: 
$\mathbf{x}'_i = H \mathbf{x}_i$.
  \STATE برای هر زوج نقطهٔ متناظر
$\mathbf{x}_i\leftrightarrow\mathbf{x}'_i$ 
ماتریس $\mathbf{A}_i$ را با استفاده از رابطهٔ \ref{eq:DLT_Ah} محاسبه کنید.
  \STATE ماتریس‌های ۹ ستونی  $\mathbf{A}_i$ را در قالب یک ماتریس $\mathbf{A}$ ۹ ستونی ترکیب کنید. 
  \STATE تجزیهٔ مقادیر منفرد \lr{(SVD)}  ماتریس $\mathbf{A}$ را بدست آورید. بردار واحد متناظر با کمترین مقدار منفرد جواب $\mathbf{h}$ خواهد بود.
  \STATE  ماتریس هوموگرافی $H$ با تغییر شکل $\mathbf{h}$ حاصل خواهد شد.
\end{algorithmic}
\end{algorithm}

\section{الگوریتم با دستورات لاتین}
الگوریتم \ref{alg:RANSAC} یک الگوریتم با دستورات لاتین است.

\begin{algorithm}[t]
\onehalfspacing
\caption{الگوریتم \lr{RANSAC} برای تخمین ماتریس هوموگرافی.} \label{alg:RANSAC}
\begin{latin}
\begin{algorithmic}[1]
\REQUIRE $n\geq4$ putative correspondences, number of estimations, $N$, distance threshold $T_{dist}$.\\
\ENSURE Set of inliers and Homography matrix $H$.
\FOR{$k = 1$ to $N$}
  \STATE Randomly choose 4 correspondence,
  \STATE Check whether these points are colinear, if so, redo the above step
  \STATE Compute the homography $H_{curr}$ by DLT algorithm from the 4 points pairs,
  \STATE $\ldots$ % الگوریتم کامل نیست
  \ENDFOR
  \STATE Refinement: re-estimate H from all the inliers using the DLT algorithm.
\end{algorithmic}
\end{latin}
\end{algorithm}

\section{نمودار}
لاتک بسته‌هایی با قابلیت‌های زیاد برای رسم انواع مختلف نمودارها دارد. مانند بسته‌های \lr{Tikz} و  \lr{PSTricks}. توضیح اینها فراتر از این پیوست کوچک است. مثالهایی از رسم نمودار را در مجموعه پارسی‌لاتک خواهید یافت. توصیه می‌کنم که حتماً مثالهایی از برخی از آنها را ببینید. راهنمای همه آنها در تک‌لایو هست. نمونه مثالهایی از بسته \lr{Tikz} را می‌توانید در \url{http://www.texample.net/tikz/examples/} ببینید.

\section{تصویر}
نمونه تصاویری در بخش قبل دیدیم. دو تصویر شیر کنار هم را هم در شکل \ref{fig:twolion} مشاهده می‌کنید.
\begin{figure}[t]
\centering 
\subfigure[شیر ۱]{ \label{fig:twolion:one}
\includegraphics[width=.3\textwidth]{lion}}
%\hspace{2mm}
\subfigure[شیر ۲]{ \label{fig:twolion:two}
\includegraphics[width=.3\textwidth]{lion}}
\caption{دو شیر}
\label{fig:twolion} %% label for entire figure
\end{figure}

%\baselineskip=.75cm
\onehalfspacing

\chapter*{واژه‌نامه فارسی به انگلیسی}\markboth{واژه‌نامه فارسی به انگلیسی}{واژه‌نامه فارسی به انگلیسی}
\addcontentsline{toc}{chapter}{واژه‌نامه فارسی به انگلیسی}
\thispagestyle{empty}
\persiangloss{آزمون‌ها}{trial}
\persiangloss{آزمون‌های واحد خودکار}{automated unit tests}
\persiangloss{آغاز گرم}{warm-starting}
\persiangloss{آگاه از زمینه}{context-aware}
\persiangloss{آگاهانه}{informed}
\persiangloss{ابرپارامترها}{Hyperparameters}
\persiangloss{ابزار}{tool}
\persiangloss{اتصال کوتاه}{skip connection}
\persiangloss{اتصالات پرشی}{skip connections}
\persiangloss{اجرا}{execution}
\persiangloss{ادبیات پژوهشی}{literature}
\persiangloss{ادبیات پژوهشی}{literature}
\persiangloss{ارتباطی}{communication protocols}
\persiangloss{ارزیابی و محک‌زنی}{Evaluation and Benchmarking}
\persiangloss{ارزیابی‌های}{evaluations}
\persiangloss{استانداردسازی}{canonicalization}
\persiangloss{استحکام}{robustness}
\persiangloss{استخراج دانش برون‌خط}{offline knowledge elicitation}
\persiangloss{استخراج دانش مبتنی بر ادبیات پژوهشی}{Literature-Driven Knowledge Extraction}
\persiangloss{استدلال}{reasoning}
\persiangloss{استدلال تقویت‌شده با بازیابی}{retrieval-augmented reasoning}
\persiangloss{استدلال‌ها}{reasoning steps}
\persiangloss{اسناد}{documents}
\persiangloss{اشتراک پارامتر}{parameter sharing}
\persiangloss{اعتبارسنجی}{validation}
\persiangloss{اعتبارسنجی دقیق}{rigorous validation}
\persiangloss{اعتبارسنجی پسین}{post-validation}
\persiangloss{الگوریتم ژنتیک}{genetic algorithm}
\persiangloss{الگوریتم‌های تکاملی}{evolutionary algorithms}
\persiangloss{الگوریتم‌های جستجو}{search algorithms}
\persiangloss{الگوهای اندک‌نمونه}{few-shot demonstrations}
\persiangloss{الگوهای طراحی}{design patterns}
\persiangloss{امتیازدهی ارتباط}{relevance scoring}
\persiangloss{انتخاب مدل}{model selection}
\persiangloss{انتقال مبتنی بر شباهت}{similarity-based transfer}
\persiangloss{باریک}{narrow}
\persiangloss{بازخورد}{feedback}
\persiangloss{بازخورد}{feedback}
\persiangloss{بازخورد زبانی}{linguistic feedback}
\persiangloss{بازخورد قابل اقدام}{actionable feedback}
\persiangloss{بازساز}{reconstructor}
\persiangloss{بازیابی}{retrieve}
\persiangloss{بازیابی مبتنی بر شباهت}{similarity-based retrieval}
\persiangloss{بازیابی مبتنی بر شباهت معنایی}{semantic similarity retrieval}
\persiangloss{بایگانی‌های کیفیت-تنوع به‌منزله حافظه}{Quality-Diversity Archives as Memory}
\persiangloss{بداعت}{novelty}
\persiangloss{برازش}{fitness}
\persiangloss{برنامه‌ریزی}{planning}
\persiangloss{برنامه‌ریزی تقویت‌شده با بازیابی}{retrieval-augmented planning}
\persiangloss{برنامه‌ریزی/تولید راه‌حل}{planning/solution generation}
\persiangloss{برنامه‌نویسی ژنتیکی}{genetic programming}
\persiangloss{بلندافق}{long-horizon}
\persiangloss{بلندمدت}{long-term}
\persiangloss{بلوک‌های ساختمانی}{building blocks}
\persiangloss{بهینه‌سازی}{optimization}
\persiangloss{بهینه‌سازی ابرپارامترها}{Hyperparameter Optimization (HPO)}
\persiangloss{بهینه‌سازی بیزی}{Bayesian optimization}
\persiangloss{بهینه‌سازی مستقیم از طریق دستوردهی تکراری}{Direct Optimization through Iterative Prompting}
\persiangloss{بهینه‌سازی کیفیت-تنوع}{quality-diversity optimization}
\persiangloss{به‌روز بودن}{up-to-dateness}
\persiangloss{بی‌نمونه}{zero-shot}
\persiangloss{تجربیات گذشته}{past experiences}
\persiangloss{تحلیل}{analyzes}
\persiangloss{تحلیل منابع دانش}{Knowledge Source Analysis}
\persiangloss{ترجمه ماشینی}{machine translation}
\persiangloss{ترکیب}{crossover}
\persiangloss{ترکیبی}{hybrid}
\persiangloss{تصمیمات ساده‌تر}{simpler decisions}
\persiangloss{تطبیق‌پذیر}{adaptive}
\persiangloss{تعامل}{interact}
\persiangloss{تعداد عملیات ممیز شناور}{FLOPs}
\persiangloss{تعداد پارامترها}{parameter count}
\persiangloss{تعمیم‌پذیری}{generalization}
\persiangloss{تعیین صریح توپولوژی}{explicit topology specification}
\persiangloss{تفسیرپذیری}{interpretability}
\persiangloss{تفکر}{Reasoning}
\persiangloss{تفکیک وظایف}{task decomposition}
\persiangloss{تقطیر دانش}{knowledge distillation}
\persiangloss{تقویت دانش}{Knowledge Augmentation}
\persiangloss{تقویت مبتنی بر فراداده}{metadata-driven augmentation}
\persiangloss{تنظیم ابرپارامترها}{hyperparameter tuning}
\persiangloss{تنظیم دستور}{prompt-tuning}
\persiangloss{توانایی‌های بازتابی}{reflective capabilities}
\persiangloss{توجه خودی}{self-attention}
\persiangloss{توصیفگرهای رفتاری}{behavioral descriptors}
\persiangloss{تولید تقویت‌شده با بازیابی}{Retrieval-Augmented Generation (RAG)}
\persiangloss{تولید خودبازگشتی}{autoregressive generation}
\persiangloss{تولید راه‌حل تک‌نمونه‌ای}{one-shot solution generation}
\persiangloss{تولید کد}{code generation}
\persiangloss{تولید کد قابل اعتبارسنجی}{Generating Verifiable Code}
\persiangloss{توهم}{hallucination}
\persiangloss{توهم}{hallucination}
\persiangloss{تکرار}{iteration}
\persiangloss{تکرار خطاها}{repeating mistakes}
\persiangloss{تک‌عاملی}{Single-Agent}
\persiangloss{تیم توسعه}{Development Team}
\persiangloss{تیم پژوهش}{Research Team}
\persiangloss{جانشین}{surrogate}
\persiangloss{جستجوی}{search}
\persiangloss{جستجوی تصادفی}{Random Search}
\persiangloss{جستجوی شبکه‌ای}{Grid Search}
\persiangloss{جستجوی معماری عصبی}{Neural Architecture Search (NAS)}
\persiangloss{جستجوی معماری عصبی}{NAS}
\persiangloss{جست‌وجوی معماری عصبی}{Neural Architecture Search (NAS)}
\persiangloss{جعبه‌سیاه}{black-box}
\persiangloss{جنگل‌های تصادفی}{random forests}
\persiangloss{جهش}{mutation}
\persiangloss{حافظه}{memory}
\persiangloss{حافظه ترتیبی}{episodic memory}
\persiangloss{حافظه رویدادی از رهگذر درخت‌های تغییر}{Episodic Memory via Modification Trees}
\persiangloss{حافظه‌های سلسله‌مراتبی}{hierarchical memory}
\persiangloss{حسگرها}{sensors}
\persiangloss{حلقه بهینه‌سازی}{optimization loop}
\persiangloss{حلقه‌های بازخورد}{feedback loops}
\persiangloss{حلقه‌های بازخورد انسانی}{Human-in-the-Loop (HITL)}
\persiangloss{حلقه‌های بازنگری}{revision loops}
\persiangloss{خروجی‌های}{outputs}
\persiangloss{خزش}{crawl}
\persiangloss{خط لوله}{pipeline}
\persiangloss{خطا}{errors}
\persiangloss{خطاهای}{errors}
\persiangloss{خطای اجرای کد}{code execution error}
\persiangloss{خطوط لوله}{Pipelines}
\persiangloss{خلاصه‌سازی متن}{text summarization}
\persiangloss{خوانایی ماشینی}{machine-readability}
\persiangloss{خود-اصلاحگر}{Self-Correcting}
\persiangloss{خودبازتابی}{self-reflection}
\persiangloss{خودمختار}{autonomous}
\persiangloss{دانش بیرونی: بازیابی از ادبیات و مخازن}{External Knowledge via Retrieval}
\persiangloss{دانش درونی: تاریخچه آزمون و بازتاب}{System-Internal Knowledge: Trials and Reflection}
\persiangloss{درخت‌های تغییر شبکه}{network modification trees}
\persiangloss{درستی نحوی}{syntactic correctness}
\persiangloss{درهم‌تنیده}{interleaved}
\persiangloss{درک زبان طبیعی}{Natural Language Understanding (NLU)}
\persiangloss{درک زمینه}{understanding context}
\persiangloss{دستور}{prompt}
\persiangloss{دستوردهی زمینه‌مند}{contextual prompting}
\persiangloss{دستوردهی مجدد}{re-prompting}
\persiangloss{دقت}{accuracy}
\persiangloss{دما}{temperature}
\persiangloss{ذخیره‌سازی هوشمند}{intelligent caching}
\persiangloss{رابط‌های برنامه‌نویسی کاربردی}{application programming interfaces (APIs)}
\persiangloss{راهبردهای باندیتی}{bandit-based strategies}
\persiangloss{راهبردهای تقویت آمیخته}{Hybrid Augmentation Strategies}
\persiangloss{راهبردهای تقویت دانش}{knowledge augmentation strategies}
\persiangloss{ردپاهای استدلالی}{reasoning traces}
\persiangloss{رقابتی}{competitive}
\persiangloss{رمزگذار}{encoder}
\persiangloss{رمزگشا}{decoder}
\persiangloss{رهگذر تفکیک کارکردی}{functional decomposition}
\persiangloss{روش‌های تکاملی}{evolutionary methods}
\persiangloss{رگرسیون}{regression}
\persiangloss{ریزتنظیم}{fine-tuning}
\persiangloss{زمینه}{context}
\persiangloss{زمینه}{context}
\persiangloss{زنجیره تفکر}{chain-of-thought}
\persiangloss{سازنده}{constructive}
\persiangloss{سازوکار بازیابی اطلاعات}{information retrieval mechanism}
\persiangloss{سبک‌گپ}{chat-style dialogues}
\persiangloss{سریال‌سازی}{serialization}
\persiangloss{سیاست}{policy}
\persiangloss{سیستم‌های عامل-محور}{Agent-based Systems}
\persiangloss{شبکه‌های عصبی عمیق}{Deep Neural Networks}
\persiangloss{شهود}{intuition}
\persiangloss{طبقه‌بندی}{classification}
\persiangloss{طراحی}{design}
\persiangloss{طرحواره‌های ازپیش‌تعریف‌شده}{predefined schemas}
\persiangloss{طرحواره‌های نرمال‌سازی}{normalization schemes}
\persiangloss{طرح‌های کدگذاری حوزه‌ای}{domain-specific encoding schemes}
\persiangloss{عامل ارزیاب}{Evaluator Agent}
\persiangloss{عامل ارزیاب}{Evaluator Agent}
\persiangloss{عامل تحلیلگر نیازمندی}{Requirement Analyst Agent}
\persiangloss{عامل معمار}{Architect Agent}
\persiangloss{عامل معمار}{Architect Agent}
\persiangloss{عامل معمار}{Architect Agent}
\persiangloss{عامل معمار}{Architect Agent}
\persiangloss{عامل منتقد}{Critic Agent}
\persiangloss{عامل منتقد/اصلاحگر}{Critic/Corrector Agent}
\persiangloss{عامل‌}{agent}
\persiangloss{عامل‌ها}{Agents}
\persiangloss{عامل‌ها}{Agents}
\persiangloss{عامل‌های}{agents}
\persiangloss{عرض‌نخست}{breadth-first}
\persiangloss{عمق‌نخست}{depth-first}
\persiangloss{عمل}{Action}
\persiangloss{عملگرهای تکاملی}{Evolutionary Operators}
\persiangloss{فرا-یادگیری}{meta-learning}
\persiangloss{فراداده ساختاریافته}{structured metadata}
\persiangloss{فراموش شده}{missing}
\persiangloss{فضای جستجو}{search space}
\persiangloss{فضای حالت و اقدام}{state-action space}
\persiangloss{فضای طراحی}{design space}
\persiangloss{قابل اقدام}{actionable}
\persiangloss{قابل موازی‌سازی}{parallelizable subtasks}
\persiangloss{قابلیت اطمینان}{reliability}
\persiangloss{قابلیت اطمینان و توهم}{Reliability and Hallucination}
\persiangloss{قطعات}{chunks}
\persiangloss{قیود منابع}{resource constraints}
\persiangloss{لایه تنگنا}{bottleneck layer}
\persiangloss{مجموعه متون}{corpus}
\persiangloss{محدودیت پنجره زمینه}{context window limitations}
\persiangloss{محک‌های}{benchmarks}
\persiangloss{محک‌های}{benchmarks}
\persiangloss{محیط}{environment}
\persiangloss{مخازن داده‌مجموعه و مدل}{Dataset and Model Repositories}
\persiangloss{مخازن کد}{code repositories}
\persiangloss{مدل‌سازی جانشین}{surrogate modeling}
\persiangloss{مدل‌های زبانی بزرگ}{Large Language Models (LLMs)}
\persiangloss{مدل‌های کوچک‌تر}{smaller models}
\persiangloss{مدیر عامل}{Agent Manager}
\persiangloss{مدیریت حافظه و وظایف بلند-افق}{Memory Management and Long-Horizon Tasks}
\persiangloss{مردمی‌سازی}{democratization}
\persiangloss{مسئله باز}{open question}
\persiangloss{معماری‌های}{architectures}
\persiangloss{معماری‌های حافظه}{memory architectures}
\persiangloss{معناداری معنایی}{semantic meaningfulness}
\persiangloss{مفاهیم}{concepts}
\persiangloss{مقیاس}{scale}
\persiangloss{منتقد}{critic}
\persiangloss{منجمد}{frozen}
\persiangloss{منسجم}{coherent}
\persiangloss{مهندسی ویژگی}{feature engineering}
\persiangloss{مولفه‌های ماژولار}{modular components}
\persiangloss{مولفه‌های ماژولار}{modular components}
\persiangloss{مکانیزم خود-اصلاحی}{Self-Correction Mechanism}
\persiangloss{مکانیزم‌های خود-اصلاحی}{self-correction mechanisms}
\persiangloss{میان‌گونه‌ای}{across modalities}
\persiangloss{ناقص}{buggy}
\persiangloss{ناهمخوانی}{inconsistencies}
\persiangloss{نتایج}{results}
\persiangloss{نتایج}{results}
\persiangloss{نمایش‌های برداری}{vector representations}
\persiangloss{نمایش‌های ساخت‌یافته کلید–مقدار}{structured key-value representations}
\persiangloss{نمایش‌های گراف}{graph representations}
\persiangloss{نمایش‌های گراف/درخت}{graph or tree representations}
\persiangloss{نمرات کنجکاوی}{curiosity scores}
\persiangloss{نمونه‌برداری نامزد}{candidate sampling}
\persiangloss{نهفتارسازی}{embedding}
\persiangloss{نهفتگی‌ها}{embeddings}
\persiangloss{هدف اصلی}{Main objective}
\persiangloss{هرس}{prune}
\persiangloss{هزینه}{cost}
\persiangloss{هزینه استنتاج}{inference cost}
\persiangloss{هزینه محاسباتی}{computational cost}
\persiangloss{هماهنگی سلسله‌مراتبی}{Hierarchical Coordination}
\persiangloss{هماهنگ‌کننده}{coordinator}
\persiangloss{هماهنگ‌کننده}{orchestrator}
\persiangloss{همریختی}{isomorphism}
\persiangloss{همکارانه}{collaborative}
\persiangloss{همکاری بین‌عاملی}{inter-agent collaboration}
\persiangloss{همکاری مبتنی بر نقش}{Role-Based Collaboration}
\persiangloss{هم‌بندی}{ensemble}
\persiangloss{وابستگی‌های دوربرد}{long-range dependencies}
\persiangloss{وجه‌های داده}{modalities}
\persiangloss{ورودی‌های نویزی}{noisy inputs}
\persiangloss{وظایف خارج از توزیع}{out-of-distribution tasks}
\persiangloss{پاسخ‌های ایستا}{static responses}
\persiangloss{پالایش}{refine}
\persiangloss{پالایش}{refine}
\persiangloss{پایداری}{robustness}
\persiangloss{پایگاه داده برداری}{vector database}
\persiangloss{پایگاه دانش}{knowledge base}
\persiangloss{پایگاه‌های داده برداری}{vector databases}
\persiangloss{پایگاه‌های دانش}{knowledge bases}
\persiangloss{پربعدتر}{higher-dimensional}
\persiangloss{پنجره زمینه}{context window}
\persiangloss{پیش‌آموزش}{pre-training}
\persiangloss{پیش‌پردازش داده}{data preprocessing}
\persiangloss{پیکربندی}{configuration}
\persiangloss{پیکربندی}{configuration}
\persiangloss{پیکره‌های کد}{code corpora}
\persiangloss{چارچوب نظری}{paradigm}
\persiangloss{چندسطوحی/چندوفایی}{multi-fidelity}
\persiangloss{چندعاملی}{Multi-Agent}
\persiangloss{کارایی}{performance}
\persiangloss{کارایی}{performance}
\persiangloss{کارایی محاسباتی}{computational efficiency}
\persiangloss{کارایی محاسباتی و هزینه}{Computational Efficiency and Cost}
\persiangloss{کارایی نمونه}{sample efficiency}
\persiangloss{کارایی نهایی}{final performance}
\persiangloss{کارت‌های داده}{data cards}
\persiangloss{کارت‌های مدل}{model cards}
\persiangloss{کارگزار بازتابنده}{reflector agent}
\persiangloss{کارگزاران خوانش تخصصی}{specialized reader agents}
\persiangloss{کد}{code}
\persiangloss{کد}{code}
\persiangloss{کد}{code}
\persiangloss{کدگذاری رشته‌ای جداکننده‌محور}{delimited string encodings}
\persiangloss{کم‌نمونه}{few-shot}
\persiangloss{کنترل‌گرهای جریان کار}{Workflow Controllers}
\persiangloss{کنترل‌گرهای جریان‌کار}{workflow controllers}
\persiangloss{کنشگرها}{actuators}
\persiangloss{کیفیت–تنوع}{quality-diversity}
\persiangloss{گذارهای بین قالب‌ها}{format transitions}
\persiangloss{گراف جهت‌دار بدون‌دور}{Directed Acyclic Graph (DAG)}
\persiangloss{گراف محاسباتی}{computational graph}
\persiangloss{گرم‌آغاز}{warmstart}
\persiangloss{گزارش وقایع}{log}
\persiangloss{گزیننده‌های ویژگی}{feature selectors}
\persiangloss{گسسته‌سازی}{discretized}
\persiangloss{گفت‌وگوهای چت}{chat-style dialogues}
\persiangloss{گونه خروجی}{output modalities}
\persiangloss{یادگیری}{learning}
\persiangloss{یادگیری از بازخورد محیطی}{Learning from Environmental Feedback}
\persiangloss{یادگیری تقویتی}{Reinforcement Learning}
\persiangloss{یادگیری تقویتی}{Reinforcement Learning}
\persiangloss{یادگیری درون‌متنی از تاریخچه بهینه‌سازی}{In-Context Learning from Optimization History}
\persiangloss{یادگیری زمینه‌ای}{In-Context Learning (ICL)}
\persiangloss{یادگیری ماشین خودکار}{Automated Machine Learning (AutoML)}

\chapter*{واژه‌نامه  انگلیسی به  فارسی}\markboth{واژه‌نامه  انگلیسی به  فارسی}{واژه‌نامه  انگلیسی به  فارسی}
\addcontentsline{toc}{chapter}{واژه‌نامه  انگلیسی به  فارسی}
\thispagestyle{empty}


% \persiangloss{مجموعه جزئاً مرتب کامل جهت‌دار}{Dcpo}
% \persiangloss{فضای تابع}{Function Space}
\persiangloss{اندازه }{Measure}
% \persiangloss{مرتب}{Ordered}
% \persiangloss{دامنه‌توانی}{Powerdomain}
% \persiangloss{احتمالی}{Probabilistic}
% \persiangloss{قطعه‌برنامه}{Program Fragment}
% \persiangloss{دامنه معنایی}{Semantic Domain}
% \persiangloss{پایدار}{Stably}
% \persiangloss{ارزیابی}{Valuation}
% \persiangloss{توپولوژی ضعیف}{Weak Topology}

\printindex
% !TeX root=main.tex
% در این فایل، عنوان پایان‌نامه، مشخصات خود و چکیده پایان‌نامه را به انگلیسی، وارد کنید.

%%%%%%%%%%%%%%%%%%%%%%%%%%%%%%%%%%%%
\baselineskip=.6cm
\begin{latin}
\latinuniversity{Iran University of Science and Technology}
\latinfaculty{Computer Engineering Department}
\latinsubject{Computer Engineering-Artificial Intelligence}
% \latinfield{Artificial Intelligence}
\latintitle{Review of deep learning methods in brain tumor segmentation in medical images}
\firstlatinsupervisor{Dr. Mohsen Soryani}
%\secondlatinsupervisor{Second Supervisor}
% \firstlatinadvisor{First Advisor}
%\secondlatinadvisor{Second Advisor}
\latinname{Morteza}
\latinsurname{Hajiabadi}
\latinthesisdate{November 2023}
\latinkeywords{Deep learning, segmentation, brain tumor, medical images, artificial intelligence}
\en-abstract{
Diagnosis and segmentation of brain tumors is one of the most challenging and critical issues in medical imaging. Accuracy in identifying tumorous areas is very important because it is essential for treatment decisions and prediction of disease outcomes.
\\
In recent years, thanks to significant advances in the field of deep learning, artificial intelligence models have been able to play an important role in improving the identification of brain tumors. Among these developments, we can mention the use of deep neural networks.
\\
Deep neural networks can analyze brain images using convolutional layers and attention layers and detect tumor areas. These networks allow doctors to determine tumor areas with high accuracy and plan more effective treatment.
\\
Overall, advances in deep learning in brain tumor diagnosis and segmentation have provided clinicians with more powerful tools for treatment decision-making. These advances have helped increase the speed and accuracy of brain tumor diagnosis, which ultimately leads to improved treatment and outcomes for patients. In this seminar report, we examine the recent developments in this field and the challenges and problems of the existing methods.
\\
Our research shows that deep learning methods have much better performance compared to traditional methods. In addition, among deep learning methods, Transformer-based methods have better performance than methods based on Convolutional layers, but they also have problems such as the need for too much data. In recent years, interactive methods for the segmentation of brain tumors have obtained very good results, which has caused more attention of specialists in this field to these methods.
}
\latinfirstPage
\end{latin}

\label{LastPage}

\end{document}