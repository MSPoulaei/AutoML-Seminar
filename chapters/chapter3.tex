\chapter{مروری بر کارهای مرتبط}
\thispagestyle{empty}

\section{مقدمه}
در سال‌های اخیر، تحقیقات زیادی با بهره‌گیری از شبکه‌های عصبی انجام شده است. در اوایل این دوره، شبکه‌های \persianfootnote{پرسپترون چندلایه}\LTRfootnote{Multi Layer Perceptron} به طور گسترده مورد استفاده قرار می‌گرفتند. اما در حوزه بینایی کامپیوتر، این نوع شبکه‌ها توانایی حفظ ساختار دو بعدی یا چند بعدی تصاویر و ورودی‌های ویدئویی را نداشته و ورودی‌ها را به یک بعد تبدیل می‌کردند. این عمل باعث از بین رفتن بخش مهمی از اطلاعات مکانی می‌شد. بنابراین، محققان به جستجوی راهکارهایی به نام شبکه‌های عصبی پیچشی پرداختند که بر خلاف شبکه‌های پرسپترون چندلایه، ساختار ورودی را حفظ می‌کنند. امروزه، شبکه‌های پیچشی به دلیل توانایی برتر خود در حل مشکلات، در حوزه‌های متعددی از بینایی کامپیوتر به نتایج قابل توجهی دست یافته‌اند.
 % از دیگر مزایای شبکه‌های عصبی می‌توان به توانایی آن‌ها در ترکیب داده‌های مختلف از جمله تصاویر اشعه X، سونوگرافی، میکروسکوپی و سی‌تی اسکن اشاره کرد.
\\
با این حال، این روش‌ها همچنان با مشکلاتی روبه‌رو هستند. یکی از مهم‌ترین این مشکلات نیاز به تعداد قابل توجهی داده‌های آموزشی است. کمبود داده‌های آموزشی مناسب توانایی شبکه‌های عصبی پیچشی را کاهش می‌دهد. هرچند برخی از تحقیقات روش‌هایی ارائه داده‌اند که با داده‌های کم نیز به دقت قابل توجهی می‌رسند، اما این مشکل هنوز باقی مانده است. مسائل دیگری مانند حفظ حریم شخصی بیماران و \persianfootnote{بیش‌برازش}\LTRfootnote{Overfitting} نیز جزو چالش‌های این روش‌ها می‌باشند\cite{biratu2021survey}.

\section{شبکه های مبتنی بر لایه های پیچشی}
شبکه های عصبی پیچشی به دلیل تطبیق پذیری و سازگاری آنها برای پرداختن به وظایف مختلف، به طور گسترده به عنوان مدل های یادگیری عمیق اساسی و محبوب شناخته می شوند. در حالی که بسیاری از مدل‌های \lr{CNN} از پیش آموزش‌دیده برای حل طیف وسیعی از چالش‌های بینایی رایانه در مجموعه داده‌های حجمی مختلف به کار گرفته شده‌اند، یک مدل \lr{CNN} سفارشی‌شده می‌تواند مزایای خاصی را در برنامه‌های کاربردی مناسب ارائه دهد و می‌تواند برای هر مورد استفاده منحصربه‌فرد \persianfootnote{تنظیم دقیق}\LTRfootnote{Fine Tune} شود. \lr{CNN}ها شامل لایه‌های مختلفی از جمله لایه‌های  پیچشی، \persianfootnote{ادغام}\LTRfootnote{Pooling}و \persianfootnote{کاملاً متصل}\LTRfootnote{Fully Connected} هستند که استخراج و انتخاب خودکار ویژگی‌ها و رده‌بندی شدت پیکسل بعدی را با استفاده از لایه‌های \persianfootnote{متراکم}\LTRfootnote{Dense} بر اساس برچسب‌های رده تسهیل می‌کنند. \persianfootnote{بهینه سازها}\LTRfootnote{Optimizer}،\persianfootnote{ توابع ضرر}\LTRfootnote{Loss Function}، و \persianfootnote{نرخ یادگیری}\LTRfootnote{Learning Rate}، هم از \persianfootnote{انتشار رو به جلو}\LTRfootnote{Forward Propagation} و هم \persianfootnote{به عقب}\LTRfootnote{Backward Propagation} پشتیبانی می کنند و امکان تنظیم وزن ماتریس های ویژگی را برای تولید نتایج بهینه فراهم می کنند. این مدل‌ها پیچیدگی معماری نسبتاً ساده‌ای دارند و به حداقل زمان آموزشی نیاز دارند، اگرچه عملکرد آنها ممکن است به اندازه برخی معماری‌های دیگر قوی نباشد\cite{das2022artificial}.
\\
یکی از رایج‌ترین معماری‌های مورد استفاده برای قطعه‌بندی تصویر، ساختارهای \persianfootnote{کدگذار}\LTRfootnote{Encoder}-\persianfootnote{کدگشا}\LTRfootnote{Decoder} است. این معماری‌ها که در سال‌های اخیر توسعه چشمگیری داشته‌اند، ثابت کرده‌اند که به طور موثر و دقیق به مسائل متعددی از این دست می‌پردازند. در ادامه به بررسی شبکه \lr{َUNet}و چند مدل از اعضای این خانواده می‌پردازیم.

\subsection{ شبکه \lr{UNet}}

در مقاله \cite{futrega2021optimized} یک مدل گسترش یافته برای مدل \lr{UNet} اولیه معرفی شد. معماری \lr{UNet} بهینه ارائه شده در این مقاله بر اساس معماری سنتی \lr{UNet} است. تفاوت اصلی بین این دو این است که \lr{UNet} بهینه شده دارای یک کدگذار عمیق تر با کانال های پیچیده تر در هر سطح است. به طور خاص، عمق کدگذار از 6 به 7 افزایش یافت و تعداد کانال ها در هر سطح به 64، 96، 128، 192، 256، 384، 512 تغییر یافت. این تغییرات برای بهبود \persianfootnote{امتیاز پایه}\LTRfootnote{‌Baseline Score}معماری سنتی \lr{UNet} اعمال شد. علاوه بر این، \lr{UNet} بهینه‌سازی شده شامل تغییرات برنامه آموزش و تغییرات معماری، مانند \persianfootnote{نظارت عمیق}\LTRfootnote{Deep Supervision}، \persianfootnote{بلوک رهاکردن}\LTRfootnote{Drop Block}، و \persianfootnote{ضرر کانونی}\LTRfootnote{Focal Loss}است. معماری این مدل بهینه شده را در شکل \ref{fig:optunet} ببینید.
\begin{figure}[h]
\centerline{\includegraphics[width=15cm]{images/optunet.pdf}}
\caption[معماری \lr{UNet} بهینه شده]{
   معماری \lr{UNet} بهینه شده. کدگذار ورودی را با کاهش ابعاد مکانی آن تغییر شکل می دهد و سپس کدگشا آن را به شکل ورودی اولیه برمی گرداند. دو سر خروجی اضافی برای ضرر نظارت عمیق (نوارهای سبز) استفاده می شود.\cite{futrega2021optimized}. }
\label{fig:optunet}
\end{figure}
\\
تغییرات اعمال شده در این مدل به صورت زیر است:
\begin{itemize}
    \item نظارت عمیق: نظارت عمیق روشی است که برای بهبود جریان گرادیان در شبکه‌های عصبی با محاسبه تابع ضرر در سطوح مختلف کدگشا استفاده می‌شود. نویسندگان \cite{futrega2021optimized} دو سر خروجی اضافی را به معماری \lr{UNet} اضافه کردند تا ضرر نظارت عمیق را محاسبه کنند. برچسب ها ابتدا با استفاده از درونیابی نزدیکترین همسایه برای مطابقت با اشکال خروجی های اضافی، نمونه برداری شدند. تابع ضرر نهایی بر اساس برچسب ها و پیش بینی ها برای هر سر خروجی محاسبه می شود.
    \item بلوک رها کردن: یک روش منظم سازی است که به طور تصادفی بلوک های پیوسته از نقشه های ویژگی را در طول آموزش رها می کند. شبیه \lr{dropout} است، اما به جای رها کردن تک تک نورون‌ها، کل بلوک‌های نورون را حذف می‌کند. \lr{UNet} بهینه شده از بلوک رها کردن با اندازه بلوک 3$\times$3$\times$3 و احتمال رها کردن $0.1$ استفاده می کند.
    \item ضرر کانونی: یک نسخه اصلاح شده از تابع \lr{CE} است که نمونه های آسان را کاهش می دهد و نمونه های سخت را افزایش می دهد. این ویژگی برای رسیدگی به مشکل عدم تعادل رده ها طراحی شده است. \lr{UNet} بهینه شده از افت کانونی با مقدار گامای $2$ و مقدار آلفا $0.75$ استفاده می کند.
    \item نویز گاوسی: یک روش افزایش داده است که در آن نویز تصادفی گاوسی به حجم ورودی اضافه می شود. در \lr{UNet} بهینه سازی شده، نویز گاوسی با احتمال $0.15$ اضافه می شود و یک انحراف استاندارد به طور یکنواخت از ($0.33$، $0$) نمونه برداری می شود.
    \item \persianfootnote{تاری}\LTRfootnote{‌Blur}گاوسی: یک روش افزایش داده است که در آن تاری گاوسی روی حجم ورودی اعمال می شود. در \lr{UNet} بهینه‌سازی شده، تاری گاوسی با احتمال $0.15$ و انحراف استاندارد هسته گاوسی به طور یکنواخت از ($1.5$، $0.5$) نمونه‌برداری می‌شود.
    \item روشنایی: یک روش افزایش داده است که در آن روشنایی حجم ورودی به طور تصادفی تنظیم می شود. در \lr{UNet} بهینه سازی شده، روشنایی با احتمال $0.15$ با ضرب وکسل های حجم ورودی در یک مقدار تصادفی نمونه برداری یکنواخت از ($1.3$، $0.7$) تنظیم می شود.
    \item کنتراست: یک روش افزایش داده است که در آن کنتراست حجم ورودی به طور تصادفی تنظیم می شود. در \lr{UNet} بهینه شده، کنتراست با احتمال $0.15$ با ضرب وکسل های حجم ورودی در یک مقدار تصادفی نمونه برداری یکنواخت از ($1.5$، $0.65$) و سپس برش مقادیر در محدوده اصلی تنظیم می شود.
\end{itemize}

برای ارزیابی مدل، آنها از  \persianfootnote{اعتبارسنجی متقاطع}\LTRfootnote{Cross Validation}\lr{5-fold}استفاده کردند و میانگین بالاترین نمره \lr{Dice} را که در هر یک از \lr{5-fold} به دست آمد، مقایسه کردند. ارزیابی مجموعه اعتبارسنجی پس از هر دوره انجام شد. برای هر \lr{fold}، آنها دو \persianfootnote{نقطه بررسی}\LTRfootnote{Checkpoint}را با بالاترین میانگین نمره \lr{Dice} در مجموعه اعتبارسنجی به دست آمده در مرحله آموزش ذخیره کرده اند. سپس در مرحله استنتاج، پیش‌بینی‌های نقطه‌های بررسی ذخیره‌شده را با میانگین‌گیری احتمالات ترکیب کردند. تاثیر تغییرات ایجاد شده در معماری ستنی مدل \lr{UNet} را در جدول های \ref{tab:optunet1}و \ref{tab:optunet2} می‌بینیم.

\begin{table}[ht]
\caption[ عملکرد مدل بهینه شده\lr{UNet }  روی مجموعه دادگان\lr{ BraTS2021 }]{میانگین نمرات \lr{Dice} رده های \lr{ET}، \lr{TC}، \lr{WT} برای هر \lr{5-fold} با مقایسه مدل های پایه \lr{UNet}، \lr{UNetR}، \lr{SegResNetVAE}\cite{futrega2021optimized}.}
\label{tab:optunet1}
\centering
\onehalfspacing
\begin{tabular}{|c|c|c|c|}
\hline 
مدل &  \lr{UNet} & \lr{UNetR} & \lr{SegResNetVAE} \\
\hline 
\lr{Fold 0} & $\mathbf{0.9087}$ & $0.9044$ & $0.9086$ \\
\hline 
\lr{Fold 1} & $\mathbf{0.9100}$ & $0.8976$ & $0.9090$ \\
\hline 
\lr{Fold 2} & $\mathbf{0.9162}$ & $0.9051$ & $0.9140$ \\
\hline 
\lr{Fold 3} & $\mathbf{0.9238}$ & $0.9111$ & $0.9219$ \\
\hline 
\lr{Fold 4} & $\mathbf{0.9061}$ & $0.8971$ & $0.9053$ \\
\hline 
میانگین \lr{Dice} & $\mathbf{0.9130}$ & $0.9031$ & $0.9118$ \\
\hline
\end{tabular}
\end{table}

\begin{table}[h]
\caption[ عملکرد مدل بهینه شده\lr{UNet }  روی مجموعه دادگان\lr{ BraTS2021 }۲]{میانگین نمرات \lr{Dice} رده های \lr{ET}، \lr{TC}، \lr{WT} برای هر \lr{5-fold} مقایسه نظارت عمیق (\lr{DS})، بلوک رهاکردن (\lr{DB}) و ضرر کانونی .\cite{futrega2021optimized}.}
\label{tab:optunet2}
\centering
\onehalfspacing
\begin{tabular}{|c|c|c|c|c|c|}
\hline
مدل & مدل پایه &  \lr{DS} &  \lr{DB} & کانونی \\
\hline
\lr{Fold 0} & $0.9087$ &  $\mathbf{0.9111}$ &  $0.9096$ & $0.9094$ \\
\hline
\lr{Fold 1} & $0.9100$ &  $\mathbf{0.9115}$ &  $0.9114$ & $0.9026$ \\
\hline
\lr{Fold 2} & $0.9162$ &  $\mathbf{0.9175}$ &  $0.9159$ & $0.9146$ \\
\hline
\lr{Fold 3} & $0.9238$ &  $\mathbf{0.9268}$ &  $0.9241$ & $0.9229$ \\
\hline
\lr{Fold 4} & $0.9061$ &  $\mathbf{0.9074}$ &  $0.9071$ & $0.9072$ \\
\hline
میانگین \lr{Dice} & $0.9130$ &  $\mathbf{0.9149}$ &  $0.9136$ & $0.9133$ \\
\hline
\end{tabular}
\end{table}
مقاله \cite{montaha2023brain} یک روش برای تقسیم‌قطعه‌بندی تومور مغزی با استفاده از معماری سه بعدی \lr{UNet} پیشنهاد می‌کند. معماری سه بعدی \lr{UNet} پیشنهادی از یک ساختار کدگذار-کدگشا با اتصالات میانبر بین لایه‌های کدگذار و کدگشا تشکیل شده است. کدگذار از 4 بلوک پیچشی تشکیل شده است که هر کدام دارای 2 لایه پیچشی است که به دنبال نرمال سازی دسته ای، فعال سازی \lr{ReLU} و حداکثر ادغام می باشد. کدگشا از 4 بلوک پیچشی تشکیل شده است که هر کدام دارای 2 لایه پیچشی است که به دنبال آن نرمال سازی دسته ای، فعال سازی \lr{ReLU} و الحاق با خروجی کدگذار مربوطه انجام می شود. لایه نهایی یک لایه پیچشی 1$\times$1 با فعال سازی سیگموئید برای تولید ماسک قطعه‌بندی دودویی است. مجموعه دادگان مورد استفاده در این مقاله \lr{BraTS2020} است. نتایج آن را در جدول \ref{tab:unet3d_result} می بینید.

\begin{table}[ht]
\caption[ عملکرد مدل \lr{UNet} ۳ بعدی]{ عملکرد مدل \lr{UNet}\cite{montaha2023brain}}
\label{tab:unet3d_result}
\centering
\onehalfspacing
\begin{tabular}{|l|l|l|l|l|l|}
\hline
توالی \lr{MRI} &  \lr{Dice} \\
\hline
\lr{FLAIR} &  $91.23$ \\
\hline
\lr{T1} & $\mathbf{93.86}$ \\
\hline
\lr{T1ce} & $85.67$ \\
\hline
\lr{T2T1ce} & $79.32$ \\
\hline
\end{tabular} 
\end{table}


\subsection{ شبکه \lr{UNet++}}
یکی از زیرمجموعه‌های شبکه \lr{UNet} که برای قطعه‌بندی تصاویر پزشکی تولید و با هدف بهبود عملکرد مدل \lr{UNet} طراحی شده است، مدل \lr{UNet++} است، که در سال ۲۰۱۸ معرفی شد. همانند \lr{UNet}، این مدل دارای دو بخش کدگذار و کدگشا است. با این حال، ظاهر کلی این شبکه شبیه به \lr{UNet} خواهد بود و از سلسله مراتبی مشابه هرم برخوردار است. اما مدل \lr{UNet++} دارای سه تفاوت اصلی نیز با مدل \lr{UNet} است، شامل بازطراحی مسیرهای میانبر، اتصالات میانبر فشرده‌تر و نظارت عمیق. \lr{UNet++} مسیرهای میانبر بیشتری را به معماری اصلی \lr{UNet} اضافه می کند که به گرفتن اطلاعات دقیق تر از تصویر ورودی کمک می کند. شکل \ref{fig:unet++}ساختار دقیق \lr{UNet++} شامل بخش‌های کدگذار و کدگشا، مسیرهای میانبر و تعداد نقشه های ویژگی را در هر مرحله از شبکه نشان می‌دهد.
\\
\begin{figure}[h]
\centerline{\includegraphics[width=13cm]{images/unet++}}
\caption[ساختار شبکه \lr{UNet++}]{ساختار شبکه \lr{UNet++}\cite{li2022unet++}.}
\label{fig:unet++}
\end{figure}
\\
مدل \lr{UNet++} از لایه‌های پیچشی بین کدگذار و کدگشا و همچنین اتصالات میانبر استفاده می‌کند. این کار به منظور کاهش فاصله و فضای بین نگاشت ویژگی‌ها در کدگذار و کدگشای یک مدل \lr{UNet} انجام می‌شود. همچنین، در فرآیند آموزش، تابع ضرر از چند مسیر، شبکه را آموزش می‌دهد. این کار به منظور جلوگیری از \persianfootnote{محو شدگی گرادیان}\LTRfootnote{Vanishing Gradient} و تأثیر بیشتر آن روی نورون‌ها انجام می‌شود\cite{li2022unet++}. 
\\
در مقاله \cite{hou2021brain} که در سال ۲۰۲۱ منتشر شد، یک ساختار بهبود یافته از مدل سنتی \lr{UNet++} معرفی شد. مدل پیشنهادی یک نسخه بهبود یافته از ساختار شبکه سنتی \lr{UNet++} برای وظیفه قطعه‌بندی تومور مغز در تصاویر \lr{MRI} است. یک ماژول باقیمانده و بودجه وزن را به \lr{UNet++} اضافه می‌کند تا عملکرد قطعه‌بندی تومور مغز را بهبود بخشد. نتایج تجربی قطعه‌بندی تصویر \lr{MRI} تومور مغز از هر دو دیدگاه ذهنی و عینی ارزیابی می‌شوند. نتایج نشان می‌دهد که مدل بهبودیافته شبکه \lr{UNet++} تا حد زیادی نسبت به مدل بهبود یافته شبکه \lr{UNet} برتری دارد و عملکرد مدل شبکه بهبودیافته \lr{++UNet} پس از پیش‌بینی وزن حتی بهتر است. مدل پیشنهادی همچنین از سه \persianfootnote{نمونه برداری پایین}\LTRfootnote{Down Sampling} و شش \persianfootnote{نمونه برداری بالا}\LTRfootnote{Up Sampling} استفاده می کند، از پیچش های سه بعدی دو بار در طول هر نمونه برداری بالا و پایین استفاده می کند، و یک ماژول باقیمانده بین آخرین نمونه برداری پایین و اولین نمونه برداری اضافه می شود تا از از دست رفتن اطلاعات شبکه و تخریب شبکه جلوگیری شود. در نهایت یک پیچش سه بعدی اضافه می شود و اندازه نهایی ذخیره شده 16×160×160×3 است. در مقایسه، ساختار شبکه سنتی \lr{UNet++} این پیشرفت ها و ویژگی ها را ندارد. معماری این شبکه را در شکل \ref{fig:optunet++} مشاهده می‌کنید. 

\begin{figure}[h]
\centerline{\includegraphics[width=13cm]{images/optunet++.pdf}}
\caption[مدل بهبود یافته \lr{UNet++}]{مدل بهبود یافته \lr{UNet++}\cite{hou2021brain}}
\label{fig:optunet++}
\end{figure}

جدول \ref{tab:optunet++} عملکرد این مدل را نشان می‌دهد.
\begin{table}[ht]
\caption{عملکرد \lr{UNet++} بهینه شده}
\label{tab:optunet++}
\centering
\onehalfspacing
\begin{tabular}{|c|c|c|c|}
\hline
معیار ارزیابی & \lr{UNet} با \lr{Residual} & \lr{UNet++} با \lr{Residual} & \lr{UNet++} با \lr{Residual} و بودجه وزنی  \\
\hline
\lr{WT Dice} & $0.8772$ & $0.8980$ & $0.8938$ \\
\hline
\lr{TC Dice} & $0.8190$ & $0.8490$ & $0.8534$ \\
\hline
\lr{ET Dice} & $0.8224$ & $0.8670$ & $0.8759$ \\
\hline
زمان آموزش & ۳۱ ساعت  & ۵۳ ساعت & ۱۰ ساعت  \\
\hline
\end{tabular}
\end{table}

\subsection{ شبکه \lr{DoubleUNet}}
این شبکه از دو شبکه \lr{UNet} تشکیل شده و اگرچه یکی از آن‌ها به طور کمی از نسخه اصلی متفاوت است، ولی به علت توانایی برجسته مدل \lr{VGG} در استخراج ویژگی‌ها، از این شبکه به عنوان کدگذار اولیه \lr{UNet} استفاده شده است. 
\\
\begin{figure}[h]
\centerline{\includegraphics[width=13cm]{images/doubleunet}}
\caption[بلوک دیاگرام معماری \lr{\lr{DoubleUNet}}]{بلوک دیاگرام معماری \lr{\lr{DoubleUNet}}\cite{jha2020doubleu}}
\label{fig:doubleunet}
\end{figure}
\\
در این شبکه، دو ماژول به نام \lr{S\&E} و \lr{ASPP} به همراه یک رویکرد نظارت میانی مورد استفاده قرار گرفته‌اند تا توانایی شبکه را تقویت کنند. ماژول \lr{S\&E} به شبکه کمک می‌کند تا ویژگی‌های مهم تر را از بین ویژگی‌های مختلفی که استخراج می‌کند تشخیص دهد، و این بهبود در عملکرد شبکه را به دنبال دارد. ماژول \lr{ASPP} نیز به شبکه در یادگیری ویژگی‌ها در مقیاس‌های مختلف کمک می‌کند و تاثیر منفی تغییر ابعاد شئ مورد نظر را کاهش می‌دهد. روش نظارت میانی هم برای تمرکز بیشتر بر آموزش اجزای مختلف این شبکه و بهره‌برداری بهتر از قابلیت‌های آن از آن استفاده می‌کند. 
\\
\begin{figure}[h]
\centerline{\includegraphics[width=16cm]{images/se}}
\caption[معماری \lr{S\&E}]{معماری \lr{S\&E} \cite{hu2018squeeze}}
\label{fig:se}
\end{figure}
\\
بلوک \lr{S\&E}\LTRfootnote{Squeeze-and-Excitation} یک واحد معماری است که رابطه کانالی را در شبکه‌های عصبی پیچشی افزایش می‌دهد. این کار با مدل‌سازی صریح وابستگی‌های متقابل بین کانال‌های ویژگی‌های پیچشی آن، به شبکه اجازه می‌دهد تا مجدداً \persianfootnote{درجه‌بندی}\LTRfootnote{Calibration} ویژگی‌ها را انجام دهد. این نتیجه از طریق \persianfootnote{سازوکار}\LTRfootnote{Mechanism} به دست می‌آید که شبکه را قادر می‌سازد تا از اطلاعات سراسری برای تأکید انتخابی بر ویژگی‌های اطلاعاتی و سرکوب موارد کمتر مفید استفاده کند. بلوک \lr{S\&E} از یک عملیات فشرده سازی تشکیل شده است که نقشه های ویژگی را در ابعاد مکانی جمع می کند تا یک توصیف کننده کانال تولید کند، به دنبال آن یک عملیات تحریک که بر تحریک هر کانال بر اساس فعال سازی های خاص نمونه که توسط سازوکار \persianfootnote{خود دروازه ای}\LTRfootnote{Self Gating} آموخته شده است، کنترل می کند. سپس خروجی بلوک \lr{S\&E} دوباره وزن می شود تا خروجی نهایی تولید شود که می تواند مستقیماً به لایه های بعدی وارد شود\cite{hu2018squeeze}
\\
\begin{figure}[h]
\centerline{\includegraphics[width=16cm]{images/aspp}}
\caption[معماری \lr{ASPP}]{ معماری \lr{ASPP} \cite{chen2017deeplab}}
\label{fig:aspp}
\end{figure}
\\
\lr{ASPP}\LTRfootnote{Atrous Spatial Pyramid Pooling} روشی است که برای قطعه‌بندی اشیا در مقیاس های چندگانه استفاده می شود. این ماژول با کاوش یک لایه ویژگی پیچشی ورودی با فیلترهایی با نرخ نمونه‌گیری چندگانه و میدان‌های دید مؤثر کار می‌کند، که به آن اجازه می‌دهد اشیا و همچنین زمینه تصویر را در مقیاس‌های مختلف ثبت کند. این امر با استفاده از چندین لایه پیچشی آتروس موازی با نرخ‌های نمونه‌گیری متفاوت به دست می‌آید، که اساساً فیلترهایی با سوراخ‌هایی هستند که به آن‌ها اجازه می‌دهد تا زمینه بیشتری را بدون افزایش تعداد پارامترها یا مقدار محاسبات ثبت کنند. با استفاده از فیلترهای متعدد با میدان دید موثر مکمل، \lr{ASPP} می‌تواند اشیاء و زمینه تصویر را در مقیاس‌های مختلف ثبت کند، که برای قطعه‌بندی \persianfootnote{قوی}\LTRfootnote{Robust} اشیاء مهم است\cite{chen2017deeplab}.
\\
این معماری قوی تر و قابل تعمیم تر در کاربردهای مختلف پزشکی نسبت به مدل های قبلی است. با ترکیب این روش‌ها، این معماری توانسته است در زمینه قطعه‌بندی تصاویر پزشکی از روش‌های موجود در آن زمان بهتر عمل کند\cite{jha2020doubleu}.
\\

% \subsection{ شبکه \lr{Res-UNet}}
% یکی از اعضای خانواده \lr{UNet} که در قطعه‌بندی تصاویر پزشکی به طور ویژه مورد توجه قرار گرفته، شبکه \lr{Res-UNet} است. \lr{Res-UNet} از نتیجه‌های شبکه‌های \persianfootnote{باقی‌مانده}\LTRfootnote{Residual} در معماری \lr{UNet} بهره می‌برد. در شکل \ref{fig:resunet}، معماری \lr{Res-UNet} نمایش داده شده است. 
% \begin{figure}[h]
% \centerline{\includegraphics[width=16cm]{images/resunet}}
% \caption[معماری \lr{Res-UNet}]{الف - ساختار \lr{Resunet} ب-بلاک \lr{Resunet} ج-بلاک \lr{PSP pooling}\cite{diakogiannis2020resunet}}
% \label{fig:resunet}
% \end{figure}
% \\
% یک نکته مهم این است که این شبکه، با تعداد پارامترهای کمتر نسبت به دیگر اعضای خانواده‌اش، عملکرد بهتری دارد، و این موضوع به دلایل زیر است:
% \begin{enumerate}
%     \item برای آموزش تمام لایه‌ها با افزایش عمق شبکه، بلوک‌های معماری \lr{UNet} با بلوک‌های باقیمانده با لایه‌های پیچشی اصلاح‌شده جایگزین شده‌اند. بلوک‌های باقیمانده به طور قابل‌توجهی مشکل ناپدید شدن گرادیان که در معماری‌های عمیق وجود دارد، را حل می‌کنند.
%     \item منطق استفاده از این لایه‌های چندمقیاسی (\lr{psp pooling})، استخراج ویژگی‌های اشیاء در مقیاس‌های مختلف است. این بلوک باعث بهبود عملکرد ‌قطعه‌بندی اشیاء مبتنی بر شناسایی ارتباطات بین اشیاء در مکان‌های مختلف تصویر و اندازه‌های متنوع می‌شود.
%     \item ادغام اطلاعات زمینه تصویر اصلی و منتقل کردن آن به لایه آخر، از ازدست‌رفتن حداقل اطلاعات اولیه جلوگیری می‌کند و باعث افزایش بهره‌وری شبکه می‌شود.
%     \item علاوه بر معماری استاندارد \lr{UNet} که دارای یک لایه ماسک ناحیه‌بندی واحد به عنوان خروجی است، در \lr{Res-UNet} دو بلوک به انتها اضافه شده که منجر به یادگیری چند وظیفه‌ای می‌شود. این الگوریتم به طور همزمان سه وظیفه مکمل را یاد می‌گیرد. اولین وظیفه مربوط به ماسک ‌قطعه‌بندی است. دومین وظیفه یافتن مرزهای مشترک بین ماسک‌های ‌قطعه‌بندی است که به بهبود عملکرد قطعه‌بندی معنایی کمک می‌کند و سومین وظیفه استفاده از معیار تبدیل فاصله برای ماسک ‌قطعه‌بندی است\cite{diakogiannis2020resunet}.
% \end{enumerate}

\subsection{ شبکه \lr{dResUnet}}
\lr{dResUNet} یک مدل مبتنی بر یادگیری عمیق است که برای قطعه‌بندی خودکار تومور مغز سه بعدی از تصاویر \lr{MRI} چندوجهی پیشنهاد شده است. \lr{dResUNet} در مقاله \cite{raza2023dresu}توسط \lr{Raza}و همکاران پیشنهاد شد. \lr{dResUNet} یک مدل ترکیبی است که معماری \lr{UNet} را با بلوک‌های باقیمانده در بخش کدگذار ترکیب می‌کند تا ویژگی‌های سطح پایین را حفظ کرده و آنها را با استفاده از اتصالات میانبر تطبیقی به سطح کدگشای مربوطه منتقل کند. این مدل از دو مسیر موازی به نام‌های کدگذار و کدگشا تشکیل شده است. در بخش کدگذار، بلوک‌های پیچشی باقیمانده برای بهره‌برداری از اتصالات میانبر برای پیش‌بینی ماسک‌های قطعه‌بندی استفاده می‌شوند. بخش کدگشا برای بازسازی تصویر اصلی از ویژگی های کدگذاری شده استفاده می شود. مدل پیشنهادی برای بهبود فرآیند کلی آموزش و غلبه بر مشکل ناپدید شدن گرادیان طراحی شده است. معماری این مدل را در شکل \ref{fig:dresunet} مشاهده می‌کنید. بلوک باقیمانده در معماری \lr{dResUNet} یادگیری مدل را با استفاده از اتصالات میانبر افزایش می دهد. این سازوکار ویژگی های محلی را حفظ می کند و به غلبه بر مشکل محو شدگی‌گرادیان کمک می کند. 
\begin{figure}[h]
\centerline{\includegraphics[width=13cm]{images/dresunet}}
\caption[\hspace{0.5em}معماری \lr{dResUNet}]{معماری \lr{dResUNet}. بالا) معماری پیشنهادی \lr{dResUNet} سه بعدی. پایین) بلوک باقیمانده با اتصال همانی میانبر.\cite{raza2023dresu}}
\label{fig:dresunet}
\end{figure}

مدل پیشنهادی \lr{dResUNet} بر روی مجموعه \lr{BraTS 2020} آموزش و ارزیابی شده است.  برای غلبه بر چالش مشکل عدم تعادل رده، این مقاله از ترکیب تابع ضرر کانونی و تابع ضرر\lr{Dice} استفاده کرده است. تابع ضرر ترکیبی به دو دلیل استفاده می‌شود: تابع ضرر \lr{Dice} برای دستیابی به حداکثر همپوشانی بین خروجی پیش‌بینی‌شده و ماسک درستی مرجع صرف نظر از رده استفاده می‌شود. تابع ضرر کانونی برای کم کردن سهم رده آسان و توجه بیشتر به رده های چالش برانگیز استفاده می شود. همچنین سلول های بافتی را با توجه به رده های اختصاص داده شده رده‌بندی می کند.

مدل پیشنهادی \lr{dResUNet} با استفاده از چارچوب یادگیری عمیق \lr{PyTorch} پیاده‌سازی شده است. این مدل بر روی یک پردازنده گرافیکی \lr{NVIDIA GeForce RTX 2080 Ti} با اندازه دسته  4 و نرخ یادگیری 0.0001 آموزش داده شده است. فرآیند آموزش برای 100 دوره انجام شده و مدل با بهترین عملکرد اعتبارسنجی برای آزمایش انتخاب شده است. عملکرد این مدل روی مجموعه داده ذکر شده را در می توانید در جدول \ref{tab:dresunet_result} مشاهده کنید.
\begin{table}[ht]
\caption[نتایج کمی آموزش و آزمایش معماری \lr{dResUNet} بر روی مجموعه داده\lr{BraTS 2020}]{نتایج کمی آموزش و آزمایش معماری \lr{dResUNet} بر روی مجموعه داده\lr{BraTS 2020}\cite{raza2023dresu}}
\label{tab:dresunet_result}
\centering
\onehalfspacing
\begin{tabular}{|c|c|c|c|c|}
\hline
نتایج روی \lr{BraTS 2020} & معیار & \lr{TC} & \lr{WT} & \lr{ET} \\
\hline
آموزش & \lr{Dice} & $0.9212$ & $0.9212$ & $0.8097$ \\
\hline
آزمایش & \lr{Dice} & $0.8357$ & $0.8660$ & $0.8004$ \\
\hline
\end{tabular}
\end{table}
چند نمونه از نتایج \lr{dResUNet} را می توانید در شکل  \ref{fig:dresunet_result} ببینید. مقایسه مدل \lr{dResUNet} با سایر روش ها در جدول \ref{tab:dresunet_comparison} قابل مشاهده است.
\begin{table}[ht]
\caption[\hspace{0.5em}مقایسه مدل \lr{dResUNet} با سایر روش ها]{مقایسه مدل \lr{dResUNet} با سایر روش ها\cite{raza2023dresu}}
\label{tab:dresunet_comparison}
\centering
\onehalfspacing
\begin{tabular}{|c|c|c|c|c|c|}
\hline
معماری & ابعاد تصویر ورودی & مجموعه دادگان & \lr{TC(Dice)} & \lr{WT(Dice)} & \lr{ET(Dice)}\\
\hline
\lr{TransBTS} & $128 \times 128 \times 128$ & \lr{BraTS 2020} & $0.8173$ & $0.9009$ & $0.7873$ \\
\hline
\lr{3D UNet} & $128 \times 128 \times 128$ & \lr{BraTS 2020} & $0.7906$ & $0.8411$ & $0.6876$ \\
\hline
\lr{Modified 3D UNet} & $192 \times 160 \times 108$ & \lr{BraTS 2020} & $0.7520$ & $0.8068$ & $0.6959$ \\
\hline
\lr{3D encoder-decoder based V-Net} & $64 \times 64 \times 64$ & \lr{BraTS 2020} & $0.7526$ & $0.8463$ & $0.6215$ \\
\hline
\lr{Lightweight 3D UNet} & $128 \times 128 \times 128$ & \lr{BraTS 2020} & $0.82$ & $0.90$ & $0.78$ \\
\hline
\lr{dResUNet} & $128 \times 128 \times 128$ & \lr{BraTS 2020} & $0.8357$ & $0.8660$ & $0.8004$ \\
\hline
\end{tabular}
\end{table}


\begin{figure}[ht]
\centerline{\includegraphics[width=10cm]{images/dresunet_result.pdf}}
\caption[\hspace{0.5em}معماری \lr{dResUNet}]{چند نمونه از نتایج کیفی \lr{dResUNet} روی مجموعه داده \lr{BraTS} از نمای محوری\cite{raza2023dresu}}
\label{fig:dresunet_result}
\end{figure}


\subsection{ شبکه \lr{ZNet}}
شبکه \lr{ZNet}\cite{ottom2022znet} برای قطعه‌بندی تومور مغزی \lr{MRI} دو بعدی، در سال ۲۰۲۲ معرفی شد و از معماری شبکه عصبی پیچشی بر اساس مدل \lr{UNet} استفاده می‌کند. معماری \lr{ZNet} به طور خاص برای تجزیه و تحلیل تصاویر پزشکی طراحی شده است و برای شناسایی تومورهای مغزی از تصاویر \lr{MRI}بهینه شده است.
\\
معماری \lr{ZNet} از یک سری لایه های پیچشی و ادغام تشکیل شده است که ویژگی ها را از تصویر ورودی استخراج می کند و به دنبال آن یک لایه کاملا متصل که طبقه بندی نهایی را انجام می دهد. این معماری همچنین شامل اتصالات میانبر است که به شبکه اجازه می دهد هر دو ویژگی محلی و سراسری تصویر ورودی را ثبت کند. معماری این شبکه در شکل \ref{fig:ZNet} نشان داده شده است.
\\
\begin{figure}[ht]
\centerline{\includegraphics[width=8cm]{images/ZNet}}
\caption[\hspace{0.5em}معماری \lr{ZNet}]{معماری \lr{ZNet}\cite{ottom2022znet}}
\label{fig:ZNet}
\end{figure}
مجموعه داده مورد استفاده در \cite{ottom2022znet} یک مجموعه داده از \persianfootnote{گلیومای درجه پایین اطلس ژنوم سرطان}\LTRfootnote{The Cancer Genome Atlas Low-Grade Glioma } (\lr{TCGA - LGG}) است. ایده پشت گردآوری \lr{TCGA – LGG} ساخت تصاویر سرطان برای اهداف تحقیقاتی و مطالعه رابطه بین فنوتیپ و \persianfootnote{نوع ژنوم}\LTRfootnote{Genotype} در تحقیقات سرطان و تصاویر پزشکی است. ژنوتیپ ترکیب ژنتیکی یک موجود زنده است، در حالی که فنوتیپ ویژگی های قابل مشاهده آن است که تحت تأثیر ژن ها و محیط قرار می گیرد. ژنوتیپ نقشه ژنتیکی است و فنوتیپ همان چیزی است که ما می بینیم. تمرکز آنها بر روی تصاویر \lr{FLAIR}  است که حاوی تصاویر تومور تقویت شده \lr{LGG} است. متخصصان به صورت دستی تصاویر \lr{FLAIR} را برای 110 بیمار بررسی، حاشیه نویسی و برچسب گذاری کردند. تعداد کل برش های تصویر به دست آمده 3929 بود که 2556 برش طبیعی و 1373 غیرطبیعی (تومور) برچسب گذاری شده بود.
\\
معماری \lr{ZNet} را می توان به دو بخش اصلی تقسیم کرد: یک بخش کدگذاری و یک بخش کدگشایی. بخش کدگذاری وظیفه نمونه برداری را بر عهده دارد و از پنج بلوک تشکیل شده است. هر بلوک کدگذار شامل پیچش های دوگانه است که با نرمال سازی دسته ای و تابع فعال سازی اصلاح شده \lr{ReLU} ترکیب می شوند. پس از این عملیات، حداکثر ادغام برای کاهش بیشتر ابعاد مکانی اعمال می شود. خروجی هر بلوک کدگذار با ورودی آن بلوک الحاق می شود و از حفظ اطلاعات ارزشمند اطمینان حاصل می کند. شایان ذکر است که ورودی بلوک کدگذار برای مطابقت با ابعاد نقشه ویژگی خروجی بلوک درون یابی می شود.

یکی از نوآوری های کلیدی شبکه \lr{ZNet}، استفاده آن از \persianfootnote{تقویت}\LTRfootnote{Amplification}داده است، که شامل تولید تصاویر \lr{MRI}مصنوعی با اعمال تبدیل های تصادفی به تصاویر اصلی است. این روش به افزایش اندازه مجموعه داده آموزشی و بهبود استحکام شبکه نسبت به تغییرات داده های ورودی کمک می کند.
\\
در طول آموزش، مدل \lr{ZNet} برای 200 دوره با استفاده از بهینه ساز \lr{ADAM}، یک الگوریتم آموزش داده می شود. این مدل بر روی سه کانال تصاویر \lr{MRI}، هر کدام دارای ابعاد 128$\times$128 پیکسل، با اندازه دسته 32 عمل می کند. مقایسه معیارهای ارزیابی بین رویکرد پیشنهادی و الگوریتم معیار \lr{UNet} را در جدول \ref{tab:znet} مشاهده می‌کنید.
\begin{table}[ht]
\caption[عملکرد \lr{ZNet}]{مقایسه معیارهای ارزیابی بین رویکرد پیشنهادی و الگوریتم معیار \lr{UNet}\cite{ottom2022znet}}
\label{tab:znet}
\centering
\onehalfspacing
\begin{tabular}{|r|c|c|}
\hline
& \lr{UNet} & \lr{ZNet} \\
\hline
\lr{Dice}آزمون & $0.8544$ & $0.9158$ \\
\hline
پارامترهای قابل یادگیری & $14,788,929$ & $44,384,833$ \\
\hline
\end{tabular}
\end{table}
یکی دیگر از موارد جدید شبکه \lr{ZNet} استفاده از یک چارچوب یادگیری چند وظیفه ای است که شبکه را قادر می سازد همزمان وظایف قطعه‌بندی و طبقه بندی را انجام دهد. این رویکرد به شبکه اجازه می دهد تا نمایش های پیچیده تری از داده های ورودی را بیاموزد و دقت نتایج قطعه‌بندی را بهبود بخشد. نمونه ای از عملکرد این مدل را می توانید در شکل \ref{fig:znet_result} مشاهده کنید\cite{ottom2022znet}.

\begin{figure}[ht]
\centerline{\includegraphics[width=10cm]{images/znet_result.pdf}}
\caption[\hspace{0.5em}نتایج بصری \lr{ZNet}]{نتایج بصری و مقایسه قطعه‌بندی تومور با استفاده از مدل پیشنهادی و مدل \lr{UNet}\cite{ottom2022znet}.}
\label{fig:znet_result}
\end{figure}


\subsection{ شبکه \lr{V-Net}}


\lr{V-Net}، یک شبکه عصبی تمام پیچشی است که توسط \lr{Milletari}و همکاران\cite{shamshad2023transformers} معرفی شد. در سال 2016، به عنوان یک پیشرفت قابل توجه در حوزه قطعه‌بندی تصاویر پزشکی حجمی است. این رویکرد نوآورانه برای رسیدگی به محدودیت‌های روش‌های سنتی قطعه‌بندی \persianfootnote{برش به برش}\LTRfootnote{slice-by-slice}با در نظر گرفتن زمینه سه‌بعدی تصاویر پزشکی طراحی شده است. \lr{V-Net} از چندین جنبه کلیدی با نسخه های قبلی خود متفاوت است، و آن را به یک رویکرد امیدوارکننده برای قطعه‌بندی تصاویر حجمی پزشکی با کارایی بالا تبدیل می کند. معماری این شبکه را در شکل \ref{fig:vnet} می توانید ببینید.
\\
\begin{figure}[ht]
\centerline{\includegraphics[width=10cm]{images/vnet}}
\caption[\hspace{0.5em}معماری \lr{VNet}]{معماری \lr{VNet}\cite{shamshad2023transformers}}
\label{fig:vnet}
\end{figure}
\\
یکی از ویژگی های متمایز \lr{V-Net} توانایی آن در پردازش کل حجم به طور همزمان با در نظر گرفتن زمینه سه بعدی تصویر است. این رویکرد سه بعدی یک مزیت واضح نسبت به روش‌های سنتی برش به برش ارائه می‌دهد، زیرا به شبکه اجازه می‌دهد تا روابط مکانی بین وکسل‌ها را در حجم یاد بگیرد. در نتیجه، این منجر به قطعه‌بندی دقیق‌تر می‌شود، به‌ویژه برای ساختارهای پیچیده‌ای که برش‌های متعددی را در بر می‌گیرند. علاوه بر این، این رویکرد پتانسیل افزایش سرعت قطعه‌بندی را با حذف نیاز به پردازش هر برش به طور مستقل دارد.
\\
اهمیت ترکیب زمینه سه بعدی در قطعه‌بندی تصویر پزشکی در دقت نتایج مشهود است. روش‌های سنتی برش به برش ممکن است روابط بین برش‌های مجاور را در نظر نگیرند که منجر به خطا در قطعه‌بندی می‌شود. در مقابل، روش‌هایی مانند \lr{V-Net} می‌توانند قطعه‌بندی‌های مطمئن‌تری را ارائه دهند، به‌ویژه زمانی که با ساختارهای تشریحی پیچیده‌ای که نیاز به درک جامعی از توزیع مکانی سه‌بعدی آن‌ها دارند، سروکار داریم.
\\
\lr{V-Net} از یک تابع هدف جدید استفاده می کند و آن را از سایر رویکردهای قطعه‌بندی متمایز می کند. این تابع دو جزء اساسی را ترکیب می کند: یک ضریب ضرر \lr{Dice} و تابع ضرر \lr{BCE}\LTRfootnote{Binary Cross Entropy}. ضریب ضرر \lr{Dice} همپوشانی بین قطعه‌بندی پیش‌بینی‌شده و درستی مرجع را اندازه‌گیری می‌کند، در حالی که \lr{BCE} شبکه را تشویق می‌کند تا هم شکل و هم مکان شی قطعه‌بندی شده را بیاموزد.
\\
این ترکیب از توابع ضرر به ویژه برای قطعه‌بندی تصویر پزشکی مفید است، جایی که توزیع‌های وکسل پیش‌زمینه و پس‌زمینه نامتعادل هستند. \lr{V-Net} با ادغام ضریب ضرر \lr{Dice}، می تواند به طور موثر چنین عدم تعادلی را کنترل کند و از قطعه‌بندی دقیق اطمینان حاصل کند حتی زمانی که ناحیه مورد نظر فقط یک منطقه کوچک در اسکن را اشغال کند.
\\
\lr{V-Net} برای افزایش بیشتر عملکرد خود، از روش های \persianfootnote{افزایش داده}\LTRfootnote{Data Augmentation}استفاده می کند. این روشها شامل چرخش‌های تصادفی، مقیاس‌بندی، و تغییر شکل‌های \persianfootnote{ٍکشسان}\LTRfootnote{Elastic}می‌شوند که همگی در هر تکرار آموزشی برای هر دسته کوچکی که به شبکه وارد می‌شود، اعمال می‌شوند.
\begin{itemize}
\item چرخش های تصادفی: چرخش تصادفی حجم حول یک محور تصادفی با یک زاویه تصادفی به شبکه اجازه می دهد تا تشخیص شی از دیدگاه های مختلف را بیاموزد، بنابراین تعمیم آن به داده های جدید را بهبود می بخشد.
\item مقیاس‌سازی: مقیاس‌گذاری تصادفی حجم در امتداد هر محور به شبکه کمک می‌کند تا قطعه‌بندی اشیاء با اندازه‌های مختلف را بیاموزد، در نتیجه تطبیق‌پذیری آن در مدیریت ساختارهای آناتومیکی مختلف افزایش می‌یابد.
\item تغییر شکل های کشسان: با اعمال تغییر شکل های تصادفی بر روی حجم با استفاده از یک میدان جابجایی، شبکه یاد می گیرد که جسم را در حضور تغییر شکل های کوچک که می تواند به دلیل حرکت بیمار یا عوامل دیگر رخ دهد، تشخیص دهد.
\end{itemize}

علاوه بر این، این افزایش‌ها در لحظه قبل از هر تکرار بهینه‌سازی انجام می‌شوند و نیاز ذخیره‌سازی را کاهش می‌دهند. علاوه بر این، \lr{V-Net} توزیع شدت داده‌ها را با استفاده از تطبیق هیستوگرام تطبیق می‌دهد تا اطمینان حاصل شود که توزیع شدت حجم‌های آموزشی با سایر اسکن‌های انتخابی تصادفی در مجموعه داده مطابقت دارد.
\\
استفاده از این روشهای افزایش داده‌ها، \lr{V-Net} را به توانایی سازگاری با شرایط مختلف و بهبود دقت قطعه‌بندی مجهز می‌کند، و آن را به یک انتخاب قانع‌کننده برای قطعه‌بندی تصویر پزشکی حجمی تبدیل می‌کند\cite{shamshad2023transformers}.

\section{شبکه های ترکیبی مبتنی بر مبدل ها}
با توجه به موفقیت‌های بی‌سابقه در حوزه وظایف زبان طبیعی، مبدل‌ها با موفقیت در حل چندین مسئله مرتبط با بینایی‌کامپیوتر نیز به کار گرفته شده و به نتیجه قابل قبولی دست‌یافته‌اند. این پیشرفت‌ها محققان را وادار به بازنگری برتری شبکه‌های عصبی پیچشی به عنوان عملگرهای واقعی کرده‌اند. با بهره‌گیری از این پیشرفت‌ها در حوزه بینایی کامپیوتر، علاقه به استفاده از مبدل‌ها در حوزه تصویربرداری پزشکی نیز به شدت افزایش یافته است.
\\
در شبکه‌های عصبی پیچشی، عملیات پیچش برای استخراج ویژگی‌های محلی از تصویر استفاده می‌شود و به کاهش تأثیرات ناهنجاری‌ها مثل نویز می‌انجامد، اما ناتوان در استخراج روابط و ارتباطات دور بین ویژگی‌ها است. حتی با پیشرفت‌های اخیر در این حوزه، روش‌های موجود هنوز قادر به استخراج ارتباطات دور بین ویژگی‌ها نبودند. به همین دلیل، شبکه‌های مبتنی بر \persianfootnote{توجه}\LTRfootnote{Attention} معرفی شدند.
\\
در مدل‌های مبدل مبتنی بر توجه به دلیل توانایی آن‌ها در کدگذاری ارتباطات دوربرد و یادگیری ارتباط بین ویژگی‌ها که بسیار تاثیر گذار است، به یک گزینه جذاب تبدیل شده‌اند. تحقیقات اخیر نشان می‌دهد که این ماژول‌های مبدل قادرند به طور کامل جایگزین پیچش‌های استاندارد در شبکه‌های عصبی عمیق که بر تصاویر عمل می‌کنند شوند و به ایجاد مدل‌های مبدل تصویر (\lr{ViTs}) منجر شوند. مدل‌های \lr{ViT} نشان داده‌اند که در زمینه‌های متعددی از وظایف بینایی مانند رده‌بندی تصویر، تشخیص اشیاء، و قطعه‌بندی معنایی، به پیشرفت‌های قابل توجهی دست‌یافته است. به علاوه، تحقیقات اخیر نشان داده‌اند که خطاهای پیش‌بینی مدل‌های مبدل تصویر به خطاهای انسانی نزدیک‌تر از مدل‌های شبکه‌های عصبی پیچشی هستند. این ویژگی‌های جذاب مدل‌های مبدل تصویر به جامعه پزشکی انگیزه داده‌اند تا از این مدل‌ها در برنامه‌های تصویربرداری پزشکی استفاده کنند.
\\
در این بخش ابتدا به بیان مقدمه ای راجع به ساختار مبدل های بینایی می‌پردازیم، سپس چند مدل مبتنی بر مبدل را معرفی می‌کنیم.
در ساختار مدل‌های مبدل تصویر (\lr{ViTs})، مبدل به شکل یک سلسله مراتبی از لایه‌های مبدل که به تدریج اطلاعات را آبشاری می‌کنند، برای تمرکز بر یک شی یا نقطه خاص در یک تصویر استفاده می‌شود. این شبکه‌ها دارای قدرت آموزش در مجموعه داده‌های آموزشی بزرگ هستند و همچنین با افزایش ظرفیت محاسباتی ارائه می‌شوند. ویژگی‌های قوی این شبکه‌ها باعث جلب توجه جامعه تصویربرداری پزشکی و انتخاب برخی از رویکردهای جدید شده است.
\\
با توجه به توانایی واحدهای توجه در استخراج روابط بین ویژگی‌ها، محققان به ایده طراحی شبکه‌ها بر اساس این واحدها دست یافتند. ساختار مبدل که شامل یک ساختار کدگذار-کدگشا است، از این ساختار بهره‌برداری می‌کند.
\\
شبکه‌های پیچشی مشابه \lr{UNet} در این نکته مشترک هستند که بخش کدگذار برای کدگذاری اطلاعات و بخش کدگشا برای کدگشایی اطلاعات با استفاده از شبکه‌های پیچشی طراحی شده‌اند. اما تفاوت اصلی اینجاست که به جای استفاده از واحدهای پیچشی در این شبکه‌ها، از واحدهای توجه استفاده می‌شود. معماری مبدل بینایی با تغییرات جزئی، از معماری مبدل برای پردازش تصاویر بهره می‌برد. در واقع، واحدهای توجه در اینجا به جای واحدهای پیچشی به کار گرفته می‌شوند.

\begin{figure}[h]
\centerline{\includegraphics[width=13cm]{images/transformer}}
\caption[\hspace{0.5em}معماری \lr{ViT}]{    تصویر به 9 قسمت تقسیم می شود و هر قسمت پس از پهن شدن به عنوان یک کلمه در نظر گرفته می شود. همانطور که در شکل نشان داده شده است، موقعیت های پیکسل نیز با استفاده از کدگذاری موقعیت به شبکه تزریق می شوند\cite{thisanke2023semantic}.
}
\label{fig:transformer}
\end{figure}
توجه به این نکته ضروری است که تفاوت قابل توجهی بین شبکه های مبتنی بر لایه های پیچشی و شبکه های مبتنی بر توجه این است که در شبکه های مبتنی بر لایه های پیچشی، هدف یافتن ویژگی های مختلف و استفاده از آنها برای استنتاج نهایی است. در مقابل، در شبکه های مبتنی بر توجه، استخراج ویژگی بر اساس روابط بین اجزای تصویر است. همچنین، شایان ذکر است که معماری مبدل بینایی دارای محدودیت‌هایی است که یکی از مهم‌ترین آن‌ها محدودیت در عناصر معنادار است که در وصله‌ها قرار نمی‌گیرند. چالش دیگر محدودیت در انتخاب وصله های کوچک است زیرا این امر تعداد این وصله ها را افزایش می دهد و شبکه را بسیار سنگین می کند\cite{thisanke2023semantic}.
\\
در زمینه پزشکی، ساختار مبدل بینایی برای قطعه‌بندی تصویر محبوبیت پیدا کرده است. در ادامه به بررسی تعدادی از این مدل ها می‌پردازیم.
\\
\subsection{ شبکه \lr{TransUNet}}
\lr{TransUNet} شامل یک کدگذار و یک کدگشا برای کدگذاری و کدگشایی اطلاعات تصویر است که برای قطعه‌بندی تصویر استفاده می شود. تفاوت آن با \lr{UNets} سنتی این است که \lr{TransUNet} از معماری ترکیبی \lr{CNN-Transformer} به عنوان کدگذار برای یادگیری اطلاعات مکانی با وضوح بالا از \lr{CNN} و اطلاعات زمینه کلی از Transformers استفاده می کند.
\\

\begin{figure}[h]
\centerline{\includegraphics[width=13cm]{images/transunet}}
\caption[\hspace{0.5em}معماری \lr{TransUNet}]{معماری \lr{TransUNet}\cite{chen2021transunet}}
\label{fig:transunet}
\end{figure}
یکی از تفاوت‌های اساسی میان \lr{TransUNet} و \lr{UNet} در اینجاست که در انتهای بخش کدگذار (که در شکل\ref{fig:transunet} به صورت بخشی سبز رنگ نمایش داده شده) از لایه‌های مبدل استفاده می‌شود. این استفاده امکان آموزش ارتباطات بین ویژگی‌ها را علاوه بر ویژگی‌هایی که توسط لایه‌های پیچشی(که ویژگی‌های محلی مختلفی را در فواصل مختلف متأثر از همسایگان استخراج می‌کنند) به دست آمده‌اند، فراهم می‌کند\cite{chen2021transunet}.

مدل پیشنهادی در مقاله \cite{chen20233d} یک نسخه 3 بعدی از \lr{TransUNet} است که بر اساس معماری پیشرفته \lr{nnUNet} ساخته شده است. هدف این مدل پیشی گرفتن از استانداردهای تعیین شده \lr{nnUNet} با استفاده از نقاط قوت \lr{UNet} و مبدل ها است.
\\
معماری ۳ بعدی \lr{TransUNet} از سه \persianfootnote{پیکربندی}\LTRfootnote{Configuration}تشکیل شده است: \lr{Encoder-only}، \lr{Decoder-only}و\lr{Full TransUNet}. در پیکربندی \lr{Encoder-only}، یک کدگذار ترکیبی \lr{CNN-Transformer} استفاده می‌شود، جایی که \lr{CNN} ابتدا به عنوان استخراج‌کننده ویژگی برای تولید نقشه ویژگی برای ورودی استفاده می‌شود. جاسازی وصله برای وصله‌های ویژگی به جای تصاویر خام اعمال می‌شود. برای مرحله کدگشایی، از کدگشای استاندارد \lr{UNet} استفاده می شود. در پیکربندی \lr{Decoder-only}، یک کدگذار \lr{CNN} معمولی برای مرحله کدگذاری استفاده می‌شود. در پیکربندی \lr{Full TransUNet}، هم از کدگذار \lr{CNN-Transformer} و هم از کدگشای \lr{Transformer} استفاده می شود. معماری این مدل را در شکل \ref{fig:transunet3d} می توانید مشاهده کنید.

\begin{figure}[h]
\centerline{\includegraphics[width=13cm]{images/transunet3d}}
\caption[\hspace{0.5em}معماری \lr{TransUNet} ۳ بعدی ]{معماری \lr{TransUNet}. کدگذار مبدل که در آن یک کدگذار \lr{CNN}ابتدا برای استخراج ویژگی‌های تصویر محلی استفاده می‌شود و سپس از یک کدگذار مبدل برای تعامل اطلاعات سراسری استفاده می‌شود. و 2) کدگشای مبدل که قطعه‌بندی در هر پیکسل را به عنوان طبقه بندی ماسک با استفاده از پرس و جوهای قابل یادگیری، که از طریق توجه متقابل با ویژگی های \lr{CNN}اصلاح می شوند، مجدداً قاب بندی می کند و از رویکرد پالایش توجه درشت به ریز برای دقت تقسیم بندی افزایش یافته استفاده می کند.\cite{raza2023dresu}}
\label{fig:transunet3d}
\end{figure}
نوآوری \lr{TransUNet}  ۳ بعدی در توانایی آن در مدیریت وابستگی های دوربرد در قطعه‌بندی تصاویر پزشکی با استفاده از نقاط قوت \lr{UNet} و مبدل نهفته است. این مدل از یک تابع ضرر ترکیبی متشکل از تابع ضرر \lr{pixel-wise croos entropy} و تابع ضرر \lr{Dice} استفاده می‌کند. 
\\
مجموعه دادگان مورد استفاده در این مقاله مجموعه دادگان \lr{BraTS2021} است. عملکرد کمی این مدل در جدول \ref{tab:transunet3d_result} قابل مشاهده است. 

\begin{table}[ht]
\caption[مقایسه عملکرد \lr{TransUNet} ۳بعدی روی \lr{BRATS2021}برای قطعه‌بندی تومور مغز با معیار  \lr{Dice} ]{}
\label{tab:transunet3d_result}
\centering
\onehalfspacing
\begin{tabular}{|c|c|c|c|c|}
\hline
مدل & \lr{ET}& \lr{TC} & \lr{WT} & \lr{Dice} میانگین \\
\hline
\lr{nnUNet} & $88.05$ & $91.92$ & $93.79$ & $91.25$ \\
\lr{AxialAttn} & $87.23$ & $91.88$ & $93.21$ & $90.77$ \\
\lr{nnUNet-Large}& $88.23$ & $92.35$ & $93.83$ & $91.47$ \\
\lr{TransUNet}& $88.85$ & $92.48$ & $93.90$ & $\mathbf{91.74}$ \\
\hline
\end{tabular}
\end{table}


\subsection{ شبکه \lr{Swin Transformer}}
شبکه \lr{Swin Transformer} پس از مدل مبدل بینایی معرفی شده و با هدف بهبود آن توسعه یافت. این شبکه با معرفی ساختار سلسله‌مراتبی، مشکلات معنایی و ارتباط‌های میان اجزا را بهبود می‌بخشد. برای این کار، واحد \persianfootnote{ادغام وصله‌}\LTRfootnote{[Patch Merging]}ها را معرفی کرده است. این واحد به ازای هر اندازه‌ی وصله، تغییرات مربوط به وصله‌ها را از چپ به راست اعمال می‌کند.

\begin{figure}[h]
\centerline{\includegraphics[width=13cm]{images/swin_patch}}
\caption[\hspace{0.5em}نحوه چینش وصله ها در \lr{SwinTransformer}]{نحوه چینش وصله‌ها : به ترتیب از الف به د نحوه ترکیب وصله‌ها عوض می‌شود\cite{liu2021swin}}
\label{fig:swin_patch}
\end{figure}
در ادامه تغییرات در واحد \persianfootnote{توجه چندگانه}\LTRfootnote{MSA} ، وزن‌ها به طرز متنوعی برای بخش‌های مختلف تنظیم شدند و خود بخش‌ها به وصله‌های کوچکتر تبدیل شدند. این تغییرات این امکان را ایجاد می‌کنند که از وصله‌های کوچکتر استفاده کنیم. تصویر به چند پنجره تقسیم می‌شود و عملیات توجه تنها روی وصله‌های هر پنجره انجام می‌شود. همانطور که در شکل\ref{fig:siwn_patch2} نشان داده شده است، مقایسه‌ای بین \lr{MSA} و \lr{W-MAS} ارائه شده است. یک نکته مهم این است که تقسیم به پنجره باعث می‌شود که ارتباط بین ‌وصله‌های مختلف در پنجره‌ها در نظر گرفته نشود. برای حل این مشکل، از وصله‌های لغزان استفاده می‌شود\cite{liu2021swin}.

\begin{figure}[h]
\centerline{\includegraphics[width=13cm]{images/swin_patch2}}
\caption[\hspace{0.5em}وصله ها در \lr{Swin Transformer}]{    الف-نحوه چینش وصله‌ها در \lr{MSA} .ب-نحوه چینش وصله‌ها در \lr{W-MAS} .ج- نحوه لغزش پردازش وصله‌ها و برقراری ارتباط بین آنها د-نحوه قرارگیری ریز وصله ها\cite{liu2021swin}}
\begin{center}
\end{center}
\label{fig:swin_patch2}
\end{figure}

\subsection{ شبکه \lr{SwinBTS}}
یکی از روش‌هایی که با بهره‌گیری از \lr{Swin} به ناحیه‌بندی تومور مغز پرداخته است، شبکه \lr{SwinBTS} است.
همانند بسیاری از شبکه‌های مورد استفاده در قطعه‌بندی تصاویر پزشکی، این شبکه نیز دارای یک ساختار مشابه \lr{UNet} است. با این حال، تفاوت آن در این است که در ساختار \lr{SwinBTS} در هر لایه از هم‌بسته‌های \lr{Swin} و هم پیچشی‌ها استفاده شده است.

\begin{figure}[h]
\centerline{\includegraphics[width=13cm]{images/swinbts}}
\caption[\hspace{0.5em}ساختار \lr{SwinBTS}]{ساختار \lr{SwinBTS}\cite{jiang2022swinbts}}
\label{fig:swinbts}
\end{figure}
همانگونه که مشاهده می‌شود، در بخش \lr{Swin} از واحدهای \lr{W-MSA} و \lr{SW-MSA} استفاده شده است. به علت ساختار پیچیده این شبکه، نیاز به توان پردازشی بالا برای آموزش دارد\cite{jiang2022swinbts}. عملکرد این مدل روی مجموعه دادگان \lr{BraTS 2019} را می توانید رد جدول \ref{tab:swinbts_result} مشاهده کنید.

\begin{table}[ht]
\caption[ عملکرد مدل \lr{SwinBTS}]{ عملکرد مدل \lr{SwinBTS} روی مجموعه دادگان \lr{BraTS 2019}}
\label{swinbts_result}
\centering
\onehalfspacing
\begin{tabular}{|c|c|c|c|c|}
\hline
\multirow{2}{*}{Method} & \multicolumn{4}{c}{Dice Score (\%)} \\
\cline{2-5}
 & \lr{ET} & \lr{TC} & \lr{WT} & \lr{AVG} \\
\hline
\lr{3D UNet} & $66.15 \pm 0.339$ & $66.94 \pm 0.322$ & $86.89 \pm 0.071$ & $73.33$ \\
\lr{Attention UNet} & $67.06 \pm 0.327$ & $71.95 \pm 0.264$ & $86.69 \pm 0.100$ & $75.23$ \\
\lr{UNetR} &  $67.19 \pm 0.346$ & $74.39 \pm 0.256$ & $88.57 \pm 0.122$ & 76.72 \\
\lr{TransBTS}  & $71.08 \pm 0.347$ & $78.67 \pm 0.207$ & $89.75 \pm 0.070$ & $79.83$ \\
\lr{VTU-Net} & $73.53 \pm 0.311$ & $78.09 \pm 0.242$ & $89.56 \pm 0.089$ & $80.39$ \\
\lr{SwinBTS} & $74.43 \pm 0.294$ & $79.28 \pm 0.232$ & $89.75 \pm 0.070$ & $81.15$ \\
\hline
\end{tabular}
\end{table}

\subsection{ شبکه \lr{UNetFormer}}
یکی از تفاوت‌های اساسی مدل \lr{UNetFormer} نسبت به خانواده‌های \lr{UNet} این است که در سه بخش کدگذار، گلوگاه و کدگشا، به جای استفاده از لایه‌های پیچشیی برای استخراج ویژگی‌های محلی، از بلوک‌های مبدل استفاده می‌شود. علاوه بر این، برای کاهش و افزایش ابعاد لایه‌ها از بلوک‌های ادغام و \persianfootnote{افکنش وصله}\LTRfootnote{Patch Projection} استفاده می‌شود که در واقع یک لایه \lr{MLP} است.

\begin{figure}[h]
\centerline{\includegraphics[width=13cm]{images/unetformer}}
\caption[\hspace{0.5em}معماری \lr{UNetFormer}]{معماری \lr{UNetFormer}\cite{hatamizadeh2022unetformer}}
\label{fig:unetformer}
\end{figure}
همچنین لازم به ذکر است که این شبکه دارای سه خروجی مختلف است که بخشی از آنها در امتداد یکدیگر قرار دارند. این امر به دلیل این است که مجموعه داده قطعه‌بندی تومور \lr{BRATS} دارای سه برچسب به نام‌های \lr{whole}، \lr{core} و \lr{enhancing} است، بنابراین برای هرکدام از این برچسب‌ها یک خروجی جداگانه در نظر گرفته شده است.
ویژگی های کلیدی مدل \lr{UNetFormer} برای قطعه‌بندی تصاویر پزشکی سه بعدی شامل ترکیب کدگذار مبتنی بر مبدل سه بعدی \lr{Swin} و شبکه عصبی پیچشی و کدگشاهای مبتنی بر مبدل است. این ترکیب امکان ثبت جزئیات ریز ساختارهای تومور را در تصاویر پزشکی فراهم می کند. علاوه بر این، این مدل یک چارچوب یکپارچه مبدل بینایی و یک روش پیش‌آموزشی برای ستون فقرات کدگذار ارائه می‌کند که طیف گسترده‌ای از الزامات مبادله بین دقت و هزینه محاسباتی را ممکن می‌سازد\cite{hatamizadeh2022unetformer}.

\begin{table}[ht]
\caption[\hspace{0.5em}مقایسه \lr{UNetFormer} با سایر روش ها]{مقایسه \lr{UNetFormer} با سایر روش ها\cite{hatamizadeh2022unetformer}}
\label{tab:unetformer}
\centering
\onehalfspacing
\begin{tabular}{|c|c|c|c|c|}
\hline
& \multicolumn{4}{|c|}{\lr{Dice}}  \\
\hline
مدل & \lr{ET} & \lr{WT} & \lr{TC} & \lr{Avg}  \\
\hline
\lr{TransBTS} & $86.60$ & $90.30$ & $89.81$ & $88.91$ \\
\hline
\lr{nnFormer} & $86.87$ & $92.68$ & $90.15$ & $89.90$ \\
\hline
\lr{SegResNet} & $88.40$ & $92.70$ & $91.70$ & $90.90$ \\
\hline
\lr{nnUNet} & $88.60$ & $92.91$ & $91.40$ & $91.01$ \\
\hline
\lr{UNetFormer+} & $88.48$ & $\mathbf{93.67}$ & $91.89$ & $91.20$  \\
\hline
\lr{UNetFormer} & $\mathbf{88.80}$ & $93.22$ & $\mathbf{92.1}$ & $\mathbf{91.54}$  \\
\hline
\end{tabular}
\end{table}


\subsection{ شبکه \lr{UNetR}}

\lr{UNetR} یک معماری نوآورانه برای قطعه‌بندی معنایی تصاویر پزشکی حجیم ارائه می‌دهد که از ترکیبی از مبدل ها و شبکه‌های عصبی پیچشی استفاده می‌کند. این رویکرد نوآورانه وظیفه قطعه‌بندی سه بعدی را به عنوان یک مسئله پیش‌بینی دنباله به دنباله یک بعدی بازفرموله می‌کند. در این معماری از یک مبدل به عنوان کدگذار برای یادگیری اطلاعات زمینه‌ای از وصله‌های ورودی استفاده می‌شود. این بازنمای‌های استخراج‌شده از کدگذار از طریق اتصال‌های میانبر در چندین وضوح با کدگشا مبتنی بر \lr{CNN}ترکیب می‌شوند تا پیش‌بینی خروجی‌های قطعه‌بندی ممکن شود.
\\
معماری \lr{UNetR} از دو عنصر اصلی تشکیل شده است: کدگذار از نوع مبدل و کدگشا مبتنی بر \lr{CNN}. کدگذار مسئول بدست‌آوردن نمایش‌های زمینه‌ای سراسری با پردازش وصله‌های ورودی است. این کدگذار شامل یک پشته از \lr{N}لایه‌ی یکسان است، هر کدام شامل یک سازوکار \persianfootnote{خود-توجه چندسر}\LTRfootnote{multi-head self-attention}و یک شبکه \persianfootnote{پیش‌رو}\LTRfootnote{feed-forward}است. سازوکار خود-توجه چندسر به مدل امکان افزایش ویژگی‌های بازه‌های بلند و به دست‌آوردن نمایش‌های زمینه‌ای سراسری در مقیاس‌های مختلف را می‌دهد. به علاوه، شبکه پیش‌رو تبدیل غیرخطی به خروجی سازوکار خود-توجه اعمال می‌کند.
\\
دیگر طرف کدگشا مبتنی بر \lr{CNN}شامل یک پشته از \lr{N} لایه‌ی یکسان است، هر کدام شامل یک لایه پیچشی سه بعدی، یک لایه \persianfootnote{نرمال‌سازی دسته}\LTRfootnote{Batch Normalization}و تابع فعال‌سازی \lr{ReLU} است. این کدگشا به تدریج تصاویر استخراج‌شده را به وضوح ورودی افزایش می‌دهد تا پیش‌بینی معنایی در مقیاس پیکسل/وکسل را انجام دهد. اتصالات میانبر اصلی خروجی کدگذار و کدگشا را در وضوح‌های مختلف ادغام می‌کنند، که امکان بازیابی اطلاعات مکانی که در فرآیند کاهش اندازه از دست می‌رود را فراهم می‌کند.
\\
همانطور که در شکل \ref{fig:unetr}نشان داده شده است، معماری \lr{UNetR} یک حجم ورودی سه بعدی را به یک دنباله از وصله های یکنواخت بدون تداخل تبدیل می‌کند. این وصله ها سپس از طریق یک لایه خطی به یک \persianfootnote{فضای جاسازی‌سازی}\LTRfootnote{Embedding Space}افکنده می‌شوند و یک جاسازی‌سازی مکانی یک‌بعدی قابل یادگیری به جاسازی‌سازی نگاشت شده اضافه می‌شود تا اطلاعات مکانی وصله های استخراج‌شده را حفظ کند.
\begin{figure}[h]
\centerline{\includegraphics[width=13cm]{images/unetr}}
\caption[\hspace{0.5em}معماری \lr{UNetR}]{    یک حجم ورودی سه بعدی (به عنوان مثال $C = 4 $کانال برای تصاویر \lr{MRI})، به دنباله ای از وصله های یکنواخت غیر همپوشانی تقسیم می شود و با استفاده از یک لایه خطی به فضای جاسازی شده نمایش داده می شود. دنباله با \persianfootnote{جاسازی موقعیت}\LTRfootnote{Positional Embedding} اضافه می شود و به عنوان ورودی به مدل مبدل استفاده می شود. نمایش های کدگذاری شده لایه های مختلف در مبدل استخراج شده و با یک کدگشا از طریق اتصالات میانبر ادغام می شوند تا قطعه‌بندی نهایی را پیش بینی کنند. اندازه خروجی برای وضوح وصله $P = 16$ و اندازه جاسازی $K = 768$ داده شده است\cite{hatamizadeh2022unetr}.
}
\label{fig:unetr}
\end{figure}

عملکرد \lr{UNetR} از طریق موفقیت در وظایف مختلف قطعه‌بندی حجیم در مدل‌های \lr{CT}و \lr{MRI}به خوبی نمایان می‌شود. \lr{UNetR} به عنوان یک معماری‌ نوآورانه در رقابت‌های استاندارد و رقابت‌های آزاد در جدول رتبه‌بندی \lr{Beyond the Cranial Vault (BTCV)}برای قطعه‌بندی چند ارگان بهترین عملکرد را ارائه داده است. باید توجه داشت که در قطعه‌بندی تومور مغز و طحال در مجموعه دادگان \lr{Medical Segmentation Decathlon (MSD)} نیز  از رقبا پیشی گرفته است.
\\
بیشتر از این، نکته‌ای قابل توجه در معماری \lr{UNetR} این است که این عملکرد را با پیچیدگی مدل متوسط به ارمغان می‌آورد. همچنین، دارای دومین زمان‌گیری متوسط بعد از \lr{nnUNet}است، که کارایی آن را نشان می‌دهد. در مقایسه با سایر مدل های مبتنی بر مبدل به طور قابل توجهی عملکرد آنها در زمینه سرعت پیش‌بینی را با چالش مواجه می‌کند\cite{hatamizadeh2022unetr}.

% در مقایسه با سایر مدل‌های مبتنی بر مبدل مانند \lr{SETR}، \lr{TransUNet}و \lr{CoTr}، \lr{UNetR} به طور قابل توجهی عملکرد آن‌ها را در زمینه سرعت پیش‌بینی با چالش مواجه می‌کند\cite{hatamizadeh2022unetr}.
% \\

\section{یادگیری فعال}
با ظهور هوش مصنوعی، دانشمندان به سوی استفاده از این الگوریتم‌ها برای کاهش هزینه‌ها و افزایش سهولت کار حرکت کردند. با پیشرفت شبکه‌های عمیق، اکثر روش‌های اخیر تلاش کرده‌اند تا بخش عمده‌ای از مسائل را با استفاده از این روش‌ها حل کنند. در حوزه پزشکی، دقت و عملکرد الگوریتم‌ها بسیار حیاتی است، چرا که نتایج مستقیماً با جان انسان‌ها در ارتباطند. با توجه به این نکته، چالش‌هایی در پردازش تصاویر وجود دارد که محققان تلاش کرده‌اند راه‌حل‌های مختلفی برای آن‌ها ارائه دهند. از مهم‌ترین چالش‌های این حوزه می‌توان به کمبود داده، اهمیت دقت نهایی الگوریتم و تفسیر پذیری الگوریتم اشاره کرد. چالش کمبود داده از آن جهت مهم است که تولید داده‌ها ممکن است بر اساس دستگاه‌های تصویربرداری مختلف متفاوت باشد و حتی بیشتر از آن، بر اساس نظر اختصاصی متخصصان، روش‌های برچسب‌گذاری متفاوتی داشته باشد. از این رو، چالشی وجود دارد که با توجه به داده‌های کم، استفاده از الگوریتم‌های با نظارت سنتی ممکن نخواهد بود و نیاز به استفاده از الگوریتم‌های جایگزین خواهد بود. چالش دوم، دستیابی به دقت نهایی قابل قبول در الگوریتم‌های مرتبط با حوزه پزشکی است. همان‌گونه که اشاره شد، با توجه به اهمیت این حوزه برای انسان‌ها، الگوریتم‌ها باید عملکرد قابل اطمینانی ارائه دهند، چرا که حتی کوچکترین اشتباه در این حوزه ممکن است منجر به فاجعه شود. چالش سوم، تفسیر‌پذیری الگوریتم‌هاست. الگوریتم‌های انتخابی باید به گونه‌ای باشند که قابلیت تفسیر پذیری داشته باشند تا بتوان به آن‌ها اعتماد کرد. مهم است بدانید که دخالت انسان در الگوریتم‌های هوش مصنوعی می‌تواند مفید باشد و به همین دلیل تلاش شده است تا از این ایده در حل چالش‌های مذکور بهره گرفته شود.
\\
\persianfootnote{یادگیری فعال}\LTRfootnote{Active Learning} یک حوزه مهم در هوش مصنوعی است که از ایده تعامل میان انسان و هوش مصنوعی بهره می‌برد. ایده اصلی در این روش این است که اگر الگوریتم به انتخاب داده‌هایی که برای آموزش استفاده می‌شود، اختیار داشته باشد، می‌تواند با داده‌های کمتر عملکرد بهتری داشته باشد. برای مثال، در شکل زیر، هر دو روش با تعداد دادگان محدود آموزش دیده‌اند. اما تفاوت آنها در این است که داده‌های شکل ج توسط الگوریتم انتخاب شده‌اند و نتیجه‌ی عملکرد بهتری نسبت به داده‌های شکل ب که به صورت تصادفی انتخاب شده‌اند، دارند.

\begin{figure}[h]
\centerline{\includegraphics[width=13cm]{images/active_learning}}
\caption[\hspace{0.5em}دادگان یادگیری فعال]{دادگان یادگیری فعال\cite{zhao2021dsal}}
\label{fig:active_learning}
\end{figure}
شکل \ref{fig:active_learning} یک نمونه از عملکرد یادگیری فعال را نشان می‌دهد. در شکل (الف)، یک مجموعه داده در فضای دو بعدی ترسیم شده است. در شکل (ب)، تنها یک تعداد محدودی از نمونه‌ها (30٪ به صورت تصادفی) برای آموزش مدل استفاده شده‌اند و نتیجهٔ مدل با یک خط نمایان‌دهنده مرز تصمیم‌گیری رده‌بندی آبی رنگ است. این مدل روی تمام نمونه‌ها اعمال می‌شود و نمونه‌هایی که به اشتباه رده‌بندی شده‌اند، جداگانه مشخص می‌شوند. در نهایت، مدل با نمونه‌های اولیه که برای آموزش انتخاب شده‌اند و نمونه‌هایی که اشتباه رده‌بندی شده‌اند، دوباره آموزش داده می‌شود و نتیجه در شکل (ج) نمایش داده شده است.
\\
استفاده از این روش مناسب است در مواقعی که دادگان آموزشی کمی در دسترس است یا وقتی که دادگان زیادی موجود هستند اما به دلایل مختلفی نظیر کندی فرآیند برچسب‌زنی یا دشواری ارائه برچسب، تعیین دسته‌بندی نمونه‌ها مشکل است. در این روش از منبع خارجی(\lr{Oracle})  به عنوان منبع مطمئن برای برچسب‌گذاری نمونه‌ها استفاده می‌شود. به طور کلی، عملکرد کلی این الگوریتم را در شکل \ref{fig:active_learning2} می‌توان مشاهده کرد.

\begin{figure}[h]
\centerline{\includegraphics[width=13cm]{images/active_learning2}}
\caption[\hspace{0.5em}فرایند کلی یادگیری فعال]{فرایند کلی یادگیری فعال\cite{zhao2021dsal}}
\label{fig:active_learning2}
\end{figure}
در این روش، ابتدا یک مدل با استفاده از داده‌های آموزشی که دارای برچسب هستند، آموزش داده می‌شود. سپس با استفاده از این مدل، داده‌هایی که انتخاب می‌شوند و به عنوان \persianfootnote{پرس و چو}\LTRfootnote{Query} مورد استفاده قرار می‌گیرند، برای برچسب‌گذاری به \lr{Oracle} ارسال می‌شوند. \lr{Oracle} به عنوان یک منبع مطمئن برای برچسب‌زنی عمل می‌کند و برچسب‌های درست برای این نمونه‌ها ارائه می‌دهد. این حلقه تکرار می‌شود تا مدل با تعداد کافی از داده‌ها آموزش ببیند یا تا به تعداد مشخصی از برچسب‌ها دست پیدا کنیم. این روش به مدل کمک می‌کند تا با تعداد کمتری از داده‌ها و برچسب‌ها عملکرد بهتری داشته باشد و از دخالت انسانی برای برچسب‌زنی داده‌ها کاسته می‌شود.
در یادگیری فعال، سه سوال اصلی برای انتخاب داده‌ها و استفاده از آنها مطرح می‌شوند:

\begin{enumerate}
    \item چگونه دادگان باید انتخاب شوند: انتخاب دقیق و مناسب داده‌ها از اهمیت بالایی برخوردار است. باید تصمیم گرفت که کدام نمونه‌ها برای برچسب‌گذاری انتخاب شوند.
    \item بر اساس چه معیارهایی دادگان مناسب باید انتخاب شوند: دادگان باید بر اساس معیارهایی منتخب شوند. این معیارها می‌توانند به \persianfootnote{عدم قطعیت}\LTRfootnote{Uncertainty}  و \persianfootnote{نمایندگی}\LTRfootnote{Representativeness} داده‌ها مرتبط باشند.
    \item چگونه از نمونه‌های برچسب‌زده شده جدید استفاده شود: باید تصمیم گرفت که نمونه‌های برچسب‌زده شده جدید چگونه در فرآیند آموزش مدل مورد استفاده قرار گیرند. در اینجا دو رویکرد اصلی وجود دارد: یکی اینکه مدل با استفاده از نمونه‌های برچسب‌زده شده جدید از ابتدا آموزش داده شود و دیگری اینکه تنها با استفاده از نمونه‌های برچسب‌زده شده جدید بهینه شود.
\end{enumerate}
این روش‌ها می‌توانند در زمینه‌های مختلفی مانند پردازش تصاویر پزشکی مورد استفاده قرار بگیرند. از آنجایی که دقت و تفسیر‌پذیری الگوریتم‌ها در این حوزه بسیار مهم است، تعامل میان انسان و مدل‌های هوش مصنوعی نقش مهمی ایفا می‌کند و متخصصان می‌توانند در بهبود و تصحیح خروجی مدل‌ها مؤثر باشند\cite{zhao2021dsal}.

\subsection{ شبکه \lr{M-VAAL}}
در مقاله \cite{khanal2023m} یک روش برای بهره گیری از یادگیری فعال در زمینه قطعه‌بندی تومور مغز معرفی شد. \lr{M-VAAL} یک روش نمونه برداری \persianfootnote{وظیفه شناس}\LTRfootnote{task-agnostic} برای یادگیری فعال است که از اطلاعات تصویر چند وجهی استفاده می کند. این روش از خط لوله ای استفاده می کند که \persianfootnote{کدگذار خودکار متغیر}\LTRfootnote{Variational Auto-Encoder}را با یادگیری خصمانه ترکیب می کند تا یک فضای پنهان با ابعاد پایین تولید کند که هم عدم قطعیت و هم بازنمایی را از داده های ورودی دریافت می کند. سپس این روش از این فضای پنهان برای انتخاب نمونه های آموزنده برای حاشیه نویسی استفاده می کند. \lr{M-VAAL} همچنین اطلاعات چند وجهی مانند علائم بالینی، گزارش‌های بیمار و اطلاعات دستگاه کمکی را برای بهبود فرآیند نمونه‌گیری ترکیب می‌کند.
\\
خط لوله \lr{M-VAAL} از سه جزء اصلی تشکیل شده است: یک \lr{VAE}، یک تمیز دهنده و یک ماژول نمونه برداری. \lr{VAE} آموزش داده شده است تا تصاویر ورودی را در یک فضای پنهان با ابعاد کم کدگذاری کند، در حالی که تمیز دهنده آموزش دیده است تا بین نمونه های کدگذاری شده و توزیع قبلی تمایز قائل شود. سپس ماژول نمونه برداری، آموزنده ترین نمونه ها را برای حاشیه نویسی بر اساس عدم قطعیت و بازنمایی ثبت شده در فضای پنهان انتخاب می کند. این خط لوله را در شکل \ref{fig:mvaal} مشاهده می‌کنید.
\begin{figure}[h]
\centerline{\includegraphics[width=13cm]{images/mvaal.pdf}}
\caption[\hspace{0.5em}خط لوله \lr{M-VAAL}]{این روش یادگیری فعال از اطلاعات چندوجهی (\lr{m1} و\lr{m2}) برای بهبود \lr{VAAL} استفاده می کند. \lr{M-VAAL} از تصاویر بدون برچسب نمونه برداری می کند و نمونه های مکمل داده های حاشیه نویسی شده را انتخاب می کند که برای حاشیه نویسی به \lr{Oracle} ارسال می شود. ترکیب اطلاعات کمکی از روش دوم (\lr{m2}) یک نمایش نهفته تعمیم‌یافته‌تر برای نمونه‌گیری ایجاد می‌کند. این روش نمایش‌های وظیفه‌شناس را می‌آموزد، بنابراین فضای پنهان را می‌توان برای کارقطعه‌بندی برای نمونه‌برداری از بهترین تصاویر بدون برچسب برای حاشیه‌نویسی استفاده کرد\cite{khanal2023m}.
}
\label{fig:mvaal}
\end{figure}

روش \lr{M-VAAL} بر روی مجموعه داده \lr{BraTS2018} برای قطعه‌بندی تومور مغزی مورد ارزیابی قرار گرفت. نتایج نشان داد که این روش از سایر روش های یادگیری فعال بهتر عمل کرد. میانگین امتیاز \lr{Dice} $0.84$ را به دست آورد، که بالاتر از نمرات به دست آمده توسط سایر روش های یادگیری فعال بود. 

\section{یادگیری تعاملی}
قطعه‌بندی تعاملی تصاویر پزشکی به پزشکان امکان می‌دهد که با بهره‌گیری از مدل‌های هوش مصنوعی، تصاویر قطعه‌بندی شده را تصحیح کرده و بهبود بخش‌هایی از ‌قطعه‌بندی که اشتباهی یا نادرست توسط مدل تولید شده، به انجام برسانند. این فرآیند تعاملی به پزشکان امکان می‌دهد که با استفاده از ورودی‌های گرافیکی مانند کلیک ماوس یا کشیدن چارچوب‌ها روی تصویر، تصویر قطعه‌بندی را بهبود دهند.
\\
در این روش، یک مدل اولیه به وسیله هوش مصنوعی روی تصویر پزشکی قطعه‌بندی انجام می‌دهد و نتایج اولیه به پزشک ارائه می‌شوند. پزشک سپس می‌تواند نواحی اشتباهی که برچسب‌گذاری نادرست داشته‌اند را مشخص کند و تصویر را تصحیح کند. پس از اعمال نظر پزشک، مدل می‌تواند پارامترهای خود را به‌روزرسانی کند و تصاویر قطعه‌بندی جدیدی تولید کند که تغییرات مورد نیاز پزشک را اعمال کرده است.
\\
این روش به پزشکان امکان می‌دهد که نقش مهمی در قطعه‌بندی تصاویر پزشکی ایفا کنند و از دقت و صحت بالاتری برخوردار شوند. همچنین، این تعامل میان پزشک و مدل‌های هوش مصنوعی می‌تواند در بهبود و تصحیح خروجی‌های مدل‌های پزشکی مؤثر باشد.

\begin{figure}[h]
\centerline{\includegraphics[width=13cm]{images/interactive}}
\caption[\hspace{0.5em}سازوکار یادگیری تعاملی]{سازوکار یادگیری تعاملی\cite{wang2022medical}}
\label{fig:interactive}
\end{figure}
مطابق شکل \ref{fig:interactive} روش قطعه‌بندی تعاملی دارای دو مرحله اصلی است:

\begin{itemize}
    \item تولید برچسب اولیه با قطعه‌بندی پایه و ورودی کاربر اول:
در این مرحله، یک مدل هوش مصنوعی ابتدا قطعه‌بندی اولیه را از تصویر پزشکی انجام می‌دهد. سپس نتایج اولیه به پزشک ارائه می‌شوند. پزشک می‌تواند اشتباهات و نواحی نادرستی که توسط مدل تولید شده‌اند را تشخیص دهد و تصحیح‌های مورد نیاز را اعمال کند. ورودی کاربر اول به مدل برای بهبود تصویر اولیه ارائه می‌شود.
    \item حلقه بهبود تعاملی:
در این مرحله، مدل به وسیلهٔ مداخلات تکراری کاربران بهبود می‌یابد. پزشکان می‌توانند تغییرات خود را با استفاده از کلیک‌های ماوس و یا کشیدن چارچوب‌ها روی تصویر اعمال کنند. تغییرات اعمالی توسط پزشکان به مدل انتقال داده می‌شود و مدل می‌تواند بر اساس نظرات و تصحیح‌های کاربران تصاویر قطعه‌بندی جدیدی تولید کند. این حلقه بهبود تعاملی می‌تواند به تکرار انجام شود تا زمانی که قطعه‌بندی نهایی رضایت‌بخش باشد.

\end{itemize}
 

این روش به پزشکان امکان می‌دهد تا به تصویر قطعه‌بندی شده دست بزنند و تصحیح‌های لازم را اعمال کنند تا قطعه‌بندی دقیق‌تری از تصویر ایجاد شود. این تعامل میان پزشک و مدل‌های هوش مصنوعی می‌تواند به بهبود و اصلاح تصاویر قطعه‌بندی پزشکی کمک کند.

استفاده از توالی از شبکه‌های عصبی پیچشیی برای قطعه‌بندی تعاملی تصاویر پزشکی توانسته است بهبودی مهم در فرآیند قطعه‌بندی و تصحیح تصاویر دستی در حوزه پزشکی ایجاد کند. این توالی از \lr{CNN} شامل دو شبکه مختلف به نام‌های \lr{P-Net} و \lr{R-Net} است:

\begin{enumerate}
    \item \lr{P-Net}: این شبکه ابتدا یک نتیجه قطعه‌بندی کلی از تصویر به دست می‌آورد. این نتیجه ابتدایی ممکن است دارای اشتباهات و نواحی نادرست باشد. در این مرحله، کاربران امکان دارند نقاط تعاملی یا خطوط کوتاهی را برای علامت‌گذاری نواحی قطعه‌بندی اشتباه ارائه دهند. این تعامل با افراد تاکید می‌کند که تصاویر را اصلاح کرده و به مدل اطلاعات تصحیح شده را ارائه دهند.
    \item \lr{R-Net}: در مرحله بعدی، از نتیجه مرحله \lr{P-Net} به عنوان ورودی برای \lr{CNN} دوم به نام \lr{R-Net} استفاده می‌شود. این شبکه به اصطلاح نتایج اصلاح شده را به دست می‌آورد. با استفاده از نقاط تعاملی و اصلاح‌های انجام شده توسط کاربران در مرحله \lr{P-Net}، \lr{R-Net} تصحیح‌های لازم را به نتایج ابتدایی اعمال می‌کند. این مرحله به بهبود دقت و اصلاح نتایج قطعه‌بندی کمک می‌کند.
\end{enumerate}

این توالی از \lr{CNN}‌ها با تعامل کاربران و مدل‌های هوش مصنوعی توانسته است بهبود مهمی در قطعه‌بندی تصاویر پزشکی و تصحیح نتایج ارائه دهد و به متخصصان پزشکی امکان دست‌اندازی و بهبود نتایج قطعه‌بندی دستی را بدهد. این رویکرد به مقایسه با روش‌های سنتی تا حد زیادی زمان فرآیند را کاهش می‌دهد و کارایی را بهبود می‌بخشد\cite{wang2022medical}.

\begin{figure}[h]
\centerline{\includegraphics[width=13cm]{images/pnet_rnet}}
\caption[\hspace{0.5em}شبکه های ویرایش کننده قطعه‌بندی تعاملی]{شبکه های ویرایش کننده قطعه‌بندی تعاملی\cite{wang2022medical}}
\label{fig:pnet_rnet}
\end{figure}

\subsection{ روش \lr{BIFSeg}}
روش \lr{BIFSeg}\cite{malhotra2022deep} در زمینه قطعه‌بندی تصاویر پزشکی با بهره‌گیری از تعامل پزشک و مدل‌های هوش مصنوعی به نحوی عمل می‌کند که توسط کاربر (پزشک) ورودی اولیه در اختیار مدل قرار داده می‌شود. عملکرد این روش به شرح زیر است:
\begin{enumerate}
    \item ترسیم جعبه مرزی: ابتدا کاربر (پزشک) یک جعبه مرزی روی تصویر پزشکی ترسیم می‌کند. این جعبه مرزی ناحیه‌ای را انتخاب می‌کند که برای تحلیل و قطعه‌بندی ورودی به شبکه اولیه (شبکه \lr{P-Net} در مثال قبلی) مناسب است.
    \item نتیجه اولیه: با استفاده از ناحیه انتخاب شده در مرحله اول، یک نتیجه اولیه از شبکه اولیه به دست می‌آید. این نتیجه اولیه ممکن است دارای اشتباهات و نواحی نادرست باشد.
    \item تعامل با نتایج: پزشک پس از دریافت نتیجه اولیه می‌تواند تصاویر را تصحیح و بهبود دهد. او اصلاح‌های لازم را انجام می‌دهد تا نتایج بهتری به دست آورد.
    \item نتیجه نهایی: در مرحله نهایی، با توجه به اصلاح‌ها و تعامل پزشک، خروجی نهایی ارائه می‌شود که بهبود‌های لازم را در قطعه‌بندی اعمال کرده و نتایج دقیق‌تری از تصاویر ارائه می‌دهد.
\end{enumerate}
مزیت اصلی این روش این است که ناحیه انتخابی توسط پزشک اولیه و با توجه به دقت و نیازهای خود او تعیین می‌شود. این به پزشک امکان می‌دهد تا تصاویر را به صورت دقیق‌تر و مورد تصویب خود قطعه‌بندی کند، و این نه تنها به کاهش تعداد مراحل تعاملی پزشک و مدل کمک می‌کند بلکه پزشک نیز از این رویکرد راضی‌تر است. این روش نشان می‌دهد که با بهره‌گیری از تعامل پزشک، تعامل‌ها و اصلاحات کمتری نیاز است تا به نتایج بهتری برسیم\cite{malhotra2022deep}.
عملکرد این روش را در جدول \ref{tab:bifsef_result} می توانید ببینید.
\begin{table}[ht]
\caption[\hspace{0.5em}مقایسه روش \lr{BIFSeg} با سایر روش ها]{مقایسه روش \lr{BIFSeg} با سایر روش ها.\lr{TC} نشان دهنده بخش \lr{Tumor Core} در \lr{T1ce} و \lr{WT} نشان دهنده بخش \lr{Whole Tumor} در \lr{Flair}\cite{wang2018interactive}}
\label{tab:bifsef_result}
\centering
\onehalfspacing
\begin{tabular}{|c|c|c|c|c|c|}
\hline
 &   & \lr{PC-Net} &  \lr{PC-Net+CRF} & \lr{BIFSeg(-w)} & \lr{BIFSeg} \\
\hline
\lr{Dice} & \lr{TC} & $82.66\pm7.78$ & $85.93\pm6.64$ & $85.88\pm7.53$ & $87.49\pm6.36$ \\
\hline
\lr{Dice} & \lr{WT} & $83.52\pm8.76$ & $85.18\pm6.78$ & $86.54\pm7.49$ & $88.11\pm6.09$ \\
\hline
\end{tabular}
\end{table}

\begin{figure}[h]
\centerline{\includegraphics[width=13cm]{images/bifseg}}
\caption[\hspace{0.5em}ساختار \lr{BIFSeg}]{قسمت بالا نحوه ورودی شبکه اولیه را نشان می‌دهد.در قسمت پایین نحوه تعامل پزشک با شبکه میانی را نمایش می‌دهد\cite{malhotra2022deep}}
\label{fig:bifseg}
\end{figure}
مقایسه بصری بین روش های تعاملی را می توانید در شکل \ref{fig:visual_comparison} مشاهده کنید.
\begin{figure}[h]
\centerline{\includegraphics[width=13cm]{images/visual_comparison.pdf}}
\caption[\hspace{0.5em}مقایسه بصری روش های تعاملی]{مقایسه بصری روش های تعاملی \cite{wang2018interactive}}
\label{fig:visual_comparison}
\end{figure}
% روش‌هایی مانند \lr{GrabCut} از تحریک‌های متنوعی استفاده می‌کنند تا نواحی مورد نظر در تصاویر پزشکی را مشخص کنند. این تحریک‌ها می‌توانند از خم‌ها، خطوط یا دیگر شیوه‌های تعاملی با تصاویر باشند. اصولاً مرحله اول از این تحریک‌ها برای تعیین نواحی تصویری که به نوعی به تصاویر اضافه شده‌اند، استفاده می‌شود. سپس تصویر با تحریک به شبکه اولیه داده می‌شود تا نتیجه اولیه قطعه‌بندی انجام شود. این نتیجه اولیه ممکن است دقت پایینی داشته باشد.

% سپس پزشک یا کاربر به تحریک‌ها اصلاح‌های لازم را اعمال می‌کند تا نواحی ناصحیح برچسب‌گذاری شده توسط شبکه اصلاح شود و نتایج بهتری به دست آید. این نوع روش‌ها به پزشکان این امکان را می‌دهند تا به عنوان ورودی یک تحریک متنوع را برای تعامل با تصویر انتخاب کنند و قطعه‌بندی نهایی توسط شبکه هوش مصنوعی تصحیح شود.

% از مزیت‌های این روش‌ها می‌توان به افزایش دقت در قطعه‌بندی تصاویر پزشکی اشاره کرد. این تحریک‌ها به پزشکان امکان می‌دهند تا به تعامل با تصاویر پزشکی بپردازند و نتایج بهتری ارائه دهند. همچنین، این روش‌ها با کاهش تعداد مراحل تعاملی پزشک و مدل، زمان و زحمت مورد نیاز برای تصحیح قطعه‌بندی تصاویر را کاهش می‌دهند.


% \begin{figure}[h]
% \centerline{\includegraphics[width=13cm]{images/grabcut}}
% \caption{تصویر الف تصویر پزشکی است. تصویر ب تصویری است تحریک توسط نقاط در اطراف ناحیه مورد نظر اعمال گردیده و تصویر ج خروجی مدل است}
% \label{fig:grabcut}
% \end{figure}
\subsection{ روش راهنمای من}
روش  راهنمای من \cite{wang2022medical}  که با استفاده از اطلاعات متنی و ورودی کاربران قطعه‌بندی تصاویر را به‌روزرسانی می‌کند، یک رویکرد جالب است. این روش از پردازش زبان طبیعی برای تبدیل جملات پزشک به بردارهای ویژگی استفاده می‌کند و سپس این بردارهای ویژگی را به عنوان ورودی به مدل قطعه‌بندی تصویر ارائه می‌دهد.

با این رویکرد، پزشکان می‌توانند با ارسال پیام‌های متنی که توصیفی از تصویر دارند، به مدل اطلاعات بیشتری ارائه دهند و خواسته‌های خود را به وضوح انتقال دهند. این امر می‌تواند به دقت و کارایی قطعه‌بندی تصاویر کمک کند. سپس مدل می‌تواند با ترکیب این اطلاعات متنی با ویژگی‌های تصویر، قطعه‌بندی بهتری ارائه دهد.

% روش \lr{GM} به نوعی ترکیب دو عنصر مهم در پردازش تصاویر و متن به کار می‌رود و این امر می‌تواند به بهبود نتایج قطعه‌بندی تصاویر و تعامل بهتر با پزشکان کمک کند. این روش نشان می‌دهد که چگونه می‌توان از اطلاعات متنی برای بهبود دقت و انطباق قطعه‌بندی تصاویر پزشکی استفاده کرد\cite{wang2022medical}.
\subsection{ چارچوب تعاملی با رابط بصری و بازخورد}
در مقاله \cite{sailunaz2023brain} یک چارچوب تعاملی با رابط بصری و بازخورد برای بهبود دقت و اعتماد ارائه شده است. فرایندی که در این چارچوب برای قطعه‌بندی تومور مغز به صورت تعاملی انجام می‌شود به صورت زیر است:
\begin{enumerate}
    % \item کاربر به برنامه وبی که آماده شده است از طریق صفحه اصلی دسترسی پیدا می کند.
    \item کاربر می تواند مستقیماً یک تصویر را برای ارزیابی آپلود کند یا از آرشیو تصاویر پزشکی بیمارستان ها استفاده کند.
    % \item در صورتی که کاربر قبلا ثبت نام کرده باشد، می تواند با شناسه و رمز عبور خود وارد سامانه شود. در غیر این صورت، آنها می توانند با یک شناسه، نام، نام خانوادگی و رمز عبور برای استفاده از سیستم در سیستم ثبت نام کنند.
    \item کاربر می تواند تصویر را برای اعمال مدل های قطعه‌بندی تومور مغز آپلود کند.
    \item این سیستم مدل \lr{UNet} یا \lr{UNet++} را بر اساس انتخاب کاربر اعمال می‌کند و از مدل‌های آموزش‌دیده ذخیره‌شده برای پیش‌بینی و ‌قطعه‌بندی تومورهای مغزی از داده‌های ورودی آپلود شده توسط کاربر استفاده می‌کند.
    \item سیستم اطلاعات قطعه‌بندی خروجی (یعنی تومور قطعه‌بندی شده) را با برخی امتیازات عملکرد برای کاربران فراهم می‌کند.
    \item کاربر می تواند تومور قطعه‌بندی شده را در برنامه وب مشاهده کند و نتایج قطعه‌بندی را با استفاده از ابزارهای تعاملی ارائه شده توسط سیستم تنظیم کند.
     \item بازخورد ارائه شده توسط کاربر در مدل‌های آموزشی گنجانده شده است تا بازخورد حرفه‌ای را برای شناسایی و قطعه‌بندی آینده در خود جای دهد.
\end{enumerate}
چارچوب کلی این سامانه را می توانید در شکل \ref{fig:interactive_framework} ببینید.
\begin{figure}[ht]
\centerline{\includegraphics[width=18cm]{images/interactive_framework.pdf}}
\caption[\hspace{0.5em}چارچوب کلی سامانه تشخیص تعاملی]{چارچوب کلی سامانه تشخیص تعاملی\cite{sailunaz2023brain}}
\label{fig:interactive_framework}
\end{figure}
بازخورد از طریق گزینه بازخورد ارائه شده در برنامه وب انجام می شود. کاربر می‌تواند با ذکر وضعیت نتیجه به‌عنوان «مساحت باید پیدا شود (\lr{FN})» برای خروجی‌های منفی کاذب و «مساحت نباید یافت شود (\lr{FP})» برای نتایج مثبت کاذب، بازخوردی در مورد نتایج قطعه‌بندی ارائه دهد. آنها همچنین می توانند در صورتی که ناحیه قطعه‌‌بندی‌شده کاملاً دقیق نباشد، خطوطی از ناحیه تومور را ارائه یا ترسیم کنند. آنها می توانند هر نظر دیگری را که ممکن است در بخش "نظر" داشته باشند اضافه کنند. بازخورد ارائه شده توسط کاربر در مدل‌های آموزشی گنجانده شده است تا بازخورد را برای شناسایی و قطعه‌بندی آینده در خود جای دهد. این سازوکار بازخورد به بهبود دقت مدل‌های قطعه‌بندی در طول زمان و ایجاد اعتماد در سامانه کمک می‌کند. شکل \ref{fig:feedback_page} صفحه مربوط به ایجاد بازخورد در این سامانه را نمایش می‌دهد.
\begin{figure}[ht]
\centerline{\includegraphics[width=15cm]{images/feedback_page.pdf}}
\caption[\hspace{0.5em}صفحه بازخورد در سامانه تشخیص تعاملی]{صفحه بازخورد در سامانه تشخیص تعاملی\cite{sailunaz2023brain}}
\label{fig:feedback_page}
\end{figure}
چارچوب تعاملی با رابط بصری و بازخورد، دقت و اعتماد را در ‌قطعه‌بندی تومور مغزی به چندین روش بهبود می‌بخشد:
\begin{itemize}
    \item رابط بصری به متخصصان مراقبت های سلامت اجازه می دهد تا نتایج قطعه‌بندی را مشاهده کرده و با آنها تعامل داشته باشند، که می تواند به آنها در درک بهتر فرآیند قطعه‌بندی و شناسایی هر گونه خطا یا عدم دقت در نتایج کمک کند.
    \item بازخورد به متخصصان مراقبت های سلامت اجازه می دهد تا در مورد نتایج قطعه‌بندی بازخورد ارائه دهند، که می تواند به بهبود دقت مدل های قطعه‌بندی در طول زمان کمک کند.
    \item این سامانه امتیازهای عملکردی را برای نتایج قطعه‌بندی ارائه می دهد که می تواند به متخصصان مراقبت های سلامت کمک کند تا صحت نتایج را ارزیابی کنند و در سیستم اعتماد ایجاد کنند.
\end{itemize}
تغییرات در مدل ها با گنجاندن بازخورد ارائه شده توسط متخصصان مراقبت های سلامت در داده های آموزشی برای مدل های یادگیری عمیق مورد استفاده در سامانه ایجاد می شود. بازخورد شامل اطلاعاتی در مورد نتایج مثبت کاذب و منفی کاذب و همچنین هرگونه نظر یا پیشنهاد دیگری است که کاربر ممکن است داشته باشد. سپس از این بازخورد برای به‌روزرسانی پارامترهای مدل‌ها استفاده می‌شود که می‌تواند دقت آنها را در طول زمان بهبود بخشد. سپس از مدل‌های به‌روزرسانی شده برای ‌قطعه‌بندی تصاویر جدید تومور مغزی استفاده می‌شود و فرآیند بازخورد و به‌روزرسانی مدل به‌طور مکرر ادامه می‌یابد. این فرآیند یادگیری تکراری به بهبود دقت مدل‌های قطعه‌بندی در طول زمان و ایجاد اعتماد در سیستم کمک می‌کند.\\
در جدول \ref{tab:2d_3d_segmentaion} و جدول \ref{tab:interactive_framework_result} می توانید مقایسه مدل های دو بعدی و سه بعدی برای قطعه‌بندی تومور مغز و همچنین عملکرد چارچوب پیشنهادی این مقاله را مشاهده کنید.
\begin{table}[ht]
\caption[\hspace{0.5em}مقایسه روش های قطعه‌بندی تومور دو بعدی و سه بعدی]{مقایسه روش های قطعه‌بندی تومور دو بعدی و سه بعدی\cite{sailunaz2023brain}}
\label{tab:2d_3d_segmentaion}
\centering
\onehalfspacing
\begin{tabular}{|c|c|c|c|}
\hline
مدل &  \lr{Dice} پیش‌بینی & \lr{Dice} آموزش &  \lr{Dice} ارزیابی \\
\hline
\lr{2D UNet} & $81.20$ & $84.48$ & $68.37$ \\
\hline
\lr{2D UNet++} & $80.96$ & $84.77$ & $61.92$ \\
\hline
\lr{3D UNet} & $96.17$ & $94.68$ & $94.74$ \\
\hline
\lr{3D UNet++} & $94.39$ & $94.28$ & $94.14$ \\
\hline
\end{tabular}
\end{table}

\begin{table}[ht]
\caption[\hspace{0.5em}مقایسه قطعه‌بندی بخش \lr{whole} تومور با \lr{BraTS 2021}]{مقایسه قطعه‌بندی کل تومور با \lr{BraTS 2021}\cite{sailunaz2023brain}}
\label{tab:interactive_framework_result}
\centering
\onehalfspacing
\begin{tabular}{|c|c|}
\hline
مدل &  \lr{Dice} \\
\hline
\lr{3D UNet} &  $93.24$ \\
\hline
\lr{3D UNet}(روش پیشنهادی مقاله) &  $96.17$ \\
\hline
\lr{3D UNet++} & $84.93$ \\
\hline
\lr{3D UNet++}(روش پیشنهادی مقاله) & $94.39$ \\
\hline
\end{tabular}
\end{table}

سامانه پیشنهادی \cite{sailunaz2023brain} گزینه‌های متعددی را برای ‌قطعه‌بندی تومور مغز، از جمله قطعه‌بندی دو بعدی و سه بعدی با استفاده از مدل‌های \lr{UNet} و \lr{UNet++} ارائه می‌کند. نتایج نشان می‌دهد که مدل‌های پیاده‌سازی‌شده در این تحقیق تقریباً 3 تا 10 درصد نمرات \lr{Dice} بالاتری نسبت به مدل های پایه \lr{UNet} و \lr{UNet++} برای مجموعه داده \lr{BraTS 2021} به دست آوردند. نشان داده شده است که فرآیند یادگیری تکراری ترکیب بازخورد از متخصصان مراقبت های سلامت در داده های آموزشی باعث بهبود دقت و عملکرد مدل های قطعه‌بندی تومور مغزی در طول زمان می شود.
% \section{یادگیری خود نظارتی}

