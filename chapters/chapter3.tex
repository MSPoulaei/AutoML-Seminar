\chapter{مروری بر کارهای مرتبط}
\thispagestyle{empty}

% \section{مقدمه}
% ادغام \persianfootnote{مدل‌های زبانی بزرگ}\LTRfootnote{Large Language Models (LLMs)} در \persianfootnote{خودکارسازی یادگیری ماشین}\LTRfootnote{Automated Machine Learning (AutoML)} به‌عنوان رویکردی دگرگون‌ساز برای بهینه‌سازی \persianfootnote{معماری‌های عصبی}\LTRfootnote{neural architectures} و \persianfootnote{فراپارامترها}\LTRfootnote{hyperparameters} پدیدار شده است. این فصل کارهای موجود را از سه منظر مکمل بررسی می‌کند: \persianfootnote{معماری عامل}\LTRfootnote{agent architecture}، \persianfootnote{راهبردهای تقویت دانش}\LTRfootnote{knowledge augmentation strategies}، و \persianfootnote{طراحی قالب خروجی}\LTRfootnote{output format design}. این ابعاد در کنار هم نشان می‌دهند که سامانه‌های گوناگون چگونه از قابلیت‌های \persianfootnote{مدل‌های زبانی بزرگ}\LTRfootnote{Large Language Models (LLMs)} بهره می‌برند و هم‌زمان محدودیت‌های ذاتی آن‌ها را مدیریت می‌کنند.
% \subsection{ظهور اولین روش های یادگیری ماشین خودکار}
% \subsection{روش های یادگیری ماشین خودکار در عصر مدل های زبانی بزرگ}

\section{مقدمه}
ادغام مدل‌های زبانی بزرگ در \persianfootnote{خودکارسازی یادگیری ماشین}\LTRfootnote{Automated Machine Learning (AutoML)} به‌عنوان رویکردی دگرگون‌ساز برای بهینه‌سازی \persianfootnote{معماری‌های عصبی}\LTRfootnote{neural architectures} و \persianfootnote{فراپارامترها}\LTRfootnote{hyperparameters} پدیدار شده است. این فصل کارهای موجود را از سه منظر مکمل بررسی می‌کند: \persianfootnote{معماری عامل}\LTRfootnote{agent architecture}، \persianfootnote{راهبردهای تقویت دانش}\LTRfootnote{knowledge augmentation strategies}، و طراحی قالب خروجی. این ابعاد در کنار هم نشان می‌دهند که سامانه‌های گوناگون چگونه از قابلیت‌های مدل‌های زبانی بزرگ بهره می‌برند و هم‌زمان محدودیت‌های ذاتی آن‌ها را مدیریت می‌کنند.

\section{ظهور نخستین روش‌های یادگیری ماشین خودکار}
خودکارسازی یادگیری ماشین با هدف \persianfootnote{مردمی‌سازی}\LTRfootnote{democratization} فرایندهای پرهزینه و زمان‌بر \persianfootnote{گزینش مدل}\LTRfootnote{model selection} و \persianfootnote{بهینه‌سازی فراپارامتر}\LTRfootnote{hyperparameter optimization (HPO)} پدید آمد. ریشه گزینش الگوریتم به سال 1976 بازمی‌گردد که در آن، مسئله انتخاب الگوریتم بهینه از میان یک مجموعه ازپیش‌تعریف‌شده به‌منظور کمینه‌سازی افت کارایی روی داده‌های اعتبارسنجی صورت‌بندی شد \cite{RICE197665}. کوشش‌های نخستین در حوزه بهینه‌سازی فراپارامتر بر روش‌های منفردی چون \persianfootnote{جست‌وجوی شبکه‌ای}\LTRfootnote{grid search} و \persianfootnote{جست‌وجوی تصادفی}\LTRfootnote{random search} متمرکز بود؛ روش‌هایی ساده اما ناکارا در فضاهای بعدبالا \cite{JMLR:v13:bergstra12a}. این روش نشان داد که جست‌وجوی تصادفی با تخصیص مؤثرتر منابع به فراپارامترهای اثرگذار، از جست‌وجوی شبکه‌ای پیشی می‌گیرد و بدین‌ترتیب زمینه را برای روش‌های پیشرفته‌تر فراهم ساخت \cite{JMLR:v13:bergstra12a}.

نقطه عطف در 2013 با معرفی مسئله CASH رخ داد؛ چارچوبی که گزینش الگوریتم و بهینه‌سازی فراپارامتر را در یک بهینه‌سازی واحد ادغام می‌کرد \cite{10.1145/2487575.2487629}. بر مبنای این ایده، Auto-WEKA به‌عنوان نخستین سامانه جامع یادگیری ماشین خودکار پدید آمد و با اتکا به SMAC — رویکردی مبتنی بر \persianfootnote{بهینه‌سازی بیزی}\LTRfootnote{Bayesian optimization} که از \persianfootnote{جنگل‌های تصادفی}\LTRfootnote{random forests} به‌عنوان \persianfootnote{جانشین}\LTRfootnote{surrogate} استفاده می‌کرد — به جست‌وجو در میان رده‌بندهای WEKA و فراپارامترهایشان پرداخت \cite{10.1145/2487575.2487629, 10.1007/978-3-642-25566-3_40}. در ادامه، auto-sklearn در 2015 پارادایم CASH را به الگوریتم‌های scikit-learn بسط داد و با به‌کارگیری \persianfootnote{فرا-یادگیری}\LTRfootnote{meta-learning} برای \persianfootnote{آغاز گرم}\LTRfootnote{warm-starting} و نیز ساخت \persianfootnote{هم‌بندی}\LTRfootnote{ensemble}‌های مقاوم، بهبودهای معناداری رقم زد \cite{NIPS2015_11d0e628}.

به‌طور موازی، \persianfootnote{الگوریتم‌های تکاملی}\LTRfootnote{evolutionary algorithms} به‌عنوان بدیلی تواناتر مطرح شدند؛ چنان‌که TPOT \persianfootnote{خطوط لوله}\LTRfootnote{ML pipelines} یادگیری ماشین را به‌صورت برنامه‌هایی درخت‌ساختار مدل می‌کند و با \persianfootnote{برنامه‌نویسی ژنتیکی}\LTRfootnote{genetic programming}، ترکیب‌های انعطاف‌پذیری از پیش‌پردازنده‌ها، \persianfootnote{گزیننده‌های ویژگی}\LTRfootnote{feature selectors} و مدل‌ها را می‌آزماید \cite{pmlr-v64-olson_tpot_2016}. روش‌های \persianfootnote{چندسطوحی/چندوفایی}\LTRfootnote{multi-fidelity} با تخصیص تدریجی منابع به پیکربندی‌های امیدبخش و با الهام از \persianfootnote{راهبردهای باندیتی}\LTRfootnote{bandit-based strategies} هزینه محاسباتی را مهار کردند \cite{pmlr-v51-jamieson16, JMLR:v18:16-558}. تمرکز این رویکردهای اولیه عمدتاً بر مسائل کلاسیک \persianfootnote{طبقه‌بندی}\LTRfootnote{classification} و \persianfootnote{رگرسیون}\LTRfootnote{regression} بود؛ و معیارهایی چون \lr{OpenML-CC18} ارزیابی‌های استانداردشده را تسهیل کردند \cite{bischl2021openmlbenchmarkingsuites}.

گذار به \persianfootnote{جست‌وجوی معماری شبکه‌های عصبی}\LTRfootnote{Neural Architecture Search (NAS)} از حوالی 2017 آغاز شد و طراحی معماری را امتداد طبیعی بهینه‌سازی فراپارامتر تلقی کرد. کارهای اولیه مبتنی بر \persianfootnote{یادگیری تقویتی}\LTRfootnote{reinforcement learning (RL)} در جست‌وجوی معماری شبکه‌های عصبی، گرچه پرهزینه، نقطه شروعی اثرگذار بودند و گونه‌های کاراتری همچون ENAS با \persianfootnote{اشتراک پارامتر}\LTRfootnote{parameter sharing} را الهام بخشیدند \cite{zoph2017neural, pmlr-v80-pham18a}. در ادامه، معیارهایی مانند \lr{NAS-Bench-101} با فراهم‌کردن ارزیابی‌های ازپیش‌محاسبه‌شده، امکان پژوهش بازتولیدپذیر را مهیا کردند \cite{pmlr-v97-ying19a}. این سیر تحول، تنظیم دستی را به خطوط لوله خودکار بدل کرد و بستر را برای همگرایی با پارادایم‌های پیشرفته‌تر فراهم ساخت.
\section{طبقه‌بندی بر اساس معماری عامل}
در چشم‌انداز رو‌به‌تحول رویکردهای مبتنی بر \persianfootnote{مدل‌های زبانی بزرگ}\LTRfootnote{Large Language Models (LLMs)} برای \persianfootnote{یادگیری ماشین خودکار}\LTRfootnote{Automated Machine Learning (AutoML)}، \persianfootnote{جست‌وجوی معماری شبکه‌های عصبی}\LTRfootnote{Neural Architecture Search (NAS)} و \persianfootnote{بهینه‌سازی ابرپارامتر}\LTRfootnote{Hyperparameter Optimization (HPO)}، یک تمایز بنیادین در \persianfootnote{معماری عامل}\LTRfootnote{Agent Architecture} برجسته است: \persianfootnote{تک‌عاملی}\LTRfootnote{Single-Agent} در برابر \persianfootnote{چندعاملی}\LTRfootnote{Multi-Agent}. این طبقه‌بندی بازتاب‌دهنده نحوه سازمان‌دهی مدل‌های زبانی برای انجام وظایف بهینه‌سازی است؛ در رویکردهای تک‌عاملی، یک نمونه منفرد از مدل زبانی کل فرایند را به‌صورت خودکار و تکرارشونده به‌عهده دارد، در حالی‌که در رویکردهای چندعاملی، مسئولیت‌ها میان چند عامل تخصصی توزیع می‌شود تا هم‌افزایی، کارایی و توان حل مسئله افزایش یابد.\\ همان‌گونه که در جدول~\ref{tab:recent-works} خلاصه شده است، این دوگانگی بر جنبه‌هایی همچون \persianfootnote{کدزایی}\LTRfootnote{Code Generation}، \persianfootnote{یکپارچه‌سازی دانش بیرونی}\LTRfootnote{External Knowledge Integration} و \persianfootnote{کاوش فضای جست‌وجو}\LTRfootnote{Search Space Exploration} اثرگذار است.

\subsection{رویکردهای تک‌عاملی}
معماری‌های تک‌عاملی بر این فرض استوارند که یک مدل زبانی واحد با ظرفیت کافی می‌تواند کل خط لوله بهینه‌سازی را مدیریت کند. این رویکردها را می‌توان بر اساس \emph{سازوکار جست‌وجو} و \emph{نحوه بهره‌گیری از بازخورد} به چهار دسته اصلی تقسیم کرد: (۱) روش‌های مبتنی بر پالایش تکرارشونده، (۲) رویکردهای تکاملی-مولد، (۳) راهبردهای مبتنی بر جانشین‌های کم‌هزینه، و (۴) چارچوب‌های یکپارچه‌سازی سراسری.

\subsubsection{پالایش تکرارشونده مبتنی بر بازخورد}
دسته‌ای از روش‌های تک‌عاملی بر \persianfootnote{حلقه‌های بازخورد}\LTRfootnote{Feedback Loops} تکیه می‌کنند که در آن مدل زبانی بر اساس نتایج ارزیابی‌های پیشین، پیکربندی‌ها را به‌صورت تکرارشونده پالایش می‌کند. در این راستا، \cite{zhang2023usingLLMforHPO} نشان داده‌اند که یک عامل منفرد می‌تواند با هدایت از طریق \persianfootnote{دستور ورودی}\LTRfootnote{Prompting} و دریافت بازخورد کارایی، ابرپارامترها را در وظایف متنوعی از رگرسیون تا مسائل ترکیباتی بهینه کند و در سناریوهای کم‌هزینه، عملکردی رقابتی با \persianfootnote{بهینه‌سازی بیزی}\LTRfootnote{Bayesian Optimization} ارائه دهد. به‌طور مشابه، \cite{zheng2023GENIUS} با تکیه بر توصیف‌های زبان طبیعی و پالایش تکرارشونده بدون اتکا به الگوریتم‌های جست‌وجوی کلاسیک، در معیارهایی چون \lr{NAS-Bench-201} نشان داده‌اند که استدلال ذاتی مدل‌های زبانی می‌تواند در پیمایش و رتبه‌بندی معماری‌ها به‌کار گرفته شود. همچنین \cite{zhang2023AutomlGPTAutomaticMachineLearning} این رویکرد را به کل خط لوله یادگیری ماشین تعمیم داده و نشان داده‌اند که یک عامل واحد می‌تواند از تولید کُد پیش‌پردازش تا انتخاب مدل و ارزیابی را به‌صورت خودکار مدیریت کند.

\emph{مزایای اصلی} این رویکردها شامل سادگی پیاده‌سازی، کاهش سربار ارتباطی، و انعطاف‌پذیری در برابر توصیف‌های زبان طبیعی است. با این حال، \emph{محدودیت بنیادین} آن‌ها در وابستگی به ظرفیت استدلالی مدل پایه و احتمال همگرایی به نقاط بهینه محلی در فضاهای جست‌وجوی پیچیده نهفته است، زیرا دیدگاه واحدی برای کاوش فضا به‌کار می‌رود.

\subsubsection{تلفیق با الگوریتم‌های تکاملی و تنوع‌محور}
برای غلبه بر محدودیت‌های کاوش تک‌دیدگاهی، دسته دوم از روش‌های تک‌عاملی مدل‌های زبانی را با \persianfootnote{الگوریتم‌های تکاملی}\LTRfootnote{Evolutionary Algorithms} و راهبردهای \persianfootnote{کیفیت-تنوع}\LTRfootnote{Quality-Diversity (QD)} ادغام می‌کنند تا تنوع جست‌وجو را افزایش دهند. \cite{LLMatic2024} با ترکیب مدل زبانی با بهینه‌سازی کیفیت-تنوع، نشان داده‌اند که تولید جهش‌های کُد-محور و ارزیابی آن‌ها در \persianfootnote{بایگانی تنوع‌محور}\LTRfootnote{Diversity-focused Archive} می‌تواند نسبت به جست‌وجوی تصادفی، تنوع و کیفیت معماری‌ها را همزمان بهبود بخشد. در رویکردی مشابه، \cite{chen2023Evoprompting} با الهام از \persianfootnote{الگوریتم‌های ژنتیکی}\LTRfootnote{Genetic Algorithms}، پرامپت‌ها را به‌صورت جمعیتی تکامل می‌دهند (از طریق جهش و ترکیب) تا کُد شبکه‌های عصبی تولید کنند، و نتایج قدرتمندی در معیارهای جست‌وجوی معماری گزارش کرده‌اند. همچنین \cite{Yu2025GPTNAS} با کُدگذاری معماری‌ها در قالب دستور ورودی و تکامل آن‌ها از طریق انتخاب و جهش، نشان داده‌اند که پیش‌آموزش مولد می‌تواند سرعت همگرایی را نسبت به الگوریتم‌های تکاملی کلاسیک بهبود بخشد.

این رویکردهای ترکیبی \emph{تعادلی} میان بهره‌گیری از توان استدلالی مدل‌های زبانی و قدرت کاوش الگوریتم‌های تکاملی برقرار می‌کنند. با این حال، آن‌ها معمولاً نیازمند تعداد بیشتری ارزیابی و تنظیم دقیق‌تر فراپارامترهای تکاملی هستند که می‌تواند هزینه محاسباتی را افزایش دهد.

\subsubsection{جست‌وجوی کارآمد با جانشین‌های کم‌هزینه}
برای کاهش هزینه محاسباتی ارزیابی‌های پرشمار، دسته سوم از روش‌ها مدل‌های زبانی را با \persianfootnote{جانشین‌های صفرهزینه}\LTRfootnote{Zero-cost Proxies} یا \persianfootnote{جست‌وجوی بی‌گرادیان}\LTRfootnote{Gradient-free Search} ادغام می‌کنند. \cite{ji2025RZNAS} چارچوب «\persianfootnote{صفرهزینهٔ بازتابی}\LTRfootnote{Reflective Zero-cost}» را معرفی کرده‌اند که در آن مدل زبانی معماری‌ها را پیشنهاد می‌کند و آن‌ها را بر پایهٔ جانشین‌های بی‌هزینه مانند \persianfootnote{جریان سیناپسی}\LTRfootnote{Synaptic Flow} سریع ارزیابی می‌کند و جهت جست‌وجو را در \persianfootnote{حلقه‌های بازتاب}\LTRfootnote{Reflection Loops} اصلاح می‌نماید؛ نتایج آن‌ها در \lr{CIFAR-10} و \lr{ImageNet} همگرایی سریع‌تر و معماری‌های بهتری نسبت به خطوط پایهٔ مبتنی بر مدل زبانی نشان می‌دهد. به‌طور مشابه، \cite{sarah2024llamaNAS} با تکیه بر جست‌وجوی بی‌گرادیان و \persianfootnote{قیود آگاه به سخت‌افزار}\LTRfootnote{Hardware-aware Constraints}، نشان داده‌اند که نمونه‌برداری و ارزیابی تکرارشونده بدون آموزش کامل می‌تواند هزینهٔ جست‌وجو را کاهش دهد و دقت رقابتی را حفظ کند.

این راهبردها \emph{کارایی محاسباتی} را به‌طور چشمگیری بهبود می‌بخشند و برای کاربردهایی که منابع محدود دارند یا نیاز به همگرایی سریع دارند، مناسب هستند. با این حال، دقت جانشین‌های کم‌هزینه در پیش‌بینی کارایی نهایی همواره محدودیتی اساسی است و ممکن است در برخی حوزه‌ها به معماری‌های زیربهینه منجر شود.

\subsubsection{جمع‌بندی و تحلیل تطبیقی رویکردهای تک‌عاملی}
رویکردهای تک‌عاملی \emph{سادگی معماری} و \emph{هزینه پیاده‌سازی کمتر} را ارائه می‌دهند، زیرا سربار هماهنگی میان عوامل را حذف می‌کنند. آن‌ها به‌ویژه در سناریوهایی که فضای جست‌وجو نسبتاً ساده است یا منابع محاسباتی محدود هستند، عملکرد خوبی دارند~\cite{zhang2023usingLLMforHPO,zheng2023GENIUS,ji2025RZNAS}. با این حال، در وظایف چندبُعدی و پیچیده که از تنوع دیدگاه‌ها یا فروکاست ماژولی سود می‌برند، ممکن است دچار کاستی شوند. همچنین، کیفیت جست‌وجو به‌شدت به ظرفیت استدلالی مدل پایه وابسته است و در صورت ضعف مدل در استدلال یا تولید کُد، کل سامانه آسیب‌پذیر می‌شود.

\subsection{رویکردهای چندعاملی}
معماری‌های چندعاملی از تفکیک مسئولیت‌ها و تخصص‌گرایی برای مدیریت پیچیدگی بهینه‌سازی بهره می‌گیرند. این رویکردها را می‌توان بر اساس \emph{الگوی سازمان‌دهی} و \emph{نحوه تعامل میان عوامل} به سه دسته اصلی تقسیم کرد: (۱) معماری‌های مبتنی بر تخصص نقشی، (۲) رویکردهای مبتنی بر ارکستراسیون سلسله‌مراتبی، و (۳) چارچوب‌های تلفیقی با الگوریتم‌های بهینه‌سازی کلاسیک.

\subsubsection{تخصص نقشی و جریان کاری مشارکتی}
رایج‌ترین الگوی سازمان‌دهی در معماری‌های چندعاملی، تفکیک مسئولیت‌ها به نقش‌های تخصصی است که معمولاً شامل \emph{برنامه‌ریزی}، \emph{کُدزایی/طراحی}، \emph{ارزیابی} و \emph{نقادی} می‌شود. \cite{xu2024largeTextToML} چارچوبی را ارائه کرده‌اند که توصیف‌های زبانی وظایف را به خط‌لوله‌های اجرایی ترجمه می‌کند و نقش‌های برنامه‌ریز، کُدزا و ارزیاب را میان عوامل توزیع می‌نماید؛ آزمایش‌های آن‌ها سودمندی هم‌افزایی عوامل را در سناریوهای بی‌نمونه نشان می‌دهد. به‌طور مشابه، \cite{liu2025agenthpo} مدلی گفت‌وگومحور را معرفی کرده‌اند که در آن عوامل پیشنهاددهنده، ارزیاب و بهینه‌ساز در حلقه‌های تکرارشونده همکاری می‌کنند و بدون نیاز به کُدزایی، تنظیمی رقابتی در شبکه‌های عمیق ارائه می‌دهند. \cite{trirat2025automlagent} این رویکرد را به کل خط لوله \lr{AutoML} گسترش داده و از عوامل تخصصی در پیش‌پردازش داده، طراحی مدل و بهینه‌سازی بهره می‌گیرند و نسبت به روش‌های تک‌عاملی کارایی برتری را در معیارهای \lr{OpenML} گزارش کرده‌اند.

\emph{مزیت کلیدی} این رویکردها در توانایی \emph{پالایش ماژولی مشارکتی} است که هر عامل می‌تواند بر جنبه خاصی از مسئله تمرکز کند و از تداخل یا تضاد میان اهداف مختلف جلوگیری شود. علاوه بر این، تخصص نقشی امکان بهره‌گیری از مدل‌های مختلف یا راهبردهای پرامپتینگ متفاوت برای هر نقش را فراهم می‌کند که می‌تواند کیفیت کلی را بهبود بخشد.

\subsubsection{ارکستراسیون سلسله‌مراتبی و یکپارچه‌سازی ابزار}
دسته دوم از روش‌های چندعاملی بر \emph{سازمان‌دهی سلسله‌مراتبی} و \emph{یکپارچه‌سازی ابزارهای بیرونی} تمرکز دارند. \cite{shen2023HuggingGPT} سامانه‌ای را معرفی کرده‌اند که در آن یک عامل مرکزی (\lr{ChatGPT}) وظایف پیچیده را به زیروظایف تفکیک می‌کند و با عوامل تخصصی که از مدل‌های \lr{Hugging Face} از طریق \lr{API} بهره می‌گیرند، هماهنگ می‌شود؛ ارزیابی‌های آن‌ها در وظایف چندوجهی نشان می‌دهد که برنامه‌ریزی عاملی و اتصال ابزارها می‌تواند خودکارسازی کارای جریان‌های پیچیده را میسر سازد. به‌طور مشابه، \cite{zhang-etal-2024-MLCopilot} چارچوبی مکالمه‌محور ارائه کرده‌اند که عوامل تحلیل داده، انتخاب مدل و رفع خطا را با یکپارچه‌سازی دانش بیرونی به‌کار می‌گیرند و نشان می‌دهند که همکاری میان عوامل حتی در فضاهای جست‌وجوی محدود نیز حل مسئله را ارتقاء می‌بخشد. \cite{Yang2025NADER} با معرفی رویکردی برای جست‌وجوی معماری که طراحی سلسله‌مراتبی را از طریق ساختارهای درختی پیش می‌برد، نشان داده‌اند که مذاکره میان عوامل و اشتراک دانش می‌تواند مقیاس‌پذیری جست‌وجو را در وظایف بینایی رایانه‌ای بهبود بخشد.

این معماری‌های سلسله‌مراتبی \emph{مقیاس‌پذیری} و \emph{قابلیت ترکیب}\LTRfootnote{Composability} بالاتری ارائه می‌دهند، زیرا عوامل جدید یا ابزارهای بیرونی را می‌توان به‌راحتی به سامانه افزود. با این حال، آن‌ها نیازمند \emph{سازوکارهای هماهنگی پیچیده‌تر} هستند و احتمال خطا در ارتباطات میان‌عاملی و انتشار خطا در سلسله‌مراتب افزایش می‌یابد.

\subsubsection{تلفیق با بهینه‌سازی بیزی و جست‌وجوی هدایت‌شده}
دسته سوم از رویکردهای چندعاملی به‌دنبال ترکیب مزایای استدلال زبانی با دقت الگوریتم‌های بهینه‌سازی کلاسیک هستند. \cite{liu2024LLAMBO} با ادغام بهینه‌سازی بیزی در سامانه‌ای چندعاملی نشان داده‌اند که عوامل می‌توانند \persianfootnote{مدل‌سازی جانشین}\LTRfootnote{Surrogate Modeling} و \persianfootnote{تابع اکتساب}\LTRfootnote{Acquisition Function} را بهبود بخشند و با کاهش شمار ارزیابی‌ها، حلقهٔ بیزی را به‌طور کارآمد راهبری کنند. این رویکرد ترکیبی نشان می‌دهد که استدلال زبانی می‌تواند به‌عنوان \emph{اکتشافی سطح بالا}\LTRfootnote{High-level Heuristic} برای هدایت الگوریتم‌های بهینه‌سازی کلاسیک به‌کار رود و تعادلی میان بهره‌برداری و کاوش برقرار کند.

این رویکردهای تلفیقی \emph{دقت بالاتر} و \emph{ضمانت‌های همگرایی بهتر} نسبت به روش‌های خالصاً مبتنی بر مدل زبانی ارائه می‌دهند، زیرا از اصول ریاضی محکم الگوریتم‌های کلاسیک بهره می‌برند. با این حال، آن‌ها پیچیدگی پیاده‌سازی بالاتر و نیاز به تخصص در هر دو حوزه مدل‌های زبانی و بهینه‌سازی را به‌همراه دارند.

\subsubsection{جمع‌بندی و تحلیل تطبیقی رویکردهای چندعاملی}
رویکردهای چندعاملی \emph{تاب‌آوری} و \emph{توان مدیریت پیچیدگی} بالاتری نسبت به معماری‌های تک‌عاملی ارائه می‌دهند~\cite{xu2024largeTextToML,trirat2025automlagent,liu2024LLAMBO}. تفکیک مسئولیت‌ها امکان بهره‌گیری از تخصص‌گرایی را فراهم می‌کند و می‌تواند به معماری‌های بهتر و راه‌حل‌های جامع‌تر منجر شود. علاوه بر این، معماری‌های چندعاملی \emph{انعطاف‌پذیری} بیشتری دارند، زیرا هر عامل می‌تواند از مدل، راهبرد پرامپتینگ یا منابع دانش متفاوتی استفاده کند. با این حال، آن‌ها با چالش‌هایی روبه‌رو هستند: (۱) \emph{سربار هماهنگی} که هزینه محاسباتی و زمانی را افزایش می‌دهد، (۲) \emph{پیچیدگی طراحی} در تعریف نقش‌ها و پروتکل‌های ارتباطی، و (۳) \emph{احتمال ناهمخوانی} در تعاملات میان عوامل که ممکن است به تصمیمات متناقض منجر شود.

\subsection{تحلیل تطبیقی و چشم‌انداز}
\label{subsec:agent-comparative}

مقایسه جامع رویکردهای تک‌عاملی و چندعاملی چندین الگوی کلیدی را آشکار می‌سازد. \textbf{اولاً}، رویکردهای تک‌عاملی در \emph{سناریوهای با فضای جست‌وجوی محدود} یا \emph{وظایف تک‌هدفه} عملکرد رقابتی ارائه می‌دهند و در عین حال هزینه پیاده‌سازی و محاسباتی کمتری دارند~\cite{zhang2023usingLLMforHPO,zheng2023GENIUS,ji2025RZNAS}. با این حال، با افزایش پیچیدگی وظیفه، محدودیت‌های آن‌ها در کاوش چندبُعدی برجسته‌تر می‌شود. \textbf{ثانیاً}، رویکردهای چندعاملی در \emph{وظایف چندوجهی} و \emph{خطوط لوله پیچیده} برتری دارند، زیرا می‌توانند مسئولیت‌ها را به‌طور مؤثر تفکیک کنند و از تخصص‌گرایی بهره بگیرند~\cite{trirat2025automlagent,liu2024LLAMBO,Yang2025NADER}. \textbf{ثالثاً}، رویکردهای ترکیبی که مدل‌های زبانی را با الگوریتم‌های کلاسیک (تکاملی یا بیزی) ادغام می‌کنند، تعادل بهتری میان کارایی محاسباتی و کیفیت جست‌وجو برقرار می‌کنند~\cite{LLMatic2024,chen2023Evoprompting,liu2024LLAMBO}.

این طبقه‌بندیِ عامل‌محور بر گرایش فزاینده به سوی \emph{ساختارهای سازگار}\LTRfootnote{Adaptive Architectures} دلالت دارد که می‌توانند به‌صورت پویا میان معماری‌های تک‌عاملی و چندعاملی جابجا شوند. مسیرهای تحقیقاتی آینده می‌توانند شامل: (۱) \emph{سازوکارهای جابجایی پویا} که بر اساس ویژگی‌های وظیفه (پیچیدگی، ابعاد فضای جست‌وجو، منابع محاسباتی) به‌طور خودکار معماری مناسب را انتخاب کنند، (۲) \emph{معماری‌های سلسله‌مراتبی ترکیبی} که در آن یک عامل هماهنگ‌کننده سطح بالا، چندین زیرعامل تخصصی را مدیریت کند و در صورت لزوم به حالت تک‌عاملی بازگردد، (۳) \emph{یادگیری تقویتی برای بهینه‌سازی تعامل میان عوامل} تا پروتکل‌های ارتباطی و تخصیص نقش‌ها به‌طور خودکار بهینه شود، و (۴) \emph{بهره‌گیری از مدل‌های متفاوت برای عوامل مختلف} به‌گونه‌ای که مدل‌های تخصصی‌تر و کارآمدتر برای هر نقش به‌کار گرفته شوند باشد.

\section[تحلیل منابع دانش]{\persianfootnote{تحلیل منابع دانش}\LTRfootnote{Knowledge Source Analysis}}

کارآمدی سامانه‌های خودکارسازی یادگیری ماشین مبتنی بر مدل‌های زبانی بزرگ به‌صورت بنیادین به چگونگی اکتساب، مدیریت و بهره‌برداری از دانش در سراسر فرایند \persianfootnote{بهینه‌سازی}\LTRfootnote{optimization} وابسته است. رویکردهای معاصر طیفی از راهبردهای \persianfootnote{تقویت دانش}\LTRfootnote{knowledge augmentation} را پوشش می‌دهند؛ از \persianfootnote{یادگیری درون‌متنی}\LTRfootnote{In-Context Learning} صرف تا چارچوب‌های \persianfootnote{تولید تقویت‌شده با بازیابی}\LTRfootnote{Retrieval-Augmented Generation (RAG)} که هر یک پیامدهای متمایزی برای کارایی \persianfootnote{جست‌وجو}\LTRfootnote{search} و \persianfootnote{تعمیم‌پذیری}\LTRfootnote{generalization} دارند.

\subsection[دانش درونی: تاریخچهٔ آزمون و بازتاب]{\persianfootnote{دانش درونی: تاریخچهٔ آزمون و بازتاب}\LTRfootnote{System-Internal Knowledge: Trials and Reflection}}
این رسته، دانشِ تولیدشده توسط خود سامانه را در بر می‌گیرد—از اتکای مستقیم به تاریخچهٔ آزمون‌ها در پنجرهٔ \persianfootnote{زمینه}\LTRfootnote{context window} تا حافظهٔ رویدادی و \persianfootnote{خودبازتابی}\LTRfootnote{self-reflection} که به‌تدریج به \persianfootnote{بازخوردِ قابلِ اقدام}\LTRfootnote{actionable feedback} و اصول طراحی تقطیر می‌شوند.
\subsubsection{\persianfootnote{یادگیری درون‌متنی از تاریخچهٔ بهینه‌سازی}\protect\LTRfootnote{In-Context Learning from Optimization History}}

یک راهبرد رایج، پنجرهٔ \persianfootnote{زمینه}\LTRfootnote{context window} مدل را همچون مخزن اصلی دانش در نظر می‌گیرد و صرفاً بر تاریخچهٔ آزمون‌های انباشته‌شده طی فرایند \persianfootnote{بهینه‌سازی}\LTRfootnote{optimization} تکیه می‌کند. سامانه‌های مبتنی بر این ایده، پیکربندی‌های پیشین و سنجه‌های کارایی متناظر را در دستور می‌گنجانند تا مدل بتواند از رهگذر بازخورد تکرارشونده پیشنهادها را پالایش کند \cite{zhang2023usingLLMforHPO, zheng2023GENIUS, liu2024LLAMBO}. این تاریخچه ممکن است به قالب‌های گوناگونی \persianfootnote{سریال‌سازی}\LTRfootnote{serialization} شود: گفت‌وگوهای \persianfootnote{سبک‌گپ}\LTRfootnote{chat-style dialogues} که توالی زمانی تعاملات را حفظ می‌کنند، \persianfootnote{خلاصه‌های فشرده}\LTRfootnote{compressed summaries} برای مهار قیود طول زمینه، \persianfootnote{الگوهای اندک‌نمونه}\LTRfootnote{few-shot demonstrations} برگزیده از جمعیت ارزیابی‌شده \cite{zhang2023usingLLMforHPO, chen2023Evoprompting}، و نیز \persianfootnote{تجربه‌های تاریخیِ استانداردسازی‌شده}\LTRfootnote{canonicalized historical experiences} که داده‌های ناهمگنِ گذشته (پیکربندی‌های \lr{JSON}، پیاده‌سازی‌های کد، سنجه‌های عددی) را به نمایش‌های یکنواختِ زبان طبیعی تبدیل می‌کنند تا پردازش و قیاس توسط مدل تسهیل شود \cite{zhang-etal-2024-MLCopilot}.

در این رویکردِ تکمیلی، ابرپارامترهای عددی برای بهبود استدلال \persianfootnote{گسسته‌سازی}\LTRfootnote{discretized} می‌شوند (مثلاً به سطوح «کم»، «متوسط»، «زیاد»)، و تجربه‌های استانداردسازی‌شده با \persianfootnote{نهفتارسازی}\LTRfootnote{embedding} و نمایه‌سازی در پایگاهِ برداری پشتیبانی می‌شوند تا \persianfootnote{بازیابی مبتنی بر شباهت}\LTRfootnote{similarity-based retrieval}، \persianfootnote{برترین‌های k}\LTRfootnote{top-k} نمونهٔ مرتبط را برای نمایش درون‌متنی برگزیند. فراتر از درجِ خامِ تجربه‌ها، \persianfootnote{استخراج دانش برون‌خط}\LTRfootnote{offline knowledge elicitation} از طریق خلاصه‌سازیِ تکرارشوندهٔ مبتنی بر مدل و \persianfootnote{اعتبارسنجیِ پسین}\LTRfootnote{post-validation} روی وظایف کنارگذاشته، اصول طراحیِ سطح‌بالا و \persianfootnote{بازخوردِ قابلِ اقدام}\LTRfootnote{actionable feedback} را تقطیر می‌کند؛ این دانشِ استخراج‌شده می‌تواند به‌صورت راهنمای سامانه یا الگوهای اندک‌نمونه در متن تزریق شود و حتی \persianfootnote{تولید راه‌حل تک‌نمونه‌ای}\LTRfootnote{one-shot solution generation} برای وظایف نو را میسر سازد \cite{zhang-etal-2024-MLCopilot}.

این پارادایم به‌ویژه در سناریوهای کم‌بودجه مؤثر است؛ جایی که پیشین‌های یادگرفته‌شدهٔ مدل و تجربه‌های استانداردسازی‌شده می‌توانند بدون دادهٔ تجربی فراوان مسیر اکتشاف را هدایت کنند. در بسترهای \persianfootnote{بهینه‌سازی بیزی}\LTRfootnote{Bayesian optimization}، مشاهدات تاریخی، آغازِ گرم، \persianfootnote{نمونه‌برداری نامزد}\LTRfootnote{candidate sampling} و \persianfootnote{مدلسازی جانشین}\LTRfootnote{surrogate modeling} را به‌طور کامل از راه \persianfootnote{سریال‌سازی زبان طبیعی}\LTRfootnote{natural language serialization} ارزیابی‌های پیشین شرطی‌سازی می‌کنند؛ و در حالی‌که در حالت‌های \persianfootnote{بی‌نمونه}\LTRfootnote{zero-shot} یا \persianfootnote{کم‌نمونه}\LTRfootnote{few-shot} عمل می‌کنند، کارایی رقابتی در قیاس با روش‌های سنتی نشان می‌دهند \cite{liu2024LLAMBO}. تجربه‌های استانداردسازی‌شده با فراهم‌سازی نمونه‌های مشابهِ تأییدشده و قواعد طراحیِ تقطیرشده، می‌توانند این مراحل را دقیق‌تر \persianfootnote{گرم‌آغاز}\LTRfootnote{warmstart} کنند و میدان جست‌وجو را به‌صورت هدایت‌شده منقبض سازند \cite{zhang-etal-2024-MLCopilot}.

محدودیت اصلی در بهینه‌سازی‌های \persianfootnote{بلندافق}\LTRfootnote{long-horizon} رخ می‌نماید؛ جایی‌که قیود پنجرهٔ زمینه مستلزم نگهداشت گزینشی یا \persianfootnote{فشرده‌سازی ازدست‌ده}\LTRfootnote{lossy compression} تاریخچه است و چه‌بسا الگوهای حیاتی برای پالایشِ مرحلهٔ پایانی را حذف کند. \persianfootnote{استانداردسازی}\LTRfootnote{canonicalization} تا حدی این معضل را با خلاصه‌های ساخت‌یافتهٔ متراکم و بازیابی هدفمندِ \lr{top-k} تخفیف می‌دهد، اما همچنان با برش اطلاعاتی، سوگیری‌های ناشی از گسسته‌سازی، و حساسیت به \persianfootnote{امتیازدهیِ ربط}\LTRfootnote{relevance scoring} و پوشش مخزن مواجه است. با این‌همه، چون مصرف نهاییِ این دانش درونِ همان پنجرهٔ زمینه صورت می‌گیرد، مرز میان «دانش درون‌متنیِ صرف» و «تقویتِ مبتنی بر بازیابی» کم‌رنگ‌تر می‌شود؛ و ادغامِ تاریخچهٔ آزمون با تجربه‌های استانداردسازی‌شده، پایایی و کاراییِ یادگیری درون‌متنی را در عمل ارتقا می‌دهد \cite{zhang2023usingLLMforHPO, zheng2023GENIUS, chen2023Evoprompting, liu2024LLAMBO, zhang-etal-2024-MLCopilot}.

% \subsubsection{\persianfootnote{حافظهٔ رویدادی از رهگذر درخت‌های تغییر}\protect\LTRfootnote{Episodic Memory via Modification Trees}}

% در بسترهای تکاملیِ \persianfootnote{جست‌وجوی معماری عصبی}\LTRfootnote{Neural Architecture Search (NAS)}، برخی چارچوب‌ها \persianfootnote{درخت‌های تغییرِ شبکه}\LTRfootnote{network modification trees} را نگه می‌دارند که کل مسیر دگرگونی‌های معماری—از روابط والد-فرزند و دگرگونی‌های اعمال‌شده تا سنجه‌های کارایی حاصل—را بایگانی می‌کنند \cite{Yang2025NADER}. این حافظهٔ رویدادی، راهبردهای کاوشِ \persianfootnote{عمق‌نخست}\LTRfootnote{depth-first} و \persianfootnote{عرض‌نخست}\LTRfootnote{breadth-first} را پشتیبانی می‌کند؛ و سازوکارهای بازیابی با جست‌وجوی شباهت، تغییرات مشابهِ گذشته را می‌یابند. در ترکیب با \persianfootnote{کارگزارِ بازتابنده}\LTRfootnote{reflector agent} که الگوهای خطای مکرر و راهبردهای موفقِ طراحی را از سوابق استخراج می‌کند، سامانه بی‌نیاز از دادهٔ آموزشی بیرونی، به‌تدریج سرانگشتانِ حوزه‌ویژه می‌پرورد.

% \subsubsection{\persianfootnote{بایگانی‌های کیفیت-تنوع به‌منزلهٔ حافظه}\protect\LTRfootnote{Quality-Diversity Archives as Memory}}

% در چارچوب‌هایی که مدل‌های زبانی را با \persianfootnote{بهینه‌سازیِ کیفیت-تنوع}\LTRfootnote{quality-diversity optimization} ادغام می‌کنند، دو بایگانی به‌منزلهٔ حافظهٔ بلندمدت عمل می‌کنند: یکی برای نگهداری معماری‌های برگزیدهٔ شبکه با \persianfootnote{توصیفگرهای رفتاری}\LTRfootnote{behavioral descriptors} (مانند نسبت پهنا-به-عمق، و \persianfootnote{تعداد عملیات ممیز شناور}\LTRfootnote{FLOPs})؛ و دیگری برای بایگانیِ راهنماهای متنی با \persianfootnote{دما}\LTRfootnote{temperature}ها و \persianfootnote{نمراتِ کنجکاوی}\LTRfootnote{curiosity scores} متناظر \cite{LLMatic2024}. پیگیریِ مشترکِ سنجه‌های \persianfootnote{برازش}\LTRfootnote{fitness} و \persianfootnote{بداعت}\LTRfootnote{novelty} در نسل‌های پیاپی، تعادلی میان بهره‌برداری از برترین‌های شناخته‌شده و اکتشاف نواحی کم‌نمونه‌برداری‌شده برقرار می‌کند؛ و دانش ازپیش‌آموختهٔ مدل دربارهٔ \persianfootnote{پیکره‌های کد}\LTRfootnote{code corpora}، فرایند جهش را به‌طور ضمنی—حتی بدون بازیابی صریح—تقویت می‌کند.

\subsection[دانش بیرونی: بازیابی از ادبیات و مخازن]{\persianfootnote{دانش بیرونی: بازیابی از ادبیات و مخازن}\LTRfootnote{External Knowledge via Retrieval}}

سامانه‌های پیشرفته‌تر، تولید تقویت‌شده با بازیابی را برای ادغام دانش بیرون از مسیر بهینه‌سازی به کار می‌گیرند. این چارچوب‌ها مخازن دانش ساخت‌یافته—عموماً \persianfootnote{پایگاه‌های دادهٔ برداری}\LTRfootnote{vector databases} نمایه‌شده با \persianfootnote{نهفتارها}\LTRfootnote{embeddings}—نگه می‌دارند تا بر پایهٔ زمینهٔ وظیفهٔ جاری، اطلاعات مرتبط را بازیابی کرده و در راهنماهای متنی تزریق کنند و بدین‌سان تصمیم‌سازی را اطلاع‌رسانی کنند.

\subsubsection{\persianfootnote{استخراج دانش مبتنی بر ادبیات پژوهشی}\protect\LTRfootnote{Literature-Driven Knowledge Extraction}}

چند رویکرد، دانش راهبردی را از ادبیات علمی برای هدایت تصمیم‌های معماری گردآوری می‌کنند. یک راهبرد از \persianfootnote{کارگزاران خوانشِ تخصصی}\LTRfootnote{specialized reader agents} بهره می‌برد که مقالات اخیر را \persianfootnote{خزش}\LTRfootnote{crawl} کرده، نکته‌های روش‌شناختی را از چکیده‌ها و بخش‌های روش استخراج می‌کنند و در پایگاه‌های دادهٔ برداری برای \persianfootnote{بازیابی مبتنی بر شباهت}\LTRfootnote{similarity-based retrieval} بایگانی می‌نمایند \cite{Yang2025NADER}. در خلال بهینه‌سازی، \persianfootnote{پیشنهادهای تغییر}\LTRfootnote{modification proposals} به‌منزلهٔ پرسش، اصول طراحی مرتبط را فراخوانی می‌کنند؛ و بدین‌ترتیب سامانه بدون آن‌که مدل پایه الزاماً بر تازه‌ترین انتشارات آموزش دیده باشد، از مرز دانش روز بهره می‌گیرد. نمونه‌ای دیگر، خلاصه‌هایی از مقالات arXiv و جست‌وجوهای وب را از طریق \persianfootnote{رابط‌های برنامه‌نویسی کاربردی}\LTRfootnote{APIs} بازیابی کرده و راهنماهای برنامه‌ریزی را با بینش‌های بیرونی پیرامون مدل‌ها، ابرپارامترها و داده‌مجموعه‌ها غنی می‌کند تا تنوع و سازگاری طرح را ارتقا دهد \cite{trirat2025automlagent}.

این استخراج دانش، برای \persianfootnote{طراحی معماری‌های عصبی}\LTRfootnote{neural architecture design} بس سودمند است؛ چراکه نوآوری‌های اخیر در \persianfootnote{ترکیب لایه‌ها}\LTRfootnote{layer compositions}، \persianfootnote{اتصالات پرشی}\LTRfootnote{skip connections} یا \persianfootnote{طرحواره‌های نرمال‌سازی}\LTRfootnote{normalization schemes} چه‌بسا در \persianfootnote{پارامترهای منجمد}\LTRfootnote{frozen parameters} مدل بازتاب نیافته باشند. با این‌همه، کیفیت دانش بازیابی‌شده به‌نحو حساس به سازوکار \persianfootnote{امتیازدهیِ ربط}\LTRfootnote{relevance scoring} و پوشش پیکرهٔ ادبیات نمایه‌شده وابسته است.

\subsubsection{\persianfootnote{مخازن داده‌مجموعه و مدل}\protect\LTRfootnote{Dataset and Model Repositories}}

در کنار بازیابی مبتنی بر ادبیات، چند سامانه از مخازن بیرونی برای فراداده‌های داده‌مجموعه‌ها و مدل‌های ازپیش‌آموزش‌دیده پرس‌وجو می‌کنند. چارچوب‌هایی که کل زنجیرهٔ \persianfootnote{خودکارسازی یادگیری ماشین}\LTRfootnote{AutoML} را راهبری می‌کنند، \persianfootnote{کارت‌های داده‌مجموعه}\LTRfootnote{dataset cards} از پلتفرم‌هایی مانند \lr{Kaggle} و \persianfootnote{کارت‌های مدل}\LTRfootnote{model cards} از \lr{HuggingFace} را بازیابی کرده و فرادادهٔ ساخت‌یافته—از جمله \persianfootnote{وجه‌های داده}\LTRfootnote{modalities}، متغیرهای هدف، معماری‌های مدل و \persianfootnote{بازه‌های ابرپارامتر}\LTRfootnote{hyperparameter ranges}—را در راهنماهای متنی می‌گنجانند تا تصمیم‌های پایین‌دستی را غنی کنند \cite{trirat2025automlagent, shen2023HuggingGPT}. برای داده‌مجموعه‌های نادیده، سازوکارهای \persianfootnote{انتقال مبتنی بر شباهت}\LTRfootnote{similarity-based transfer} با محاسبهٔ همبستگی میان \persianfootnote{کدگذاریِ کارت‌های داده}\LTRfootnote{data card encodings} (با مدل‌هایی مانند \lr{CLIP}) مسائل مشابه را شناسایی کرده و ابرپارامترها یا الگوهای معماری را از تجربه‌های تاریخی منتقل می‌سازند \cite{zhang2023AutomlGPTAutomaticMachineLearning}.

این \persianfootnote{تقویتِ مبتنی بر فراداده}\LTRfootnote{metadata-driven augmentation} تعمیم‌پذیری میان حوزه‌های گوناگون را بدون نیاز به آموزش‌های خاصِ وظیفه ممکن می‌کند؛ هرچند به دسترس‌پذیری مخازن خوش‌سامان و برچسب‌گذاری دقیق فراداده متکی است.

\subsection[راهبردهای تقویتِ آمیخته]{\persianfootnote{راهبردهای تقویتِ آمیخته}\LTRfootnote{Hybrid Augmentation Strategies}}

سامانه‌های پیشرفتهٔ روز، غالباً چند منبع دانش را برای بهره‌گیری از قوت‌های مکمل با هم ترکیب می‌کنند. چارچوب‌های \persianfootnote{چندکارگزاره}\LTRfootnote{multi-agent frameworks} ممکن است \persianfootnote{برنامه‌ریزیِ تقویت‌شده با بازیابی}\LTRfootnote{retrieval-augmented planning}—که دانش ادبیات و مخازن را برای راهبردهای سطح‌بالا به کار می‌گیرد—را با \persianfootnote{حافظهٔ رویدادی}\LTRfootnote{episodic memory} برآمده از گزارش‌های تجربی که اجرای عملی را صیقل می‌دهد، جفت کنند \cite{trirat2025automlagent, Yang2025NADER}. به‌همین سیاق، تاریخچهٔ آزمونِ درون‌متنی می‌تواند با تجربه‌های استانداردسازی‌شدهٔ بازیابی‌شده غنی شود تا \persianfootnote{گرم‌آغاز}\LTRfootnote{warmstart} بهینه‌سازی را—به‌ویژه در مواجهه با وظایف نو با ارزیابی‌های اولیهٔ محدود—تسریع کند \cite{zhang-etal-2024-MLCopilot}.

گزینش معماریِ تقویت دانش، به‌طرزِ حساس با پارادایم عملیاتی کارگزار برهم‌کنش دارد: \persianfootnote{پرومپت‌دهیِ تکرارشونده}\LTRfootnote{iterative prompting} بیشترین سود را از \persianfootnote{خلاصه‌های فشردهٔ تاریخی}\LTRfootnote{compressed historical summaries} می‌برد؛ \persianfootnote{عملگرهای تکاملی}\LTRfootnote{evolutionary operators} برای نگهداشت تنوع به بایگانی‌های کیفیت-تنوع اتکا دارند؛ و \persianfootnote{کنترل‌گرهای جریان‌کار}\LTRfootnote{workflow controllers} برای هماهنگ‌سازی مراحل ناهمگونِ خط لوله، به فرادادهٔ ساخت‌یافته نیازمندند. با گسترش ظرفیت‌های پنجرهٔ زمینه و پختگیِ سازوکارهای بازیابی، مرز میان دانش درون‌متنی و بیرونی هرچه بیشتر محو می‌شود و ادغامی غنی‌تر از پیشین‌های آموخته، شواهد تجربی و خبرگیِ حوزه را امکان‌پذیر می‌سازد.

\section{تحلیل بر مبنای قالب خروجی مدل}

\subsection{خروجی‌های سبک واژنامه‌ای}

\persianfootnote{نمایش‌های ساخت‌یافته کلید–مقدار}\LTRfootnote{structured key-value representations} کدگذاری بی‌ابهام فراپارامترها یا انتخاب‌های معماری را با خوانایی ماشینی مستقیم فراهم می‌کنند. سامانه‌هایی که پیکربندی‌های قالب JSON تولید می‌کنند، پارامترهایی مانند نرخ یادگیری، اندازه دسته و ابعاد شبکه را مشخص می‌سازند \cite{zhang2023usingLLMforHPO, liu2025agenthpo}. گونه‌های پیشرفته، استدلال \persianfootnote{زنجیره تفکر}\LTRfootnote{chain-of-thought} را پیش از خروجی ساخت‌یافته می‌گنجانند تا کیفیت پیشنهادها را با استدلال میانی صریح بهبود دهند، در حالی‌که مشخصات نهایی همچنان قابل تجزیه باقی می‌ماند. محدودیت اصلی، مقیدشدن اکتشاف به \persianfootnote{طرحواره‌های ازپیش‌تعریف‌شده}\LTRfootnote{predefined schemas} است که می‌تواند کشف الگوهای طراحی نو را محدود کند. شکل \ref{fig:hpo-config} نمونه‌ای از خروجی قالب JSON را نشان می‌دهد.
\begin{figure}[h!]
    \centering
    \includegraphics[width=0.9\textwidth]{images/hpo-config.png}
    \caption[نمونه ای از خروجی قالب واژنامه‌ای]{نمونه‌ای از خروجی قالب JSON برای پیکربندی بهینه‌سازی فراپارامتر در سامانه مبتنی بر مدل زبانی بزرگ \cite{zhang2023usingLLMforHPO}
    }
    \label{fig:hpo-config}
\end{figure}
\subsection{تولید کد برنامه}

خروجی مبتنی بر کد با تکیه بر بیان‌پذیری کامل زبان‌های برنامه‌نویسی، از قیود فضاهای پیکربندی ازپیش‌تعریف‌شده می‌گریزد. چندین سامانه پیاده‌سازی‌های Python از شبکه‌های عصبی و خطوط لوله یادگیری ماشین را به‌صورت برنامه‌های کامل یا \persianfootnote{مولفه‌های ماژولار}\LTRfootnote{modular components} تولید می‌کنند و با آزمون‌های واحد خودکار سازگاری را می‌سنجند \cite{xu2024largeTextToML, LLMatic2024, chen2023Evoprompting}. برخی، تولید را بر معیارهای هدف با دستوردهی چندنمونه و نمونه‌هایی برگرفته از جمعیت‌های ارزیابی‌شده شرطی می‌کنند؛ برخی دیگر، عملگرهای تکاملی مانند جهش و ترکیب را مستقیماً بر نمایش‌های کدی اعمال می‌کنند. گونه‌های ترکیبی، تولید کد برای پیاده‌سازی معماری را با فهرست‌های ساخت‌یافته فراپارامتر و متن قالب‌بندی‌شده \persianfootnote{گزارش وقایع}\LTRfootnote{log} آموزشی پیش‌بینی‌شده درمی‌آمیزند \cite{zhang2023AutomlGPTAutomaticMachineLearning, trirat2025automlagent}. شکل \ref{fig:evoprompting} نمونه‌ای از خروجی تولید کد را نشان می‌دهد.
\begin{figure}[h!]
    \centering
    \includegraphics[width=0.9\textwidth]{images/evoprompting.png}
    \caption[نمونه ای از خروجی تولید کد]{
        نمونه‌ای از خروجی تولید کد در سامانه \lr{Evoprompting} که از مدل زبانی بزرگ برای تولید و بهینه‌سازی کدهای Python شبکه‌های عصبی استفاده می‌کند \cite{chen2023Evoprompting}
    }
    \label{fig:evoprompting}
\end{figure}
تولید کد، انعطاف طراحی را به حداکثر می‌رساند و امکان جست‌وجو در معماری‌های نامقید را بدون تعریف صریح اجزای ابتدایی فراهم می‌آورد. بااین‌حال، این رویکرد چالش‌های اعتبارسنجی به همراه دارد: \persianfootnote{درستی نحوی}\LTRfootnote{syntactic correctness} لزوماً آموزش‌پذیری، رعایت \persianfootnote{قیود منابع}\LTRfootnote{resource constraints} یا \persianfootnote{معناداری معنایی}\LTRfootnote{semantic meaningfulness} را تضمین نمی‌کند. ازاین‌رو، سامانه‌ها به محیط‌های اجرا برای ارزیابی نیاز دارند و سازوکارهای مدیریت خطا را برای مواجهه با خروجی‌های نامعتبر پیاده می‌کنند؛ از جمله \persianfootnote{دستوردهی مجدد}\LTRfootnote{re-prompting} با استفاده از پیام‌های خطا به‌عنوان بازخورد.

\subsection{خروجی‌های درخت‌ساختار}

\persianfootnote{نمایش‌های گراف/درخت}\LTRfootnote{graph or tree representations} بر روابط ترکیبی درون معماری‌ها تأکید می‌کنند و برای وظایفی که به \persianfootnote{تعیین صریح توپولوژی}\LTRfootnote{explicit topology specification} نیاز دارند سودمندند؛ بی‌آن‌که جزئیات پیاده‌سازی کدی که می‌تواند از استدلال ساختاری منحرف کند تحمیل شود. برخی سامانه‌ها نمایش متنی \persianfootnote{گراف جهت‌دار بدون‌دور}\LTRfootnote{Directed Acyclic Graph (DAG)} با گره‌های شماره‌گذاری‌شده برای عملیات و اتصالات برمی‌گزینند \cite{Yang2025NADER}؛ برخی دیگر از \persianfootnote{کدگذاری رشته‌ای جداکننده‌محور}\LTRfootnote{delimited string encodings} برای ویژگی‌های لایه سازگار با \persianfootnote{تولید خودبازگشتی}\LTRfootnote{autoregressive generation} استفاده می‌کنند \cite{Yu2025GPTNAS}. خروجی‌ها معمولاً با فرایندهای تجزیه تخصصی به کد اجرایی تبدیل می‌شوند و ابزارهای راستی‌آزمایی \persianfootnote{گراف محاسباتی}\LTRfootnote{computational graph} علاوه بر صحت نحوی، \persianfootnote{همریختی}\LTRfootnote{isomorphism} با طرح‌های موجود را نیز می‌سنجند. شکل \ref{fig:nader-tree} نمونه‌ای از خروجی درخت‌ساختار را نشان می‌دهد.
\begin{figure}[h!]
    \centering
    \includegraphics[width=0.9\textwidth]{images/nader-tree.png}
    \caption[نمونه ای از خروجی درخت ساختار]{
        نمونه‌ای از نمایش گراف‌محور معماری شبکه عصبی. سمت چپ: تصویرسازی گراف جهت‌دار بدون‌دور . سمت راست: نمایش متنی گراف جهت‌دار بدون‌دور برای فهم مدل زبانی بزرگ. \cite{Yang2025NADER}
    }
    \label{fig:nader-tree}
\end{figure}
این قالب‌ها استدلال ترکیبی و راهبردهای تغییر سلسله‌مراتبی را تسهیل می‌کنند و به مدل امکان می‌دهند بر توپولوژی معماری مستقل از جزئیات پیاده‌سازی تمرکز کند. به‌کارگیری آن‌ها به \persianfootnote{طرح‌های کدگذاری حوزه‌ای}\LTRfootnote{domain-specific encoding schemes} و رویه‌های اعتبارسنجی ویژه نیاز دارد، اما با کاهش پیچیدگی وظیفه تولید از طریق سطح تجرید مناسب، کیفیت تولید را بهبود می‌بخشد.

\subsection{خروجی‌های ترکیبی}

سامانه‌های پیشرفته، چندین \persianfootnote{گونه خروجی}\LTRfootnote{output modalities} را ترکیب می‌کنند تا از قوت‌های مکمل قالب‌های مختلف در مراحل گوناگون خط لوله بهره ببرند. الگوی رایج، مشخصات ساخت‌یافته را با توضیحات زبان طبیعی همراه می‌کند تا هم اجرای ماشینی و هم تفسیر انسانی میسر شود \cite{liu2025agenthpo, zhang2023usingLLMforHPO}. طرح‌های مفصل‌تر برای مراحل متمایز از قالب‌های متفاوت بهره می‌گیرند: JSON برای نیازمندی‌های قابل اعتبارسنجی صوری، زبان طبیعی برای برنامه‌ریزی و تحلیل منعطف، و کد اجرایی برای پیاده‌سازی‌های نهایی \cite{trirat2025automlagent, zhang2023AutomlGPTAutomaticMachineLearning}. برخی سامانه‌ها کدگذاری‌های ساختاری فشرده را در توضیحات زبان طبیعی می‌گنجانند تا مشخصات دقیق را با استدلال‌های قابل تفسیر تلفیق کنند \cite{ji2025RZNAS, Yang2025NADER}. شکل \ref{fig:rznas} نمونه‌ای از خروجی ترکیبی را نشان می‌دهد.
\begin{figure}[h!]
    \centering
    \includegraphics[width=0.9\textwidth]{images/rznas.png}
    \caption[نمونه ای از خروجی ترکیبی واژنامه‌ای و کد]{
        نمونه‌ای از خروجی ترکیبی در سامانه \lr{RZNAS} که از قالب‌های JSON و کد Python برای طراحی معماری شبکه عصبی استفاده می‌کند \cite{ji2025RZNAS}
    }
    \label{fig:rznas}
\end{figure}
این روش بازتاب این واقعیت است که هیچ قالب یگانه‌ای به‌تنهایی برای همه جنبه‌های یادگیری ماشین خودکار بهینه نیست: خروجی ساخت‌یافته برای تجزیه و اعتبارسنجی مناسب است؛ کد، پیاده‌سازی منعطف را ممکن می‌سازد؛ گراف‌ها استدلال ترکیبی را تقویت می‌کنند؛ و زبان طبیعی تفسیرپذیری و زمینه غنی را فراهم می‌کند و با موارد دشوار صوری‌سازی روبه‌رو می‌شود. ادغام موفق مستلزم طراحی دقیق \persianfootnote{گذارهای بین قالب‌ها}\LTRfootnote{format transitions}، رویه‌های اعتبارسنجی \persianfootnote{میان‌گونه‌ای}\LTRfootnote{across modalities} و راهبردهایی برای مدیریت \persianfootnote{ناهمخوانی}\LTRfootnote{inconsistencies} در صورت تعارض نمایش‌ها است.
% \section{طبقه بندی بر اساس دانش خارجی}
% \subsection{با استفاده از تولید تقویت شده با بازیابی}
% \subsection{با استفاده از ابزار های جستجو}
% \section{طبقه بندی بر اساس نوع خروجی مدل}
% \subsection{خروجی بصورت واژه‌نامه}
% \subsection{خروجی بصورت کد برنامه}
% \subsection{خروجی بصورت درخت}
% \subsection{خروجی بصورت ترکیبی از واژه‌نامه و کد برنامه}
\section{طبقه بندی کار های مرتبط}
همانطور که در جدول \ref{tab:recent-works} مشاهده می‌شود، اکثر روش‌های بررسی شده از مدل‌های زبانی بزرگ چندعاملی استفاده می‌کنند. این رویکرد به دلیل توانایی در تقسیم وظایف و همکاری بین عوامل مختلف، معمولاً عملکرد بهتری را در مسائل پیچیده ارائه می‌دهد. همچنین، بیشتر روش‌های جدید از دانش خارجی بهره می‌برند که می‌تواند به بهبود دقت و کارایی مدل کمک کند. در زمینه نوع خروجی، روش‌های متنوعی وجود دارد که بسته به نیاز مسئله، می‌توانند انتخاب شوند. برای مثال، خروجی بصورت کد برنامه برای مسائل نیازمند پیاده‌سازی عملی مناسب‌تر است، در حالی که خروجی بصورت واژه‌نامه ممکن است برای مسائل تحلیلی کاربردی‌تر باشد. این تنوع در رویکردها نشان‌دهنده انعطاف‌پذیری و قابلیت تطبیق مدل‌های زبانی بزرگ با نیازهای مختلف در حوزه جستجوی معماری شبکه عصبی و بهینه‌سازی ابرپارامتر است.
\begin{table}[t]
    \centering
    \footnotesize
    \setlength{\tabcolsep}{3pt}
    \renewcommand{\arraystretch}{1.2}
    \begin{tabularx}{\textwidth}{@{} Y c c c c c c c l @{}}
        \toprule
        عنوان                                                            & عامل & روش قدیمی & ایجاد کد                      & ابزار  & دانش خارجی   & فضای جستجو & بدون آموزش & وظیفه        \\
        \midrule
        \lr{LLM for HPO}\cite{zhang2023usingLLMforHPO}             & تک   & —         & \xmark                        & \xmark & \xmark & اختیاری    & —          & \lr{HPO}\    \\
        \lr{GENIUS}\cite{zheng2023GENIUS}                                & تک   & —         & \xmark                        & \xmark & \xmark & بله        & —          & \lr{NAS}     \\
        \lr{LLMATIC}\cite{LLMatic2024}                                   & تک   & EA        & \cmark                        & \xmark & \xmark & خیر        & —          & \lr{NAS}     \\
        \lr{Text to ML}\cite{xu2024largeTextToML}                        & چند  & —         & \cmark                        & \xmark & \xmark & خیر        & \cmark     & \lr{AutoML}  \\
        \lr{AgentHPO}\cite{liu2025agenthpo}                              & چند  & —         & \xmark                        & \cmark & \xmark & بله        & —          & \lr{HPO}     \\
        \lr{AutoML-Agent}\cite{trirat2025automlagent}                    & چند  & —         & \cmark                        & \xmark & \cmark & خیر        & \cmark     & \lr{AutoML}  \\
        \lr{LLAMBO}\cite{liu2024LLAMBO}                                  & چند  & BO        & \xmark                        & \xmark & \xmark & بله        & \cmark     & \lr{HPO}     \\
        \lr{Nader}\cite{Yang_2025_NADER}                                 & چند  & —         & \xmark\textsuperscript{\dag}  & \cmark & \cmark & بله        & —          & \lr{NAS} \\
        \lr{RZ-NAS}\cite{ji2025RZNAS}                                    & تک   & EA        & \cmark\textsuperscript{\ddag} & \xmark & \xmark & بله        & \cmark     & \lr{NAS}     \\
        \lr{EvoPrompting}\cite{chen2023Evoprompting}                     & تک   & EA        & \cmark                        & \xmark & \xmark & خیر        & —          & \lr{NAS}     \\
        \lr{AutoML-GPT}\cite{zhang2023AutomlGPTAutomaticMachineLearning} & تک   & —         & \cmark                        & \xmark & \xmark & بله        & —          & \lr{AutoML}  \\
        \lr{HuggingGPT}\cite{shen2023HuggingGPT}                         & چند  & —         & \cmark                        & \xmark & \cmark & خیر        & —          & \lr{AutoML}  \\
        \lr{GPT-NAS}\cite{Yu2025GPTNAS}                                  & تک   & EA        & —                             & \xmark & \xmark & خیر        & —          & \lr{NAS}     \\
        \lr{ML Copilot}\cite{zhang-etal-2024-MLCopilot}                  & چند  & —         & \xmark                        & \xmark & \cmark & خیر        & —          & \lr{AutoML}  \\
        \bottomrule
    \end{tabularx}
    \caption[مقایسهٔ فشردهٔ مقالات مبتنی بر \lr{LLM}]{مقایسهٔ فشردهٔ مقالات مبتنی بر \lr{LLM}. \lr{EA}=الگوریتم‌های تکاملی، \lr{BO}=بهینه‌سازی بیزی. نشانه‌ها: \textsuperscript{\dag}=ساختار درختی به‌جای تولید کد؛ \textsuperscript{\ddag}=تولید کد + تنظیمات.}
    \label{tab:recent-works}
\end{table}
