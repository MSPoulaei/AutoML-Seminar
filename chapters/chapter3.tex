\chapter{مروری بر کارهای مرتبط}
\thispagestyle{empty}

\section{مقدمه}
\subsection{ظهور اولین روش های یادگیری ماشین خودکار}
\subsection{روش های یادگیری ماشین خودکار در عصر مدل های زبانی بزرگ}

\section{مقایسه بر اساس عامل}
در این بخش، به مقایسه روش‌های مختلف بر اساس نوع عامل (multi-agent یا single-agent) پرداخته می‌شود. این مقایسه می‌تواند به درک بهتر مزایا و معایب هر رویکرد کمک کند و زمینه‌ساز انتخاب روش مناسب برای مسائل خاص باشد.

\section{مقایسه بر اساس دانش خارجی}
\subsection{با استفاده از تولید تقویت شده با بازیابی}
\subsection{با استفاده از ابزار های جستجو}
\section{مقایسه بر اساس نوع خروجی مدل}
\subsection{خروجی بصورت واژه‌نامه}
\subsection{خروجی بصورت کد برنامه}
\subsection{خروجی بصورت درخت}
\subsection{خروجی بصورت ترکیبی از واژه‌نامه و کد برنامه}
\section{مقایسه کار های مرتبط}
همانطور که در جدول \ref{tab:recent-works} مشاهده می‌شود، اکثر روش‌های بررسی شده از مدل‌های زبانی بزرگ چندعاملی (multi-agent) استفاده می‌کنند. این رویکرد به دلیل توانایی در تقسیم وظایف و همکاری بین عوامل مختلف، معمولاً عملکرد بهتری را در مسائل پیچیده ارائه می‌دهد. همچنین، بیشتر روش‌های جدید از دانش خارجی بهره می‌برند که می‌تواند به بهبود دقت و کارایی مدل کمک کند. در زمینه نوع خروجی، روش‌های متنوعی وجود دارد که بسته به نیاز مسئله، می‌توانند انتخاب شوند. برای مثال، خروجی بصورت کد برنامه برای مسائل نیازمند پیاده‌سازی عملی مناسب‌تر است، در حالی که خروجی بصورت واژه‌نامه ممکن است برای مسائل تحلیلی کاربردی‌تر باشد. این تنوع در رویکردها نشان‌دهنده انعطاف‌پذیری و قابلیت تطبیق مدل‌های زبانی بزرگ با نیازهای مختلف در حوزه جستجوی معماری شبکه عصبی و بهینه‌سازی ابرپارامتر است.
\begin{table}[t]
    \centering
    \footnotesize
    \setlength{\tabcolsep}{3pt}
    \renewcommand{\arraystretch}{1.2}
    \begin{tabularx}{\textwidth}{@{} Y c c c c c c c l @{}}
        \toprule
        عنوان                                                            & عامل & روش قدیمی & ایجاد کد                      & ابزار  & دانش خارجی   & فضای جستجو & بدون آموزش & وظیفه        \\
        \midrule
        \lr{LLM for HPO}\cite{zhang2023usingLLMforHPO}             & تک   & —         & \xmark                        & \xmark & \xmark & اختیاری    & —          & \lr{HPO}\    \\
        \lr{GENIUS}\cite{zheng2023GENIUS}                                & تک   & —         & \xmark                        & \xmark & \xmark & بله        & —          & \lr{NAS}     \\
        \lr{LLMATIC}\cite{LLMatic2024}                                   & تک   & EA        & \cmark                        & \xmark & \xmark & خیر        & —          & \lr{NAS}     \\
        \lr{Text to ML}\cite{xu2024largeTextToML}                        & چند  & —         & \cmark                        & \xmark & \xmark & خیر        & \cmark     & \lr{AutoML}  \\
        \lr{AgentHPO}\cite{liu2025agenthpo}                              & چند  & —         & \xmark                        & \cmark & \xmark & بله        & —          & \lr{HPO}     \\
        \lr{AutoML-Agent}\cite{trirat2025automlagent}                    & چند  & —         & \cmark                        & \xmark & \cmark & خیر        & \cmark     & \lr{AutoML}  \\
        \lr{LLAMBO}\cite{liu2024LLAMBO}                                  & چند  & BO        & \xmark                        & \xmark & \xmark & بله        & \cmark     & \lr{HPO}     \\
        \lr{Nader}\cite{Yang_2025_NADER}                                 & چند  & —         & \xmark\textsuperscript{\dag}  & \cmark & \cmark & بله        & —          & \lr{NAS} \\
        \lr{RZ-NAS}\cite{ji2025RZNAS}                                    & تک   & EA        & \cmark\textsuperscript{\ddag} & \xmark & \xmark & بله        & \cmark     & \lr{NAS}     \\
        \lr{EvoPrompting}\cite{chen2023Evoprompting}                     & تک   & EA        & \cmark                        & \xmark & \xmark & خیر        & —          & \lr{NAS}     \\
        \lr{AutoML-GPT}\cite{zhang2023AutomlGPTAutomaticMachineLearning} & تک   & —         & \cmark                        & \xmark & \xmark & بله        & —          & \lr{AutoML}  \\
        \lr{HuggingGPT}\cite{shen2023HuggingGPT}                         & چند  & —         & \cmark                        & \xmark & \cmark & خیر        & —          & \lr{AutoML}  \\
        \lr{GPT-NAS}\cite{Yu2025GPTNAS}                                  & تک   & EA        & —                             & \xmark & \xmark & خیر        & —          & \lr{NAS}     \\
        \lr{ML Copilot}\cite{zhang-etal-2024-MLCopilot}                  & چند  & —         & \xmark                        & \xmark & \cmark & خیر        & —          & \lr{AutoML}  \\
        \bottomrule
    \end{tabularx}
    \caption[مقایسهٔ فشردهٔ مقالات مبتنی بر \lr{LLM}]{مقایسهٔ فشردهٔ مقالات مبتنی بر \lr{LLM}. \lr{EA}=الگوریتم‌های تکاملی، \lr{BO}=بهینه‌سازی بیزی. نشانه‌ها: \textsuperscript{\dag}=ساختار درختی به‌جای تولید کد؛ \textsuperscript{\ddag}=تولید کد + تنظیمات.}
    \label{tab:recent-works}
\end{table}
