\chapter{مروری بر کارهای مرتبط}
\thispagestyle{empty}

\section{مقدمه}
\subsection{ظهور اولین روش های یادگیری ماشین خودکار}
\subsection{روش های یادگیری ماشین خودکار در عصر مدل های زبانی بزرگ}


\section{تحلیل بر مبنای معماری عامل}
\subsection{سامانه‌های تک‌عاملی}

بیشینهٔ سامانه‌های مبتنی بر مدل‌های زبانی بزرگ برای خودکارسازی یادگیری ماشین، معماری‌های تک‌عاملی را برمی‌گزینند که در آن یک عامل منفرد کل جریان بهینه‌سازی را مدیریت می‌کند. بر مبنای چگونگی ادغام مدل زبانی در فرایند بهینه‌سازی، این سامانه‌ها در چند پارادایم عملیاتی متمایز قرار می‌گیرند.

\subsubsection{\persianfootnote{بهینه‌سازی مستقیم از طریق دستوردهی تکراری}\protect\LTRfootnote{Direct Optimization through Iterative Prompting}}
در این رویکرد برجسته، مدل‌های زبانی به‌منزلهٔ بهینه‌سازهای \persianfootnote{جعبه‌سیاه}\LTRfootnote{black-box} به‌کار می‌روند که پیکربندی‌ها را پیشنهاد می‌کنند و از راه \persianfootnote{حلقه‌های بازخورد}\LTRfootnote{feedback loops} آن‌ها را پالایش می‌کنند. مدل با تکیه بر تاریخچهٔ \persianfootnote{آزمون‌ها}\LTRfootnote{trial} که به‌صورت \persianfootnote{گفت‌وگوهای چت}\LTRfootnote{chat-style dialogues} یا خلاصه‌های فشرده انباشته می‌شود، \persianfootnote{زمینه}\LTRfootnote{context} را حفظ می‌کند و پالایش تکراری مبتنی بر معیارهای اعتبارسنجی را ممکن می‌سازد \cite{zhang2023usingLLMforHPO, zheng2023GENIUS}. این پارادایم به چارچوب‌های بهینه‌سازی \persianfootnote{بیزی}\LTRfootnote{Bayesian} نیز بسط یافته است؛ جایی که مدل‌های آغاز گرم، نمونه‌گیری از نامزدها و \persianfootnote{مدل‌سازی جانشین}\LTRfootnote{surrogate modeling} را با اتکاء به استدلال زبان طبیعی و مشروط بر تاریخچهٔ بهینه‌سازی انجام می‌دهند \cite{liu2024LLAMBO}. این رویکردها در تنظیمات با بودجه کم کارایی رقابتی نشان می‌دهند و بی‌آن‌که به \persianfootnote{ریزتنظیم}\LTRfootnote{fine-tuning} نیاز داشته باشند، بر \persianfootnote{یادگیری زمینه ای}\LTRfootnote{in-context learning} از توصیف مسئله و بازخورد تجربی تکیه می‌کنند.

\subsubsection{\persianfootnote{عملگرهای تکاملی}\protect\LTRfootnote{Evolutionary Operators}}
راهبردی دیگر، مدل زبانی را به‌عنوان عملگرهای \persianfootnote{جهش}\LTRfootnote{mutation} یا \persianfootnote{ترکیب}\LTRfootnote{crossover} درون چارچوب‌های \persianfootnote{تکاملی}\LTRfootnote{evolutionary} جا می‌دهد. به‌جای جایگزینی الگوریتم‌های جست‌وجوی سنتی، این سامانه‌ها آن‌ها را با تنوع‌های تولیدشده توسط مدل زبانی تقویت می‌کنند. مدل می‌تواند تغییرات معماری مبتنی بر کد را در چارچوب‌های \persianfootnote{کیفیت–تنوع}\LTRfootnote{quality-diversity} بسازد \cite{LLMatic2024} یا به‌صورت عملگرهای \persianfootnote{تطبیقی}\LTRfootnote{adaptive operators} که میان نسل‌ها با \persianfootnote{تنظیم دستور}\LTRfootnote{prompt-tuning} پالایش می‌شوند عمل کند \cite{chen2023Evoprompting}. برخی پیاده‌سازی‌ها \persianfootnote{توانایی‌های بازتابی}\LTRfootnote{reflective capabilities} را نیز می‌گنجانند؛ به این معنا که مدل پیامدهای جهش را تحلیل می‌کند و \persianfootnote{بازخورد زبانی}\LTRfootnote{linguistic feedback} برای هدایت تکرارهای بعدی تولید می‌کند \cite{ji2025RZNAS}. این ادغام، \persianfootnote{پایداری}\LTRfootnote{robustness} جست‌وجوی تکاملی را حفظ می‌کند و در عین حال از \persianfootnote{خلاقیت}\LTRfootnote{creativity} مدل در تولید تنوع‌های معنادار بهره می‌گیرد. نمونه‌ای شاخص، GPT-NAS است که در آن \persianfootnote{مدل زبانی}\LTRfootnote{GPT} به‌منزلهٔ یک \persianfootnote{بازساز}\LTRfootnote{reconstructor} معماری عمل کرده و با ماسک‌گذاری و بازتولید لایه‌ها، نامزدهای نمونه‌برداری‌شده توسط \persianfootnote{الگوریتم ژنتیک}\LTRfootnote{genetic algorithm} را بهبود می‌دهد؛ بنابراین عملاً نقشی هم‌ارز با یک عملگر جهشِ آگاه از زمینه ایفا می‌کند بی‌آن‌که راهبرد جست‌وجوی تکاملی را جایگزین کند \cite{Yu2025GPTNAS}.

\subsubsection{\persianfootnote{کنترل‌گرهای جریان کار}\protect\LTRfootnote{Workflow Controllers}}
در این پارادایم، مدل‌های زبانی به‌مثابه \persianfootnote{هماهنگ‌کننده}\LTRfootnote{orchestrator} برای مدیریت اجزای \persianfootnote{خطّ لوله}\LTRfootnote{pipeline} به کار می‌روند. سامانه با ترکیب دستور‌هایی که \persianfootnote{فرادادهٔ ساختاریافته}\LTRfootnote{structured metadata}-از جمله \persianfootnote{کارت‌های داده}\LTRfootnote{data cards} و \persianfootnote{کارت‌های مدل}\LTRfootnote{model cards}-را در خود دارند، مدل را در مراحل پیاپی از پردازش داده، انتخاب مدل تا تنظیم فراپارامتر هدایت می‌کند \cite{zhang2023AutomlGPTAutomaticMachineLearning, shen2023HuggingGPT}. برخی پیاده‌سازی‌ها برنامه‌های پیچیدهٔ یادگیری ماشین را به \persianfootnote{مولفه‌های ماژولار}\LTRfootnote{modular components} تجزیه می‌کنند که به‌طور جداگانه تولید و با \persianfootnote{آزمون‌های واحد خودکار}\LTRfootnote{automated unit tests} راستی‌آزمایی می‌شوند تا سازگاری تضمین گردد \cite{xu2024largeTextToML}. این رویکرد با شکستن خطوط لولهٔ طولانی و ناهمگون به زیروظایف قابل مدیریت و اتکاء به \persianfootnote{دستوردهی زمینه‌مند}\LTRfootnote{contextual prompting}، انسجام کلّی را حفظ می‌کند.


\subsection{سامانه‌های چندعاملی}

معماری‌های چندعاملی با تفکیک نقش، زیروظایف یادگیری ماشین خودکار را میان عامل‌هایی با قابلیت‌های مکمل توزیع می‌کنند. این سامانه‌ها از \persianfootnote{رهگذر تفکیک کارکردی}\LTRfootnote{functional decomposition} و \persianfootnote{همکاری بین‌عاملی}\LTRfootnote{inter-agent collaboration}، مدیریت پیچیدگی و استدلال پیچیده‌تر را ممکن می‌سازند.

\subsubsection{\persianfootnote{همکاری مبتنی بر نقش}\protect\LTRfootnote{Role-Based Collaboration}}
الگوی پایه، دو عامل تخصصی با مسئولیت‌های متمایز را به‌کار می‌گیرد. در یک ساختار، تولید \persianfootnote{پیکربندی}\LTRfootnote{configuration} از اجرای تجربی جدا می‌شود: عامل سازنده نیازمندی‌ها را تفسیر و پیکربندی‌های پیشنهادی همراه با استدلال ارائه می‌کند و عامل اجراکننده آموزش را انجام داده و نتایج را در گزارش‌های مشترک می‌گنجاند تا چرخه‌های بعدی پیشنهادهای سازنده را تغذیه کند \cite{liu2025agenthpo}. این تقسیم کار بازتاب گردش‌کار متخصصان است و با \persianfootnote{حافظهٔ ترتیبی}\LTRfootnote{episodic memory} انباشته، به عملکرد خودگردان بدون مداخلهٔ انسانی می‌انجامد.

\subsubsection{\persianfootnote{هماهنگی سلسله‌مراتبی}\protect\LTRfootnote{Hierarchical Coordination}}
در ساختارهای پیچیده‌تر، چند عامل تخصصی تحت نظارت یک \persianfootnote{مدیر عامل}\LTRfootnote{Agent Manager} سازمان می‌یابند. مدیر با \persianfootnote{استدلال تقویت‌شده با بازیابی}\LTRfootnote{retrieval-augmented reasoning} طرح‌های متنوعی می‌سازد، آن‌ها را به زیروظایف \persianfootnote{قابل موازی‌سازی}\LTRfootnote{parallelizable subtasks} واگشایی و به عامل مناسب تخصیص می‌دهد و از رهگذر راستی‌آزمایی چندمرحله‌ای و \persianfootnote{حلقه‌های بازنگری}\LTRfootnote{revision loops} نتایج را اعتبارسنجی می‌کند \cite{trirat2025automlagent}. این معماری کلّ خط لولهٔ AutoML را از بازیابی داده تا \persianfootnote{استقرار}\LTRfootnote{deployment} به‌صورت کارآمد پیش می‌برد و \persianfootnote{وابستگی‌های بین‌گامی}\LTRfootnote{inter-step dependencies} را با \persianfootnote{پروتکل‌های هماهنگی ساختاریافته}\LTRfootnote{structured coordination protocols} مدیریت می‌کند.

\subsubsection{\persianfootnote{تیم‌های پژوهش–توسعه}\protect\LTRfootnote{Research-Development Teams}.}
پیشرفته‌ترین ساختار سازمانی، عامل‌ها را در تیم‌های کارکردی تقسیم می‌کند \cite{Yang2025NADER}. \persianfootnote{تیم پژوهش}\LTRfootnote{Research Team} دانش را از \persianfootnote{ادبیات پژوهشی}\LTRfootnote{literature} استخراج می‌کند و با اتکاء به بینش‌های بازیابی‌شده، پیشنهادهای تغییر را می‌سازد؛ \persianfootnote{تیم توسعه}\LTRfootnote{Development Team} این تغییرها را بر \persianfootnote{نمایش‌های گراف}\LTRfootnote{graph representations} اعمال کرده و هم بازخورد فوری و هم استخراج تجربهٔ بلندمدت را فراهم می‌کند. گفت‌وگوهای چندمرحله‌ای میان تیم‌ها یادگیری مستمر از تاریخچهٔ طراحی را ممکن می‌کند و \persianfootnote{پایگاه‌های دادهٔ برداری}\LTRfootnote{vector databases} با \persianfootnote{بازیابی مبتنی بر شباهت}\LTRfootnote{similarity-based retrieval}، دانش و تجربه‌های گذشتهٔ مرتبط را برای هدایت اکتشاف فراخوانی می‌کنند.

پارادایم چندعاملی با افزودن پیچیدگی هماهنگی، در برابر \persianfootnote{توضیح‌پذیری}\LTRfootnote{interpretability} بهبود‌یافته از طریق تفکیک صریح نقش‌ها و کارایی بهتر بهینه‌سازی از رهگذر استدلال تخصصی معامله می‌کند؛ هرچند به مدیریت حافظه پیشرفته برای حفظ سازگاری در تعاملات عامل‌ها و طراحی دقیق پروتکل‌های ارتباطی برای جلوگیری از شکست‌های هماهنگی نیاز دارد.

\section[تحلیل منابع دانش]{\persianfootnote{تحلیل منابع دانش}\LTRfootnote{Knowledge Source Analysis}}

کارآمدی سامانه‌های خودکارسازی یادگیری ماشین مبتنی بر مدل‌های زبانی بزرگ به‌صورت بنیادین به چگونگی اکتساب، مدیریت و بهره‌برداری از دانش در سراسر فرایند \persianfootnote{بهینه‌سازی}\LTRfootnote{optimization} وابسته است. رویکردهای معاصر طیفی از راهبردهای \persianfootnote{تقویت دانش}\LTRfootnote{knowledge augmentation} را پوشش می‌دهند؛ از \persianfootnote{یادگیری درون‌متنی}\LTRfootnote{In-Context Learning} صرف تا چارچوب‌های \persianfootnote{تولید تقویت‌شده با بازیابی}\LTRfootnote{Retrieval-Augmented Generation (RAG)} که هر یک پیامدهای متمایزی برای کارایی \persianfootnote{جست‌وجو}\LTRfootnote{search} و \persianfootnote{تعمیم‌پذیری}\LTRfootnote{generalization} دارند.

\subsection[دانش درونی: تاریخچهٔ آزمون و بازتاب]{\persianfootnote{دانش درونی: تاریخچهٔ آزمون و بازتاب}\LTRfootnote{System-Internal Knowledge: Trials and Reflection}}
این رسته، دانشِ تولیدشده توسط خود سامانه را در بر می‌گیرد—از اتکای مستقیم به تاریخچهٔ آزمون‌ها در پنجرهٔ \persianfootnote{زمینه}\LTRfootnote{context window} تا حافظهٔ رویدادی و \persianfootnote{خودبازتابی}\LTRfootnote{self-reflection} که به‌تدریج به \persianfootnote{بازخوردِ قابلِ اقدام}\LTRfootnote{actionable feedback} و اصول طراحی تقطیر می‌شوند.
\subsubsection{\persianfootnote{یادگیری درون‌متنی از تاریخچهٔ بهینه‌سازی}\protect\LTRfootnote{In-Context Learning from Optimization History}}

یک راهبرد رایج، پنجرهٔ \persianfootnote{زمینه}\LTRfootnote{context window} مدل را همچون مخزن اصلی دانش در نظر می‌گیرد و صرفاً بر تاریخچهٔ آزمون‌های انباشته‌شده طی فرایند \persianfootnote{بهینه‌سازی}\LTRfootnote{optimization} تکیه می‌کند. سامانه‌های مبتنی بر این ایده، پیکربندی‌های پیشین و سنجه‌های کارایی متناظر را در دستور می‌گنجانند تا مدل بتواند از رهگذر بازخورد تکرارشونده پیشنهادها را پالایش کند \cite{zhang2023usingLLMforHPO, zheng2023GENIUS, liu2024LLAMBO}. این تاریخچه ممکن است به قالب‌های گوناگونی \persianfootnote{سریال‌سازی}\LTRfootnote{serialization} شود: گفت‌وگوهای \persianfootnote{سبک‌گپ}\LTRfootnote{chat-style dialogues} که توالی زمانی تعاملات را حفظ می‌کنند، \persianfootnote{خلاصه‌های فشرده}\LTRfootnote{compressed summaries} برای مهار قیود طول زمینه، \persianfootnote{الگوهای اندک‌نمونه}\LTRfootnote{few-shot demonstrations} برگزیده از جمعیت ارزیابی‌شده \cite{zhang2023usingLLMforHPO, chen2023Evoprompting}، و نیز \persianfootnote{تجربه‌های تاریخیِ استانداردسازی‌شده}\LTRfootnote{canonicalized historical experiences} که داده‌های ناهمگنِ گذشته (پیکربندی‌های \lr{JSON}، پیاده‌سازی‌های کد، سنجه‌های عددی) را به نمایش‌های یکنواختِ زبان طبیعی تبدیل می‌کنند تا پردازش و قیاس توسط مدل تسهیل شود \cite{zhang-etal-2024-MLCopilot}.

در این رویکردِ تکمیلی، ابرپارامترهای عددی برای بهبود استدلال \persianfootnote{گسسته‌سازی}\LTRfootnote{discretized} می‌شوند (مثلاً به سطوح «کم»، «متوسط»، «زیاد»)، و تجربه‌های استانداردسازی‌شده با \persianfootnote{نهفتارسازی}\LTRfootnote{embedding} و نمایه‌سازی در پایگاهِ برداری پشتیبانی می‌شوند تا \persianfootnote{بازیابی مبتنی بر شباهت}\LTRfootnote{similarity-based retrieval}، \persianfootnote{برترین‌های k}\LTRfootnote{top-k} نمونهٔ مرتبط را برای نمایش درون‌متنی برگزیند. فراتر از درجِ خامِ تجربه‌ها، \persianfootnote{استخراج دانش برون‌خط}\LTRfootnote{offline knowledge elicitation} از طریق خلاصه‌سازیِ تکرارشوندهٔ مبتنی بر مدل و \persianfootnote{اعتبارسنجیِ پسین}\LTRfootnote{post-validation} روی وظایف کنارگذاشته، اصول طراحیِ سطح‌بالا و \persianfootnote{بازخوردِ قابلِ اقدام}\LTRfootnote{actionable feedback} را تقطیر می‌کند؛ این دانشِ استخراج‌شده می‌تواند به‌صورت راهنمای سامانه یا الگوهای اندک‌نمونه در متن تزریق شود و حتی \persianfootnote{تولید راه‌حل تک‌نمونه‌ای}\LTRfootnote{one-shot solution generation} برای وظایف نو را میسر سازد \cite{zhang-etal-2024-MLCopilot}.

این پارادایم به‌ویژه در سناریوهای کم‌بودجه مؤثر است؛ جایی که پیشین‌های یادگرفته‌شدهٔ مدل و تجربه‌های استانداردسازی‌شده می‌توانند بدون دادهٔ تجربی فراوان مسیر اکتشاف را هدایت کنند. در بسترهای \persianfootnote{بهینه‌سازی بیزی}\LTRfootnote{Bayesian optimization}، مشاهدات تاریخی، آغازِ گرم، \persianfootnote{نمونه‌برداری نامزد}\LTRfootnote{candidate sampling} و \persianfootnote{مدلسازی جانشین}\LTRfootnote{surrogate modeling} را به‌طور کامل از راه \persianfootnote{سریال‌سازی زبان طبیعی}\LTRfootnote{natural language serialization} ارزیابی‌های پیشین شرطی‌سازی می‌کنند؛ و در حالی‌که در حالت‌های \persianfootnote{بی‌نمونه}\LTRfootnote{zero-shot} یا \persianfootnote{کم‌نمونه}\LTRfootnote{few-shot} عمل می‌کنند، کارایی رقابتی در قیاس با روش‌های سنتی نشان می‌دهند \cite{liu2024LLAMBO}. تجربه‌های استانداردسازی‌شده با فراهم‌سازی نمونه‌های مشابهِ تأییدشده و قواعد طراحیِ تقطیرشده، می‌توانند این مراحل را دقیق‌تر \persianfootnote{گرم‌آغاز}\LTRfootnote{warmstart} کنند و میدان جست‌وجو را به‌صورت هدایت‌شده منقبض سازند \cite{zhang-etal-2024-MLCopilot}.

محدودیت اصلی در بهینه‌سازی‌های \persianfootnote{بلندافق}\LTRfootnote{long-horizon} رخ می‌نماید؛ جایی‌که قیود پنجرهٔ زمینه مستلزم نگهداشت گزینشی یا \persianfootnote{فشرده‌سازی ازدست‌ده}\LTRfootnote{lossy compression} تاریخچه است و چه‌بسا الگوهای حیاتی برای پالایشِ مرحلهٔ پایانی را حذف کند. \persianfootnote{استانداردسازی}\LTRfootnote{canonicalization} تا حدی این معضل را با خلاصه‌های ساخت‌یافتهٔ متراکم و بازیابی هدفمندِ \lr{top-k} تخفیف می‌دهد، اما همچنان با برش اطلاعاتی، سوگیری‌های ناشی از گسسته‌سازی، و حساسیت به \persianfootnote{امتیازدهیِ ربط}\LTRfootnote{relevance scoring} و پوشش مخزن مواجه است. با این‌همه، چون مصرف نهاییِ این دانش درونِ همان پنجرهٔ زمینه صورت می‌گیرد، مرز میان «دانش درون‌متنیِ صرف» و «تقویتِ مبتنی بر بازیابی» کم‌رنگ‌تر می‌شود؛ و ادغامِ تاریخچهٔ آزمون با تجربه‌های استانداردسازی‌شده، پایایی و کاراییِ یادگیری درون‌متنی را در عمل ارتقا می‌دهد \cite{zhang2023usingLLMforHPO, zheng2023GENIUS, chen2023Evoprompting, liu2024LLAMBO, zhang-etal-2024-MLCopilot}.

% \subsubsection{\persianfootnote{حافظهٔ رویدادی از رهگذر درخت‌های تغییر}\protect\LTRfootnote{Episodic Memory via Modification Trees}}

% در بسترهای تکاملیِ \persianfootnote{جست‌وجوی معماری عصبی}\LTRfootnote{Neural Architecture Search (NAS)}، برخی چارچوب‌ها \persianfootnote{درخت‌های تغییرِ شبکه}\LTRfootnote{network modification trees} را نگه می‌دارند که کل مسیر دگرگونی‌های معماری—از روابط والد-فرزند و دگرگونی‌های اعمال‌شده تا سنجه‌های کارایی حاصل—را بایگانی می‌کنند \cite{Yang2025NADER}. این حافظهٔ رویدادی، راهبردهای کاوشِ \persianfootnote{عمق‌نخست}\LTRfootnote{depth-first} و \persianfootnote{عرض‌نخست}\LTRfootnote{breadth-first} را پشتیبانی می‌کند؛ و سازوکارهای بازیابی با جست‌وجوی شباهت، تغییرات مشابهِ گذشته را می‌یابند. در ترکیب با \persianfootnote{کارگزارِ بازتابنده}\LTRfootnote{reflector agent} که الگوهای خطای مکرر و راهبردهای موفقِ طراحی را از سوابق استخراج می‌کند، سامانه بی‌نیاز از دادهٔ آموزشی بیرونی، به‌تدریج سرانگشتانِ حوزه‌ویژه می‌پرورد.

% \subsubsection{\persianfootnote{بایگانی‌های کیفیت-تنوع به‌منزلهٔ حافظه}\protect\LTRfootnote{Quality-Diversity Archives as Memory}}

% در چارچوب‌هایی که مدل‌های زبانی را با \persianfootnote{بهینه‌سازیِ کیفیت-تنوع}\LTRfootnote{quality-diversity optimization} ادغام می‌کنند، دو بایگانی به‌منزلهٔ حافظهٔ بلندمدت عمل می‌کنند: یکی برای نگهداری معماری‌های برگزیدهٔ شبکه با \persianfootnote{توصیفگرهای رفتاری}\LTRfootnote{behavioral descriptors} (مانند نسبت پهنا-به-عمق، و \persianfootnote{تعداد عملیات ممیز شناور}\LTRfootnote{FLOPs})؛ و دیگری برای بایگانیِ راهنماهای متنی با \persianfootnote{دما}\LTRfootnote{temperature}ها و \persianfootnote{نمراتِ کنجکاوی}\LTRfootnote{curiosity scores} متناظر \cite{LLMatic2024}. پیگیریِ مشترکِ سنجه‌های \persianfootnote{برازش}\LTRfootnote{fitness} و \persianfootnote{بداعت}\LTRfootnote{novelty} در نسل‌های پیاپی، تعادلی میان بهره‌برداری از برترین‌های شناخته‌شده و اکتشاف نواحی کم‌نمونه‌برداری‌شده برقرار می‌کند؛ و دانش ازپیش‌آموختهٔ مدل دربارهٔ \persianfootnote{پیکره‌های کد}\LTRfootnote{code corpora}، فرایند جهش را به‌طور ضمنی—حتی بدون بازیابی صریح—تقویت می‌کند.

\subsection[دانش بیرونی: بازیابی از ادبیات و مخازن]{\persianfootnote{دانش بیرونی: بازیابی از ادبیات و مخازن}\LTRfootnote{External Knowledge via Retrieval}}

سامانه‌های پیشرفته‌تر، \persianfootnote{تولید تقویت‌شده با بازیابی}\LTRfootnote{RAG} را برای ادغام دانش بیرون از مسیر بهینه‌سازی به کار می‌گیرند. این چارچوب‌ها مخازن دانش ساخت‌یافته—عموماً \persianfootnote{پایگاه‌های دادهٔ برداری}\LTRfootnote{vector databases} نمایه‌شده با \persianfootnote{نهفتارها}\LTRfootnote{embeddings}—نگه می‌دارند تا بر پایهٔ زمینهٔ وظیفهٔ جاری، اطلاعات مرتبط را بازیابی کرده و در راهنماهای متنی تزریق کنند و بدین‌سان تصمیم‌سازی را اطلاع‌رسانی کنند.

\subsubsection{\persianfootnote{استخراج دانش مبتنی بر ادبیات پژوهشی}\protect\LTRfootnote{Literature-Driven Knowledge Extraction}}

چند رویکرد، دانش راهبردی را از ادبیات علمی برای هدایت تصمیم‌های معماری گردآوری می‌کنند. یک راهبرد از \persianfootnote{کارگزاران خوانشِ تخصصی}\LTRfootnote{specialized reader agents} بهره می‌برد که مقالات اخیر را \persianfootnote{خزش}\LTRfootnote{crawl} کرده، نکته‌های روش‌شناختی را از چکیده‌ها و بخش‌های روش استخراج می‌کنند و در پایگاه‌های دادهٔ برداری برای \persianfootnote{بازیابی مبتنی بر شباهت}\LTRfootnote{similarity-based retrieval} بایگانی می‌نمایند \cite{Yang2025NADER}. در خلال بهینه‌سازی، \persianfootnote{پیشنهادهای تغییر}\LTRfootnote{modification proposals} به‌منزلهٔ پرسش، اصول طراحی مرتبط را فراخوانی می‌کنند؛ و بدین‌ترتیب سامانه بدون آن‌که مدل پایه الزاماً بر تازه‌ترین انتشارات آموزش دیده باشد، از مرز دانش روز بهره می‌گیرد. نمونه‌ای دیگر، خلاصه‌هایی از مقالات arXiv و جست‌وجوهای وب را از طریق \persianfootnote{رابط‌های برنامه‌نویسی کاربردی}\LTRfootnote{APIs} بازیابی کرده و راهنماهای برنامه‌ریزی را با بینش‌های بیرونی پیرامون مدل‌ها، ابرپارامترها و داده‌مجموعه‌ها غنی می‌کند تا تنوع و سازگاری طرح را ارتقا دهد \cite{trirat2025automlagent}.

این استخراج دانش، برای \persianfootnote{طراحی معماری‌های عصبی}\LTRfootnote{neural architecture design} بس سودمند است؛ چراکه نوآوری‌های اخیر در \persianfootnote{ترکیب لایه‌ها}\LTRfootnote{layer compositions}، \persianfootnote{اتصالات پرشی}\LTRfootnote{skip connections} یا \persianfootnote{طرحواره‌های نرمال‌سازی}\LTRfootnote{normalization schemes} چه‌بسا در \persianfootnote{پارامترهای منجمد}\LTRfootnote{frozen parameters} مدل بازتاب نیافته باشند. با این‌همه، کیفیت دانش بازیابی‌شده به‌نحو حساس به سازوکار \persianfootnote{امتیازدهیِ ربط}\LTRfootnote{relevance scoring} و پوشش پیکرهٔ ادبیات نمایه‌شده وابسته است.

\subsubsection{\persianfootnote{مخازن داده‌مجموعه و مدل}\protect\LTRfootnote{Dataset and Model Repositories}}

در کنار بازیابی مبتنی بر ادبیات، چند سامانه از مخازن بیرونی برای فراداده‌های داده‌مجموعه‌ها و مدل‌های ازپیش‌آموزش‌دیده پرس‌وجو می‌کنند. چارچوب‌هایی که کل زنجیرهٔ \persianfootnote{خودکارسازی یادگیری ماشین}\LTRfootnote{AutoML} را راهبری می‌کنند، \persianfootnote{کارت‌های داده‌مجموعه}\LTRfootnote{dataset cards} از پلتفرم‌هایی مانند \lr{Kaggle} و \persianfootnote{کارت‌های مدل}\LTRfootnote{model cards} از \lr{HuggingFace} را بازیابی کرده و فرادادهٔ ساخت‌یافته—از جمله \persianfootnote{وجه‌های داده}\LTRfootnote{modalities}، متغیرهای هدف، معماری‌های مدل و \persianfootnote{بازه‌های ابرپارامتر}\LTRfootnote{hyperparameter ranges}—را در راهنماهای متنی می‌گنجانند تا تصمیم‌های پایین‌دستی را غنی کنند \cite{trirat2025automlagent, shen2023HuggingGPT}. برای داده‌مجموعه‌های نادیده، سازوکارهای \persianfootnote{انتقال مبتنی بر شباهت}\LTRfootnote{similarity-based transfer} با محاسبهٔ همبستگی میان \persianfootnote{کدگذاریِ کارت‌های داده}\LTRfootnote{data card encodings} (با مدل‌هایی مانند \lr{CLIP}) مسائل مشابه را شناسایی کرده و ابرپارامترها یا الگوهای معماری را از تجربه‌های تاریخی منتقل می‌سازند \cite{zhang2023AutomlGPTAutomaticMachineLearning}.

این \persianfootnote{تقویتِ مبتنی بر فراداده}\LTRfootnote{metadata-driven augmentation} تعمیم‌پذیری میان حوزه‌های گوناگون را بدون نیاز به آموزش‌های خاصِ وظیفه ممکن می‌کند؛ هرچند به دسترس‌پذیری مخازن خوش‌سامان و برچسب‌گذاری دقیق فراداده متکی است.

\subsection[راهبردهای تقویتِ آمیخته]{\persianfootnote{راهبردهای تقویتِ آمیخته}\LTRfootnote{Hybrid Augmentation Strategies}}

سامانه‌های پیشرفتهٔ روز، غالباً چند منبع دانش را برای بهره‌گیری از قوت‌های مکمل با هم ترکیب می‌کنند. چارچوب‌های \persianfootnote{چندکارگزاره}\LTRfootnote{multi-agent frameworks} ممکن است \persianfootnote{برنامه‌ریزیِ تقویت‌شده با بازیابی}\LTRfootnote{retrieval-augmented planning}—که دانش ادبیات و مخازن را برای راهبردهای سطح‌بالا به کار می‌گیرد—را با \persianfootnote{حافظهٔ رویدادی}\LTRfootnote{episodic memory} برآمده از گزارش‌های تجربی که اجرای عملی را صیقل می‌دهد، جفت کنند \cite{trirat2025automlagent, Yang2025NADER}. به‌همین سیاق، تاریخچهٔ آزمونِ درون‌متنی می‌تواند با تجربه‌های استانداردسازی‌شدهٔ بازیابی‌شده غنی شود تا \persianfootnote{گرم‌آغاز}\LTRfootnote{warmstart} بهینه‌سازی را—به‌ویژه در مواجهه با وظایف نو با ارزیابی‌های اولیهٔ محدود—تسریع کند \cite{zhang-etal-2024-MLCopilot}.

گزینش معماریِ تقویت دانش، به‌طرزِ حساس با پارادایم عملیاتی کارگزار برهم‌کنش دارد: \persianfootnote{پرومپت‌دهیِ تکرارشونده}\LTRfootnote{iterative prompting} بیشترین سود را از \persianfootnote{خلاصه‌های فشردهٔ تاریخی}\LTRfootnote{compressed historical summaries} می‌برد؛ \persianfootnote{عملگرهای تکاملی}\LTRfootnote{evolutionary operators} برای نگهداشت تنوع به بایگانی‌های کیفیت-تنوع اتکا دارند؛ و \persianfootnote{کنترل‌گرهای جریان‌کار}\LTRfootnote{workflow controllers} برای هماهنگ‌سازی مراحل ناهمگونِ خط لوله، به فرادادهٔ ساخت‌یافته نیازمندند. با گسترش ظرفیت‌های پنجرهٔ زمینه و پختگیِ سازوکارهای بازیابی، مرز میان دانش درون‌متنی و بیرونی هرچه بیشتر محو می‌شود و ادغامی غنی‌تر از پیشین‌های آموخته، شواهد تجربی و خبرگیِ حوزه را امکان‌پذیر می‌سازد.

\section{تحلیل بر مبنای قالب خروجی مدل}

\subsection{خروجی‌های سبک واژنامه‌ای}

\persianfootnote{نمایش‌های ساخت‌یافته کلید–مقدار}\LTRfootnote{structured key-value representations} کدگذاری بی‌ابهام فراپارامترها یا انتخاب‌های معماری را با خوانایی ماشینی مستقیم فراهم می‌کنند. سامانه‌هایی که پیکربندی‌های قالب JSON تولید می‌کنند، پارامترهایی مانند نرخ یادگیری، اندازه دسته و ابعاد شبکه را مشخص می‌سازند \cite{zhang2023usingLLMforHPO, liu2025agenthpo}. گونه‌های پیشرفته، استدلال \persianfootnote{زنجیره تفکر}\LTRfootnote{chain-of-thought} را پیش از خروجی ساخت‌یافته می‌گنجانند تا کیفیت پیشنهادها را با استدلال میانی صریح بهبود دهند، در حالی‌که مشخصات نهایی همچنان قابل تجزیه باقی می‌ماند. محدودیت اصلی، مقیدشدن اکتشاف به \persianfootnote{طرحواره‌های ازپیش‌تعریف‌شده}\LTRfootnote{predefined schemas} است که می‌تواند کشف الگوهای طراحی نو را محدود کند. شکل \ref{fig:hpo-config} نمونه‌ای از خروجی قالب JSON را نشان می‌دهد.
\begin{figure}[h!]
    \centering
    \includegraphics[width=0.9\textwidth]{images/hpo-config.png}
    \caption[نمونه ای از خروجی قالب واژنامه‌ای]{نمونه‌ای از خروجی قالب JSON برای پیکربندی بهینه‌سازی فراپارامتر در سامانه مبتنی بر مدل زبانی بزرگ \cite{zhang2023usingLLMforHPO}
    }
    \label{fig:hpo-config}
\end{figure}
\subsection{تولید کد برنامه}

خروجی مبتنی بر کد با تکیه بر بیان‌پذیری کامل زبان‌های برنامه‌نویسی، از قیود فضاهای پیکربندی ازپیش‌تعریف‌شده می‌گریزد. چندین سامانه پیاده‌سازی‌های Python از شبکه‌های عصبی و خطوط لوله یادگیری ماشین را به‌صورت برنامه‌های کامل یا \persianfootnote{مولفه‌های ماژولار}\LTRfootnote{modular components} تولید می‌کنند و با آزمون‌های واحد خودکار سازگاری را می‌سنجند \cite{xu2024largeTextToML, LLMatic2024, chen2023Evoprompting}. برخی، تولید را بر معیارهای هدف با دستوردهی چندنمونه و نمونه‌هایی برگرفته از جمعیت‌های ارزیابی‌شده شرطی می‌کنند؛ برخی دیگر، عملگرهای تکاملی مانند جهش و ترکیب را مستقیماً بر نمایش‌های کدی اعمال می‌کنند. گونه‌های ترکیبی، تولید کد برای پیاده‌سازی معماری را با فهرست‌های ساخت‌یافته فراپارامتر و متن قالب‌بندی‌شده \persianfootnote{گزارش وقایع}\LTRfootnote{log} آموزشی پیش‌بینی‌شده درمی‌آمیزند \cite{zhang2023AutomlGPTAutomaticMachineLearning, trirat2025automlagent}. شکل \ref{fig:evoprompting} نمونه‌ای از خروجی تولید کد را نشان می‌دهد.
\begin{figure}[h!]
    \centering
    \includegraphics[width=0.9\textwidth]{images/evoprompting.png}
    \caption[نمونه ای از خروجی تولید کد]{
        نمونه‌ای از خروجی تولید کد در سامانه \lr{Evoprompting} که از مدل زبانی بزرگ برای تولید و بهینه‌سازی کدهای Python شبکه‌های عصبی استفاده می‌کند \cite{chen2023Evoprompting}
    }
    \label{fig:evoprompting}
\end{figure}
تولید کد، انعطاف طراحی را به حداکثر می‌رساند و امکان جست‌وجو در معماری‌های نامقید را بدون تعریف صریح اجزای ابتدایی فراهم می‌آورد. بااین‌حال، این رویکرد چالش‌های اعتبارسنجی به همراه دارد: \persianfootnote{درستی نحوی}\LTRfootnote{syntactic correctness} لزوماً آموزش‌پذیری، رعایت \persianfootnote{قیود منابع}\LTRfootnote{resource constraints} یا \persianfootnote{معناداری معنایی}\LTRfootnote{semantic meaningfulness} را تضمین نمی‌کند. ازاین‌رو، سامانه‌ها به محیط‌های اجرا برای ارزیابی نیاز دارند و سازوکارهای مدیریت خطا را برای مواجهه با خروجی‌های نامعتبر پیاده می‌کنند؛ از جمله \persianfootnote{دستوردهی مجدد}\LTRfootnote{re-prompting} با استفاده از پیام‌های خطا به‌عنوان بازخورد.

\subsection{خروجی‌های درخت‌ساختار}

\persianfootnote{نمایش‌های گراف/درخت}\LTRfootnote{graph or tree representations} بر روابط ترکیبی درون معماری‌ها تأکید می‌کنند و برای وظایفی که به \persianfootnote{تعیین صریح توپولوژی}\LTRfootnote{explicit topology specification} نیاز دارند سودمندند؛ بی‌آن‌که جزئیات پیاده‌سازی کدی که می‌تواند از استدلال ساختاری منحرف کند تحمیل شود. برخی سامانه‌ها نمایش متنی \persianfootnote{گراف جهت‌دار بدون‌دور}\LTRfootnote{Directed Acyclic Graph (DAG)} با گره‌های شماره‌گذاری‌شده برای عملیات و اتصالات برمی‌گزینند \cite{Yang2025NADER}؛ برخی دیگر از \persianfootnote{کدگذاری رشته‌ای جداکننده‌محور}\LTRfootnote{delimited string encodings} برای ویژگی‌های لایه سازگار با \persianfootnote{تولید خودبازگشتی}\LTRfootnote{autoregressive generation} استفاده می‌کنند \cite{Yu2025GPTNAS}. خروجی‌ها معمولاً با فرایندهای تجزیه تخصصی به کد اجرایی تبدیل می‌شوند و ابزارهای راستی‌آزمایی \persianfootnote{گراف محاسباتی}\LTRfootnote{computational graph} علاوه بر صحت نحوی، \persianfootnote{همریختی}\LTRfootnote{isomorphism} با طرح‌های موجود را نیز می‌سنجند. شکل \ref{fig:nader-tree} نمونه‌ای از خروجی درخت‌ساختار را نشان می‌دهد.
\begin{figure}[h!]
    \centering
    \includegraphics[width=0.9\textwidth]{images/nader-tree.png}
    \caption[نمونه ای از خروجی درخت ساختار]{
        نمونه‌ای از نمایش گراف‌محور معماری شبکه عصبی. سمت چپ: تصویرسازی گراف جهت‌دار بدون‌دور . سمت راست: نمایش متنی گراف جهت‌دار بدون‌دور برای فهم مدل زبانی بزرگ. \cite{Yang2025NADER}
    }
    \label{fig:nader-tree}
\end{figure}
این قالب‌ها استدلال ترکیبی و راهبردهای تغییر سلسله‌مراتبی را تسهیل می‌کنند و به مدل امکان می‌دهند بر توپولوژی معماری مستقل از جزئیات پیاده‌سازی تمرکز کند. به‌کارگیری آن‌ها به \persianfootnote{طرح‌های کدگذاری حوزه‌ای}\LTRfootnote{domain-specific encoding schemes} و رویه‌های اعتبارسنجی ویژه نیاز دارد، اما با کاهش پیچیدگی وظیفه تولید از طریق سطح تجرید مناسب، کیفیت تولید را بهبود می‌بخشد.

\subsection{خروجی‌های ترکیبی}

سامانه‌های پیشرفته، چندین \persianfootnote{گونه خروجی}\LTRfootnote{output modalities} را ترکیب می‌کنند تا از قوت‌های مکمل قالب‌های مختلف در مراحل گوناگون خط لوله بهره ببرند. الگوی رایج، مشخصات ساخت‌یافته را با توضیحات زبان طبیعی همراه می‌کند تا هم اجرای ماشینی و هم تفسیر انسانی میسر شود \cite{liu2025agenthpo, zhang2023usingLLMforHPO}. طرح‌های مفصل‌تر برای مراحل متمایز از قالب‌های متفاوت بهره می‌گیرند: JSON برای نیازمندی‌های قابل اعتبارسنجی صوری، زبان طبیعی برای برنامه‌ریزی و تحلیل منعطف، و کد اجرایی برای پیاده‌سازی‌های نهایی \cite{trirat2025automlagent, zhang2023AutomlGPTAutomaticMachineLearning}. برخی سامانه‌ها کدگذاری‌های ساختاری فشرده را در توضیحات زبان طبیعی می‌گنجانند تا مشخصات دقیق را با استدلال‌های قابل تفسیر تلفیق کنند \cite{ji2025RZNAS, Yang2025NADER}. شکل \ref{fig:rznas} نمونه‌ای از خروجی ترکیبی را نشان می‌دهد.
\begin{figure}[h!]
    \centering
    \includegraphics[width=0.9\textwidth]{images/rznas.png}
    \caption[نمونه ای از خروجی ترکیبی واژنامه‌ای و کد]{
        نمونه‌ای از خروجی ترکیبی در سامانه \lr{RZNAS} که از قالب‌های JSON و کد Python برای طراحی معماری شبکه عصبی استفاده می‌کند \cite{ji2025RZNAS}
    }
    \label{fig:rznas}
\end{figure}
این روش بازتاب این واقعیت است که هیچ قالب یگانه‌ای به‌تنهایی برای همه جنبه‌های یادگیری ماشین خودکار بهینه نیست: خروجی ساخت‌یافته برای تجزیه و اعتبارسنجی مناسب است؛ کد، پیاده‌سازی منعطف را ممکن می‌سازد؛ گراف‌ها استدلال ترکیبی را تقویت می‌کنند؛ و زبان طبیعی تفسیرپذیری و زمینه غنی را فراهم می‌کند و با موارد دشوار صوری‌سازی روبه‌رو می‌شود. ادغام موفق مستلزم طراحی دقیق \persianfootnote{گذارهای بین قالب‌ها}\LTRfootnote{format transitions}، رویه‌های اعتبارسنجی \persianfootnote{میان‌گونه‌ای}\LTRfootnote{across modalities} و راهبردهایی برای مدیریت \persianfootnote{ناهمخوانی}\LTRfootnote{inconsistencies} در صورت تعارض نمایش‌ها است.
% \section{طبقه بندی بر اساس دانش خارجی}
% \subsection{با استفاده از تولید تقویت شده با بازیابی}
% \subsection{با استفاده از ابزار های جستجو}
% \section{طبقه بندی بر اساس نوع خروجی مدل}
% \subsection{خروجی بصورت واژه‌نامه}
% \subsection{خروجی بصورت کد برنامه}
% \subsection{خروجی بصورت درخت}
% \subsection{خروجی بصورت ترکیبی از واژه‌نامه و کد برنامه}
\section{طبقه بندی کار های مرتبط}
همانطور که در جدول \ref{tab:recent-works} مشاهده می‌شود، اکثر روش‌های بررسی شده از مدل‌های زبانی بزرگ چندعاملی استفاده می‌کنند. این رویکرد به دلیل توانایی در تقسیم وظایف و همکاری بین عوامل مختلف، معمولاً عملکرد بهتری را در مسائل پیچیده ارائه می‌دهد. همچنین، بیشتر روش‌های جدید از دانش خارجی بهره می‌برند که می‌تواند به بهبود دقت و کارایی مدل کمک کند. در زمینه نوع خروجی، روش‌های متنوعی وجود دارد که بسته به نیاز مسئله، می‌توانند انتخاب شوند. برای مثال، خروجی بصورت کد برنامه برای مسائل نیازمند پیاده‌سازی عملی مناسب‌تر است، در حالی که خروجی بصورت واژه‌نامه ممکن است برای مسائل تحلیلی کاربردی‌تر باشد. این تنوع در رویکردها نشان‌دهنده انعطاف‌پذیری و قابلیت تطبیق مدل‌های زبانی بزرگ با نیازهای مختلف در حوزه جستجوی معماری شبکه عصبی و بهینه‌سازی ابرپارامتر است.
\begin{table}[t]
    \centering
    \footnotesize
    \setlength{\tabcolsep}{3pt}
    \renewcommand{\arraystretch}{1.2}
    \begin{tabularx}{\textwidth}{@{} Y c c c c c c c l @{}}
        \toprule
        عنوان                                                            & عامل & روش قدیمی & ایجاد کد                      & ابزار  & دانش خارجی   & فضای جستجو & بدون آموزش & وظیفه        \\
        \midrule
        \lr{LLM for HPO}\cite{zhang2023usingLLMforHPO}             & تک   & —         & \xmark                        & \xmark & \xmark & اختیاری    & —          & \lr{HPO}\    \\
        \lr{GENIUS}\cite{zheng2023GENIUS}                                & تک   & —         & \xmark                        & \xmark & \xmark & بله        & —          & \lr{NAS}     \\
        \lr{LLMATIC}\cite{LLMatic2024}                                   & تک   & EA        & \cmark                        & \xmark & \xmark & خیر        & —          & \lr{NAS}     \\
        \lr{Text to ML}\cite{xu2024largeTextToML}                        & چند  & —         & \cmark                        & \xmark & \xmark & خیر        & \cmark     & \lr{AutoML}  \\
        \lr{AgentHPO}\cite{liu2025agenthpo}                              & چند  & —         & \xmark                        & \cmark & \xmark & بله        & —          & \lr{HPO}     \\
        \lr{AutoML-Agent}\cite{trirat2025automlagent}                    & چند  & —         & \cmark                        & \xmark & \cmark & خیر        & \cmark     & \lr{AutoML}  \\
        \lr{LLAMBO}\cite{liu2024LLAMBO}                                  & چند  & BO        & \xmark                        & \xmark & \xmark & بله        & \cmark     & \lr{HPO}     \\
        \lr{Nader}\cite{Yang_2025_NADER}                                 & چند  & —         & \xmark\textsuperscript{\dag}  & \cmark & \cmark & بله        & —          & \lr{NAS} \\
        \lr{RZ-NAS}\cite{ji2025RZNAS}                                    & تک   & EA        & \cmark\textsuperscript{\ddag} & \xmark & \xmark & بله        & \cmark     & \lr{NAS}     \\
        \lr{EvoPrompting}\cite{chen2023Evoprompting}                     & تک   & EA        & \cmark                        & \xmark & \xmark & خیر        & —          & \lr{NAS}     \\
        \lr{AutoML-GPT}\cite{zhang2023AutomlGPTAutomaticMachineLearning} & تک   & —         & \cmark                        & \xmark & \xmark & بله        & —          & \lr{AutoML}  \\
        \lr{HuggingGPT}\cite{shen2023HuggingGPT}                         & چند  & —         & \cmark                        & \xmark & \cmark & خیر        & —          & \lr{AutoML}  \\
        \lr{GPT-NAS}\cite{Yu2025GPTNAS}                                  & تک   & EA        & —                             & \xmark & \xmark & خیر        & —          & \lr{NAS}     \\
        \lr{ML Copilot}\cite{zhang-etal-2024-MLCopilot}                  & چند  & —         & \xmark                        & \xmark & \cmark & خیر        & —          & \lr{AutoML}  \\
        \bottomrule
    \end{tabularx}
    \caption[مقایسهٔ فشردهٔ مقالات مبتنی بر \lr{LLM}]{مقایسهٔ فشردهٔ مقالات مبتنی بر \lr{LLM}. \lr{EA}=الگوریتم‌های تکاملی، \lr{BO}=بهینه‌سازی بیزی. نشانه‌ها: \textsuperscript{\dag}=ساختار درختی به‌جای تولید کد؛ \textsuperscript{\ddag}=تولید کد + تنظیمات.}
    \label{tab:recent-works}
\end{table}
