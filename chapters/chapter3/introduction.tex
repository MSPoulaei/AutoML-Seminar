\section{مقدمه}
ادغام مدل‌های زبانی بزرگ در خودکارسازی یادگیری ماشین به‌عنوان رویکردی دگرگون‌ساز برای بهینه‌سازی معماری‌های عصبی و فراپارامترها پدیدار شده است. این فصل کارهای موجود را از سه منظر مکمل بررسی می‌کند: معماری عامل، \persianfootnote{راهبردهای تقویت دانش}\LTRfootnote{knowledge augmentation strategies}، و طراحی قالب خروجی. این ابعاد در کنار هم نشان می‌دهند که سامانه‌های گوناگون چگونه از قابلیت‌های مدل‌های زبانی بزرگ بهره می‌برند و هم‌زمان محدودیت‌های ذاتی آن‌ها را مدیریت می‌کنند.
در شکل \ref{fig:category}، دسته‌بندی مقالات مرورشده بر مبنای این سه منظر ارائه شده است. بخش‌های بعدی به‌تفصیل هر یک از این ابعاد را بررسی می‌کنند و نمونه‌های شاخص را برای روشن‌ساختن رویکردهای مختلف ارائه می‌دهند.
\begin{figure}[h!]
    \centering
    \includegraphics[width=0.9\textwidth]{diagram/category.pdf}
    \caption[دسته بندی مقالات]{
        دسته بندی مقالات بر مبنای معماری عامل، استفاده از روش های سنتی، و قالب خروجی مدل زبانی بزرگ.
    }
    \label{fig:category}

\end{figure}
\section{ظهور نخستین روش‌های یادگیری ماشین خودکار}
خودکارسازی یادگیری ماشین با هدف مردمی‌سازی فرایندهای پرهزینه و زمان‌بر گزینش مدل و بهینه‌سازی فراپارامتر پدید آمد. ریشه گزینش الگوریتم به سال 1976 بازمی‌گردد که در آن، مسئله انتخاب الگوریتم بهینه از میان یک مجموعه ازپیش‌تعریف‌شده به‌منظور کمینه‌سازی افت کارایی روی داده‌های اعتبارسنجی صورت‌بندی شد \cite{RICE197665}. کوشش‌های نخستین در حوزه بهینه‌سازی فراپارامتر بر روش‌های منفردی چون جست‌وجوی شبکه‌ای و جست‌وجوی تصادفی متمرکز بود؛ روش‌هایی ساده اما ناکارا در فضاهای بعدبالا \cite{JMLR:v13:bergstra12a}. این روش نشان داد که جست‌وجوی تصادفی با تخصیص مؤثرتر منابع به فراپارامترهای اثرگذار، از جست‌وجوی شبکه‌ای پیشی می‌گیرد و بدین‌ترتیب زمینه را برای روش‌های پیشرفته‌تر فراهم ساخت \cite{JMLR:v13:bergstra12a}.

نقطه عطف در 2013 با معرفی مسئله CASH رخ داد؛ چارچوبی که گزینش الگوریتم و بهینه‌سازی فراپارامتر را در یک بهینه‌سازی واحد ادغام می‌کرد \cite{10.1145/2487575.2487629}. بر مبنای این ایده، Auto-WEKA به‌عنوان نخستین سامانه جامع یادگیری ماشین خودکار پدید آمد و با اتکا به SMAC   رویکردی مبتنی بر بهینه‌سازی بیزی که از \persianfootnote{جنگل‌های تصادفی}\LTRfootnote{random forests} به‌عنوان \persianfootnote{جانشین}\LTRfootnote{surrogate} استفاده می‌کرد   به جست‌وجو در میان رده‌بندهای WEKA و فراپارامترهایشان پرداخت \cite{10.1145/2487575.2487629, 10.1007/978-3-642-25566-3_40}. در ادامه، auto-sklearn در 2015 روش CASH را به الگوریتم‌های scikit-learn بسط داد و با به‌کارگیری \persianfootnote{فرا-یادگیری}\LTRfootnote{meta-learning} برای شروع تازه و نیز ساخت \persianfootnote{هم‌بندی}\LTRfootnote{ensemble}‌های مقاوم، بهبودهای معناداری رقم زد \cite{NIPS2015_11d0e628}.

به‌طور موازی، الگوریتم‌های تکاملی به‌عنوان بدیلی تواناتر مطرح شدند؛ چنان‌که TPOT خطوط لوله یادگیری ماشین را به‌صورت برنامه‌هایی درخت‌ساختار مدل می‌کند و با \persianfootnote{برنامه‌نویسی ژنتیکی}\LTRfootnote{genetic programming}، ترکیب‌های انعطاف‌پذیری از پیش‌پردازنده‌ها، \persianfootnote{گزیننده‌های ویژگی}\LTRfootnote{feature selectors} و مدل‌ها را می‌آزماید \cite{pmlr-v64-olson_tpot_2016}. روش‌های \persianfootnote{چندسطوحی/چندوفایی}\LTRfootnote{multi-fidelity} با تخصیص تدریجی منابع به پیکربندی‌های امیدبخش و با الهام از \persianfootnote{راهبردهای باندیتی}\LTRfootnote{bandit-based strategies} هزینه محاسباتی را مهار کردند \cite{pmlr-v51-jamieson16, JMLR:v18:16-558}. تمرکز این رویکردهای اولیه عمدتاً بر مسائل کلاسیک \persianfootnote{طبقه‌بندی}\LTRfootnote{classification} و \persianfootnote{رگرسیون}\LTRfootnote{regression} بود؛ و معیارهایی چون \lr{OpenML-CC18} ارزیابی‌های استانداردشده را تسهیل کردند \cite{bischl2021openmlbenchmarkingsuites}.

گذار به جست‌وجوی معماری شبکه‌های عصبی از حوالی 2017 آغاز شد و طراحی معماری را امتداد طبیعی بهینه‌سازی فراپارامتر تلقی کرد. کارهای اولیه مبتنی بر یادگیری تقویتی در جست‌وجوی معماری شبکه‌های عصبی، گرچه پرهزینه، نقطه شروعی اثرگذار بودند و گونه‌های کاراتری همچون ENAS با \persianfootnote{اشتراک پارامتر}\LTRfootnote{parameter sharing} را الهام بخشیدند \cite{zoph2017neural, pmlr-v80-pham18a}. در ادامه، معیارهایی مانند \lr{NAS-Bench-101} با فراهم‌کردن ارزیابی‌های ازپیش‌محاسبه‌شده، امکان پژوهش بازتولیدپذیر را مهیا کردند \cite{pmlr-v97-ying19a}. این سیر تحول، تنظیم دستی را به خطوط لوله خودکار بدل کرد و بستر را برای همگرایی با روش‌های پیشرفته‌تر فراهم ساخت.