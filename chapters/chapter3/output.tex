
\section{تحلیل بر مبنای قالب خروجی مدل}

\subsection{خروجی‌های سبک واژنامه‌ای}

\persianfootnote{نمایش‌های ساخت‌یافته کلید–مقدار}\LTRfootnote{structured key-value representations} کدگذاری بی‌ابهام فراپارامترها یا انتخاب‌های معماری را با خوانایی ماشینی مستقیم فراهم می‌کنند. سامانه‌هایی که پیکربندی‌های قالب JSON تولید می‌کنند، پارامترهایی مانند نرخ یادگیری، اندازه دسته و ابعاد شبکه را مشخص می‌سازند \cite{zhang2023usingLLMforHPO, liu2025agenthpo}. گونه‌های پیشرفته، استدلال \persianfootnote{زنجیره تفکر}\LTRfootnote{chain-of-thought} را پیش از خروجی ساخت‌یافته می‌گنجانند تا کیفیت پیشنهادها را با استدلال میانی صریح بهبود دهند، در حالی‌که مشخصات نهایی همچنان قابل تجزیه باقی می‌ماند. محدودیت اصلی، مقیدشدن اکتشاف به \persianfootnote{طرحواره‌های ازپیش‌تعریف‌شده}\LTRfootnote{predefined schemas} است که می‌تواند کشف الگوهای طراحی نو را محدود کند. شکل \ref{fig:hpo-config} نمونه‌ای از خروجی قالب JSON را نشان می‌دهد.
\begin{figure}[h!]
    \centering
    \includegraphics[width=0.9\textwidth]{images/hpo-config.png}
    \caption[نمونه ای از خروجی قالب واژنامه‌ای]{نمونه‌ای از خروجی قالب JSON برای پیکربندی بهینه‌سازی فراپارامتر در سامانه مبتنی بر مدل زبانی بزرگ \cite{zhang2023usingLLMforHPO}
    }
    \label{fig:hpo-config}
\end{figure}
\subsection{تولید کد برنامه}

خروجی مبتنی بر کد با تکیه بر بیان‌پذیری کامل زبان‌های برنامه‌نویسی، از قیود فضاهای پیکربندی ازپیش‌تعریف‌شده می‌گریزد. چندین سامانه پیاده‌سازی‌های Python از شبکه‌های عصبی و خطوط لوله یادگیری ماشین را به‌صورت برنامه‌های کامل یا \persianfootnote{مولفه‌های ماژولار}\LTRfootnote{modular components} تولید می‌کنند و با آزمون‌های واحد خودکار سازگاری را می‌سنجند \cite{xu2024largeTextToML, LLMatic2024, chen2023Evoprompting}. برخی، تولید را بر معیارهای هدف با دستوردهی چندنمونه و نمونه‌هایی برگرفته از جمعیت‌های ارزیابی‌شده شرطی می‌کنند؛ برخی دیگر، عملگرهای تکاملی مانند جهش و ترکیب را مستقیماً بر نمایش‌های کدی اعمال می‌کنند. گونه‌های ترکیبی، تولید کد برای پیاده‌سازی معماری را با فهرست‌های ساخت‌یافته فراپارامتر و متن قالب‌بندی‌شده \persianfootnote{گزارش وقایع}\LTRfootnote{log} آموزشی پیش‌بینی‌شده درمی‌آمیزند \cite{zhang2023AutomlGPTAutomaticMachineLearning, trirat2025automlagent}. شکل \ref{fig:evoprompting} نمونه‌ای از خروجی تولید کد را نشان می‌دهد.
\begin{figure}[h!]
    \centering
    \includegraphics[width=0.9\textwidth]{images/evoprompting.png}
    \caption[نمونه ای از خروجی تولید کد]{
        نمونه‌ای از خروجی تولید کد در سامانه \lr{Evoprompting} که از مدل زبانی بزرگ برای تولید و بهینه‌سازی کدهای Python شبکه‌های عصبی استفاده می‌کند \cite{chen2023Evoprompting}
    }
    \label{fig:evoprompting}
\end{figure}
تولید کد، انعطاف طراحی را به حداکثر می‌رساند و امکان جست‌وجو در معماری‌های نامقید را بدون تعریف صریح اجزای ابتدایی فراهم می‌آورد. بااین‌حال، این رویکرد چالش‌های اعتبارسنجی به همراه دارد: \persianfootnote{درستی نحوی}\LTRfootnote{syntactic correctness} لزوماً آموزش‌پذیری، رعایت \persianfootnote{قیود منابع}\LTRfootnote{resource constraints} یا \persianfootnote{معناداری معنایی}\LTRfootnote{semantic meaningfulness} را تضمین نمی‌کند. ازاین‌رو، سامانه‌ها به محیط‌های اجرا برای ارزیابی نیاز دارند و سازوکارهای مدیریت خطا را برای مواجهه با خروجی‌های نامعتبر پیاده می‌کنند؛ از جمله \persianfootnote{دستوردهی مجدد}\LTRfootnote{re-prompting} با استفاده از پیام‌های خطا به‌عنوان بازخورد.

\subsection{خروجی‌های درخت‌ساختار}

\persianfootnote{نمایش‌های گراف/درخت}\LTRfootnote{graph or tree representations} بر روابط ترکیبی درون معماری‌ها تأکید می‌کنند و برای وظایفی که به \persianfootnote{تعیین صریح توپولوژی}\LTRfootnote{explicit topology specification} نیاز دارند سودمندند؛ بی‌آن‌که جزئیات پیاده‌سازی کدی که می‌تواند از استدلال ساختاری منحرف کند تحمیل شود. برخی سامانه‌ها نمایش متنی \persianfootnote{گراف جهت‌دار بدون‌دور}\LTRfootnote{Directed Acyclic Graph (DAG)} با گره‌های شماره‌گذاری‌شده برای عملیات و اتصالات برمی‌گزینند \cite{Yang2025NADER}؛ برخی دیگر از \persianfootnote{کدگذاری رشته‌ای جداکننده‌محور}\LTRfootnote{delimited string encodings} برای ویژگی‌های لایه سازگار با \persianfootnote{تولید خودبازگشتی}\LTRfootnote{autoregressive generation} استفاده می‌کنند \cite{Yu2025GPTNAS}. خروجی‌ها معمولاً با فرایندهای تجزیه تخصصی به کد اجرایی تبدیل می‌شوند و ابزارهای راستی‌آزمایی \persianfootnote{گراف محاسباتی}\LTRfootnote{computational graph} علاوه بر صحت نحوی، \persianfootnote{همریختی}\LTRfootnote{isomorphism} با طرح‌های موجود را نیز می‌سنجند. شکل \ref{fig:nader-tree} نمونه‌ای از خروجی درخت‌ساختار را نشان می‌دهد.
\begin{figure}[h!]
    \centering
    \includegraphics[width=0.9\textwidth]{images/nader-tree.png}
    \caption[نمونه ای از خروجی درخت ساختار]{
        نمونه‌ای از نمایش گراف‌محور معماری شبکه عصبی. سمت چپ: تصویرسازی گراف جهت‌دار بدون‌دور . سمت راست: نمایش متنی گراف جهت‌دار بدون‌دور برای فهم مدل زبانی بزرگ. \cite{Yang2025NADER}
    }
    \label{fig:nader-tree}
\end{figure}
این قالب‌ها استدلال ترکیبی و راهبردهای تغییر سلسله‌مراتبی را تسهیل می‌کنند و به مدل امکان می‌دهند بر توپولوژی معماری مستقل از جزئیات پیاده‌سازی تمرکز کند. به‌کارگیری آن‌ها به \persianfootnote{طرح‌های کدگذاری حوزه‌ای}\LTRfootnote{domain-specific encoding schemes} و رویه‌های اعتبارسنجی ویژه نیاز دارد، اما با کاهش پیچیدگی وظیفه تولید از طریق سطح تجرید مناسب، کیفیت تولید را بهبود می‌بخشد.

\subsection{خروجی‌های ترکیبی}

سامانه‌های پیشرفته، چندین \persianfootnote{گونه خروجی}\LTRfootnote{output modalities} را ترکیب می‌کنند تا از قوت‌های مکمل قالب‌های مختلف در مراحل گوناگون خط لوله بهره ببرند. الگوی رایج، مشخصات ساخت‌یافته را با توضیحات زبان طبیعی همراه می‌کند تا هم اجرای ماشینی و هم تفسیر انسانی میسر شود \cite{liu2025agenthpo, zhang2023usingLLMforHPO}. طرح‌های مفصل‌تر برای مراحل متمایز از قالب‌های متفاوت بهره می‌گیرند: JSON برای نیازمندی‌های قابل اعتبارسنجی صوری، زبان طبیعی برای برنامه‌ریزی و تحلیل منعطف، و کد اجرایی برای پیاده‌سازی‌های نهایی \cite{trirat2025automlagent, zhang2023AutomlGPTAutomaticMachineLearning}. برخی سامانه‌ها کدگذاری‌های ساختاری فشرده را در توضیحات زبان طبیعی می‌گنجانند تا مشخصات دقیق را با استدلال‌های قابل تفسیر تلفیق کنند \cite{ji2025RZNAS, Yang2025NADER}. شکل \ref{fig:rznas} نمونه‌ای از خروجی ترکیبی را نشان می‌دهد.
\begin{figure}[h!]
    \centering
    \includegraphics[width=0.9\textwidth]{images/rznas.png}
    \caption[نمونه ای از خروجی ترکیبی واژنامه‌ای و کد]{
        نمونه‌ای از خروجی ترکیبی در سامانه \lr{RZNAS} که از قالب‌های JSON و کد Python برای طراحی معماری شبکه عصبی استفاده می‌کند \cite{ji2025RZNAS}
    }
    \label{fig:rznas}
\end{figure}
این روش بازتاب این واقعیت است که هیچ قالب یگانه‌ای به‌تنهایی برای همه جنبه‌های یادگیری ماشین خودکار بهینه نیست: خروجی ساخت‌یافته برای تجزیه و اعتبارسنجی مناسب است؛ کد، پیاده‌سازی منعطف را ممکن می‌سازد؛ گراف‌ها استدلال ترکیبی را تقویت می‌کنند؛ و زبان طبیعی تفسیرپذیری و زمینه غنی را فراهم می‌کند و با موارد دشوار صوری‌سازی روبه‌رو می‌شود. ادغام موفق مستلزم طراحی دقیق \persianfootnote{گذارهای بین قالب‌ها}\LTRfootnote{format transitions}، رویه‌های اعتبارسنجی \persianfootnote{میان‌گونه‌ای}\LTRfootnote{across modalities} و راهبردهایی برای مدیریت \persianfootnote{ناهمخوانی}\LTRfootnote{inconsistencies} در صورت تعارض نمایش‌ها است.