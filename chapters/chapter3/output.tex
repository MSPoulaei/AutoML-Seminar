
\section{طبقه‌بندی بر اساس نوع قالب خروجی}
یکی از ابعاد بنیادین در طراحی سامانه‌های مبتنی بر مدل‌های زبانی برای یادگیری ماشین خودکار، نحوهٔ بازنمایی و صدور نتایج بهینه‌سازی است. قالب خروجی نه‌تنها بر کارایی و قابلیت تفسیر تعیین می‌کند، بلکه بر یکپارچگی با خطوط لوله یادگیری ماشین، امکان ارزیابی خودکار و انعطاف‌پذیری در مواجهه با وظایف متنوع نیز اثرگذار است. چهار قالب اصلی در آثار مرورشده بروز می‌یابد: \persianfootnote{خروجی واژه‌نامه‌ای}\LTRfootnote{Dictionary-based Output} که پیکربندی‌های ساختاریافته (مانند \lr{JSON}) را ارائه می‌دهد؛ \persianfootnote{خروجی کُد برنامه}\LTRfootnote{Code-based Output} که معماری‌ها یا خطوط لوله را به‌صورت اسکریپت‌های اجرایی تولید می‌کند؛ \persianfootnote{خروجی درختی}\LTRfootnote{Tree-based Output} که بازنمایی‌های سلسله‌مراتبی را رمزگذاری می‌کند؛ و \persianfootnote{خروجی ترکیبی}\LTRfootnote{Hybrid Output} که چندین قالب را یکپارچه می‌سازد. در ادامه، هر رویکرد از منظر قابلیت‌ها، مبادلات و کاربردهای نمونه مورد بررسی قرار می‌گیرد.

\subsection{خروجی به‌صورت واژه‌نامه (پیکربندی ساختاریافته)}
رویکردهای مبتنی بر واژه‌نامه، خروجی را به‌صورت جفت کلید-مقدار ساختاریافته (اغلب در قالب \lr{JSON} یا \lr{YAML}) بازنمایی می‌کنند که پیکربندی‌های ابرپارامتر، گزینه‌های معماری یا انتخاب مدل را مشخص می‌کنند. این قالب به‌ویژه در سناریوهایی که اجرای خودکار، اعتبارسنجی و یکپارچه‌سازی با چارچوب‌های یادگیری ماشین موجود حیاتی است، سودمند است.

رویکرد پیشنهادی در~\cite{zhang2023usingLLMforHPO} نمونه‌ای از این قالب است؛ مدل زبانی پیکربندی‌های ابرپارامتری را به‌صورت واژه‌نامه‌های ساختاریافته صادر می‌کند که مستقیماً به حل‌کننده‌های بهینه‌سازی تغذیه می‌شوند؛ این امر ارزیابی و پالایش تکرارشونده را در حلقهٔ بسته تسهیل می‌کند. به‌گونه‌ای مشابه، \lr{AgentHPO}~\cite{liu2025agenthpo} از گفت‌وگوهای میان‌عاملی برای پیشنهاد و پالایش پیکربندی‌های ابرپارامتری در قالب واژه‌نامه بهره می‌گیرد؛ خروجی ساختاریافته امکان اعتبارسنجی خودکار و بررسی محدودیت‌ها را قبل از ارزیابی پرهزینه فراهم می‌آورد. چارچوب \lr{LLAMBO}~\cite{liu2024LLAMBO} نیز که بهینه‌سازی بیزی را با استدلال مدل زبانی ترکیب می‌کند، پیشنهادهای پارامتر را به‌صورت واژه‌نامه صادر می‌کند تا با مدل‌سازی جانشین و توابع اکتساب یکپارچه شود.

مزایای اصلی این رویکرد شامل سهولت پردازش خودکار، تطابق با چارچوب‌های یادگیری ماشین استاندارد (مانند \lr{scikit-learn}، \lr{PyTorch}) و توانایی اعمال محدودیت‌های سخت (برای مثال محدوده‌های پارامتر، وابستگی‌های شرطی) در زمان تولید یا اعتبارسنجی است. با این حال، این قالب انعطاف‌پذیری محدودتری برای تعریف معماری‌های سفارشی یا خطوط لوله پیچیده دارد؛ زیرا ساختارهای از پیش تعریف‌شده را فرض می‌کند و ممکن است در کاوش فضاهای جست‌وجوی باز یا نوآورانه محدودیت ایجاد کند~\cite{zhang2023usingLLMforHPO,liu2025agenthpo}.

\subsection{خروجی به‌صورت کُد برنامه}
رویکردهای مبتنی بر کُدزایی، معماری‌ها، خطوط لوله یا اسکریپت‌های آموزش را مستقیماً به‌صورت کُد اجرایی (معمولاً پایتون) تولید می‌کنند. این قالب حداکثر انعطاف‌پذیری را ارائه می‌دهد و امکان تعریف معماری‌های خلاقانه، جریان‌های کاری سفارشی و منطق‌های شرطی پیچیده را فراهم می‌آورد.

چارچوب \lr{AutoML-GPT}~\cite{zhang2023AutomlGPTAutomaticMachineLearning} نمونه‌ای بارز است؛ این سامانهٔ تک‌عاملی کُد پایتون را برای پیش‌پردازش، انتخاب مدل، آموزش و ارزیابی تولید می‌کند و به‌صورت تکرارشونده آن را بر پایهٔ بازخورد اجرا پالایش می‌کند. خروجی کُد امکان یکپارچه‌سازی یکپارچه با کتابخانه‌های یادگیری ماشین، اشکال‌زدایی و سفارشی‌سازی توسط کاربران خبره را میسر می‌سازد. \lr{LLMatic}~\cite{LLMatic2024} از جهش‌های کُد-محور برای کاوش فضای معماری بهره می‌گیرد؛ مدل زبانی تغییرات معماری را به‌صورت تکه‌های کُد تولید می‌کند که در یک بایگانی تنوع‌محور ارزیابی و تکامل می‌یابند. به‌گونه‌ای مشابه، \lr{EvoPrompting}~\cite{chen2023Evoprompting} کُد شبکه‌های عصبی را از طریق تکامل تکرارشونده پرامپت‌ها تولید می‌کند؛ جمعیت کُدها از طریق جهش و ترکیب پالایش می‌شوند. چارچوب \lr{AutoMLAgent}~\cite{trirat2025automlagent} نیز از عوامل تخصصی برای تولید ماژول‌های کُد در پیش‌پردازش، طراحی مدل و بهینه‌سازی بهره می‌گیرد.

مزایای کُدزایی شامل حداکثر قدرت بیان، امکان تعریف معماری‌های بی‌سابقه و سهولت یکپارچه‌سازی با جریان‌های کاری توسعه نرم‌افزار است. با این حال، این رویکرد به توانایی‌های قوی کُدزایی مدل زبانی وابسته است و در معرض خطاهای نحوی، منطقی یا زمان‌اجرا قرار دارد که نیاز به اعتبارسنجی دقیق و مکانیسم‌های بازیابی خطا دارند؛ علاوه بر این، ارزیابی کُد تولیدشده می‌تواند پرهزینه‌تر از اعتبارسنجی پیکربندی‌های ساختاریافته باشد~\cite{zhang2023AutomlGPTAutomaticMachineLearning,LLMatic2024}.

\subsection{خروجی به‌صورت درختی (بازنمایی سلسله‌مراتبی)}
رویکردهای مبتنی بر درخت، معماری‌ها یا فضاهای جست‌وجو را به‌صورت ساختارهای سلسله‌مراتبی رمزگذاری می‌کنند؛ اغلب با استفاده از درخت‌های تصمیم، گراف‌های محاسباتی یا بازنمایی‌های رشته‌ای که ماژول‌ها و اتصالات آن‌ها را مشخص می‌کنند. این قالب برای وظایف جست‌وجوی معماری که طراحی شبکه‌های عصبی ذاتاً ساختاریافته و ترکیبی است، مناسب است.

چارچوب \lr{NADER}~\cite{Yang2025NADER} نمونه‌ای برجسته است؛ این رویکرد چندعاملی معماری‌ها را به‌صورت درخت‌های سلسله‌مراتبی بازنمایی می‌کند که عوامل آن‌ها را به‌صورت تکرارشونده با بازیابی دانش بیرونی پالایش می‌کنند. ساختار درختی امکان کاوش مرحله‌ای (از کلان‌ساختار به میکرو-اتصالات)، هرس کارآمد فضای جست‌وجو و استفاده مجدد از زیردرخت‌ها را فراهم می‌آورد. به‌گونه‌ای مشابه، \lr{LLaMa-NAS}~\cite{sarah2024llamaNAS} از بازنمایی‌های ساختاریافته (که می‌تواند درختی باشد) برای کدگذاری معماری‌های فشرده برای مدل‌های زبانی بهره می‌گیرد؛ جست‌وجوی بی‌گرادیان در این فضای ساختاریافته امکان کاوش کارآمد با قیود آگاه به سخت‌افزار را می‌دهد.

مزایای بازنمایی درختی شامل تطابق طبیعی با ساختار شبکه‌های عصبی، امکان عملیات جست‌وجوی سلسله‌مراتبی و کارآمد (برای مثال هرس، تقسیم‌وحل) و تسهیل تفسیرپذیری از طریق بصری‌سازی است. با این حال، رمزگذاری و دستکاری ساختارهای درختی پیچیده‌تر از قالب‌های مسطح است؛ و بازنمایی‌های درختی ممکن است برای معماری‌های با اتصالات پیچیده (برای مثال گراف‌های دوری، اتصالات بازگشتی) کافی نباشند~\cite{Yang2025NADER}.

\subsection{خروجی به‌صورت ترکیبی از واژه‌نامه و کُد برنامه}
تعدادی از رویکردهای پیشرفته از قالب‌های خروجی ترکیبی بهره می‌گیرند که مزایای بازنمایی‌های ساختاریافته و انعطاف‌پذیری کُدزایی را ترکیب می‌کنند. این راهبردهای ترکیبی اغلب شامل صدور پیکربندی‌های سطح‌بالا به‌صورت واژه‌نامه و تولید کُد سفارشی برای اجزای تخصصی یا پیچیده است.

برای مثال، چارچوب \lr{Text-to-ML}~\cite{xu2024largeTextToML} توصیف‌های زبانی وظایف را به خطوط لوله اجرایی ترجمه می‌کند؛ عوامل هم پیکربندی‌های سطح‌بالا (انتخاب مدل، پارامترهای پیش‌پردازش) را به‌صورت واژه‌نامه صادر می‌کنند و هم کُد پایتون سفارشی برای تبدیلات ویژگی یا معماری‌های خاص تولید می‌کنند. به‌گونه‌ای مشابه، \lr{GENIUS}~\cite{zheng2023GENIUS} اگرچه عمدتاً بر تولید معماری مبتنی بر توصیف زبان طبیعی تمرکز دارد، می‌تواند هم بازنمایی‌های ساختاریافته (برای جست‌وجوی اولیه) و هم کُد اجرایی (برای اجرای نهایی) صادر کند.

رویکردهای ترکیبی امکان بهره‌گیری از تطابق پیکربندی‌های ساختاریافته را برای وظایف استاندارد و انعطاف‌پذیری کُدزایی برای سفارشی‌سازی‌های خاص فراهم می‌آورند. این امر می‌تواند سهولت استفاده را برای کاربران غیرخبره (از طریق واسط‌های پیکربندی) و قدرت برای کاربران پیشرفته (از طریق ویرایش کُد) موازنه کند. با این حال، مدیریت تعامل و سازگاری میان قالب‌های مختلف خروجی پیچیدگی اضافی را به سامانه می‌افزاید و نیاز به هماهنگی دقیق میان تولید، اعتبارسنجی و اجرای اجزای مختلف دارد~\cite{xu2024largeTextToML,zheng2023GENIUS}.

\paragraph{جمع‌بندی و چشم‌انداز.}
انتخاب قالب خروجی بازتاب‌دهندهٔ مبادله‌ای بنیادین میان ساختار، انعطاف‌پذیری و کارایی است. خروجی‌های واژه‌نامه‌ای برای تنظیم ابرپارامتر ساختاریافته و یکپارچه‌سازی یکپارچه مناسب‌اند؛ خروجی‌های کُد حداکثر قدرت بیان را برای کاوش معماری خلاقانه فراهم می‌آورند؛ بازنمایی‌های درختی با ساختار ذاتی شبکه‌های عصبی هم‌راستا هستند؛ و رویکردهای ترکیبی سعی در موازنه این نیازهای متنوع دارند. مسیرهای آینده می‌تواند شامل انتخاب خودکار قالب خروجی بر حسب مشخصات وظیفه، توسعهٔ قالب‌های میانی یکپارچه که تبدیل یکپارچه میان بازنمایی‌ها را امکان‌پذیر می‌سازند، و طراحی مکانیسم‌های اعتبارسنجی و بازیابی خطای قوی‌تر برای کُدزایی و بازنمایی‌های ساختاریافته باشد. علاوه بر این، ادغام قابلیت‌های تفسیرپذیری و تبیین‌پذیری در قالب‌های خروجی می‌تواند اعتماد و قابلیت بررسی در سامانه‌های مبتنی بر مدل‌های زبانی برای یادگیری ماشین خودکار را افزایش دهد.