\section[تحلیل منابع دانش]{\persianfootnote{تحلیل منابع دانش}\LTRfootnote{Knowledge Source Analysis}}

کارآمدی سامانه‌های خودکارسازی یادگیری ماشین مبتنی بر مدل‌های زبانی بزرگ به‌صورت بنیادین به چگونگی اکتساب، مدیریت و بهره‌برداری از دانش در سراسر فرایند بهینه‌سازی وابسته است. رویکردهای معاصر طیفی از راهبردهای تقویت دانش را پوشش می‌دهند؛ از یادگیری درون‌متنی صرف تا چارچوب‌های تولید تقویت‌شده با بازیابی که هر یک پیامدهای متمایزی برای کارایی جست‌وجو و \persianfootnote{تعمیم‌پذیری}\LTRfootnote{generalization} دارند.

\subsection[دانش درونی: تاریخچه آزمون و بازتاب]{\persianfootnote{دانش درونی: تاریخچه آزمون و بازتاب}\LTRfootnote{System-Internal Knowledge: Trials and Reflection}}
این رسته، دانش تولیدشده توسط خود سامانه را در بر می‌گیرد از اتکای مستقیم به تاریخچه آزمون‌ها در \persianfootnote{پنجره زمینه}\LTRfootnote{context window} تا حافظه رویدادی و \persianfootnote{خودبازتابی}\LTRfootnote{self-reflection} که به‌تدریج به \persianfootnote{بازخورد قابل اقدام}\LTRfootnote{actionable feedback} و اصول طراحی تقطیر می‌شوند.
\begin{figure}[h!]
    \centering
    \includegraphics[width=0.9\textwidth]{images/knowledge-icl.png}
    \caption[]{یادگیری درون‌متنی از تاریخچه بهینه‌سازی~\cite{liu2025agenthpo}.
    }
    \label{fig:knowledge-icl}
\end{figure}
\subsubsection{\persianfootnote{یادگیری درون‌متنی از تاریخچه بهینه‌سازی}\protect\LTRfootnote{In-Context Learning from Optimization History}}

یک راهبرد رایج، پنجره زمینه مدل را همچون مخزن اصلی دانش در نظر می‌گیرد و صرفاً بر تاریخچه آزمون‌های انباشته‌شده طی فرایند \persianfootnote{بهینه‌سازی}\LTRfootnote{optimization} تکیه می‌کند. سامانه‌های مبتنی بر این ایده، پیکربندی‌های پیشین و سنجه‌های کارایی متناظر را در دستور می‌گنجانند تا مدل بتواند از رهگذر بازخورد تکرارشونده پیشنهادها را پالایش کند \cite{zhang2023usingLLMforHPO, zheng2023GENIUS, liu2024LLAMBO}. این تاریخچه ممکن است به قالب‌های گوناگونی \persianfootnote{سریال‌سازی}\LTRfootnote{serialization} شود: گفت‌وگوهای \persianfootnote{سبک‌گپ}\LTRfootnote{chat-style dialogues} که توالی زمانی تعاملات را حفظ می‌کنند، خلاصه‌های فشرده برای مهار قیود طول زمینه، \persianfootnote{الگوهای اندک‌نمونه}\LTRfootnote{few-shot demonstrations} برگزیده از جمعیت ارزیابی‌شده \cite{zhang2023usingLLMforHPO, chen2023Evoprompting}، و نیز تجربه‌های تاریخی استانداردسازی‌شده که داده‌های ناهمگن گذشته (پیکربندی‌های \lr{JSON}\LTRfootnote{JavaScript Object Notation}، پیاده‌سازی‌های کد، سنجه‌های عددی) را به نمایش‌های یکنواخت زبان طبیعی تبدیل می‌کنند تا پردازش و قیاس توسط مدل تسهیل شود \cite{zhang-etal-2024-MLCopilot}. شکل \ref{fig:knowledge-icl} این رویکردها را به‌مثابه طیفی از راهبردهای یادگیری درون‌متنی نشان می‌دهد که استفاده از تاریخچه بهینه سازی را شامل می‌شود \cite{liu2025agenthpo}.

در این رویکرد تکمیلی، ابرپارامترهای عددی برای بهبود استدلال \persianfootnote{گسسته‌سازی}\LTRfootnote{discretized} می‌شوند (مثلاً به سطوح «کم»، «متوسط»، «زیاد»)، و تجربه‌های استانداردسازی‌شده با \persianfootnote{نهفتارسازی}\LTRfootnote{embed} و نمایه‌سازی در پایگاه برداری پشتیبانی می‌شوند تا بازیابی مبتنی بر شباهت، چند برترین های نمونه مرتبط را برای نمایش درون‌متنی برگزیند. فراتر از درج خام تجربه‌ها، \persianfootnote{استخراج دانش برون‌خط}\LTRfootnote{offline knowledge elicitation} از طریق خلاصه‌سازی تکرارشونده مبتنی بر مدل و \persianfootnote{اعتبارسنجی پسین}\LTRfootnote{post-validation} روی وظایف کنارگذاشته، اصول طراحی سطح‌بالا و بازخورد قابل اقدام را تقطیر می‌کند؛ این دانش استخراج‌شده می‌تواند به‌صورت راهنمای سامانه یا الگوهای اندک‌نمونه در متن تزریق شود و حتی \persianfootnote{تولید راه‌حل تک‌نمونه‌ای}\LTRfootnote{one-shot solution generation} برای وظایف نو را میسر سازد \cite{zhang-etal-2024-MLCopilot}.

این روش به‌ویژه در سناریوهای کم‌بودجه مؤثر است؛ جایی که پیشین‌های یادگرفته‌شده مدل و تجربه‌های استانداردسازی‌شده می‌توانند بدون داده تجربی فراوان مسیر اکتشاف را هدایت کنند. در بسترهای بهینه‌سازی بیزی، مشاهدات تاریخی، آغاز گرم، \persianfootnote{نمونه‌برداری نامزد}\LTRfootnote{candidate sampling} و مدلسازی جانشین را به‌طور کامل از راه سریال‌سازی زبان طبیعی ارزیابی‌های پیشین شرطی‌سازی می‌کنند؛ و در حالی‌که در حالت‌های \persianfootnote{بی‌نمونه}\LTRfootnote{zero-shot} یا \persianfootnote{کم‌نمونه}\LTRfootnote{few-shot} عمل می‌کنند، کارایی رقابتی در قیاس با روش‌های سنتی نشان می‌دهند \cite{liu2024LLAMBO}. تجربه‌های استانداردسازی‌شده با فراهم‌سازی نمونه‌های مشابه تأییدشده و قواعد طراحی تقطیرشده، می‌توانند این مراحل را دقیق‌تر \persianfootnote{شروع تازه}\LTRfootnote{warmstart} کنند و میدان جست‌وجو را به‌صورت هدایت‌شده منقبض سازند \cite{zhang-etal-2024-MLCopilot}.

محدودیت اصلی در بهینه‌سازی‌های \persianfootnote{بلندافق}\LTRfootnote{long-horizon} رخ می‌نماید؛ جایی‌که قیود پنجره زمینه مستلزم نگهداشت گزینشی یا فشرده‌سازی با از دست رفتن اطلاعات تاریخچه است و چه‌بسا الگوهای حیاتی برای پالایش مرحله پایانی را حذف کند. \persianfootnote{استانداردسازی}\LTRfootnote{canonicalization} تا حدی این معضل را با خلاصه‌های ساخت‌یافته متراکم و بازیابی هدفمند تخفیف می‌دهد، اما همچنان با برش اطلاعاتی، سوگیری‌های ناشی از گسسته‌سازی، و حساسیت به \persianfootnote{امتیازدهی ارتباط}\LTRfootnote{relevance scoring} و پوشش مخزن مواجه است. با این‌همه، چون مصرف نهایی این دانش درون همان پنجره زمینه صورت می‌گیرد، مرز میان «دانش درون‌متنی صرف» و «تقویت مبتنی بر بازیابی» کم‌رنگ‌تر می‌شود؛ و ادغام تاریخچه آزمون با تجربه‌های استانداردسازی‌شده، پایایی و کارایی یادگیری درون‌متنی را در عمل ارتقا می‌دهد \cite{zhang2023usingLLMforHPO, zheng2023GENIUS, chen2023Evoprompting, liu2024LLAMBO, zhang-etal-2024-MLCopilot}.

\subsection[دانش بیرونی: بازیابی از ادبیات و مخازن]{\persianfootnote{دانش بیرونی: بازیابی از ادبیات و مخازن}\LTRfootnote{External Knowledge via Retrieval}}

سامانه‌های پیشرفته‌تر، تولید تقویت‌شده با بازیابی را برای ادغام دانش بیرون از مسیر بهینه‌سازی به کار می‌گیرند. این چارچوب‌ها مخازن دانش ساخت‌یافته عموماً پایگاه‌های داده برداری نمایه‌شده با نهفتارها نگه می‌دارند تا بر پایه زمینه وظیفه جاری، اطلاعات مرتبط را بازیابی کرده و در راهنماهای متنی تزریق کنند و بدین‌سان تصمیم‌سازی را اطلاع‌رسانی کنند.

\subsubsection{\persianfootnote{استخراج دانش مبتنی بر ادبیات پژوهشی}\protect\LTRfootnote{Literature-Driven Knowledge Extraction}}

چند رویکرد، دانش راهبردی را از ادبیات علمی برای هدایت تصمیم‌های معماری گردآوری می‌کنند. یک راهبرد از \persianfootnote{کارگزاران خوانش تخصصی}\LTRfootnote{specialized reader agents} بهره می‌برد که مقالات اخیر را \persianfootnote{خزش}\LTRfootnote{crawl} کرده، نکته‌های روش‌شناختی را از چکیده‌ها و بخش‌های روش استخراج می‌کنند و در پایگاه‌های داده برداری برای بازیابی مبتنی بر شباهت بایگانی می‌نمایند \cite{Yang2025NADER}. در خلال بهینه‌سازی، پیشنهادهای تغییر به‌منزله پرسش، اصول طراحی مرتبط را فراخوانی می‌کنند؛ و بدین‌ترتیب سامانه بدون آن‌که مدل پایه الزاماً بر تازه‌ترین انتشارات آموزش دیده باشد، از مرز دانش روز بهره می‌گیرد. نمونه‌ای دیگر، خلاصه‌هایی از مقالات arXiv و جست‌وجوهای وب را از طریق \persianfootnote{رابط‌های برنامه‌نویسی کاربردی}\LTRfootnote{application programming interfaces (APIs)} بازیابی کرده و راهنماهای برنامه‌ریزی را با بینش‌های بیرونی پیرامون مدل‌ها، ابرپارامترها و داده‌مجموعه‌ها غنی می‌کند تا تنوع و سازگاری طرح را ارتقا دهد \cite{trirat2025automlagent}.

این استخراج دانش، برای طراحی معماری‌های عصبی بس سودمند است؛ چراکه نوآوری‌های اخیر در ترکیب لایه‌ها، \persianfootnote{اتصالات پرشی}\LTRfootnote{skip connections} یا \persianfootnote{طرحواره‌های نرمال‌سازی}\LTRfootnote{normalization schemes} چه‌بسا در پارامترهای \persianfootnote{منجمد}\LTRfootnote{frozen} مدل بازتاب نیافته باشند. با این‌همه، کیفیت دانش بازیابی‌شده به‌نحو حساس به سازوکار امتیازدهی ارتباط و پوشش پیکره ادبیات نمایه‌شده وابسته است.

\subsubsection{\persianfootnote{مخازن داده‌مجموعه و مدل}\protect\LTRfootnote{Dataset and Model Repositories}}

در کنار بازیابی مبتنی بر ادبیات، چند سامانه از مخازن بیرونی برای فراداده‌های داده‌مجموعه‌ها و مدل‌های ازپیش‌آموزش‌دیده پرس‌وجو می‌کنند. چارچوب‌هایی که کل زنجیره خودکارسازی یادگیری ماشین را راهبری می‌کنند، کارت‌های داده‌مجموعه از پلتفرم‌هایی مانند \lr{Kaggle} و کارت‌های مدل از \lr{HuggingFace} را بازیابی کرده و فراداده ساخت‌یافته از جمله \persianfootnote{وجه‌های داده}\LTRfootnote{modalities}، متغیرهای هدف، معماری‌های مدل و بازه‌های ابرپارامتر را در راهنماهای متنی می‌گنجانند تا تصمیم‌های پایین‌دستی را غنی کنند (شکل \ref{fig:huggingGPT}) \cite{trirat2025automlagent, shen2023HuggingGPT}. برای داده‌مجموعه‌های نادیده، سازوکارهای \persianfootnote{انتقال مبتنی بر شباهت}\LTRfootnote{similarity-based transfer} با محاسبه همبستگی میان کدگذاری کارت‌های داده (با مدل‌هایی مانند \lr{CLIP}) مسائل مشابه را شناسایی کرده و ابرپارامترها یا الگوهای معماری را از تجربه‌های تاریخی منتقل می‌سازند \cite{zhang2023AutomlGPTAutomaticMachineLearning}.

این \persianfootnote{تقویت مبتنی بر فراداده}\LTRfootnote{metadata-driven augmentation} تعمیم‌پذیری میان حوزه‌های گوناگون را بدون نیاز به آموزش‌های خاص وظیفه ممکن می‌کند؛ هرچند به دسترس‌پذیری مخازن خوش‌سامان و برچسب‌گذاری دقیق فراداده متکی است.
\begin{figure}[h!]
    \centering
    \includegraphics[width=0.9\textwidth]{images/huggingGPT.png}
    \caption[ چارچوب HuggingGPT برای خودکارسازی یادگیری ماشین]{
        چارچوب \lr{HuggingGPT} که از مخازن مدل \lr{HuggingFace} برای خودکارسازی وظایف یادگیری ماشین بهره می‌گیرد~
        \cite{shen2023HuggingGPT}.
    }
    \label{fig:huggingGPT}

\end{figure}
\subsection[راهبردهای تقویت آمیخته]{\persianfootnote{راهبردهای تقویت آمیخته}\LTRfootnote{Hybrid Augmentation Strategies}}

سامانه‌های پیشرفته روز، غالباً چند منبع دانش را برای بهره‌گیری از قوت‌های مکمل با هم ترکیب می‌کنند. چارچوب‌های چندکارگزاره ممکن است \persianfootnote{برنامه‌ریزی تقویت‌شده با بازیابی}\LTRfootnote{retrieval-augmented planning} که دانش ادبیات و مخازن را برای راهبردهای سطح‌بالا به کار می‌گیرد را با حافظه رویدادی برآمده از گزارش‌های تجربی که اجرای عملی را صیقل می‌دهد، جفت کنند \cite{trirat2025automlagent, Yang2025NADER}. به‌همین سیاق، تاریخچه آزمون درون‌متنی می‌تواند با تجربه‌های استانداردسازی‌شده بازیابی‌شده غنی شود تا گرم‌آغاز بهینه‌سازی را به‌ویژه در مواجهه با وظایف نو با ارزیابی‌های اولیه محدود تسریع کند \cite{zhang-etal-2024-MLCopilot}.

گزینش معماری تقویت دانش، به‌طرز حساس با روش عملیاتی کارگزار برهم‌کنش دارد: دستوردهی تکرارشونده بیشترین سود را از خلاصه‌های فشرده تاریخی می‌برد؛ عملگرهای تکاملی برای نگهداشت تنوع به بایگانی‌های کیفیت-تنوع اتکا دارند؛ و \persianfootnote{کنترل‌گرهای جریان‌کار}\LTRfootnote{workflow controllers} برای هماهنگ‌سازی مراحل ناهمگون خط لوله، به فراداده ساخت‌یافته نیازمندند. با گسترش ظرفیت‌های پنجره زمینه و پختگی سازوکارهای بازیابی، مرز میان دانش درون‌متنی و بیرونی هرچه بیشتر محو می‌شود و ادغامی غنی‌تر از پیشین‌های آموخته، شواهد تجربی و خبرگی حوزه را امکان‌پذیر می‌سازد.