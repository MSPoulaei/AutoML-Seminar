\section{طبقه‌بندی بر اساس یکپارچه‌سازی دانش بیرونی}
در کنار توانایی‌های ذاتیِ پارامتریِ مدل‌های زبانی بزرگ، یکپارچه‌سازی دانش بیرونی نقشی حیاتی در ارتقای دقت، تازگی و قابلیت تعمیم رویکردهای مبتنی بر مدل‌های زبانی برای یادگیری ماشین خودکار ایفا می‌کند. این یکپارچه‌سازی به دو شیوهٔ اصلی صورت می‌پذیرد: \persianfootnote{تولید تقویت‌شده با بازیابی}\LTRfootnote{Retrieval-Augmented Generation (RAG)} که دانش ساختاریافته یا اسنادی را از پایگاه‌های دانش بیرونی استخراج و در فرایند استدلال مدل زبانی تزریق می‌کند، و \persianfootnote{ابزارهای جست‌وجو و واسط‌های برنامه‌نویسی}\LTRfootnote{Search Tools \& APIs} که امکان تعامل پویا با منابع محاسباتی، مخازن مدل و موتورهای جست‌وجوی برخط را فراهم می‌آورند. این تمایز بازتاب‌دهندهٔ یک طیف از استراتژی‌هاست: از سیستم‌های مستقلِ مبتنی بر دانش پارامتری تا چارچوب‌های زمینه‌آگاه و افزوده‌شدهٔ بیرونی. همان‌گونه که در ادامه تشریح می‌شود، انتخاب میان این رویکردها بر کارایی، تازگی دانش و پیچیدگی پیاده‌سازی اثرگذار است.

\subsection{رویکردهای مبتنی بر تولید تقویت‌شده با بازیابی}
رویکردهای مبتنی بر \lr{RAG} دانش بیرونی را از پایگاه‌های اسنادی، مخازن معماری‌های شبکه‌های عصبی یا مستندات تخصصی بازیابی کرده و آن را با دستورات ورودی ترکیب می‌کنند تا استدلال مدل زبانی را زمینه‌مند و غنی‌تر سازند. این راهبرد به‌ویژه در سناریوهایی که دانش حوزه‌ای تخصصی، معماری‌های پیش‌طراحی‌شده یا بهترین شیوه‌های به‌روز در دسترس‌اند، سودمند است.

چارچوب \lr{NADER}~\cite{Yang2025NADER} نمونه‌ای برجسته از این رویکرد است؛ این سامانه با بازیابی دانش معماری‌های سلسله‌مراتبی از یک پایگاهِ ساختاریافته، فرایند جست‌وجوی چندمرحله‌ای را راهنمایی می‌کند. عوامل تخصصی از توصیف‌های معماری‌های پیشین و الگوهای طراحی بهره می‌گیرند تا فضای جست‌وجو را هرس کنند و پیشنهادهای معماری را پالایش کنند؛ این امر منجر به همگرایی سریع‌تر و بهبود کیفیت معماری‌ها در وظایف بینایی رایانه‌ای می‌شود. به‌گونه‌ای مشابه، \lr{MLCopilot}~\cite{zhang-etal-2024-MLCopilot} از یکپارچه‌سازی دانش بیرونی برای تحلیل داده و انتخاب مدل بهره می‌گیرد؛ این چارچوب چندعاملی با بازیابی بهترین شیوه‌ها و راهنماهای حوزه‌ای، جریان‌های کاری مکالمه‌محور را پشتیبانی می‌کند و قابلیت حل مسئله را حتی در فضاهای جست‌وجوی محدود ارتقاء می‌دهد. چارچوب \lr{AutoMLAgent}~\cite{trirat2025automlagent} نیز از دانش بیرونی برای راهنمایی پیش‌پردازش داده، طراحی مدل و بهینه‌سازی استفاده می‌کند؛ عوامل تخصصی با بازیابی اسناد و نمونه‌های مرتبط، پالایش ماژولی و مشارکتی را ممکن می‌سازند.

مزیت اصلی رویکردهای مبتنی بر \lr{RAG} در توانایی آن‌ها برای تزریق دانش تازه و حوزه‌ویژه بدون نیاز به آموزش مجدد مدل زبانی نهفته است؛ این امر قابلیت تعمیم را در حوزه‌های نوظهور یا تخصصی افزایش می‌دهد. با این حال، کارایی این رویکردها به شدت به کیفیت، پوشش و ساختار پایگاه دانش بیرونی وابسته است؛ بازیابی نامرتبط یا ناقص می‌تواند منجر به استدلال نادرست یا پیشنهادهای زیربهینه شود~\cite{Yang2025NADER,zhang-etal-2024-MLCopilot}.

\subsection{رویکردهای مبتنی بر ابزارهای جست‌وجو و واسط‌های برنامه‌نویسی}
در مقابل، رویکردهای مبتنی بر ابزارهای جست‌وجو و واسط‌های برنامه‌نویسی بر تعامل پویا با منابع محاسباتی بیرونی، مخازن مدل و خدمات وب تمرکز دارند؛ بدین‌ترتیب مدل‌های زبانی نقش هماهنگ‌کنندهٔ دسترسی به ابزارهای تخصصی یا مدل‌های پیش‌آموزش‌دیده را ایفا می‌کنند تا وظایف پیچیده را حل کنند.

نمونهٔ بارز این رویکرد، \lr{HuggingGPT}~\cite{shen2023HuggingGPT} است؛ این سامانهٔ چندعاملی با محوریت \lr{ChatGPT}، وظایف پیچیده را به زیروظایف تفکیک کرده و با فراخوانی مدل‌های تخصصی از \lr{Hugging Face} از طریق واسط‌های برنامه‌نویسی، آن‌ها را به‌صورت پویا یکپارچه می‌سازد. این رویکرد امکان برنامه‌ریزی و اجرای جریان‌های کاریِ چندوجهی (برای مثال ترکیب بینایی، زبان و صوت) را بدون پیش‌طراحی خطوط لوله میسر می‌کند و مقیاس‌پذیری و انعطاف‌پذیری را از طریق اتصال ابزارهای بیرونی نشان می‌دهد. به‌گونه‌ای مشابه، \lr{AutoML-GPT}~\cite{zhang2023AutomlGPTAutomaticMachineLearning} اگرچه عمدتاً بر کُدزایی تکیه دارد، می‌تواند با ابزارهای بیرونی (مانند کتابخانه‌های پایتون) برای پیش‌پردازش، انتخاب مدل و ارزیابی تعامل کند؛ مدل زبانی نقش اجراکنندهٔ کُد و هماهنگ‌کنندهٔ منابع محاسباتی را ایفا می‌کند.

مزیت کلیدی این رویکردها در توانایی بهره‌گیری از اکوسیستم‌های گستردهٔ مدل‌های پیش‌آموزش‌دیده، ابزارها و خدمات است؛ بدین‌ترتیب نیاز به آموزش یا توسعهٔ مدل‌های اختصاصی کاهش می‌یابد. با این حال، این رویکردها سربار ارتباط شبکه، مسائل مربوط به تأخیر و پیچیدگی‌های مدیریت خطا در محیط‌های توزیع‌شده را به‌همراه دارند؛ علاوه بر این، اتکا به واسط‌های برنامه‌نویسی بیرونی می‌تواند استقلال و قابلیت بازتولید را محدود کند~\cite{shen2023HuggingGPT}.

\subsection{رویکردهای مستقل بدون دانش بیرونی}
بخش قابل‌توجهی از رویکردهای موجود بر دانش پارامتریِ مدل‌های زبانی بزرگ تکیه می‌کنند بدون آنکه دانش بیرونی را یکپارچه سازند. این رویکردهای مستقل از پیش‌آموزش گستردهٔ مدل‌های زبانی بر روی پیکره‌های متنوع (شامل کُد، مستندات علمی و مقالات فنی) سود می‌برند تا دانش ضمنی دربارهٔ معماری‌های یادگیری ماشین، ابرپارامترها و بهترین شیوه‌ها را استخراج کنند.

برای مثال، رویکرد پیشنهادی در~\cite{zhang2023usingLLMforHPO} از دستورات ورودی ساختاریافته برای راهنمایی یک عامل منفرد در پیشنهاد و پالایش تکرارشوندهٔ پیکربندی‌های ابرپارامتری استفاده می‌کند، صرفاً بر پایهٔ توصیف‌های داده و مدل؛ نتایج نشان می‌دهد که حتی بدون بازیابی بیرونی، عملکردی رقابتی با بهینه‌سازی بیزی در وظایف کم‌هزینه حاصل می‌شود. به‌گونه‌ای مشابه، \lr{GENIUS}~\cite{zheng2023GENIUS} از توانایی بی‌نمونهٔ مدل‌های زبانی برای تولید و رتبه‌بندی معماری‌ها بر پایهٔ توصیف‌های زبان طبیعی بهره می‌گیرد، بدون اتکا به پایگاه‌های دانش معماری. \lr{EvoPrompting}~\cite{chen2023Evoprompting} نیز با تکامل جمعیتی پرامپت‌ها از طریق الگوریتم‌های ژنتیکی، کُد شبکه‌های عصبی تولید می‌کند بدون نیاز به دانش بیرونی.

این رویکردهای مستقل از سادگی پیاده‌سازی، استقلال از زیرساخت‌های بیرونی و قابلیت بازتولید بهره‌مند می‌شوند. با این حال، ممکن است در حوزه‌های تخصصی یا نوظهور که دانش آن‌ها در داده‌های پیش‌آموزش کم‌نمایندگی شده، عملکرد ضعیف‌تری داشته باشند؛ علاوه بر این، قدیمی‌شدن دانش پارامتری (به‌سبب عدم به‌روزرسانی مدل) می‌تواند دقت را در کاربردهای حساس به زمان محدود کند~\cite{zhang2023usingLLMforHPO,zheng2023GENIUS}.

\paragraph{جمع‌بندی و چشم‌انداز.}
انتخاب راهبرد یکپارچه‌سازی دانش بیرونی نقش تعیین‌کننده‌ای در موفقیت رویکردهای مبتنی بر مدل‌های زبانی برای یادگیری ماشین خودکار ایفا می‌کند. رویکردهای \lr{RAG} برای حوزه‌های غنی از دانش ساختاریافته و نیازمند تازگی دانش مناسب‌اند؛ رویکردهای مبتنی بر ابزار برای وظایف پیچیدِ چندوجهی که از اکوسیستم‌های موجود سود می‌برند کارآمدند؛ و رویکردهای مستقل در سناریوهای محدود به منابع یا زمانی که سادگی و قابلیت بازتولید اولویت دارند، برتری دارند. مسیرهای آتی تحقیق می‌تواند شامل راهبردهای ترکیبی باشد که به‌صورت سازگار میان این رویکردها بر حسب مشخصات وظیفه، در دسترس‌بودن دانش و محدودیت‌های محاسباتی جابجا شوند؛ همچنین توسعهٔ پایگاه‌های دانش سازگار با یادگیری ماشین خودکار و ابزارهای بازیابی بهینه‌شده برای استدلال مدل‌های زبانی می‌تواند کارایی رویکردهای مبتنی بر \lr{RAG} را بیشتر ارتقا دهد.
