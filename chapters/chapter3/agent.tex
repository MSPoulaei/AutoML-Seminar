\section{طبقه‌بندی بر اساس معماری عامل}
در چشم‌انداز رو‌به‌تحول رویکردهای مبتنی بر \persianfootnote{مدل‌های زبانی بزرگ}\LTRfootnote{Large Language Models (LLMs)} برای \persianfootnote{یادگیری ماشین خودکار}\LTRfootnote{Automated Machine Learning (AutoML)}، \persianfootnote{جست‌وجوی معماری شبکه‌های عصبی}\LTRfootnote{Neural Architecture Search (NAS)} و \persianfootnote{بهینه‌سازی ابرپارامتر}\LTRfootnote{Hyperparameter Optimization (HPO)}، یک تمایز بنیادین در \persianfootnote{معماری عامل}\LTRfootnote{Agent Architecture} برجسته است: \persianfootnote{تک‌عاملی}\LTRfootnote{Single-Agent} در برابر \persianfootnote{چندعاملی}\LTRfootnote{Multi-Agent}. این طبقه‌بندی بازتاب‌دهنده نحوه سازمان‌دهی مدل‌های زبانی برای انجام وظایف بهینه‌سازی است؛ در رویکردهای تک‌عاملی، یک نمونه منفرد از مدل زبانی کل فرایند را به‌صورت خودکار و تکرارشونده به‌عهده دارد، در حالی‌که در رویکردهای چندعاملی، مسئولیت‌ها میان چند عامل تخصصی توزیع می‌شود تا هم‌افزایی، کارایی و توان حل مسئله افزایش یابد.\\ همان‌گونه که در جدول~\ref{tab:recent-works} خلاصه شده است، این دوگانگی بر جنبه‌هایی همچون \persianfootnote{کدزایی}\LTRfootnote{Code Generation}، \persianfootnote{یکپارچه‌سازی دانش بیرونی}\LTRfootnote{External Knowledge Integration} و \persianfootnote{کاوش فضای جست‌وجو}\LTRfootnote{Search Space Exploration} اثرگذار است.

\subsection{رویکردهای تک‌عاملی}
معماری‌های تک‌عاملی بر این فرض استوارند که یک مدل زبانی واحد با ظرفیت کافی می‌تواند کل خط لوله بهینه‌سازی را مدیریت کند. این رویکردها را می‌توان بر اساس \emph{سازوکار جست‌وجو} و \emph{نحوه بهره‌گیری از بازخورد} به چهار دسته اصلی تقسیم کرد: (۱) روش‌های مبتنی بر پالایش تکرارشونده، (۲) رویکردهای تکاملی-مولد، (۳) راهبردهای مبتنی بر جانشین‌های کم‌هزینه، و (۴) چارچوب‌های یکپارچه‌سازی سراسری.

\subsubsection{پالایش تکرارشونده مبتنی بر بازخورد}
دسته‌ای از روش‌های تک‌عاملی بر \persianfootnote{حلقه‌های بازخورد}\LTRfootnote{Feedback Loops} تکیه می‌کنند که در آن مدل زبانی بر اساس نتایج ارزیابی‌های پیشین، پیکربندی‌ها را به‌صورت تکرارشونده پالایش می‌کند. در این راستا، \cite{zhang2023usingLLMforHPO} نشان داده‌اند که یک عامل منفرد می‌تواند با هدایت از طریق \persianfootnote{دستور ورودی}\LTRfootnote{Prompting} و دریافت بازخورد کارایی، ابرپارامترها را در وظایف متنوعی از رگرسیون تا مسائل ترکیباتی بهینه کند و در سناریوهای کم‌هزینه، عملکردی رقابتی با \persianfootnote{بهینه‌سازی بیزی}\LTRfootnote{Bayesian Optimization} ارائه دهد. به‌طور مشابه، \cite{zheng2023GENIUS} با تکیه بر توصیف‌های زبان طبیعی و پالایش تکرارشونده بدون اتکا به الگوریتم‌های جست‌وجوی کلاسیک، در معیارهایی چون \lr{NAS-Bench-201} نشان داده‌اند که استدلال ذاتی مدل‌های زبانی می‌تواند در پیمایش و رتبه‌بندی معماری‌ها به‌کار گرفته شود. همچنین \cite{zhang2023AutomlGPTAutomaticMachineLearning} این رویکرد را به کل خط لوله یادگیری ماشین تعمیم داده و نشان داده‌اند که یک عامل واحد می‌تواند از تولید کُد پیش‌پردازش تا انتخاب مدل و ارزیابی را به‌صورت خودکار مدیریت کند.

\emph{مزایای اصلی} این رویکردها شامل سادگی پیاده‌سازی، کاهش سربار ارتباطی، و انعطاف‌پذیری در برابر توصیف‌های زبان طبیعی است. با این حال، \emph{محدودیت بنیادین} آن‌ها در وابستگی به ظرفیت استدلالی مدل پایه و احتمال همگرایی به نقاط بهینه محلی در فضاهای جست‌وجوی پیچیده نهفته است، زیرا دیدگاه واحدی برای کاوش فضا به‌کار می‌رود.

\subsubsection{تلفیق با الگوریتم‌های تکاملی و تنوع‌محور}
برای غلبه بر محدودیت‌های کاوش تک‌دیدگاهی، دسته دوم از روش‌های تک‌عاملی مدل‌های زبانی را با \persianfootnote{الگوریتم‌های تکاملی}\LTRfootnote{Evolutionary Algorithms} و راهبردهای \persianfootnote{کیفیت-تنوع}\LTRfootnote{Quality-Diversity (QD)} ادغام می‌کنند تا تنوع جست‌وجو را افزایش دهند. \cite{LLMatic2024} با ترکیب مدل زبانی با بهینه‌سازی کیفیت-تنوع، نشان داده‌اند که تولید جهش‌های کُد-محور و ارزیابی آن‌ها در \persianfootnote{بایگانی تنوع‌محور}\LTRfootnote{Diversity-focused Archive} می‌تواند نسبت به جست‌وجوی تصادفی، تنوع و کیفیت معماری‌ها را همزمان بهبود بخشد. در رویکردی مشابه، \cite{chen2023Evoprompting} با الهام از \persianfootnote{الگوریتم‌های ژنتیکی}\LTRfootnote{Genetic Algorithms}، پرامپت‌ها را به‌صورت جمعیتی تکامل می‌دهند (از طریق جهش و ترکیب) تا کُد شبکه‌های عصبی تولید کنند، و نتایج قدرتمندی در معیارهای جست‌وجوی معماری گزارش کرده‌اند. همچنین \cite{Yu2025GPTNAS} با کُدگذاری معماری‌ها در قالب دستور ورودی و تکامل آن‌ها از طریق انتخاب و جهش، نشان داده‌اند که پیش‌آموزش مولد می‌تواند سرعت همگرایی را نسبت به الگوریتم‌های تکاملی کلاسیک بهبود بخشد.

این رویکردهای ترکیبی \emph{تعادلی} میان بهره‌گیری از توان استدلالی مدل‌های زبانی و قدرت کاوش الگوریتم‌های تکاملی برقرار می‌کنند. با این حال، آن‌ها معمولاً نیازمند تعداد بیشتری ارزیابی و تنظیم دقیق‌تر فراپارامترهای تکاملی هستند که می‌تواند هزینه محاسباتی را افزایش دهد.

\subsubsection{جست‌وجوی کارآمد با جانشین‌های کم‌هزینه}
برای کاهش هزینه محاسباتی ارزیابی‌های پرشمار، دسته سوم از روش‌ها مدل‌های زبانی را با \persianfootnote{جانشین‌های صفرهزینه}\LTRfootnote{Zero-cost Proxies} یا \persianfootnote{جست‌وجوی بی‌گرادیان}\LTRfootnote{Gradient-free Search} ادغام می‌کنند. \cite{ji2025RZNAS} چارچوب «\persianfootnote{صفرهزینهٔ بازتابی}\LTRfootnote{Reflective Zero-cost}» را معرفی کرده‌اند که در آن مدل زبانی معماری‌ها را پیشنهاد می‌کند و آن‌ها را بر پایهٔ جانشین‌های بی‌هزینه مانند \persianfootnote{جریان سیناپسی}\LTRfootnote{Synaptic Flow} سریع ارزیابی می‌کند و جهت جست‌وجو را در \persianfootnote{حلقه‌های بازتاب}\LTRfootnote{Reflection Loops} اصلاح می‌نماید؛ نتایج آن‌ها در \lr{CIFAR-10} و \lr{ImageNet} همگرایی سریع‌تر و معماری‌های بهتری نسبت به خطوط پایهٔ مبتنی بر مدل زبانی نشان می‌دهد. به‌طور مشابه، \cite{sarah2024llamaNAS} با تکیه بر جست‌وجوی بی‌گرادیان و \persianfootnote{قیود آگاه به سخت‌افزار}\LTRfootnote{Hardware-aware Constraints}، نشان داده‌اند که نمونه‌برداری و ارزیابی تکرارشونده بدون آموزش کامل می‌تواند هزینهٔ جست‌وجو را کاهش دهد و دقت رقابتی را حفظ کند.

این راهبردها \emph{کارایی محاسباتی} را به‌طور چشمگیری بهبود می‌بخشند و برای کاربردهایی که منابع محدود دارند یا نیاز به همگرایی سریع دارند، مناسب هستند. با این حال، دقت جانشین‌های کم‌هزینه در پیش‌بینی کارایی نهایی همواره محدودیتی اساسی است و ممکن است در برخی حوزه‌ها به معماری‌های زیربهینه منجر شود.

\subsubsection{جمع‌بندی و تحلیل تطبیقی رویکردهای تک‌عاملی}
رویکردهای تک‌عاملی \emph{سادگی معماری} و \emph{هزینه پیاده‌سازی کمتر} را ارائه می‌دهند، زیرا سربار هماهنگی میان عوامل را حذف می‌کنند. آن‌ها به‌ویژه در سناریوهایی که فضای جست‌وجو نسبتاً ساده است یا منابع محاسباتی محدود هستند، عملکرد خوبی دارند~\cite{zhang2023usingLLMforHPO,zheng2023GENIUS,ji2025RZNAS}. با این حال، در وظایف چندبُعدی و پیچیده که از تنوع دیدگاه‌ها یا فروکاست ماژولی سود می‌برند، ممکن است دچار کاستی شوند. همچنین، کیفیت جست‌وجو به‌شدت به ظرفیت استدلالی مدل پایه وابسته است و در صورت ضعف مدل در استدلال یا تولید کُد، کل سامانه آسیب‌پذیر می‌شود.

\subsection{رویکردهای چندعاملی}
معماری‌های چندعاملی از تفکیک مسئولیت‌ها و تخصص‌گرایی برای مدیریت پیچیدگی بهینه‌سازی بهره می‌گیرند. این رویکردها را می‌توان بر اساس \emph{الگوی سازمان‌دهی} و \emph{نحوه تعامل میان عوامل} به سه دسته اصلی تقسیم کرد: (۱) معماری‌های مبتنی بر تخصص نقشی، (۲) رویکردهای مبتنی بر ارکستراسیون سلسله‌مراتبی، و (۳) چارچوب‌های تلفیقی با الگوریتم‌های بهینه‌سازی کلاسیک.

\subsubsection{تخصص نقشی و جریان کاری مشارکتی}
رایج‌ترین الگوی سازمان‌دهی در معماری‌های چندعاملی، تفکیک مسئولیت‌ها به نقش‌های تخصصی است که معمولاً شامل \emph{برنامه‌ریزی}، \emph{کُدزایی/طراحی}، \emph{ارزیابی} و \emph{نقادی} می‌شود. \cite{xu2024largeTextToML} چارچوبی را ارائه کرده‌اند که توصیف‌های زبانی وظایف را به خط‌لوله‌های اجرایی ترجمه می‌کند و نقش‌های برنامه‌ریز، کُدزا و ارزیاب را میان عوامل توزیع می‌نماید؛ آزمایش‌های آن‌ها سودمندی هم‌افزایی عوامل را در سناریوهای بی‌نمونه نشان می‌دهد. به‌طور مشابه، \cite{liu2025agenthpo} مدلی گفت‌وگومحور را معرفی کرده‌اند که در آن عوامل پیشنهاددهنده، ارزیاب و بهینه‌ساز در حلقه‌های تکرارشونده همکاری می‌کنند و بدون نیاز به کُدزایی، تنظیمی رقابتی در شبکه‌های عمیق ارائه می‌دهند. \cite{trirat2025automlagent} این رویکرد را به کل خط لوله \lr{AutoML} گسترش داده و از عوامل تخصصی در پیش‌پردازش داده، طراحی مدل و بهینه‌سازی بهره می‌گیرند و نسبت به روش‌های تک‌عاملی کارایی برتری را در معیارهای \lr{OpenML} گزارش کرده‌اند.

\emph{مزیت کلیدی} این رویکردها در توانایی \emph{پالایش ماژولی مشارکتی} است که هر عامل می‌تواند بر جنبه خاصی از مسئله تمرکز کند و از تداخل یا تضاد میان اهداف مختلف جلوگیری شود. علاوه بر این، تخصص نقشی امکان بهره‌گیری از مدل‌های مختلف یا راهبردهای پرامپتینگ متفاوت برای هر نقش را فراهم می‌کند که می‌تواند کیفیت کلی را بهبود بخشد.

\subsubsection{ارکستراسیون سلسله‌مراتبی و یکپارچه‌سازی ابزار}
دسته دوم از روش‌های چندعاملی بر \emph{سازمان‌دهی سلسله‌مراتبی} و \emph{یکپارچه‌سازی ابزارهای بیرونی} تمرکز دارند. \cite{shen2023HuggingGPT} سامانه‌ای را معرفی کرده‌اند که در آن یک عامل مرکزی (\lr{ChatGPT}) وظایف پیچیده را به زیروظایف تفکیک می‌کند و با عوامل تخصصی که از مدل‌های \lr{Hugging Face} از طریق \lr{API} بهره می‌گیرند، هماهنگ می‌شود؛ ارزیابی‌های آن‌ها در وظایف چندوجهی نشان می‌دهد که برنامه‌ریزی عاملی و اتصال ابزارها می‌تواند خودکارسازی کارای جریان‌های پیچیده را میسر سازد. به‌طور مشابه، \cite{zhang-etal-2024-MLCopilot} چارچوبی مکالمه‌محور ارائه کرده‌اند که عوامل تحلیل داده، انتخاب مدل و رفع خطا را با یکپارچه‌سازی دانش بیرونی به‌کار می‌گیرند و نشان می‌دهند که همکاری میان عوامل حتی در فضاهای جست‌وجوی محدود نیز حل مسئله را ارتقاء می‌بخشد. \cite{Yang2025NADER} با معرفی رویکردی برای جست‌وجوی معماری که طراحی سلسله‌مراتبی را از طریق ساختارهای درختی پیش می‌برد، نشان داده‌اند که مذاکره میان عوامل و اشتراک دانش می‌تواند مقیاس‌پذیری جست‌وجو را در وظایف بینایی رایانه‌ای بهبود بخشد.

این معماری‌های سلسله‌مراتبی \emph{مقیاس‌پذیری} و \emph{قابلیت ترکیب}\LTRfootnote{Composability} بالاتری ارائه می‌دهند، زیرا عوامل جدید یا ابزارهای بیرونی را می‌توان به‌راحتی به سامانه افزود. با این حال، آن‌ها نیازمند \emph{سازوکارهای هماهنگی پیچیده‌تر} هستند و احتمال خطا در ارتباطات میان‌عاملی و انتشار خطا در سلسله‌مراتب افزایش می‌یابد.

\subsubsection{تلفیق با بهینه‌سازی بیزی و جست‌وجوی هدایت‌شده}
دسته سوم از رویکردهای چندعاملی به‌دنبال ترکیب مزایای استدلال زبانی با دقت الگوریتم‌های بهینه‌سازی کلاسیک هستند. \cite{liu2024LLAMBO} با ادغام بهینه‌سازی بیزی در سامانه‌ای چندعاملی نشان داده‌اند که عوامل می‌توانند \persianfootnote{مدل‌سازی جانشین}\LTRfootnote{Surrogate Modeling} و \persianfootnote{تابع اکتساب}\LTRfootnote{Acquisition Function} را بهبود بخشند و با کاهش شمار ارزیابی‌ها، حلقهٔ بیزی را به‌طور کارآمد راهبری کنند. این رویکرد ترکیبی نشان می‌دهد که استدلال زبانی می‌تواند به‌عنوان \emph{اکتشافی سطح بالا}\LTRfootnote{High-level Heuristic} برای هدایت الگوریتم‌های بهینه‌سازی کلاسیک به‌کار رود و تعادلی میان بهره‌برداری و کاوش برقرار کند.

این رویکردهای تلفیقی \emph{دقت بالاتر} و \emph{ضمانت‌های همگرایی بهتر} نسبت به روش‌های خالصاً مبتنی بر مدل زبانی ارائه می‌دهند، زیرا از اصول ریاضی محکم الگوریتم‌های کلاسیک بهره می‌برند. با این حال، آن‌ها پیچیدگی پیاده‌سازی بالاتر و نیاز به تخصص در هر دو حوزه مدل‌های زبانی و بهینه‌سازی را به‌همراه دارند.

\subsubsection{جمع‌بندی و تحلیل تطبیقی رویکردهای چندعاملی}
رویکردهای چندعاملی \emph{تاب‌آوری} و \emph{توان مدیریت پیچیدگی} بالاتری نسبت به معماری‌های تک‌عاملی ارائه می‌دهند~\cite{xu2024largeTextToML,trirat2025automlagent,liu2024LLAMBO}. تفکیک مسئولیت‌ها امکان بهره‌گیری از تخصص‌گرایی را فراهم می‌کند و می‌تواند به معماری‌های بهتر و راه‌حل‌های جامع‌تر منجر شود. علاوه بر این، معماری‌های چندعاملی \emph{انعطاف‌پذیری} بیشتری دارند، زیرا هر عامل می‌تواند از مدل، راهبرد پرامپتینگ یا منابع دانش متفاوتی استفاده کند. با این حال، آن‌ها با چالش‌هایی روبه‌رو هستند: (۱) \emph{سربار هماهنگی} که هزینه محاسباتی و زمانی را افزایش می‌دهد، (۲) \emph{پیچیدگی طراحی} در تعریف نقش‌ها و پروتکل‌های ارتباطی، و (۳) \emph{احتمال ناهمخوانی} در تعاملات میان عوامل که ممکن است به تصمیمات متناقض منجر شود.

\subsection{تحلیل تطبیقی و چشم‌انداز}
\label{subsec:agent-comparative}

مقایسه جامع رویکردهای تک‌عاملی و چندعاملی چندین الگوی کلیدی را آشکار می‌سازد. \textbf{اولاً}، رویکردهای تک‌عاملی در \emph{سناریوهای با فضای جست‌وجوی محدود} یا \emph{وظایف تک‌هدفه} عملکرد رقابتی ارائه می‌دهند و در عین حال هزینه پیاده‌سازی و محاسباتی کمتری دارند~\cite{zhang2023usingLLMforHPO,zheng2023GENIUS,ji2025RZNAS}. با این حال، با افزایش پیچیدگی وظیفه، محدودیت‌های آن‌ها در کاوش چندبُعدی برجسته‌تر می‌شود. \textbf{ثانیاً}، رویکردهای چندعاملی در \emph{وظایف چندوجهی} و \emph{خطوط لوله پیچیده} برتری دارند، زیرا می‌توانند مسئولیت‌ها را به‌طور مؤثر تفکیک کنند و از تخصص‌گرایی بهره بگیرند~\cite{trirat2025automlagent,liu2024LLAMBO,Yang2025NADER}. \textbf{ثالثاً}، رویکردهای ترکیبی که مدل‌های زبانی را با الگوریتم‌های کلاسیک (تکاملی یا بیزی) ادغام می‌کنند، تعادل بهتری میان کارایی محاسباتی و کیفیت جست‌وجو برقرار می‌کنند~\cite{LLMatic2024,chen2023Evoprompting,liu2024LLAMBO}.

این طبقه‌بندیِ عامل‌محور بر گرایش فزاینده به سوی \emph{ساختارهای سازگار}\LTRfootnote{Adaptive Architectures} دلالت دارد که می‌توانند به‌صورت پویا میان معماری‌های تک‌عاملی و چندعاملی جابجا شوند. مسیرهای تحقیقاتی آینده می‌توانند شامل: (۱) \emph{سازوکارهای جابجایی پویا} که بر اساس ویژگی‌های وظیفه (پیچیدگی، ابعاد فضای جست‌وجو، منابع محاسباتی) به‌طور خودکار معماری مناسب را انتخاب کنند، (۲) \emph{معماری‌های سلسله‌مراتبی ترکیبی} که در آن یک عامل هماهنگ‌کننده سطح بالا، چندین زیرعامل تخصصی را مدیریت کند و در صورت لزوم به حالت تک‌عاملی بازگردد، (۳) \emph{یادگیری تقویتی برای بهینه‌سازی تعامل میان عوامل} تا پروتکل‌های ارتباطی و تخصیص نقش‌ها به‌طور خودکار بهینه شود، و (۴) \emph{بهره‌گیری از مدل‌های متفاوت برای عوامل مختلف} به‌گونه‌ای که مدل‌های تخصصی‌تر و کارآمدتر برای هر نقش به‌کار گرفته شوند باشد.
