\chapter{تعاریف و مفاهیم مبنایی }
\thispagestyle{empty}
\section{مقدمه}
در اینجا، تلاش شده است تا مفاهیم اصلی و تعاریف مرتبط با موضوع به طور خلاصه معرفی شوند. هدف اصلی این است که یک نگاه کلی به مهم‌ترین مفاهیم و اصطلاحات مورد استفاده در این فصل داده شود.
\\
استفاده از سامانه‌های \persianfootnote{تشخیص به کمک رایانه}\LTRfootnote{Computer-Aided Diagnosis(CAD)} می‌تواند به پزشکان در تشخیص و درمان بیماری‌ها کمک کند. این سیستم‌ها با استفاده از الگوریتم‌ها و فنون پردازش تصویر و داده، اطلاعات بالینی و تصاویر پزشکی را تحلیل می‌کنند و نتایجی ارائه می‌دهند که به پزشک در تصمیم‌گیری و تشخیص بیماری کمک می‌کنند. از مزایای استفاده از این سیستم‌ها می‌توان به دسترسی به اطلاعات جامع (این سیستم‌ها قادرند به طور همزمان به مجموعه‌ای بزرگ از اطلاعات پزشکی و تصاویر دسترسی داشته باشند و از این طریق می‌توانند به پزشکان در تشخیص بهتر و سریع‌تر بیماری‌ها کمک کنند)، افزایش دقت تشخیص، کاهش خطای پزشکی و افزایش سرعت تشخیص اشاره کرد.
\\
با این حال، باید توجه داشت که سامانه‌های تشخیص به کمک رایانه، فقط به عنوان ابزاری اولیه و پشتیبان برای پزشکان عمل می‌کنند و تصمیم‌گیری نهایی و درمان بر عهده پزشک است. همچنین، دقت و کارایی این سامانه‌ها بستگی به کیفیت و دقت تصاویر پزشکی، داده‌های ورودی و الگوریتم‌های استفاده شده دارد. بنابراین، نیاز است تا سامانه‌های تشخیص به کمک رایانه به طور مداوم بهبود و توسعه یابند تا بتوانند بهترین کارایی را در تشخیص و درمان اختلالات پزشکی ارائه دهند\cite{pandey2022comprehensive}.
\\
هرچند شناسایی بیماری به پزشک کمک می کند اما در بسیاری از موارد تعیین موضع بیماری می تواند در درمان بهتر کمک کننده باشد. لذا ‌سامانه‌های قطعه‌بندی که علاوه بر تشخیص، ناحیه مورد نظر را نیز تعیین می کنند می توانند به تیم پزشکی کمک کنند تا با سرعت و دقت بیشتری به فرآیند درمان بپردازند. این سیستم‌ها در طول درمان از قابلیت بالایی برای کمک به پزشکان برخوردارند و مطالعات گسترده ای بر روی آنها در حال انجام است.
\\
تازه‌ترین پیشرفت‌ها در فناوری پردازش گرافیک و یادگیری ماشین، به ویژه روش‌های یادگیری عمیق، تغییرات قابل توجهی را در زمینه قطعه‌بندی تصویربرداری پزشکی و تحلیل تصاویر مغز به همراه داشته است. روش‌های معمول در قطعه‌بندی تصویر مغز معمولاً بر اساس ویژگی‌های ساده طراحی می‌شوند و توسط انسان انجام می‌شوند. با این حال، روش‌های یادگیری عمیق با استفاده از \persianfootnote{شبکه‌های عصبی پیچشی}\LTRfootnote{Convolutional Neural Networks} و \persianfootnote{مبدل ها}\LTRfootnote{Transformers} توانایی بالاتری در خودکارسازی قطعه‌بندی تصاویر را ارائه می‌دهند.
\\
مزایای استفاده از روش‌های یادگیری عمیق شامل مقاومت بیشتر در برابر شرایط مختلف تصاویر، دقت بالا در قطعه‌بندی و توانایی تشخیص ناحیه‌های پیچیده‌تر است. این شبکه‌ها توانمندی دارند که ویژگی‌های متنوع و مهم را به صورت خودکار بدون نیاز به دستکاری انسانی در طراحی ویژگی‌ها فرابگیرند و تشخیص دهند. به طور مشابه، در زمینه قطعه‌بندی تومور مغز، روش‌های مبتنی بر یادگیری عمیق به عنوان جایگزینی کارآمد برای مدل‌سازی صریح و طراحی دستی ویژگی‌ها به کار گرفته شده‌اند. این پیشرفت‌ها بهبود معناداری در دقت و کارایی قطعه‌بندی تصاویر تومور مغز ارائه داده‌اند\cite{you2022eg}.
\\
با این توصیف، در فصل‌های بعدی می‌توانیم به تفصیل به تحلیل موضوعات مرتبط با این مفاهیم بپردازیم و اطلاعات بیشتری را ارائه دهیم. این مرحله اولیه معمولاً برای آشنایی اولیه با موضوع و تسلط به اصطلاحات و مفاهیم مورد نیاز است.

\section{ساختار و اهمیت مغز}
 مغز اندام پیچیده ای است که فکر، حافظه، احساسات، قدرت لمس، مهارت های حرکتی، بینایی، تنفس، دما، گرسنگی و هر فرآیندی که بدن ما را تنظیم می کند را کنترل می کند. مغز و \persianfootnote{نخاع}\LTRfootnote{Spinal Cord} که از آن امتداد می‌یابند، با هم، \persianfootnote{سامانه عصبی مرکزی}\LTRfootnote{Central Nervous System} را تشکیل می‌دهند.
\\
\persianfootnote{ماده خاکستری و سفید}\LTRfootnote{Gray and White Matter} دو ناحیه متفاوت از دستگاه عصبی مرکزی هستند. در مغز، ماده خاکستری به قسمت بیرونی و تیره‌تر اشاره دارد، در حالی که ماده سفید به بخش داخلی و روشن‌تر زیر آن اشاره می‌کند. در نخاع، این ترتیب معکوس است: ماده سفید در خارج است، و ماده خاکستری در داخل قرار دارد.
\\
\begin{figure}[ht]
\centerline{\includegraphics[width=15cm]{images/gray_white_matter.png}}
\caption[برش های مغز]{برش های مغز و نخاع که ماده خاکستری و سفید را نشان می دهد\cite{brain_anatomy}}
\label{fig:gray-white-matter}
\end{figure}
\\
ماده خاکستری عمدتاً از \persianfootnote{بدنه‌های عصبی}\LTRfootnote{Neuron Somas} (جسم سلول مرکزی گرد) و ماده سفید عمدتاً از آکسون ها (ساقه های بلندی که نورون ها را به یکدیگر متصل می کند) ساخته شده است که در \persianfootnote{ماده چربی پوشش دهنده آکسون}\LTRfootnote{Myelin} (پوشش محافظ) پیچیده شده است. ترکیب متفاوت قطعات نورون به همین دلیل است که این دو به صورت سایه های جداگانه دراسکن های خاص ظاهر می شوند.
\\
\begin{figure}[h]
\centerline{\includegraphics[width=8cm]{images/neuron_anatomy.png}}
\caption[ساختمان نورون]{ساختمان نورون.بدنه سلول همراه با هسته داخلی و آکسون بلند عایق شده توسط ماده میلین\cite{brain_anatomy}}
\label{fig:neuron-anatomy}
\end{figure}
\\
در دستگاه عصبی، هر منطقه و ناحیه وظیفه‌ها و نقش‌های مختلفی را ایفا می‌کند. ماده خاکستری و ماده سفید از جمله دو عنصر اصلی در ساختار مغز هستند و هر یک وظیفه خاصی دارند. ماده خاکستری از نورون‌ها و سلول‌های عصبی تشکیل شده است و در درجه اول برای پردازش و تفسیر اطلاعات از منابع مختلف مسئولیت دارد. این منطقه برای فعالیت‌های مرتبط با تفکر، ادراک، حافظه، و کارهای شناختی مهم است. ماده سفید وظیفه انتقال آخرین اطلاعات از یک منطقه به منطقه دیگر در دستگاه عصبی در سریعترین زمان ممکن را عهده‌دار است. این ماده شامل میلین است که یک عایق از جنس چربی بر روی آکسون ها است و سرعت انتقال سیگنال‌های عصبی را افزایش می‌دهد. از این طریق، هر دو ماده خاکستری و ماده سفید به تعادل و عملکرد صحیح دستگاه عصبی کمک می‌کنند، و هر یک نقش مهمی در انتقال و پردازش اطلاعات در دستگاه عصبی دارند\cite{brain_anatomy}.
\section{ساختار و انواع تومورهای مغزی}
تومور مغزی با رشد غیرطبیعی سلول ها در مغز مشخص می شود. مغز انسان دارای \persianfootnote{ساختار}\LTRfootnote{Anatomy} بسیار پیچیده ای است و مناطق مختلفی مسئول عملکردهای مختلف در دستگاه عصبی هستند. تومورهای مغزی می توانند در هر ناحیه ای از مغز یا جمجمه ظاهر شوند، از جمله پوشش محافظ، قاعده جمجمه، ساقه مغز، سینوس ها، حفره بینی و بسیاری از مناطق دیگر. طبقه بندی تومورهای مغزی متنوع است، با بیش از 120 نوع مختلف، بسته به بافت خاصی که از آن منشأ می گیرند\cite{brain_tumor}. انواع این تومورها را می توانید در جدول \ref{tab:tumor-types} ببینید.
\\
\begin{table}[h]
\caption[انواع تومورهای مغزی]{انواع تومورهای مغزی\cite{ranjbarzadeh}}
\label{tab:tumor-types}
\centering
\onehalfspacing
\begin{tabular}{|c|c|}
\hline انواع تومور مغزی & زیرنوع\\ 
\hline 
\multirow{9}{*}{گلیوما} & آستروسیتوم\\
                        &آستروسیتوم پیلوسیتیک (درجه \lr{I})\\
                        &آستروسیتومای منتشر (درجه \lr{II})\\
                        &آستروسیتوم آناپلاستیک (درجه \lr{III})\\
                        &گلیوبلاستوما مولتی فرم (درجه \lr{IV})\\
                        &الیگودندروگلیوما (درجه \lr{II})\\
                        &الیگودندروگلیوم آناپلاستیک (درجه \lr{III})\\
                        &اپندیموما (درجه \lr{II})\\
                        &اپندیموم آناپلاستیک (درجه \lr{III})\\
\hline
\multirow{1}{*}{کرانیوفارژیوم} & -\\
\hline
\multirow{1}{*}{اپیدرموئید} & -\\
\hline
\multirow{1}{*}{لنفوم} & -\\
\hline
\multirow{1}{*}{مننژیوم} & -\\
\hline
\multirow{1}{*}{شوانوما (نوروما)} & -\\
\hline
\multirow{1}{*}{آدنوم هیپوفیز} & -\\
\hline
\multirow{1}{*}{پینه آلوما (پینئوسیتوما، پینئوبلاستوما)} & -\\
\hline
\end{tabular} 
\end{table}

گلیومایکی از شایع‌ترین نوع تومورهای مغزی است و از سلول‌های گلیال، که سلول‌های حمایتی مغز هستند، نشأت می‌گیرد. این نوع تومور حدود 33 درصد از تمام تومورهای مغز و دستگاه عصبی مرکزی و تقریباً 80 درصد از تومورهای بدخیم مغزی را شامل می‌شود. \persianfootnote{سازمان بهداشت جهانی}\LTRfootnote{WHO}  این نوع تومورها را بر اساس ویژگی‌های \persianfootnote{ذره بینی}\LTRfootnote{Microscopic} و رفتار تومور به چهار درجه مختلف تقسیم بندی می‌کند.
\begin{enumerate}
    \item درجه \lr{I} و \lr{II}: این نوع گلیوماها با درجه پایین  شناخته می‌شوند و دارای رشد آهسته‌ای هستند که نزدیک به خوش‌خیمی است. آنها به طور معمول کمتر خطرناک هستند.
    \item درجه \lr{III} و \lr{IV}: گلیوماهای با درجه بالا در این دسته قرار می‌گیرند و به عنوان سرطانی و تهاجمی شناخته می‌شوند. تهاجمی‌ترین نوع گلیوماها هستند. معمولاً پس از عمل جراحی به مرگ منتهی می‌شوند و پیشرفت توده سرطانی در آنها پس از عمل جراحی حتی بیشتر می‌شود.
\end{enumerate}
در جدول \ref{tab:glioma-grades} درجات گلیوما و ویژگی های آنها را می توانید مشاهده کنید.
\begin{table}[ht]
\caption[درجات گلیوما و ویژگی های آنها]{درجات گلیوما و ویژگی های آنها\cite{ranjbarzadeh}}
\label{tab:glioma-grades}
\centering
\onehalfspacing
\begin{tabular}{|c|r|}
\hline درجه & ویژگی ها\\ 
\hline 
\multirow{4}{*}{\lr{I}} & ظاهری نزدیک به معمولی \\
                    & کمترین بدخیم \\
                    & رشد آهسته سلول ها \\
                    & معمولا نشان دهنده بقای طولانی مدت است\\
\hline
\multirow{4}{*}{\lr{II}} &سلول ها نسبتاً آهسته رشد می کنند\\
                    &ظاهر نسبتاً غیر طبیعی\\
                    &قادر به حمله به بافت مجاور است\\
                    &در برخی موارد به عنوان درجه بالاتر تکرار می شود\\

\hline
\multirow{4}{*}{\lr{III}} &به طور فعال سلول های غیر طبیعی ایجاد می کند\\
                    &ظاهر غیر طبیعی\\
                    &به بافت طبیعی نفوذ کنید\\
                    &تمایل به عود، اغلب به عنوان درجه بالاتر\\

\hline
\multirow{4}{*}{\lr{IV}} & تکثیر سریع سلول های غیر طبیعی\\
                    &ظاهر بسیار غیر طبیعی\\
                    &ناحیه سلول های مرده (نکروز) در مرکز\\
                    &ساختن عروق خونی جدید برای ادامه رشد\\
\hline
\end{tabular} 
\end{table}

بیمارانی که تومور از نوع گلیوما دارند، معمولاً نیاز به درمان‌های مختلفی دارند، از جمله شیمی درمانی و پرتو درمانی، که به کاهش سرعت رشد توده سرطانی و کاهش اندازه آن کمک می‌کند. این نوع تومورها نیازمند درمان‌های مکمل و متخصصین متعدد حوزه پزشکی هستند\cite{KOMORI2022126}.
\\
% تومورهای مغزی و دستگاه عصبی مرکزی شامل تومورهای مننژیوم، تومورهای غلاف عصبی و تومورهای هیپوفیز هستند. این تومورها برخی از شایع‌ترین تومورهای مغزی هستند و تقریباً همیشه خوش‌خیم (غیرسرطانی) هستند. تومورهای این دسته‌ها به شکل‌ها و مکان‌های مختلف در مغز و دستگاه عصبی مرکزی ظاهر می‌شوند.
% \begin{itemize}
%     \item تومورهای مننژیوم: این تومورها از مننژها، که پوشش‌های حفاظتی مغز و نخاع هستند، نشأت می‌گیرند. علائم این تومورها معمولاً با تغییرات در فشار مغزی و سایر علائم مرتبط با مغز مرتبط هستند. بیشتر این تومورها خوش‌خیم هستند\cite{brunasso2022spotlight}.
%     \item تومورهای غلاف عصبی: این تومورها از غلاف عصبی نشأت می‌گیرند و معمولاً به عنوان شهاب‌سنگ‌های عصبی شناخته می‌شوند. آنها اغلب خوش‌خیم هستند و علائم کمتری ایجاد می‌کنند\cite{somatilaka2022malignant}.
%     \item تومورهای هیپوفیز: هیپوفیز یک غدد ترشح کننده مهم در مغز است و تومورهای هیپوفیز ممکن است تولید بیش‌افزایی هورمون‌ها و اختلالات مرتبط با کنترل هورمونی ایجاد کنند. این تومورها نیز معمولاً خوش‌خیم هستند\cite{asa2022overview}.
% \end{itemize}
قطعه‌بندی تصویر نقش مهمی در تشخیص و درمان تومورهای مغزی ایفا می‌کند. به عنوان مثال، تشخیص دقیق نوع گلیوما و دقیق‌ترین تصویر برداری ممکن می‌تواند به برنامه‌ریزی جراحی، پیگیری بعد از عمل جراحی و تعیین شدت بقا کمک کند. این روش‌ها معمولاً توسط تیم‌های پزشکی و تخصصی در این زمینه انجام می‌شوند.



\section{روش‌های تصویربرداری از تومور گلیوما}
تشخیص زودهنگام گلیوما نقش مهمی در روند درمان دارد. روش‌های تصویربرداری پزشکی مختلف مانند \persianfootnote{توموگرافی کامپیوتری}\LTRfootnote{CT} ، \persianfootnote{توموگرافی کامپیوتری تک فوتونی}\LTRfootnote{SPECT}،\persianfootnote{توموگرافی انتشار پوزیترون}\LTRfootnote{PE}،\persianfootnote{طیف سنجی تشدید مغناطیسی}\LTRfootnote{MRS}،و \persianfootnote{تصویربرداری تشدید مغناطیسی}\LTRfootnote{MRI} اطلاعات ارزشمندی در مورد شکل، اندازه، محل و \persianfootnote{سوخت و ساز}\LTRfootnote{Metabolism} تومورهای مغزی ارائه می کنند که به امر تشخیص کمک می‌کند. در ادامه جزئیات بیشتری در مورد چند روش پراستفاده تر ارائه می‌دهیم: \\
تصویربرداری با \persianfootnote{پرتوهای ایکس}\LTRfootnote{X-ray}  یکی از روش‌های قدیمی برای مشاهده استخوان‌های بدن انسان است. این روش بر پایه انتقال نور است و از \persianfootnote{تابش یونیزان‌کننده}\LTRfootnote{Ionizing Radiation} برای تهیه تصاویر دوبعدی از بافت‌های انسان استفاده می‌کند. در این روش، پرتوهای ایکس به مواد با چگالی مختلف جذب می‌شود. تصاویر پرتوهای ایکس عموماً سریع و ساده هستند. با این حال، این روش قادر به مشاهده تومورها پشت استخوان‌های جمجمه یا نخاع نیست. بنابراین، از تصویربرداری از پرتوهای ایکس برای پیش‌بینی وجود تومور مغزی به ندرت استفاده می‌شود.
\begin{figure}[ht]
\centerline{\includegraphics[width=12cm]{images/x-ray.jpg}}
\caption[تصویر برداری با پرتو های ایکس]{تصویر برداری با پرتو های ایکس\cite{ranjbarzadeh}}
\label{fig:x-ray}
\end{figure}

تصویربرداری کامپیوتری توموگرافی از یک سری از تصاویر پرتوهای ایکس از زوایا و زمان های مختلف استفاده می‌کند. این روش برای اسکن اعضای داخلی بدن استفاده می‌شود و اطلاعاتی را در مورد بافت‌های نرم و عروق خونی در اعضا ارائه می‌دهد. \lr{CT} یک روش پیشرفته‌تر از پرتوهای ایکس است و اطلاعات بیشتری در مورد آسیب استخوان، نواحی غیرطبیعی مغز و همچنین مکان تومورهای مغزی ارائه می‌دهد. فرآیند \lr{CT} از اطلاعات به دست آمده در فرآیند جذب تصاویر دوبعدی عرضی از عضو استفاده می‌کند و سپس آنها را ترکیب کرده تا تصویر سه‌بعدی ایجاد کند و یک دید بهتر از عضو را ارائه دهد. \lr{CT} محاسبه‌شده در مواردی مانند خونریزی، لخته خون یا سرطان توسط اکثر متخصصان پزشکی توصیه می‌شود. با این حال، تصاویر \lr{CT} از پرتوهای ایکس قوی‌تری استفاده می‌کنند که تابش یونیزان‌کننده را منتشر می‌کنند و پتانسیل برای تأثیرگذاری بر بافت‌های زنده و افزایش خطر سرطان را دارند. تحقیقات نشان داده‌اند که خطر تابش در تصاویر \lr{CT} در مقایسه با تشخیص‌های عادی پرتو های ایکس حداقل ۱۰۰ برابر بیشتر است\cite{soomro2022image}.
\\
\begin{figure}[h]
\centerline{\includegraphics[width=9cm]{images/ct.png}}
\caption[مجموعه ای از تصاویر \lr{CT}]{مجموعه ای از تصاویر توموگرافی کامپیوتری با افزایش کنتراست از کمترین به بالاترین کنتراست\cite{soomro2022image}}
\label{fig:ct}
\end{figure}
\\
\lr{MRI} از این واقعیت فیزیکی استفاده می‌کند که پروتون هایی که در هسته اتم هیدروژن قرار گرفته‌اند مانند کره زمین در حول محور و با سرعت زیادی می‌چرخند و در نتیجه یک میدان معناطیسی در اطراف خود تشکیل می‌دهند.
\\
 \lr{MRI} یکی از مهمترین روش‌های تصویربرداری در پزشکی مدرن است و بسته به پروتکل ثبت تصویر، می‌تواند تا حدود زیادی به کنتراست بهتری بین بافت‌های نرم مغز و دیگر ساختارها نسبت به \lr{CT} برسد. این روش تصویربرداری بر خلاف روش‌هایی مانند \lr{CT}، \lr{PET} و \lr{SPECT}، از تابش یونیزان کننده جلوگیری می‌کند و برای تولید تصاویر پزشکی از میدان مغناطیسی خارجی قوی و امواج \persianfootnote{فرکانس رادیویی}\LTRfootnote{RF} بهره می‌برد.
\\

\section{تصویر برداری تشدید مغناطیسی}
در تصویربرداری \lr{MRI}، اتم‌های هیدروژن در بافت‌ها درمعرض تابش مغناطیسی قرار می‌گیرند و سپس با ارسال امواج \lr{RF} تحریک می‌شوند. این امواج \lr{RF} به میزان ویژه‌ای از اتم‌های هیدروژن انرژی می‌دهند و وقتی اتم‌ها به حالت اولیه خود بازگشت می‌کنند، یک سیگنال که تابعی از میزان انرژی تولید شده است رامنتشر می‌کنند. این سیگنال‌ها در دستگاه کامپیوتری تجزیه و تحلیل می‌شوند تا تصاویر مختلف بافت‌ها و اعضای داخلی بدن ایجاد شوند.
\\
از مزایای اصلی \lr{MRI} نسبت به \lr{CT} این است که از تابش یونیزان کننده معاف است و می‌تواند کنتراست بهتری برای تصویرگری بافت‌های نرم و ساختارهای مختلف ارائه دهد. به همین دلیل، این روش به عنوان یک ابزار ارزشمند در تشخیص و تصویرگری مشکلات پزشکی در اعضای مختلف بدن استفاده می‌شود.
\\
 \lr{MRI} بر اساس خواص مغناطیسی هسته های هیدروژن کار می کند. این خواص مغناطیسی تابعی از زمان آسایش و پالس های فرکانسی رادیویی که به آنها اعمال می گردد هستند و نوع عکسبرداری \lr{MRI} باتوجه به این پارامترها مشخص می شوند. در تصویر برداری از مغز غالباً از دو زمان آسایش \lr{T1} و \lr{T2} استفاده می‌کنند که بستگی زیادی به نوع بافتی دارد که پروتونهای هیدروژن در آن قرار دارند. از این رو، \lr{MRI} قادر است تصاویر ساختمان اعضایی از بدن انسان که حاوی هیدروژن هستند را بهتر ترسیم کند.
% \begin{itemize}
%     \item زمان آسایش (\lr{T1}): زمانی که هیدروژن‌های هسته‌ها پس از تحریک با پالس \lr{RF} به حالت اولیه باز می‌گردند. \lr{T1} به ترتیب به تفکیک و کنتراست میان بافت‌ها کمک می‌کند. برای مثال، بافت‌هایی که دارای \lr{T1} بلندتر هستند (معمولاً بافتهای چربی) به خوبی در تصاویر \lr{T1} تبدیل می‌شوند.
%     \item زمان آسایش (\lr{T2}): زمانی که هیدروژن‌های هسته‌ها پس از تحریک با پالس \lr{RF} به حالت اولیه بازگشت می‌یابند. \lr{T2} به تبدیل و تفکیک بافت‌ها و افزایش کنتراست میان آنها کمک می‌کند. به عنوان مثال، بافت‌هایی که دارای \lr{T2} بلندتر هستند (معمولاً بافتهای آبی) در تصاویر \lr{T2} بهتر مشخص می‌شوند.
% \end{itemize}
با استفاده از این دو زمان آسایش \lr{T1} و \lr{T2}، \lr{MRI} قادر به ترسیم بهتر ساختمان اعضای مختلف بدن انسان است، زیرا هر بافت دارای خواص مغناطیسی خاص خود است که در تصاویر نمایان می‌شود. این امکان به پزشکان و متخصصان پزشکی کمک می‌کند تا مشکلات و تغییرات در بافت‌ها را به دقت تشخیص دهند و برنامه‌های درمانی مناسب را ارائه دهند.
\\
\lr{MRI} یک روش ارزشمند در تجزیه و تحلیل تصاویر پزشکی برای مشاهده و بخش‌بندی اطلاعات در مورد بافت مغز و نیز ساختار بدن است. تصاویر \lr{MRI} نتایج بهتری برای تشخیص تومورهای مغزی نسبت به روش‌های غیر تهاجمی دیگر ارائه می‌دهد. \lr{MRI} به دلیل کنتراست خوبی که فراهم می‌کند، اطلاعات واضح‌تری از بافت نرم مغز ارائه می‌دهد. این تصاویر شامل سه تایی از صفحات \persianfootnote{محوری}\LTRfootnote{Axial}، \persianfootnote{سهمی}\LTRfootnote{Sagittal} و \persianfootnote{تاجی}\LTRfootnote{Coronal} برای اطلاعات دقیق در مورد مغز، نخاع فشرده و ساختمان عروق مغز می‌باشد. یکی از مزایای \lr{MRI} عدم استفاده از تابش است، بنابراین این روش ایمنی بیشتری نسبت به \lr{CT} دارد. علاوه بر این، \lr{MRI} به دلیل کنتراست خوب آن اطلاعات دقیقی از مغز و ساختمان عروق را ارائه می‌دهد. صفحات اصلی تصویربرداری  \lr{MRI} در شکل \ref{fig:mr-plans} نشان داده شده است.
\\
\begin{figure}[ht]
\centerline{\includegraphics[width=13cm]{images/mr-plans.png}}
\caption[سه صفحه \lr{MRI} مغز]{چپ نمای محوری، وسط نمای سهمی، راست نمای تاجی\cite{soomro2022image}}
\label{fig:mr-plans}
\end{figure}
این سه صفحه امکان مشاهده ساختمان مغز را فراهم می‌کند.
% رایج ترین دنباله‌های \lr{MRI} برای تجزیه و تحلیل مغز، دنباله‌های \lr{T1-weighted}، \lr{T2-weighted} و \lr{fluid-attenuated reversal recovery} (\lr{FLAIR}) هستند. دنباله \lr{T1-weighted} کنتراست خاکستری و سفیدی فراهم می‌کند. دنباله \lr{T2-weighted} حساس به محتوای آب است و بنابراین برای بیماری‌هایی مناسب است که آب درون بافت مغز جمع می‌کند. تصاویر \lr{T1} و \lr{T2 weighted} همچنین برای تفکیک مایع مغزی مورد استفاده قرار می‌گیرند. مایع مغزی بی‌رنگ است و در مغز و نخاع فشرده وجود دارد و در تصاویر \lr{T1-weighted} و \lr{T2-weighted} به صورت تاریک نمایان می‌شود. سومین دنباله \lr{MRI} تصویربرداری \lr{FLAIR} است، چرا که به تصویربرداری \lr{T2-weighted} شباهت دارد. فرآیند \lr{FLAIR} تمایز میان مایع مغزی و ناهنجاری‌های مغزی در تصویربرداری \lr{MR} است. محدوده تورم CSF در دنباله \lr{FLAIR} \lr{MRI} قابل مشخص شدن است. 
با تغییر ترتیب ارسال و جمع آوری پالس های \lr{RF} می توان اشکال مختلفی از تصاویر را تولید کرد. اسکن های \lr{T1-wighted} و \lr{T2-weighted} توالی های \lr{MRI} هستند. \persianfootnote{زمان کوتاه برای اکو }\LTRfootnote{ Short
Time to Echo}(\lr{TE}) و \persianfootnote{زمان تکرار}\LTRfootnote{Repetition Time} (\lr{TR}) برای ایجاد تصاویر \lr{T1} استفاده می شود. ویژگی های \lr{T1} بافت در درجه اول مسئول تعیین کنتراست و روشنایی تصویر است. از سوی دیگر، زمان های \lr{TE} و \lr{TR} طولانی تری برای ایجاد تصاویر \lr{T2} استفاده می شود. ویژگی های \lr{T2} بافت، کنتراست و روشنایی را در این تصاویر تعیین می کند. \persianfootnote{بازیابی وارونگی ضعیف شده با سیال }\LTRfootnote{Fluid
Attenuated Inversion Recovery}(\lr{Flair}) سومین دنباله ای است که اغلب مورد استفاده قرار می گیرد. زمان بندی \lr{TE} و \lr{TR} دنباله \lr{Flair} به طور قابل توجهی طولانی تر از زمان بندی یک تصویر \lr{T2} است. علاوه بر این ۳ دنباله، معمولا یک دنباله دیگر هم در تصاویر \lr{MRI} تومور مغز به نام \lr{T1ce} وجود دارد. \lr{T1ce} یک روش تصویربرداری با کنتراست است که برای برجسته کردن و تجسم نواحی جریان خون غیرطبیعی، تومورها، التهاب و سایر شرایط در بدن با استفاده از ماده حاجب مبتنی بر گادولینیوم استفاده می‌شود\cite{ranjbarzadeh}.
شکل \ref{fig:mr-modals} توالی های مختلف مربوط به \lr{MRI} مغز را نشان می‌دهد.
\lr{MRI} مغز با تومور و بدون تومور در شکل \ref{fig:mr-tumor} نشان داده شده است.
\\
\begin{figure}[h]
\centerline{\includegraphics[width=16cm]{images/mr-modals.pdf}}
\caption[نمونه ای از دنباله های مختلف \lr{MRI} مغز]{    (الف) برچسب ‌قطعه‌بندی درستی مرجع (\lr{GT}) ارائه شده توسط متخصصان حوزه . (ب) روش \lr{FLAIR}. (پ) روش \lr{T1}. (ت) روش \lr{T1ce}. (ث) روش \lr{T2}\cite{zhu2023brain}.
}
\label{fig:mr-modals}
\end{figure}
\begin{figure}
\centerline{\includegraphics[width=8cm]{images/mr-tumor.png}}
\caption[مقایسه \lr{MRI} مغز با تومور و بدون تومور]{الف)\lr{MRI}مغز بدون تومور. ب) \lr{MRI} مغز با تومور\cite{soomro2022image}.}
\label{fig:mr-tumor}
\end{figure}
\\
با توجه مطالب گفته شده در این قسمت، \lr{MRI} به دلایل زیر برای بررسی بافت های نرم و دستگاه عصبی بهترین انتخاب ممکن است:

\begin{enumerate}
    \item\lr{MRI} نیازی به فرکانس بالا یا تابش یونیزان کننده که برای بدن انسان مضر است ندارد؛ در حالی که \lr{CT}، \lr{SPECT} و \lr{PET} از تابش‌های یونیزان‌کننده استفاده می‌کنند.
    \item \lr{MRI} نسبت به دیگر روش های تصویربرداری مشابه، تصاویر با کنتراست بالاتری تولید می‌کند.
    \item \lr{MRI} قادر به تولید هم تصاویر ۲ بعدی و هم تصاویر ۳ بعدی است؛ به ویژه تصاویر ‌۳ بعدی جزئیات بیشتری از ناحیه ناهنجار مغز ارائه می‌دهند.
    \item \lr{MRI} حاوی فرآیند گرفتن اطلاعات مربوط به هر دو جنبه عملکردی و ساختمانی تومور مغز در طول یک فرآیند اسکن مشترک است.
\end{enumerate}

\section{قطعه‌بندی تومورهای مغزی}
قطعه‌بندی تصویر یک عمل مهم در تجزیه و تحلیل تصویر است که به صورت خودکار یا دستی تصویر را به قطعات کوچکتر تقسیم می‌کند. این قطعات کوچک، که به عنوان نواحی یا مناطق شناخته می‌شوند، ممکن است تنها یک پیکسل یا چند پیکسل (یا وکسل\LTRfootnote{Voxel}) را شامل شوند و می‌توانند با کلاس‌های خاص مثل یک کلاس مشخص برچسب‌گذاری شوند.
\\
در پردازش تصویر پزشکی، قطعه‌بندی تصاویر پزشکی می‌تواند به تشخیص و افرازهای خاص مانند اعضا در تصویر بپردازد. به عبارت دیگر، در تصاویر پزشکی معمولاً بخش‌هایی از تصویر که ناحیه‌های مختلف مغز را نشان می‌دهند، شناسایی می‌شوند. این اقدام به پزشکان کمک می‌کند تا به ویژگی‌ها و مشخصات مرتبط با هر ناحیه (مانند اندازه، شکل و موقعیت مکانی آن‌ها و غیره) دسترسی داشته باشند. استفاده از قطعه‌بندی در تصاویر پزشکی امکانات متعددی را فراهم می‌کند، اما قطعه‌بندی دستی معمولاً وقت‌بر و پر زحمت است. به همین دلیل، تلاش‌های زیادی در جهت توسعه روش‌های خودکار قطعه‌بندی تصویر در حوزه پزشکی انجام شده است. با استفاده از الگوریتم‌ها و فنون پیشرفته پردازش تصویر، می‌توان روش‌های قطعه‌بندی خودکار را ارائه داد که به پزشکان در تحلیل تصاویر پزشکی و استخراج اطلاعات مرتبط کمک می‌کنند.
\\
با پیشرفت روش‌های هوش مصنوعی و یادگیری عمیق، قطعه‌بندی تصویر به صورت خودکار و با دقت بالا امکان‌پذیر شده است. این روش‌ها می‌توانند زمان و زحمت پزشکان را کاهش داده و به دقت و سرعت تشخیص بیماری‌ها و مشخصات مرتبط با آن‌ها کمک کنند\cite{xu2022medical}.
\\
شبکه \lr{UNet} بر اساس شبکه کاملاًمتصل ساخته شده است و معماری آن برای کار با تصاویر آموزشی محدود و ارائه بخش های دقیق تر اصلاح و توسعه یافته است. ایده اصلی \lr{FCN} جایگزینی اپراتورهای ادغام با اپراتورهای نمونه برداری در یک شبکه معمولی است. در نتیجه، این لایه ها وضوح خروجی را افزایش می دهند. برای محلی‌سازی دقیق، ویژگی‌های با وضوح بالا تولید شده از مسیر \persianfootnote{انقباض}\LTRfootnote{Contraction} با خروجی‌های نمونه‌سازی شده ترکیب می‌شوند. سپس، یک لایه پیچشی متوالی یاد می گیرد که خروجی دقیق تری ارائه دهد.

\begin{figure}[h]
\centerline{\includegraphics[width=13cm]{images/unet}}
\caption[معماری \lr{UNet}]{معماری \lr{UNet}\cite{punn2022modality}.}
\label{fig:unet}
\end{figure}
یکی از اصلاحات قابل توجه در معماری \lr{UNet} وجود تعداد زیادی کانال ویژگی در بخش نمونه‌برداری است که به شبکه اجازه می‌دهد اطلاعات ریزدانه را در لایه‌های با وضوح بالاتر توزیع کند. در نتیجه، مسیر \persianfootnote{انبساط}\LTRfootnote{Expansion} در مقایسه با مسیر انقباض، تقارن‌های بیشتر یا کمتری را نشان می‌دهد و یک معماری \lr{U} شکل ایجاد می‌کند. هیچ اتصال کاملاً متصل بین لایه‌ها در این شبکه وجود ندارد و فقط از بخش‌های معتبر هر لایه پیچشی استفاده می‌شود. این به این معنی است که نقشه قطعه بندی فقط حاوی پیکسل هایی است که یک پس زمینه کامل در تصویر ورودی دارند. این استراتژی،  قطعه‌بندی یکپارچه تصاویر بزرگ و دلخواه را از طریق استراتژی \persianfootnote{کاشی همپوشانی}\LTRfootnote{Overlap-Tile} امکان پذیر می کند. برای پیش‌بینی پیکسل‌ها در ناحیه مرزی تصویر، اطلاعات زمینه از دست رفته با استفاده از \persianfootnote{آینه‌سازی}\LTRfootnote{Mirroring} برون‌یابی می‌شود\cite{walsh2022using}.
\\
بلوک کدگذار، که به عنوان مسیر انقباض شناخته می شود، از یک سری عملیات شامل پیچش های$3\times3$ و به دنبال آن تابع فعال سازی \lr{ReLU}، همانطور که در شکل \ref{fig:unet_operation} الف نشان داده شده است، استفاده می کند. برای پردازش تصاویر بزرگ در کاشی های جداگانه، یک حاشیه 1 پیکسل قربانی می شود. نقشه‌های ویژگی به‌دست‌آمده، که با ترکیب لایه پیچشی و \lr{ReLU} به دست می‌آیند، سپس از طریق عملیات max-pooling کوچک می‌شوند، همانطور که در شکل \ref{fig:unet_operation} ب نشان داده شده است. پس از هر لایه پیچشی، فعال‌سازی و ادغام حداکثر، تعداد کانال‌های ویژگی دو برابر می‌شود و به دلیل استفاده از ادغام حداکثر ابعاد مکانی آن کاهش می‌یابد.
\\
این نقشه‌های ویژگی استخراج‌شده از طریق یک لایه \persianfootnote{گلوگاه}\LTRfootnote{Bottleneck} که از لایه‌های پیچشی آبشاری استفاده می‌کند، به بلوک کدگشا منتقل می‌شوند. بلوک کدگشا که به عنوان مسیر گسترش نیز شناخته می شود، شامل یک سری لایه های \persianfootnote{پیچشی بالا}\LTRfootnote{UpConv} می باشد که در شکل \ref{fig:unet_operation} پ  نشان داده شده است. این ‌لایه های پیچشی بالا، هر بردار ویژگی را با استفاده از یک هسته به یک پنجره خروجی $2\times2$ ترسیم می‌کنند و به دنبال آن یک تابع فعال‌سازی \lr{ReLU}. در نهایت، لایه خروجی یک ماسک قطعه‌بندی با دو کانال تولید می کند که رده های پس زمینه و پیش زمینه را نشان می دهد.
\\
\begin{figure}[h]
\centerline{\includegraphics[width=15cm]{images/unet_operation}}
\caption[خلاصه ای از عملیات در \lr{UNet}]{
    الف) لایه پیچشی با فیلتر $3\times3$ و تابع فعالسازی \lr{ReLU}.ب)لایه ادغام حداکثر با فیلتر $2\times2$.پ)عملیات لایه پیچشی بالا با فیلتر $2\times2$ \cite{punn2022modality}. }
\label{fig:unet_operation}
\end{figure}
فاز انقباض به تمایل استخراج ویژگی‌های سطح بالا و پایین می‌پردازد، در حالی که فاز انبساط از ویژگی‌های آموخته شده در فاز انقباض مربوطه (\persianfootnote{اتصال‌های میانبر}\LTRfootnote{Skip Connection}) پیروی می‌کند تا تصویر را به ابعاد دلخواه با کمک پیچش‌های جابجا شده یا عملیات نمونه‌برداری بازسازی کند. مدل \lr{UNet} در چالش \lr{ISBI 2015} برنده شد و از مدل‌های قبلی خود بهتر عمل کرد.
با توجه به توانایی مدل \lr{UNet} در آخرین پیشرفت های علمی، نمونه های مختلف بر اساس تنوع در عملیات پیچش و ادغام، اتصالات میانبر، ترتیب مؤلفه‌ها در هر لایه و رویکردهای ترکیبی که از مدل‌های یادگیری عمیق جدیدتر استفاده می‌کند، برای مقابله با چالش‌های مرتبط با برنامه‌های مختلف پیشنهاد شده‌اند\cite{punn2022modality}.
\\

