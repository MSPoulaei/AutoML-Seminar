
\chapter{تعاریف و مفاهیم مبنایی }
\thispagestyle{empty}
\section{مقدمه}
برای درک عمیق روش‌های نوین ارائه شده در این گزارش، ضروری است که با مفاهیم و تعاریف پایه‌ای در حوزه‌های یادگیری ماشین خودکار، مدل‌های زبانی بزرگ و سیستم‌های عامل-محور آشنا شویم. این فصل به مرور این مبانی نظری اختصاص دارد و به عنوان سنگ بنایی برای تحلیل‌های ارائه شده در فصل سوم عمل خواهد کرد.

\section{یادگیری خودکار ماشین}
یادگیری ماشین خودکار به فرایند خودکارسازی وظایف تکراری و تخصصی در طراحی و استقرار مدل‌های یادگیری ماشین اطلاق می‌شود. هدف نهایی یادگیری ماشین خودکار، کاهش نیاز به دخالت متخصصان انسانی و تسریع فرایند کشف و اعتبارسنجی مدل‌های کارآمد است. این فرایند معمولاً شامل مراحلی چون \persianfootnote{پیش‌پردازش داده}\LTRfootnote{data preprocessing}، مهندسی ویژگی، انتخاب مدل و بهینه‌سازی ابرپارامتر می‌باشد.

\subsection{بهینه‌سازی ابرپارامتر}
\persianfootnote{ابرپارامترها}\LTRfootnote{Hyperparameters} پارامترهایی در مدل یادگیری ماشین هستند که مقدار آن‌ها پیش از آغاز فرایند آموزش تنظیم می‌شود (برخلاف پارامترها یا وزن‌ها که در طول آموزش یادگرفته می‌شوند). بهینه‌سازی ابرپارامتر فرایند یافتن ترکیبی از ابرپارامترها است که منجر به بهترین \persianfootnote{کارایی}\LTRfootnote{performance} مدل بر روی یک مجموعه داده اعتبارسنجی می‌شود. روش‌های متداول برای این کار شامل \persianfootnote{جستجوی شبکه‌ای}\LTRfootnote{Grid Search}، \persianfootnote{جستجوی تصادفی}\LTRfootnote{Random Search} و بهینه‌سازی بیزی است. این فرایند به دلیل \persianfootnote{هزینه محاسباتی}\LTRfootnote{computational cost} بالای ارزیابی هر \persianfootnote{پیکربندی}\LTRfootnote{configuration}، بسیار چالش‌برانگیز است.

\begin{align}
\lambda^\star &= \arg\min_{\lambda}\; \mathcal{L}_V^\star(\lambda)
= \arg\min_{\lambda}\; \mathcal{L}_V\!\big(\lambda, \mathbf{w}^\star(\lambda)\big), \\
\text{s.t.}\quad \mathbf{w}^\star(\lambda) &= \arg\min_{\mathbf{w}}\; \mathcal{L}_T(\lambda, \mathbf{w}) .
\end{align}

توضیح کوتاه:
در این فرمول:
- $\lambda$: بردار ابرپارامترها
- $\mathbf{w}$: پارامترهای قابل‌آموزش مدل
- $\mathcal{L}_T$: تابع زیان روی داده‌های آموزش (Train)
- $\mathcal{L}_V$: تابع زیان/معیار روی داده‌های اعتبارسنجی (Validation)
- $\mathbf{w}^\star(\lambda)$: وزن‌های بهینه‌ی حاصل از آموزش با ابرپارامترِ $\lambda$
معنا: ابتدا برای هر $\lambda$ مدل را با کمینه‌کردن $\mathcal{L}_T$ آموزش می‌دهیم تا $\mathbf{w}^\star(\lambda)$ به‌دست آید؛
سپس $\lambda$ای را برمی‌گزینیم که $\mathcal{L}_V$ روی مجموعه‌ی اعتبارسنجی را کمینه کند.

\subsection{جستجوی معماری شبکه عصبی}
جستجوی معماری عصبی یکی از زیرشاخه‌های یادگیری ماشین خودکار است که به طور خاص بر خودکارسازی طراحی معماری \persianfootnote{شبکه‌های عصبی عمیق}\LTRfootnote{Deep Neural Networks} تمرکز دارد. به جای تکیه بر \persianfootnote{شهود}\LTRfootnote{intuition} و طراحی دستی توسط متخصصان، جستجوی معماری عصبی از \persianfootnote{الگوریتم‌های جستجو}\LTRfootnote{search algorithms} (مانند \persianfootnote{یادگیری تقویتی}\LTRfootnote{Reinforcement Learning} یا \persianfootnote{روش‌های تکاملی}\LTRfootnote{evolutionary methods}) برای کاوش در \persianfootnote{فضای طراحی}\LTRfootnote{design space} وسیع معماری‌ها استفاده می‌کند. هدف، یافتن معماری‌ای است که بهترین توازن را میان دقت و \persianfootnote{کارایی محاسباتی}\LTRfootnote{computational efficiency} برقرار کند.

\section{مدل های زبانی بزرگ}
مدل‌های زبانی بزرگ مدل‌های زبانی هستند که با استفاده از معماری ترنسفورمر و بر روی حجم عظیمی از داده‌های متنی آموزش دیده‌اند. این مدل‌ها دارای میلیاردها پارامتر هستند و توانایی‌های شگفت‌انگیزی در \persianfootnote{درک زمینه}\LTRfootnote{understanding context}، تولید متن \persianfootnote{منسجم}\LTRfootnote{coherent} و استدلال از خود نشان می‌دهند.

\subsection{تاریخچه}
توسعه مدل‌های زبانی بزرگ با معرفی معماری ترنسفورمر \cite{vaswani2017attention} شتاب گرفت. مدل‌هایی مانند BERT \cite{devlin2019bert} و GPT \cite{brown2020language} \persianfootnote{چارچوب نظری}\LTRfootnote{paradigm} \persianfootnote{پیش‌آموزش}\LTRfootnote{pre-training} و \persianfootnote{ریزتنظیم}\LTRfootnote{fine-tuning} را تثبیت کردند. نسل‌های بعدی این مدل‌ها، با افزایش چشمگیر \persianfootnote{مقیاس}\LTRfootnote{scale} داده و پارامترها، به توانایی‌های \persianfootnote{یادگیری درون-متنی}\LTRfootnote{In-Context Learning (ICL)} دست یافتند که به آن‌ها اجازه می‌دهد وظایف جدید را بدون ریزتنظیم و تنها با دیدن چند مثال در \persianfootnote{دستور}\LTRfootnote{prompt} انجام دهند.

\subsection{معماری}
پایه و اساس اکثر مدل‌های زبانی بزرگ مدرن، معماری ترنسفورمر است که بر مکانیزم \persianfootnote{توجه خودی}\LTRfootnote{self-attention} تکیه دارد. این مکانیزم به مدل اجازه می‌دهد تا \persianfootnote{وابستگی‌های دوربرد}\LTRfootnote{long-range dependencies} را در متن مدل کند و به بخش‌های مختلف ورودی وزن‌های متفاوتی اختصاص دهد. مدل‌ها معمولاً از پشته‌ای از لایه‌های \persianfootnote{رمزگذار}\LTRfootnote{encoder} (مانند \lr{BERT}) یا \persianfootnote{رمزگشا}\LTRfootnote{decoder} (مانند \lr{GPT}) یا هر دو (مانند \lr{T5}) تشکیل شده‌اند \cite{vaswani2017attention}.

\subsection{کاربردها}
مدل‌های زبانی بزرگ کاربردهای متنوعی از جمله \persianfootnote{ترجمه ماشینی}\LTRfootnote{machine translation}~\cite{wang-etal-2023-document-level}، \persianfootnote{خلاصه‌سازی متن}\LTRfootnote{text summarization}، پرسش و پاسخ و تولید محتوا دارند~\cite{minaee2024large}. اخیراً، توانایی آن‌ها در \persianfootnote{تولید کد}\LTRfootnote{code generation}~\cite{gao2023pal} و حل مسائل منطقی~\cite{pan-etal-2023-logic}، درهای جدیدی را برای استفاده از آن‌ها در حوزه‌های فنی‌تر مانند مهندسی نرم‌افزار و یادگیری ماشین خودکار گشوده است.

\subsection{تفکر و عمل}
فراتر از تولید \persianfootnote{پاسخ‌های ایستا}\LTRfootnote{static responses}، مدل‌های زبانی بزرگ می‌توانند برای \persianfootnote{تفکر}\LTRfootnote{Reasoning} و \persianfootnote{عمل}\LTRfootnote{Action} نیز به کار روند. چارچوب‌هایی مانند ReAct \cite{yao2023react} نشان دادند که چگونه یک مدل زبانی بزرگ می‌تواند به صورت \persianfootnote{درهم‌تنیده}\LTRfootnote{interleaved}، \persianfootnote{ردپاهای استدلالی}\LTRfootnote{reasoning traces} (تفکر) و اقدامات (مانند جستجو در وب یا اجرای دستور) تولید کند. این قابلیت، سنگ بنای استفاده از مدل‌های زبانی بزرگ به عنوان هسته تصمیم‌گیرنده در عامل‌های خودمختار است.

\section{عامل}
در زمینه هوش مصنوعی، عامل به سیستمی اطلاق می‌شود که در یک \persianfootnote{محیط}\LTRfootnote{environment} قرار دارد، آن را از طریق \persianfootnote{حسگرها}\LTRfootnote{sensors} ادراک می‌کند و از طریق \persianfootnote{کنشگرها}\LTRfootnote{actuators} بر آن تأثیر می‌گذارد تا به اهداف خود دست یابد. عامل‌های زبانی نوع خاصی از عامل‌ها هستند که از مدل‌های زبانی بزرگ به عنوان موتور استدلال اصلی خود برای پردازش ادراکات (اغلب متنی)، \persianfootnote{برنامه‌ریزی}\LTRfootnote{planning} و انتخاب اقدام استفاده می‌کنند \cite{wang2024survey}.

\subsection[تک‌عاملی]{\persianfootnote{تک‌عاملی}\LTRfootnote{Single-Agent}}
یک سیستم  شامل یک عامل واحد است که تمام وظایف ادراک، استدلال و عمل را به تنهایی انجام می‌دهد. در زمینه یادگیری ماشین خودکار، این می‌تواند یک عامل مبتنی بر مدل زبانی بزرگ باشد که کل خط لوله بهینه‌سازی را از ابتدا تا انتها مدیریت می‌کند \cite{wang2024survey}.

\subsection[چندعاملی]{\persianfootnote{چندعاملی}\LTRfootnote{Multi-Agent}}
سیستم‌های چندعاملی شامل دو یا چند عامل هستند که در یک محیط مشترک با یکدیگر \persianfootnote{تعامل}\LTRfootnote{interact} می‌کنند. این تعامل می‌تواند \persianfootnote{همکارانه}\LTRfootnote{collaborative} (برای دستیابی به یک هدف مشترک) یا \persianfootnote{رقابتی}\LTRfootnote{competitive} باشد. در یادگیری ماشین خودکار، می‌توان از سیستم‌های چندعاملی برای \persianfootnote{تفکیک وظایف}\LTRfootnote{task decomposition} استفاده کرد؛ برای مثال، یک عامل متخصص تحلیل داده، یک عامل متخصص \persianfootnote{تولید معماری}\LTRfootnote{architecture generation} و یک عامل \persianfootnote{منتقد}\LTRfootnote{critic} برای ارزیابی نتایج \cite{wang2024survey}.

\section[تولید تقویت‌شده با بازیابی]{\persianfootnote{تولید تقویت‌شده با بازیابی}\LTRfootnote{Retrieval-Augmented Generation (RAG)}}
تولید تقویت‌شده با بازیابی \cite{Lewis2020RAG} روشی است که مدل‌های زبانی بزرگ را با یک \persianfootnote{سازوکار بازیابی اطلاعات}\LTRfootnote{information retrieval mechanism} خارجی ترکیب می‌کند. به جای تکیه صرف بر دانش پارامتری (ذخیره شده در وزن‌های مدل)، تولید تقویت‌شده با بازیابی ابتدا اطلاعات مرتبط را از یک \persianfootnote{مجموعه متون}\LTRfootnote{corpus} یا \persianfootnote{پایگاه دانش}\LTRfootnote{knowledge base} بازیابی می‌کند و سپس این اطلاعات را به مدل زبانی بزرگ می‌دهد تا پاسخ نهایی را بر اساس آن تولید کند. این روش به کاهش \persianfootnote{توهم}\LTRfootnote{hallucination} و افزایش دقت و \persianfootnote{به‌روز بودن}\LTRfootnote{up-to-dateness} اطلاعات کمک می‌کند \cite{xia2025ragselfreasoning}.

\subsection{پایگاه دانش}
پایگاه دانش در تولید تقویت‌شده با بازیابی معمولاً مجموعه‌ای از \persianfootnote{اسناد}\LTRfootnote{documents} است. این اسناد اغلب به \persianfootnote{قطعات}\LTRfootnote{chunks} کوچکتر تقسیم شده و به صورت \persianfootnote{نمایش‌های برداری}\LTRfootnote{vector representations} (یا \persianfootnote{نهفتگی‌ها}\LTRfootnote{embeddings}) در یک \persianfootnote{پایگاه داده برداری}\LTRfootnote{vector database} ذخیره می‌شوند تا \persianfootnote{بازیابی مبتنی بر شباهت معنایی}\LTRfootnote{semantic similarity retrieval} به سرعت انجام شود.

\subsection{ترکیب با عامل}
عامل‌های هوشمند می‌توانند از تولید تقویت‌شده با بازیابی به عنوان یک \persianfootnote{ابزار}\LTRfootnote{tool} کلیدی استفاده کنند. زمانی که یک عامل یادگیری ماشین خودکار با یک مجموعه داده جدید روبرو می‌شود، می‌تواند از تولید تقویت‌شده با بازیابی برای جستجو در پایگاه دانش استفاده کند. این دانش بازیابی‌شده به عامل کمک می‌کند تا تصمیمات آگاهانه‌تری در مورد مسئله اتخاذ کند \cite{singh2025agenticrag}.

