\pagenumbering{arabic}

\chapter{مقدمه}
\thispagestyle{empty}
\section{شرح مسأله}
تجزیه و تحلیل تصاویر پزشکی از نظر تحقیقات اولیه پزشکی و درمان بالینی همواره نقش مهمی داشته و به متخصصان پزشکی کمک کرده است تابیماری‌ها را بهتر درک کنند و چالش‌های بالینی به منظور بهبود کیفیت مراقبت‌های بهداشتی را بررسی کنند . در این میان، \persianfootnote{قطعه‌بندی تومورهای مغزی}\LTRfootnote{‌Brain Tumor Segmentation} به طور ویژه توجه محققان را به خود جلب کرده و به عنوان یک چالش کلیدی مورد بررسی قرار گرفته است.
\\
با وجود تلاش‌های بسیار محققان و پزشکان، قطعه‌بندی دقیق تومورهای مغز همچنان با چالش‌های مختلفی مواجه است. این چالش‌ها شامل عدم قطعیت در مکان تومور، عدم قطعیت \persianfootnote{شکل‌شناسی}\LTRfootnote{Morphology} آن، تصویربرداری با کنتراست کم، \persianfootnote{سوگیری حاشیه‌نویسی}\LTRfootnote{Annotation Bias} و \persianfootnote{عدم تعادل داده‌ها}\LTRfootnote{Data Imbalance} می‌شود.
\\
با ورود فناوری های پیشرفته و روش‌های \persianfootnote{یادگیری عمیق}\LTRfootnote{Deep Learning} به عرصه تجزیه و تحلیل تصاویر پزشکی، شرایط امیدوارکننده‌ای به وجود آمده است. روش‌های مبتنی بر یادگیری عمیق با عملکرد قدرتمند خود، به قطعه‌بندی تومورهای مغزی پرداخته و توانایی استخراج خودکار \persianfootnote{نمایش‌های ویژگی}\LTRfootnote{Feature Representation} از داده‌ها را دارند. این ابزارها به پزشکان و محققان این امکان را می‌دهد تا به نتایج دقیق‌تری برسند و از عملکرد پایدار و بهتری در تحلیل تومورهای مغزی بهره‌مند شوند. این پیشرفت‌ها در تجزیه و تحلیل تصاویر پزشکی نشان دهنده اهمیت روزافزون این حوزه در ارتقاء تشخیص و درمان بیماران دارای تومورهای مغزی است و امیدواریم که با ادامه تحقیقات و توسعه فناوری‌ها، موفقیت‌های بیشتری در این زمینه به دست آوریم.
\\
مغز به عنوان یکی از اعضای حیاتی بدن، دارای صد میلیارد سلول عصبی به نام \persianfootnote{نورون}\LTRfootnote{Neuron} است. بر اساس گزارشات علمی، تومورهای مغز به عنوان دهمین علت اصلی مرگ و میر در میان جمعیت بزرگسالان و کودکان در کشورهای توسعه‌یافته تلقی می‌شوند. پیش‌بینی شده بود که در سال 2022، تومورهای مغزی باعث مرگ حدود 18280 نفر از بزرگسالان در ایالات متحده آمریکا خواهند شد. این تومورها، که به عنوان \persianfootnote{تومورهای جمجمه‌ای}\LTRfootnote{Cranial Tumors} شناخته می‌شوند، شامل مجموعه‌ای متنوع از سلول‌های سرطانی هستند که در بافت‌های جمجمه‌ای مغز شروع به رشد می‌کنند و می‌توانند در شرایط مختلف از خوش‌خیم تا پیشرفته باشند. تومور مغز به علت افزایش نرخ تقسیم سلول‌ها و ناپایداری آنها آغاز می‌شود. هر قسمتی از مغز یا جمجمه می‌تواند تومور مغزی تولید کند. بیش از 150 نوع مختلف تومور مغزی وجود دارد که به دو دسته اصلی تومورهای سرطانی و غیرسرطانی تقسیم می‌شوند\cite{ranjbarzadeh}.
%، شامل پوشش محافظتی مغز، پایه جمجمه، سینوسهای سر، تنفسی، و مکان‌های مختلف دیگر.

\section{معرفی حوزه سمینار}
 در این گزارش ابتدا به بررسی ساختار و ویژگی های تومور های مغزی و تصاویر تومور های مغزی خواهیم پرداخت. پس از آن علاوه بر بررسی جنبه های فنی مانند طراحی معماری شبکه و بررسی نتایج آنها، روش های مختلف را با هم مقایسه خواهیم کرد.
\begin{figure}[ht]
\centerline{\includegraphics[width=11cm]{images/structure.pdf}}
\caption{ساختار موضوعات بررسی شده در گزارش}
\label{fig:shir}
\end{figure}



\section{ساختار گزارش}
این گزارش در ۴ فصل تهیه شده است. در فصل اول به بیان مقدمه و شرح مسئله پرداخته ایم. در فصل دوم مبانی و مفاهیم حوزه تصویر برداری پزشکی، تومور های مغزی و روش های یادگیری عمیق شرح داده شده است. در فصل سوم روش های مختلف برای قطعه‌بندی تومور های مغزی بررسی و مقایسه شده است. در فصل چهارم نیز به جمع‌بندی و نتیجه‌گیری از مطالب آورده شده در گزارش پرداخته‌ایم.
