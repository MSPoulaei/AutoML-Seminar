\chapter{نتیجه گیری و کارهای آینده}
\thispagestyle{empty}
\section{نتیجه‌گیری}
این گزارش سمینار به بررسی و تحلیل رویکردهای نوظهور در یادگیری ماشین خودکار پرداخت که از قابلیت‌های مدل‌های زبانی بزرگ در قالب سیستم‌های عامل-محور بهره می‌برند. مرور ادبیات نشان داد که این حوزه به سرعت در حال فاصله گرفتن از بهینه‌سازهای جعبه‌سیاه سنتی و حرکت به سوی روندهای بر اساس دانش، تفسیرپذیر و خودمختار است.

ما روش‌های موجود را از سه منظر کلیدی طبقه‌بندی کردیم: معماری عامل (تک‌عاملی در برابر چندعاملی)، منابع دانش (درونی در برابر بیرونی) و قالب خروجی (ساختاریافته، کد، یا ترکیبی).

یافته‌ها حاکی از آن است که عامل‌های تک‌عاملی، به‌ویژه آن‌هایی که از دستوردهی تکراری یا عملگرهای تکاملی استفاده می‌کنند، برای بهینه‌سازی ابرپارامترها و جستجوی معماری عصبی محدود مؤثر هستند. در مقابل، معماری‌های چندعاملی با تفکیک نقش (مانند پژوهشگر و توسعه‌دهنده)، پتانسیل بیشتری برای مدیریت خطوط لوله پیچیده و بلند-افق یادگیری ماشین خودکار از خود نشان می‌دهند.

ادغام تولید تقویت‌شده با بازیابی یک پیشرفت کلیدی است که به عامل‌ها اجازه می‌دهد تا از دانش ایستای خود فراتر رفته و از ادبیات پژوهشی، \persianfootnote{مخازن کد}\LTRfootnote{code repositories} و نتایج آزمایش‌های گذشته برای اتخاذ تصمیمات آگاهانه‌تر استفاده کنند. در نهایت، قالب خروجی نشان‌دهنده یک توازن میان \persianfootnote{خوانایی ماشینی}\LTRfootnote{machine-readability} (مانند \lr{JSON}) و بیانگری (مانند تولید کد کامل) است، که رویکردهای ترکیبی به عنوان راه‌حلی میانه در حال ظهور هستند. در مجموع، یادگیری ماشین خودکار عامل-محور یک حوزه تحقیقاتی بسیار پویا است که نویدبخش خودکارسازی هوشمندانه‌تر و کارآمدتر فرایندهای علم داده است.

\section{مسائل باز و کارهای قابل انجام}
حوزه یادگیری ماشین خودکار مبتنی بر عامل‌های زبانی، با وجود پیشرفت‌های سریع، همچنان در مراحل اولیه تکامل خود قرار دارد و مملو از مسائل باز و زمینه‌های پژوهشی است. بر اساس تحلیل‌های ارائه‌شده در این گزارش، می‌توان کارهای آتی را در چند محور اصلی دسته‌بندی کرد:

\textbf{بهینه‌سازی برای منابع محدود و افزایش کارایی:} اکثر روش‌های فعلی بر مدل‌های زبانی بزرگ و پرهزینه (مانند خانواده های \lr{GPT, Claude, Gemini}) متکی هستند. یک مسئله باز کلیدی، تطبیق این رویکردها برای سناریوهایی با منابع محاسباتی محدود است. کارهای آینده می‌تواند شامل تحقیق بر روی \textbf{ریزتنظیم کردن مدل‌های زبانی با پارامتر کم} (مثلاً \lr{Gemma-3-12B} یا مدل‌های کوچک‌تر) برای وظایف خاص یادگیری ماشین خودکار مانند بهینه‌سازی ابرپارامترها باشد تا به جای اتکای صرف به یادگیری درون-متنی، دانش تخصصی مستقیماً در پارامترهای مدل تزریق شود؛ همچنین استفاده از تکنیک‌های \persianfootnote{تقطیر دانش}\LTRfootnote{knowledge distillation} برای انتقال قابلیت‌های استدلال یک مدل زبانی بزرگ به یک مدل کوچک‌تر و کارآمدتر که بتواند به عنوان یک عامل یادگیری ماشین خودکار سبک عمل کند، مسیر مهمی به شمار می‌رود.

\textbf{یکپارچه‌سازی و مدیریت دانش پیشرفته:} همانطور که در تحلیل‌ها مشاهده شد، بسیاری از سیستم‌ها هنوز به دانش درونی (تاریخچه بهینه‌سازی) محدود هستند. غنی‌سازی عامل‌ها با دانش خارجی یک مرز پژوهشی مهم است، از جمله طراحی \textbf{پایگاه‌های دانش پویا و خود-بهبودگر} که نه تنها مقالات علمی را شامل شوند، بلکه از مخازن کد (مانند \lr{GitHub})، بحث‌های فنی (مانند \lr{Stack Overflow}) و نتایج بنچمارک‌های عمومی (مانند \lr{Papers with Code}) نیز تغذیه شوند؛ نیز توسعه راهبردهایی برای \textbf{مدیریت دانش متناقض}، به این صورت که اگر یک مقاله روشی را پیشنهاد دهد اما نتایج تجربی عامل چیز دیگری را نشان دهد، عامل بتواند این تضاد را حل کند.

\textbf{سیستم‌های چندعاملی با تیم‌های پژوهشی موازی (مسابقه عامل‌ها برای ریزتنظیم):} طراحی یک چارچوب که در آن چند «تیم» عامل (با نقش‌های متمایز مانند پژوهشگر، مهندس داده و مربی) هر کدام یک مدل را برای یک تسک مشخص فاین‌تیون کنند و در نهایت بهترین مدل انتخاب یا تجمیع شود، مستلزم وجود \textbf{هماهنگ‌کننده مرکزی} برای تشکیل تیم‌ها، تخصیص نقش‌ها، تعریف قرارداد رابط (مشخصات داده، بودجه، محدودیت‌ها) و زمان‌بندی اجرای آزمایش‌ها است؛ علاوه بر این، باید \textbf{راهبرد انتخاب یا تجمیع} مشخص شود تا بر اساس معیارهای چندهدفه (دقت، زمان، حافظه) بهترین مدل انتخاب گردد و در صورت نزدیک‌بودن عملکردها، \lr{Ensembling} سبک (مثلاً \lr{logit averaging}) و تحلیل جبهه پارتو ارزیابی شود؛ همچنین \textbf{اشتراک دانش بین تیم‌ها} از طریق حافظه مشترک مبتنی بر بازیابی (\lr{RAG}) برای دسترسی به یافته‌ها، تنظیمات موفق و خطاهای رایج و استانداردسازی ثبت آزمایش‌ها جهت تکرارپذیری اهمیت دارد؛ و نهایتاً تضمین \textbf{ایمنی و بازتولیدپذیری} با تثبیت بذر تصادفی، قفل‌کردن نسخه‌ وابستگی‌ها، نظارت بر نشتی داده و تعریف آزمون‌های سلامت برای جلوگیری از خطاهای کدنویسی عامل‌ها ضروری است.

\textbf{سامانه عامل‌محور برای انتخاب و تنظیم بهینه مدل در شرایط }. تمرکز بر \textbf{سه تصمیم کلیدی} در انتقال یادگیری است: انتخاب مدل پایه، استراتژی ریزتنظیم و داده‌افزایی؛ و در این میان، تولید یک \textbf{طرح پیکربندی} ساختاریافته (\lr{JSON}) به‌عنوان خروجی میانی پیشنهاد می‌شود. معماری \textbf{چندعاملی} شامل عامل تحلیل‌گرِ دیتاست، عامل استراتژیست برای برنامه‌ریزی و عامل اجراکننده است که به \textbf{موتور اجرا} متصل می‌شود تا پیکربندی تولیدشده را به کد قابل‌اعتماد تبدیل کند.

\section{معرفی موضوع مورد نظر برای پایان نامه}

با توجه به تحلیل‌های صورت‌گرفته در این سمینار و بررسی مسائل باز موجود، موضوع زیر که بر اساس ایده سوم پیشنهادی تدوین شده است، به عنوان یک حوزه پژوهشی نوآورانه، کاربردی و دارای عمق کافی برای یک پایان‌نامه کارشناسی ارشد انتخاب می‌گردد.

\subsection*{عنوان پیشنهادی}
\textbf{طراحی و پیاده‌سازی یک سیستم یادگیری ماشین خودکار عامل-محور برای انتخاب و تنظیم بهینه مدل }

در بسیاری از کاربردهای عملی یادگیری عمیق، دسترسی به داده‌های برچسب‌دار انبوه امکان‌پذیر نیست. در چنین شرایطی، رویکرد غالب، استفاده از \persianfootnote{انتقال یادگیری}\LTRfootnote{Transfer Learning} از طریق ریزتنظیم کردن مدل‌های از پیش‌آموزش‌دیده است. با این حال، موفقیت این رویکرد به شدت به مجموعه‌ای از تصمیمات پیچیده و به هم وابسته بستگی دارد:

\textbf{انتخاب \persianfootnote{مدل پایه}\LTRfootnote{backbone}:} کدام مدل از میان ده‌ها مدل موجود در استخر مدل ها (مثلاً \lr{ResNet}, \lr{ViT}, \lr{EfficientNet}) برای دیتاست و تسک مورد نظر مناسب‌تر است؟

\textbf{انتخاب استراتژی ریزتنظیم:} آیا باید کل مدل را ریزتنظیم کرد، فقط لایه‌های آخر را آموزش داد، یا از روش‌های کارآمد پارامتری مانند \lr{LoRA} و دیگر تکنیک‌های \lr{PEFT} استفاده نمود؟

\textbf{انتخاب روش‌های پیش‌پردازش و داده‌افزایی:} کدام تکنیک‌های داده‌افزایی (مانند \lr{CutMix}, \lr{Mixup}, \lr{RandAugment}) با توجه به ویژگی‌های دیتاست، بیشترین بهبود را در عملکرد مدل ایجاد می‌کنند؟

فضای جستجوی حاصل از ترکیب این انتخاب‌ها بسیار بزرگ و گسسته است و روش‌های سنتی یادگیری ماشین خودکار برای کاوش مؤثر در آن با چالش مواجه هستند. این پژوهش قصد دارد با بهره‌گیری از یک سیستم چندعاملی مبتنی بر مدل زبانی بزرگ، این فرآیند تصمیم‌گیری را خودکار و هوشمند سازد. هدف اصلی، ساخت عاملی است که بتواند با تحلیل ویژگی‌های دیتاست ورودی، یک \textbf{\persianfootnote{طرح اجرایی}\LTRfootnote{Configuration Plan}} بهینه تولید کند که بهترین ترکیب از مدل، روش ریزتنظیم و تکنیک‌های داده‌افزایی را برای دستیابی به حداکثر دقت با حداقل منابع ممکن، مشخص نماید.

\textbf{طراحی معماری عامل-محور:} یک گردش کار مبتنی بر عامل‌های مدل زبانی بزرگ طراحی می‌شود که وظایف را به صورت منطقی تقسیم کند؛ این معماری می‌تواند شامل یک "عامل تحلیل‌گر" برای استخراج فراداده از دیتاست، یک "عامل استراتژیست" برای تولید طرح پیکربندی بر اساس تحلیل‌ها و دانش پیشین، و یک "عامل اجراکننده" برای اجرای کد مبتنی بر کانفیگ تولید شده باشد.

\textbf{توسعه مکانیزم تصمیم‌گیری مبتنی بر مدل زبانی بزرگ:} \persianfootnote{مهندسی دستور}\LTRfootnote{Prompt Engineering} و طراحی ساختار ورودی/خروجی به گونه‌ای صورت می‌گیرد که مدل زبانی بزرگ بتواند بر اساس ویژگی‌های دیتاست (مانند اندازه، تعداد کلاس‌ها، نوع داده) و محدودیت‌های منابع، استدلال کرده و تصمیمات آگاهانه بگیرد. خروجی مدل زبانی بزرگ یک فایل پیکربندی ساختاریافته (مثلاً در قالب \lr{JSON}) خواهد بود.

\textbf{پیاده‌سازی پل ارتباطی بین استدلال و اجرا:} یک موتور اجرایی ساخته می‌شود که فایل پیکربندی تولیدشده توسط مدل زبانی بزرگ را تفسیر کرده و آن را به کد پایتون قابل اجرا تبدیل و اجرا نماید. این رویکرد، استدلال سطح بالای مدل زبانی بزرگ را از اجرای سطح پایین و مستعد خطای کد جدا می‌کند.

\textbf{ارزیابی جامع سیستم:} عملکرد سیستم پیشنهادی بر روی چندین دیتاست بنچمارک با اندازه‌های متفاوت ارزیابی می‌شود و نتایج (دقت و هزینه محاسباتی) با روش‌های پایه مانند انتخاب تصادفی، یک استراتژی ریزتنظیم ثابت و در صورت امکان، سایر ابزارهای یادگیری ماشین خودکار مقایسه خواهد شد.

این پژوهش در چند جنبه دارای نوآوری است: (۱) به جای تمرکز بر یک جزء منفرد مانند بهینه‌سازی ابرپارامترها یا جستجوی معماری عصبی، یک \textbf{خط لوله کامل و یکپارچه} برای انتقال یادگیری را هدف قرار می‌دهد. (۲) از مدل زبانی بزرگ نه به عنوان یک بهینه‌ساز جعبه-سیاه، بلکه به عنوان یک \textbf{موتور استدلال و برنامه‌ریزی} برای تولید یک طرح اجرایی شفاف و قابل تفسیر استفاده می‌کند. (۳) استفاده از یک \textbf{فایل پیکربندی به عنوان خروجی میانی}، یک راهکار نوین برای ترکیب قدرت استدلال مدل زبانی بزرگ با قابلیت اطمینان و استحکام سیستم‌های نرم‌افزاری کدمحور است. موفقیت این پروژه می‌تواند گام مهمی در جهت "مردمی‌سازی" استفاده بهینه از مدل‌های پایه باشد و به محققان و مهندسان کمک کند تا با سرعت و کارایی بیشتری به نتایج مطلوب دست یابند.