\chapter{نتیجه گیری و کارهای آینده}
\thispagestyle{empty}
\section{نتیجه‌گیری}
در این گزارش، به بررسی روش‌های یادگیری عمیق در زمینه قطعه‌بندی تومور مغزی در تصاویر پزشکی پرداختیم. قطعه‌بندی تومور مغزی وظیفه ای بسیار حیاتی‌ را در تصمیم‌گیری‌های بالینی و برنامه‌ریزی درمانی ایفا می‌کند. 
\\
ابتدا و قبل از هر چیز، اهمیت قطعه‌بندی دقیق و کارآمد تومور مغزی قابل تاکید است. شناسایی دقیق و محدود کردن مناطق تومور دار مبنایی است برای تصمیم‌گیری‌های بالینی و مدیریت بیماران. روش‌های یادگیری عمیق مورد بررسی در این گزارش قدرت‌های چشمگیری دارند که به طور قابل‌توجهی دقت این فرآیند را افزایش می‌دهند و احتمال کم‌کردن خطاهای انسانی و بهبود کلی بهره‌وری تشخیص را فراهم می‌کنند.
\\
ظهور معماری‌های مبتنی بر مبدل نشانگر تغییر قابل توجهی در منظر کلی تصویربرداری پزشکی است. مدل‌هایی همچون \lr{UNetR}و \lr{UNetFormer} که قادر به گرفتن وابستگی‌های دورتر و ارتباطات پیچیده در تصاویر پزشکی هستند، نشان دهنده چشم انداز بزرگی هستند. این مدل‌ها قطعه‌بندی تومور مغزی را بهبود داده و بهبود در دقت قطعه‌بندی و کارآیی محاسباتی را ارائه می‌دهند.
\\
علاوه بر این، بحث یادگیری فعال و یادگیری تعاملی، استراتژی‌های نوآورانه‌ای را برجسته کرد که از همکاری انسان و رایانه برای افزایش نتایج قطعه‌بندی استفاده می‌کنند. روش \lr{BIFSeg}، با توالی ساده از مراحل، و رویکرد راهنمای من، که به کاربران قدرت می دهد تا نیازهای خود را از طریق متن منتقل کنند، توانایی تغییر دهنده چنین تعاملاتی را نشان می دهد. این روش‌ها نه تنها به چالش‌های کمبود داده و تفسیرپذیری می‌پردازند، بلکه منجر به نتایج قطعه‌بندی مطمئن‌تر و دقیق‌تر در کاربردهای پزشکی حیاتی می‌شوند.
\\
آینده قطعه‌بندی تومور مغز در تصاویر پزشکی نویدبخش است. انتظار می‌رود روش‌های قطعه‌بندی خودکار، که توسط روش‌های یادگیری عمیق هدایت می‌شوند، نقش مهمی ایفا کنند. این روش‌ها برای تسریع و اصلاح فرآیند تشخیص، در نهایت منجر به تشخیص سریع‌تر و دقیق‌تر بیماری‌ها و علائم مرتبط می‌شوند.
\\
در پایان، اکتشاف روش‌های یادگیری عمیق در قطعه‌بندی تومور مغز به عنوان شاهدی بر تکامل و نوآوری مداوم در زمینه تصویربرداری پزشکی است. با درک اصول اساسی، پذیرش پیشرفت‌ها در معماری‌های مبتنی بر مبدل، و بهره‌گیری از توانایی یادگیری فعال و یادگیری تعاملی، جامعه پزشکی در موقعیت مناسبی قرار می‌گیرد تا دقت و کارایی قطعه‌بندی تومور مغز را افزایش دهد، در نتیجه مراقبت از بیمار را ارتقا بخشد. 
\section{مسائل باز و کارهای قابل انجام}
با توجه به مطالبی که بررسی شد و سایر آموخته هایی که از قبل داریم، می توان به موارد زیر اشاره کرد:
\begin{itemize}
    \item داده های ناکافی برای مدل های یادگیری عمیق : یکی از چالش های اصلی در زمینه قطعه‌بندی تصاویر پزشکی، دسترسی محدود به تصاویر پزشکی حاشیه‌نویسی شده است. مدل‌های یادگیری عمیق، به‌ویژه آن‌هایی که مبتنی بر مبدل هستند، اغلب به مجموعه داده‌های بزرگی برای آموزش نیاز دارند. پرداختن به موضوع کمبود داده و توسعه روش‌هایی برای آموزش مدل کارآمد با داده‌های محدود می تواند مفید باشد.
    \item قابلیت همکاری و یکپارچه سازی : ادغام سامانه‌های قطعه‌بندی مبتنی بر یادگیری عمیق در جریان کار بالینی و سامانه‌های پزشکی موجود می تواند چالش برانگیز باشد. اطمینان از قابلیت همکاری این سامانه‌ها با سامانه‌های اطلاعات بیمارستانی و آرشیو تصاویر و سیستم های ارتباطی یک مسئله مرتبط است. کارهای آینده می تواند بر ایجاد رابط ها و پروتکل های یکپارچه برای ادغام این فناوری ها در عمل بالینی متمرکز شود. به طور مثال ایجاد دستگاه های دستیار جراحی می‌تواند یک اقدام مفید در این زمینه باشد.
    \item تفسیرپذیری مدل: مدل های یادگیری عمیق، به ویژه شبکه های مبتنی بر مبدل، اغلب جعبه سیاه در نظر گرفته می شوند. درک تصمیمات اتخاذ شده توسط این مدل ها برای جلب اعتماد متخصصان پزشکی ضروری است. تحقیق در مورد روش‌ها و روش‌های تفسیرپذیری مدل برای توضیح نتایج قطعه‌بندی همچنان مهم خواهد بود.
    % \item استحکام به داده های متنوع : تصاویر پزشکی می توانند به طور قابل توجهی از نظر کیفیت، وضوح و ارائه متفاوت باشند. اطمینان از اینکه مدل های یادگیری عمیق در برابر تغییرات کیفیت تصویر و ارائه قوی هستند، یک چالش مداوم است. تحقیقات آینده می‌تواند روش‌هایی را برای بهبود سازگاری و قابلیت‌های تعمیم این مدل‌ها کشف کند.
    \item پردازش بلادرنگ : در محیط های بالینی، پردازش بلادرنگ یا تقریبا واقعی تصاویر پزشکی برای تشخیص و تصمیم گیری به موقع بسیار مهم است. تحقیق در مورد بهینه سازی مدل های یادگیری عمیق برای عملکرد بلادرنگ و شتاب سخت افزاری ارزشمند خواهد بود. مانند کاری که در مدل \lr{UNetR} شاهد بودیم.
    % \item ادغام داده های چندوجهی: تشخیص تومور مغزی اغلب شامل استفاده از روش های تصویربرداری متعدد مانند اسکن MRI، CT و PET است. توسعه مدل‌های یادگیری عمیق که می‌توانند اطلاعات را از روش‌های مختلف برای قطعه‌بندی و تشخیص دقیق‌تر ادغام کنند، یک حوزه تحقیقاتی نوظهور است.
    \item پرداختن به انواع تومورهای نادر : در حالی که بسیاری از مدل های یادگیری عمیق برای انواع تومورهای مغزی رایج موثر هستند، در پرداختن به انواع تومورهای نادر یا کمتر مطالعه شده جا برای بهبود وجود دارد. کار آینده می تواند بر گسترش قابلیت های این مدل ها برای پوشش طیف وسیع تری از تومورهای مغزی متمرکز شود.
    \item همکاری انسان و ماشین: تقویت هم افزایی بین مدل های هوش مصنوعی و کارشناسان پزشکی از طریق یادگیری فعال و یادگیری تعاملی یک جهت امیدوارکننده است. تحقیقات بیشتر در زمینه توسعه ابزارها و پلتفرم های کاربر پسند برای متخصصان پزشکی برای همکاری موثر با سیستم های هوش مصنوعی بسیار مهم است.

\end{itemize}
\section{معرفی موضوع مورد نظر برای پایان نامه}
با توجه به موضوعات بررسی شده در این گزارش و ترکیب چالش های موجود، استفاده ترکیبی از روش های گفته شده و همچنین استفاده از بعضی روش ها که تاکنون روی قطعه‌بندی تومور مغز ارزیابی نشده‌اند می تواند منجر به نتایج بهتری شود. همانطور که گفته شد یکی از مشکلات موجود، کمبود داده های برچسب خورده است. از سویی مدل های مبتنی بر مبدل نیاز زیادی به داده های برچسب خورده دارند. استفاده از یادگیری تعاملی می تواند در این زمینه مفید باشد چرا که هم بر تفسیری پذیری و قابلیت اطمینان مدل ها برای پزشکان می افزاید و هم با چالش کمبود داده های برچسب نخورده مقابله می‌کند.